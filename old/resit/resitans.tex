\documentclass{amsart}
\usepackage{a4wide}
\usepackage[pdftex]{graphicx}


\newcommand{\adj}{\operatorname{adj}}
\newcommand{\half}      {{\textstyle\frac{1}{2}}}
\newcommand{\om}{\omega}
\newcommand{\xra}{\xrightarrow}
\renewcommand{\:}       {\colon}

\newcommand{\seen}      {\mbox{\bf [seen]}}
\newcommand{\unseen}    {\mbox{\bf [unseen]}}
\newcommand{\simseen}   {\mbox{\bf [similar examples seen]}}
\newcommand{\bwk}       {\mbox{\bf [bookwork]}}

\newcounter{markcounter}
\setcounter{markcounter}{0}
\newcounter{qmarkcounter}
\setcounter{qmarkcounter}{0}

\newcommand{\question}{%
 \typeout{qmarkcounter = \arabic{qmarkcounter}}%
 \setcounter{qmarkcounter}{0} %
 \item %
}

\newcommand{\mks}[1]{
 \mbox{\bf [#1]}%
 \addtocounter{markcounter}{#1}%
 \addtocounter{qmarkcounter}{#1}%
}

\newcommand{\mkrem}{\mks}
\newcommand{\mk}{\mks{1}}

\renewcommand{\theenumii}{\roman{enumii}}
\renewcommand{\theenumiii}{\alph{enumiii}}


\begin{document}


\begin{center}
{\Large Pure Mathematics Core --- Exam solutions}
\end{center}

\renewcommand{\theenumi}{A\arabic{enumi}}

\begin{enumerate}
 \question % pfrac 3
  The general form is
  \[ \frac{x^2}{(x+2)^2} = A + \frac{B}{x+2} + \frac{C}{(x+2)^2}.\mks{2}
  \]
  Multiplying by $(x+2)^2$ gives
  \begin{align*}
   x^2 &= A(x+2)^2 + B(x+2) + C  = Ax^2 +4Ax + 4A + Bx + 2B + C \\
       &= Ax^2 + (4A+B)x + (4A+2B+C), \mk
  \end{align*}
  so $A=1$ and $4A+B=0$ and $4A+2B+C=0$, which gives $B=-4$ and $C=4$. \mk
  This means that
  \begin{align*}
   \frac{x^2}{(x+2)^2} &= 1 - \frac{4}{x+2} + \frac{4}{(x+2)^2} \\
   \int \frac{x^2}{(x+2)^2}\,dx &= x - 4\log(x+2) - \frac{4}{(x+2)}.\mks{2}
  \end{align*}

 \question % inverse 1
  If $x=f(y)=\log(1+y^2)$ \mk then $e^x=1+y^2$ \mk, so
  $e^x-1=y^2$, so $f^{-1}(x)=y=\sqrt{e^x-1}$ \mk.

 \question % composite 2
  \[ (\log\circ f\circ\exp)(x) = \log(f(\exp(x))) \mk = 
      \log(2(e^x)^3) \mk = \log(2) + 3\log(e^x) \mk = \log(2) + 3x \mk.
  \]

 \question % log 1
  We note that $1000=10^3=\sqrt{10}^6$ \mk, so
  $\sqrt{10}=(1000)^{1/6}$, so $\log_{1000}(\sqrt{10})=1/6$ \mk.

 \question % trig value
  Note that $\tan(x)$ repeats with period $\pi$ \mk, so
  \[ \tan(9999\pi/4) = \tan(9999\pi/4 - 2500\pi) = 
      \tan(-\pi/4)\mk  = -1 \mk.
  \]

 \question % identity 1
  Put $u=e^x$.  Then 
  \[ \tanh(x) = \frac{\sinh(x)}{\cosh(x)} = 
      \frac{(u-u^{-1})/2}{(u+u^{-1})/2} = \frac{u-u^{-1}}{u+u^{-1}}, \mks{2}
  \]
  so
  \begin{align*}
   1+\tanh(x)^2 &= 
    1 + \left(\frac{u-u^{-1}}{u+u^{-1}}\right)^2 = 
    1 + \frac{u^2-2+u^{-2}}{u^2+2+u^{-2}} \\
    &= \frac{(u^2+2+u^{-2})+(u^2-2+u^{-2})}{u^2+2+u^{-2}} = 
       \frac{2u^2+2u^{-2}}{u^2+2+u^{-2}} \mks{2} \\
   1-\tanh(x)^2 &= 
    1 - \left(\frac{u-u^{-1}}{u+u^{-1}}\right)^2 = 
    1 - \frac{u^2-2+u^{-2}}{u^2+2+u^{-2}} \\
    &= \frac{(u^2+2+u^{-2})-(u^2-2+u^{-2})}{u^2+2+u^{-2}} = 
       \frac{4}{u^2+2+u^{-2}} \mk \\
\intertext{ so }
   \frac{1+\tanh(x)^2}{1-\tanh(x)^2} &= 
    \frac{2u^2+2u^{-2}}{4} = \frac{e^{2x}+e^{-2x}}{2} = \cosh(2x) \mks{2}.
  \end{align*}

 \question % diff rat 2
  Put $u=x^p-x^q$, so $y=u^{1/pq}$ \mk.  Then 
  \[ du/dx=px^{p-1}-qx^{q-1} = x^{-1}(px^p-qx^q) \mk \]
  and 
  \[ \frac{dy}{du} = \frac{1}{pq} u^{1/pq - 1} = 
      \frac{1}{pq}(x^p-y^q)^{1/pq - 1} \mk.
  \]
  We therefore have
  \begin{align*}
   x(x^p-x^q)\frac{dy}{dx} &=
    x u \frac{dy}{du} \frac{du}{dx} \\
    &= x (x^p-x^q) \frac{1}{pq} (x^p-x^q)^{1/pq-1} x^{-1}(px^p-qx^q)\\
    &= (x^p-x^q)^{1/pq}(px^p-qx^q)/(pq) \\
    &= (x^p-x^q)^{1/pq}(x^p/q - x^q/p) \mks{3}. 
  \end{align*}
 
 \question % diff log 
  Put $u=x+2x^2+3x^3+4x^4$ and $y=\log(u)$, so 
  \[ y' = \frac{u'}{u} =
      \frac{1+4x+9x^2+16x^3}{x+2x^2+3x^3+4x^4} \mks{2}.
  \]

 \question % diff quotient
  The quotient rule gives
  \begin{align*}
   \frac{d}{dx}\left(\frac{x^2}{\log(x)}\right) 
    &= \frac{2x.\log(x) - x^2.\log'(x)}{\log(x)^2} \mks{2} 
     = \frac{2x\log(x) - x^2.x^{-1}}{\log(x)^2}  \mk \\
    &= \frac{2x}{\log(x)} - \frac{x}{\log(x)^2} \mk.
  \end{align*}

 \question % diff packet
  First put $u=-1/(x+a)$, so $du/dx=1/(x+a)^2=(x+a)^{-2}$ \mk.  Then put
  $v=\exp(u)=e^{-1/(x+a)}$, so the chain rule gives
  \[ \frac{dv}{dx} = (x+a)^{-2}e^{-1/(x+a)} \mk. \]
  Finally, we apply the product rule:
  \begin{align*}
   f'(x) &= \frac{d}{dx}(x^2\,v) = 
            2x.v + x^2\frac{dv}{dx} \\
         &= (2x+x^2(x+a)^{-2})e^{-1/(x+a)}  \mks{2} \\
         &= (2x^2+(4a+1)x+2a^2)x(x+a)^{-2} e^{-1/(x+a)}.
  \end{align*}

 \question % int poly exp
  We know that
  \[ \int (x^2-x+1) e^{-x}\, dx = (ax^2+bx+c) e^{-x} \]
  for some constants $a$, $b$ and $c$ \mks{2}.  To find these, we
  differentiate to get 
  \begin{align*}
   (x^2-x+1) e^{-x} &= \frac{d}{dx}((ax^2+bx+c) e^{-x}) 
    = (2ax+b)e^{-x} - (ax^2+bx+c) e^{-x} \mk \\
    &= (-ax^2+(2a-b)x+(b-c))e^{-x}. \mk
  \end{align*}
  We equate coefficients to see that $-a=1$ and $2a-b=-1$ and $b-c=1$
  \mk, which gives $a=-1$ and $b=-1$ and $c=-2$.  We conclude that
  \[ \int (x^2-x+1) e^x\, dx = -(x^2+x+2) e^{-x}. \mk \]

  Alternatively, one can integrate by parts:
  \begin{align*}
   \int (x^2-x+1)e^{-x}\, dx 
    &= (x^2-x+1)(-e^{-x}) - \int (2x-1)(-e^{-x})\,dx \\
    &= (-x^2+x-1)e^{-x} + \int(2x-1)e^{-x}\,dx \\
    &= (-x^2+x-1)e^{-x} + (2x-1)(-e^{-x}) - \int 2(-e^{-x})\,dx \\
    &= (-x^2+x-1)e^{-x} + (-2x+1)e^{-x} + \int 2e^{-x}\,dx \\
    &= (-x^2+x-1)e^{-x} + (-2x+1)e^{-x} - 2e^{-x} \\
    &= -(x^2+x+2)e^{-x}.
  \end{align*}

 \question % int exp trig 2
  We know that 
  \[ \int e^{-3x}\cos(4x)\,dx = e^{-3x}(A\cos(4x) + B\sin(4x)) \]
  for some $A$ and $B$ \mks{2}.  To find these, we differentiate and
  equate coefficients: 
  \begin{align*}
   e^{-3x}\cos(4x) &=
    \frac{d}{dx}\left(e^{-3x}(A\cos(4x) + B\sin(4x))\right) \\
    &= -3e^{-3x}(A\cos(4x) + B\sin(4x)) +
        e^{-3x}(-4A\sin(4x) + 4B\cos(4x)) \\
    &= e^{3x}((-3A+4B)\cos(4x) + (-3B-4A)\sin(4x)) \mk,
  \end{align*}
  so $-3A+4B=1$ and $-3B-4A=0$ \mk.  These equations give $A=-3/25$
  and $B=4/25$ \mk, so 
  \[ \int e^{-3x}\cos(4x)\,dx = e^{-3x}(-3\cos(4x) + 4\sin(4x))/25. \mk \]

 \question % linear 1
  The augmented matrix is
  \[ \left[\begin{array}{cccc|c}
      1 &  1 &  1 & -1 & -2 \\
      1 &  1 & -1 & -1 &  0 \\
      1 & -1 & -1 & -1 &  2 
     \end{array}\right] \mk
  \]
  This can be row-reduced as follows:
  \[ \left[\begin{array}{cccc|c}
      1 &  1 &  1 & -1 & -2 \\
      1 &  1 & -1 & -1 &  0 \\
      1 & -1 & -1 & -1 &  2 
     \end{array}\right]
     \xra{}
     \left[\begin{array}{cccc|c}
      1 &  1 &  1 & -1 & -2 \\
      0 &  0 & -2 &  0 &  2 \\
      0 & -2 & -2 &  0 &  4 
     \end{array}\right]
     \xra{}
     \left[\begin{array}{cccc|c}
      1 &  1 &  1 & -1 & -2 \\
      0 & -2 & -2 &  0 &  4 \\
      0 &  0 & -2 &  0 &  2 
     \end{array}\right]
     \xra{}
   \] \[
     \left[\begin{array}{cccc|c}
      1 &  1 &  1 & -1 & -2 \\
      0 &  1 &  1 &  0 & -2 \\
      0 &  0 &  1 &  0 & -1 
     \end{array}\right]
     \xra{}
     \left[\begin{array}{cccc|c}
      1 &  0 &  0 & -1 &  0 \\
      0 &  1 &  1 &  0 & -2 \\
      0 &  0 &  1 &  0 & -1 
     \end{array}\right]
     \xra{}
     \left[\begin{array}{cccc|c}
      1 &  0 &  0 & -1 &  0 \\
      0 &  1 &  0 &  0 & -1 \\
      0 &  0 &  1 &  0 & -1 
     \end{array}\right].\mks{2}
  \]
  There is no pivot in the last column, so the variable $z$ is
  independent \mk.  The final matrix corresponds to the equations
  $w-z=0$ and $x=y=-1$, so $(w,x,y,z)=(z,-1,-1,z)$ \mk.

  Alternatively (and more efficiently), one can manipulate the
  equations directly.  Subtracting adjacent equations gives $x=y=-1$,
  and the rest is then easy.
 \question % inverse matrix
  To get the adjugate, we write down the matrix of minors, transpose
  it, and then multiply by the associated signs:
  \[ \left[\begin{array}{ccc}
      -9 & -25 & 12 \\ -24 & 0 & -18 \\ 9 & -25 & -12
     \end{array}\right] \xra{}
     \left[\begin{array}{ccc}
      -9 & -24 & 9 \\ -25 & 0 & -25 \\ 12 & -18 & -12
     \end{array}\right]\mks{2} \xra{}
     \left[\begin{array}{ccc}
      -9 & 24 & 9 \\ 25 & 0 & 25 \\ 12 & 18 & -12
     \end{array}\right]\mks{2}.
  \]
  The determinant is the dot product of the first row of $A$ with the
  first column of $\adj(A)$, which is
  \[ \det(A) = (-3,3,4).(-9,25,12) = 150. \mks{2} \]
  This gives
  \[ A^{-1} = \frac{1}{150} 
     \left[\begin{array}{ccc}
      -9 & 24 & 9 \\ 25 & 0 & 25 \\ 12 & 18 & -12
     \end{array}\right] = 
     \left[\begin{array}{ccc}
      -3/50 & 4/25 & 3/50 \\ 1/6 & 0 & 1/6 \\ 2/25 & 3/25 & -2/25
     \end{array}\right].\mks{2}
  \]

  Alternatively, we can write down the augmented matrix and row-reduce
  it as follows:
  \[ \left[\begin{array}{ccc|ccc}
      -3 & 3 &  4 & 1 & 0 & 0 \\ 
       4 & 0 &  3 & 0 & 1 & 0 \\ 
       3 & 3 & -4 & 0 & 0 & 1 
     \end{array}\right] \xra{}
     \left[\begin{array}{ccc|ccc}
         1 &  -1 & -4/3 & -1/3 &    0 &    0 \\ 
         4 &   0 &    3 &    0 &    1 &    0 \\ 
         3 &   3 &   -4 &    0 &    0 &    1 
     \end{array}\right] \xra{}
   \] \[
     \left[\begin{array}{ccc|ccc}
         1 &  -1 & -4/3 & -1/3 &    0 &    0 \\ 
         0 &   4 & 25/3 &  4/3 &    1 &    0 \\ 
         0 &   6 &    0 &    1 &    0 &    1 
     \end{array}\right] \xra{}
     \left[\begin{array}{ccc|ccc}
         1 &  -1 & -4/3 & -1/3 &    0 &    0 \\ 
         0 &   1 &    0 &  1/6 &    0 &  1/6 \\
         0 &   4 & 25/3 &  4/3 &    1 &    0 
     \end{array}\right] \xra{}
   \] \[
     \left[\begin{array}{ccc|ccc}
         1 &   0 & -4/3 & -1/6 &    0 &  1/6 \\ 
         0 &   1 &    0 &  1/6 &    0 &  1/6 \\
         0 &   0 & 25/3 &  2/3 &    1 & -2/3
     \end{array}\right] \xra{}
     \left[\begin{array}{ccc|ccc}
         1 &   0 & -4/3 & -1/6 &    0 &  1/6 \\ 
         0 &   1 &    0 &  1/6 &    0 &  1/6 \\
         0 &   0 &    1 & 2/25 & 3/25 & -2/25
     \end{array}\right] \xra{}
   \] \[
     \left[\begin{array}{ccc|ccc}
         1 &   0 &    0 & -3/50& 4/25 & 3/50 \\ 
         0 &   1 &    0 &  1/6 &    0 &  1/6 \\
         0 &   0 &    1 & 2/25 & 3/25 & -2/25
     \end{array}\right] 
  \]
  At the final stage, the left hand block is the identity, so
  the right hand block is the inverse of the original matrix.

\end{enumerate}

\renewcommand{\theenumi}{B\arabic{enumi}}

\begin{enumerate}
 \question % range 4
  Observe that $f(x)=(x+1)^2+2$.  As $x$ runs from $-2$ to $1$ 
  (including the endpoints), $x+1$ increases from $-1$ to $2$.  Thus
  $(x+1)^2+2$ decreases from $3$ to $2$ and then increases again to
  $6$.  The range is thus $[2,6]$.  Alternatively, one can read this
  off from the graph.\mks{4}
 
 \question % int trig
  Note that $\sin(x)\cos(x)=\sin(2x)/2$ \mks{2}, so
  \[ \sin(x)^2\cos(x)^2 = \sin(2x)^2/4 = (1-\cos(4x))/8 \mks{2}.
  \]
  Thus
  \begin{align*}
   \int \sin(x)^2\cos(x)^2\, dx &=
    \tfrac{1}{8} \int 1-\cos(4x)\, dx \mks{2} \\
    &= \frac{x}{8} - \frac{\sin(4x)}{32} = \frac{4x-\sin(4x)}{32} \mks{2}.
  \end{align*}

 \question % int subs 2
  Put $u=\sin(x)$, so $du=\cos(x)\,dx$ \mks{2}.  Then
  \begin{align*}
   \int\cos(x)\log(\sin(x))\,dx &= 
    \int \log(u)\,du = u\log(u)-u \mks{3} =  \\
    &= \sin(x)(\log(\sin(x)) - 1) \mk.
  \end{align*}

 \question % int misc
  We first note that
  \begin{align*}
   \frac{d}{dx}\left(x^4(a\log(x)^2 + b\log(x) + c)\right) 
    &= 4x^3(a\log(x)^2 + b\log(x) + c) + 
       x^4(2a\log(x)/x+b/x)  \mks{2} \\
    &= x^3(4a\log(x)^2 + (4b+2a)\log(x) + (4c+b)). \mk
  \end{align*}
  This must also be equal to $x^3\log(x)^2$ for all $x$, \mk so we must
  have  
  \begin{align*}
   4a &= 1 \\
   4b+2a &= 0 \\
   4c+b &= 0, \mk
  \end{align*}
  so $a=1/4$ and $b=-1/8$ and $c=1/32$,  \mk giving
  \[ \int x^3\log(x)^2\,dx = x^4(\log(x)^2/4-\log(x)/8+1/32)
      = x^4(8\log(x)^2-4\log(x)+1)/32. \mk
  \]
  It follows that 
  \begin{align*}
   \int_1^e x^3\log(x)^2\, dx &= 
    \left[ x^4(\log(x)^2/4-\log(x)/8+1/32) \right]_1^e  \mk\\
   &= e^4(1/4-1/8+1/32) - 1^3(0/4-0/8+1/32) \\
   &= (5e^4 - 1)/32. \mk
  \end{align*}

 \question % determinant
  Put
  \[ A = \left[\begin{array}{cccc}
      -a & a & 1 \\
      1 & 0 & -a \\
      a & a & -1
     \end{array}\right].
  \]
  Then
  \begin{align*}
   \det(A) &=
    -a \det\left[\begin{array}{cc} 0 & -a \\ a & -1 \end{array}\right] 
    -a \det\left[\begin{array}{cc} 1 & -a \\ a & -1 \end{array}\right] 
    +  \det\left[\begin{array}{cc} 1 & 0 \\ a & a \end{array}\right] \mks{3}\\
   &= -a(0-(-a^2)) -a(-1-(-a^2)) + (a-0) = -a^3 + a - a^3 + a \\
   &= 2a(1-a^2). \mk
  \end{align*}
  The matrix is invertible for all $a$ where $2a(1-a^2)\neq 0$, or in
  other words all $a$ except $a=0$ or $a=\pm 1$. \mks{3}
\end{enumerate} 

\typeout{Total marks: \arabic{markcounter}}

\end{document}
