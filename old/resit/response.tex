\documentclass{amsart}

\begin{document}
\begin{center}
 {\Large PMA101R --- Response to checker's comments}
\end{center}

As you say, there is no real difference between Section~A and
Section~B.  This was also the case in the January exam, and in the
mock exam that I distributed previously.  At the beginning of last
semester I announced that there would be two sections, but the planned
distinction between them did not work well when I came to write the
exam, so I abandoned it.

\begin{itemize}
 \item[A3] I used $\log$ for $\log_e$ systematically in the course.
 \item[A4] To do this by calculator one has to know that
  $\log_{1000}(\sqrt{10})=\log(\sqrt{10})/\log(1000)$.  Two marks is a
  bit generous for that, but not absurdly so.  I would give one mark
  for a decimal approximation.
 \item[A6] The mark scheme is meant to be interpreted in the way that
  you suggest, although I did not say so explicitly.  All similar
  problems in the lecture notes etc.\ introduce the symbol $u=e^x$, so
  it is natural to write the solution in this way.
 \item[A7] I gave this an extra mark, and also specified that
  $p,q\neq 0$.
 \item[A8] The solution has been corrected.
 \item[A10] I moved the ``[\textbf{2}]'' to the penultimate line, to
  indicate that it is acceptable as a final answer.  The alternative
  form on the last line is merely recorded for marking convenience.  I
  also moved a mark to A9.
 \item[A11] The method indicated is one of the few non-A-level topics
  that I attempted to teach, so I prefer to write the model solution
  that way.  It is true, however, that many students will just
  integrate by parts.  I have included this as an alternative in the
  solutions.  The lecture notes have a careful discussion of constants
  of integration; doubtless the students absorbed only the punchline,
  that I did not expect them to be included.
 \item[A12] I added a mark for solving the simultaneous equations.
 \item[A13] I left this as it was, but reduced the credit by two
  marks. 
 \item[A14] I modified the question t ask explicitly for the
  determinant, and added a hint suggestng that students use the
  cofactor method for the inverse.  With this guidance, I think that
  the denominator should not cause trouble.  I also added an extra
  mark. 
 \item[B2] This is identical to a question on the original paper, for
  which full solutions are on the web.  Nonetheless, I have added a
  hint that the double angle formulae are relevant.
 \item[B3] I reduced the final two marks to one mark.
 \item[B5] There are various ways to save work by some initial row or
  column operations, and the students have been shown a number of
  examples of that approach.  I am of course happy to give full credit
  for any such method.
\end{itemize}
\end{document}
