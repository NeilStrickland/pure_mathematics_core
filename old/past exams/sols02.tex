\documentclass{amsart}
\usepackage{a4wide}
\usepackage[pdftex]{graphicx}


\newcommand{\half}      {{\textstyle\frac{1}{2}}}
\newcommand{\om}{\omega}
\newcommand{\xra}{\xrightarrow}
\renewcommand{\:}       {\colon}
\newcommand{\R}         {{\mathbb{R}}}
\newcommand{\st}        {\;|\;}


\renewcommand{\theenumii}{\roman{enumii}}
\renewcommand{\theenumiii}{\alph{enumiii}}

\begin{document}


\begin{center}
{\Large Pure Mathematics Core --- Solutions for January 2002 exam}
\end{center}


\begin{itemize}
 \item[1]
  \begin{itemize}
   \item[(i)]
    Put $x=f(y)=(3-y)/(5-2y)$, so $f^{-1}(x)=y$.  We have
    $(5-2y)x=3-y$ so $5x-2xy=3-y$ so $5x-3=2xy-y=(2x-1)y$.  Dividing
    by $2x-1$ gives $f^{-1}(x)=y=(5x-3)/(2x-1)$.
   \item[(ii)]
    \begin{itemize}
     \item[(a)] By the product rule, we have
      \[ \frac{d}{dx}(e^{-x}\sin(x+1)) = 
           \frac{d}{dx}(e^{-x})\,\sin(x+1) + 
           e^{-x}\,\frac{d}{dx}\sin(x+1) = 
           -e^{-x}\sin(x+1) +e^{-x}\cos(x+1) = 
           e^{-x}(\cos(x+1)-\sin(x+1)).
      \]
     \item[(b)] Put $u=x^2+3x+1$ and $y=\cos(x^2+3x+1)=\cos(u)$.  Then
      $du/dx=2x+3$ and $dy/du=-\sin(u)=-\sin(x^2+3x+1)$ so the chain
      rule gives 
      \[ \frac{dy}{dx} = \frac{dy}{du}\frac{du}{dx} = 
           -(2x+3)\sin(x^2+3x+1).
      \]
     \item[(c)] Put $u=e^x$ and $v=2+\sin(x)$, so $du/dx=e^x$ and
      $dv/dx=\cos(x)$.  Then
      \[ \frac{d}{dx}\left(\frac{e^x}{2+\sin(x)}\right) = 
          \frac{d}{dx}\left(\frac{u}{v}\right) = 
          \frac{(du/dx) v - u(dv/dx)}{v^2} = 
          \frac{e^x(2+\sin(x)) - e^x\cos(x)}{(2+\sin(x))^2} = 
          \frac{2+\sin(x)-\cos(x)}{(2+\sin(x))^2}e^x.
      \]
    \end{itemize}
   \item[(iii)]
    \begin{itemize}
     \item[(a)] Put $y=\sinh^{-1}(x)=\text{arcsinh}(x)$, so $x=\sinh(y)$.
      We then have $dx/dy=\cosh(y)$.  Using the relation
      $\cosh(y)^2-\sinh(y)^2=1$ we get
      $\cosh(y)=\sqrt{1+\sinh(y)^2}=\sqrt{1+x^2}$, so
      $dx/dy=\sqrt{1+x^2}$.  It follows that
      \[ \frac{dy}{dx} = \left(\frac{dx}{dy}\right)^{-1} = 
          \frac{1}{\sqrt{1+x^2}}.
      \]
     \item[(b)] Put $y=x^{\sin(x)}$.  Then $\log(y)=\sin(x)\log(x)$,
      so 
      \[ \frac{1}{y}\,\frac{dy}{dx} = \frac{d}{dx}\log(y) = 
          \cos(x)\log(x) + \sin(x)x^{-1}.
      \]
      It follows that  
      \[ \frac{dy}{dx} = y(\cos(x)\log(x) + \sin(x)x^{-1}) = 
           x^{\sin(x)}((\cos(x)\log(x) + \sin(x)x^{-1}).
      \]
    \end{itemize}
  \end{itemize}
 \item[2]
  \begin{itemize}
   \item[(i)]
    \begin{itemize}
     \item[(a)] Put $u=x^2$, so $du=2x\,dx$.  Then
      \[ \int x\sin(x^2)\,dx = 
         \int x\sin(u)\frac{du}{2x} = \frac{1}{2}\int\sin(u)\,du =
         -\frac{1}{2}\cos(u) = -\cos(x^2)/2.
      \]
     \item[(b)] Put $v=x^2+4$, so $dv=2x\,dx$.  Then
      \[ \int \frac{3x}{x^2+4}\,dx = 
         \frac{3}{2}\int\frac{dv}{v} = 3\log(v)/2 = 3\log(x^2+4)/2.
      \]
      On the other hand, if we put $t=x/2$ we get $dt=dx/2$ and so
      \[ \int \frac{2}{x^2+4}dx = 
         \int \frac{2}{4t^2+4}2dt = \int\frac{dt}{1+t^2} = 
         \arctan(t) = \arctan(x/2).
      \]
      Putting these together, we get
      \[ \int\frac{3x+2}{x^2+4}\,dx =
          \frac{3}{2}\log(x^2+4) + \arctan(x/2).
      \]
    \end{itemize}
   \item[(ii)]
    Put $u=\log(x)$, so $du=(dx)/x$.  Note that when $x=1$ we have
    $u=\log(1)=0$, and when $x=e$ we have $u=\log(e)=1$.  We thus have
    \[ \int_{x=1}^e\frac{dx}{x(1+\log(x)^2)} =
       \int_{u=0}^1\frac{du}{1+u^2} = 
       \left[\arctan(u)\right]_0^1 = \pi/4-0 = \pi/4.
    \]

   \item[(iii)]
    \begin{itemize}
     \item[(a)]
      Put $I_n=\int_0^3 x^ne^{3x}\,dx$.  Write $u=x^n$ (so
      $du/dx=nx^{n-1}$) and $dv/dx=e^{3x}$ (so $v=e^{3x}/3$).  We can
      then integrate by parts:
      \begin{align*}
       I_n &= \int u\frac{dv}{dx}\,dx = 
              \left[uv\right]_0^3 - \int_0^3\frac{du}{dx}v\,dx  \\
           &= \left[x^ne^{3x}/3\right]_0^3 - 
              \int_0^3 nx^{n-1}e^{3x}/3\,dx \\
           &= 3^ne^{3.3}/3 - n I_{n-1}/3 
            = 3^{n-1}e^9 - \frac{n}{3}I_{n-1} .
      \end{align*}
     \item[(b)]
      We have
      $I_0=\int_0^3e^{3x}\,dx=\left[e^{3x}/3\right]_0^3=(e^9-1)/3$,
      so 
      \begin{align*}
       \int_0^3 xe^{3x}\,dx   &= I_1 = 3^{1-1}e^9-I_0/3 
                               = e^9-e^9/9+1/9 = (8e^9 +1)/9 \\
       \int_0^3 x^2e^{3x}\,dx &= I_2 = 3e^9-2I_1/3
                               = (81 e^9 - 16e^9 - 2)/27 
                               = (65 e^9 - 2)/27. 
      \end{align*}
    \end{itemize}
  \end{itemize}
 \item[3]
  \begin{itemize}
   \item[(i)]
    The given matrix can be row-reduced as follows:
    {\tiny
    \[
     \left[\begin{array}{ccccc}
      1 & 2 & 0 & 2 & 7 \\ 
      0 & 3 & -3 & 4 & 5 \\ 
      2 & 1 & 3 & -1 & 10 \\ 
      -1 & -3 & 1 & -5 & -7
     \end{array}\right] \xra{1}
     \left[\begin{array}{ccccc}
      1 & 2 & 0 & 2 & 7 \\ 
      0 & 3 & -3 & 4 & 5 \\ 
      0 & -3 & 3 & -5 & -4 \\ 
      0 & -1 & 1 & -3 & 0
     \end{array}\right] \xra{2}
     \left[\begin{array}{ccccc}
      1 & 2 & 0 & 2 & 7 \\ 
      0 & 3 & -3 & 4 & 5 \\ 
      0 & -3 & 3 & -5 & -4 \\ 
      0 & 1 & -1 & 3 & 0
     \end{array}\right] \xra{3}
    \] \[
     \left[\begin{array}{ccccc}
      1 & 0 & 2 & -4 & 7 \\ 
      0 & 0 & 0 & -5 & 5 \\ 
      0 & 0 & 0 & 4 & -4 \\ 
      0 & 1 & -1 & 3 & 0
     \end{array}\right] \xra{4}
     \left[\begin{array}{ccccc}
      1 & 0 & 2 & -4 & 7 \\ 
      0 & 0 & 0 & -5 & 5 \\ 
      0 & 0 & 0 & 1 & -1 \\ 
      0 & 1 & -1 & 3 & 0
     \end{array}\right] \xra{5}
     \left[\begin{array}{ccccc}
      1 & 0 & 2 & 0 & 3 \\ 
      0 & 0 & 0 & 0 & 0 \\
      0 & 0 & 0 & 1 & -1 \\ 
      0 & 1 & -1 & 0 & 3
     \end{array}\right] \xra{6}
    \] \[
     \left[\begin{array}{ccccc}
      1 & 0 & 2 & 0 & 3 \\ 
      0 & 1 & -1 & 0 & 3 \\
      0 & 0 & 0 & 1 & -1 \\ 
      0 & 0 & 0 & 0 & 0
     \end{array}\right]
    \]}
    (In step 1 we subtracted $2R_1$ from $R_3$, and added $R_1$ to
    $R_4$. In step 2 we multiplied $R_4$ by $-1$. In step 3 we
    subtracted $2R_4$ from $R_1$, subtracted $3R_4$ from $R_2$, and
    added $3R_4$ to $R_3$.  In step 4 we divided $R_3$ by $4$.  In
    step 5 we added $4R_3$ to $R_1$, added $5R_3$ to $R_2$, and
    subtracted $3R_3$ from $R_4$.  Finally, in step 6 we reordered the
    rows.)

    The initial matrix was the augmented matrix for the system of
    equations in the question, and the final matrix is the augmented
    matrix for the equations
    \begin{align*}
     x + 2 z &= 3 \\
     y - z   &= 3 \\
     w       &= -1.
    \end{align*}
    It follows that $z$ is independent, and the remaining variables
    are given by 
    \begin{align*}
     x &= 3 - 2z \\
     y &= 3 + z \\
     w &= -1.
    \end{align*}
   \item[(ii)]
    {\tiny \[
     AB = 
      \left[\begin{array}{ccc}
       1 & 1 & 1 \\ 1 & 0 & 1 \\ -1 & -1 & 1
      \end{array}\right]
      \left[\begin{array}{ccc}
       1 & -1 & 1 \\ 0 & -1 & 1 \\ 1 & 0 & 1
      \end{array}\right] =
      \left[\begin{array}{ccc}
       2 & -2 & 3 \\ 2 & -1 & 2 \\ 0 & 2 & -1
      \end{array}\right]
    \] }
    {\tiny \[
     BA = 
      \left[\begin{array}{ccc}
       1 & -1 & 1 \\ 0 & -1 & 1 \\ 1 & 0 & 1
      \end{array}\right] 
      \left[\begin{array}{ccc}
       1 & 1 & 1 \\ 1 & 0 & 1 \\ -1 & -1 & 1
      \end{array}\right]=
      \left[\begin{array}{ccc}
       -1 & 0 & 1 \\ -2 & -1 & 0 \\ 0 & 0 & 2
      \end{array}\right]
    \] }
    {\tiny \[
     AB - BA = 
      \left[\begin{array}{ccc}
       2 & -2 & 3 \\ 2 & -1 & 2 \\ 0 & 2 & -1
      \end{array}\right] - 
      \left[\begin{array}{ccc}
       1 & 0 & 1 \\ -2 & -1 & 0 \\ 0 & 0 & 2
      \end{array}\right] = 
      \left[\begin{array}{ccc}
       3 & -2 & 2 \\ 4 & 0 & 2 \\ 0 & 2 & -3 
      \end{array}\right]
    \] }
   \item[(iii)]
    The matrix of coefficients for the given system of equations is 
    \[ C = \left[\begin{array}{ccc}
              1 & a &  0 \\
              2 & 1 & -1 \\
             -a & 2 &  3 
           \end{array}\right]
    \]
    We find that 
    \begin{align*}
      \det(C) &=
        \det\left[\begin{array}{cc} 1 & -1 \\ 2 & 3 \end{array}\right]
     -a \det\left[\begin{array}{cc} 2 & -1 \\ -a & 3 \end{array}\right]
     +0 \det\left[\begin{array}{cc} 2 & 1 \\ -a & 2 \end{array}\right] \\
      &= 5 - a(6-a) = a^2-6a + 5 = (a-1)(a-5).
    \end{align*}
    The equations have a unique solution unless $\det(C)=0$, which
    happens when $a=1$ or $a=5$.
  \end{itemize}
 \item[4]
  \begin{itemize}
   \item[(i)]
    \begin{itemize}
     \item[(a)] 
      We have 
      \[ \frac{7x^2-2x+13}{(x+2)(x^2-x+3)} = 
          \frac{Ax+B}{x^2-x+3} + \frac{C}{x+2} 
      \]
      (for all $x$) if and only if 
      \begin{align*}
        7x^2-2x+13 &= (Ax+B)(x+2) + C(x^2-x+3) \\
                   &= Ax^2 + 2Ax + Bx + 2B + Cx^2 - C x + 3C \\
                   &= (A+C)x^2 + (2A+B-C)x + (2B+3C).
      \end{align*}
      Putting $x=-2$ gives 
      \[ 7.(-2)^2 -2.(-2) + 13 = C((-2)^2 -(-2) + 3), \]
      or $45=9C$, so $C=5$.  Comparing coefficients in the previous
      equation gives $A+C=7$ and $2B+3C=13$, so $A=7-C=2$ and
      $B=(13-3C)/2=-1$.  We thus have
      \[ \frac{7x^2-2x+13}{(x+2)(x^2-x+3)} = 
          \frac{2x-1}{x^2-x+3} + \frac{5}{x+2}. 
      \]
     \item[(b)]
      We now integrate the above equation.  For the first term, put
      $u=x^2-x+3$, so $du=(2x-1)dx$, so
      \[ \int \frac{2x-1}{x^2-x+3} dx = \int \frac{du}{u} = 
          \log(u) = \log(x^2-x+3).
      \]
      We also have $\int(x+2)^{-1}dx=\log(x+2)$, so
      \[ \int \frac{7x^2-2x+13}{(x+2)(x^2-x+3)}dx =
          \log(x^2-x+3) + 5 \log(x+2).
      \] 
      \begin{align*}
        \int_1^2 \frac{7x^2-2x+13}{(x+2)(x^2-x+3)}dx &=
          \left[\log(x^2-x+3) + 5 \log(x+2)\right]_1^2 \\ &= 
          \log(5) + 5 \log(4) - \log(3) - 5 \log(3) = 
          \log(5) + 10\log(2) - 6\log(3).
      \end{align*} 
    \end{itemize}
   \item[(ii)]
    \begin{itemize}
     \item[(a)] We have $|x-2|<2$ iff the distance from $x$ to $2$ is
      less than $2$.  This happens when $0<x<4$, so
      $\{x\in\R:|x-2|<2\}=(0,4)$
     \item[(b)] We have $x^2-5x+4=(x-1)(x-4)$.  When $x\leq 1$, both
      factors are less than or equal to zero so the product is
      nonnegative.  When $1<x<4$ we see that $x-4<0<x-1$ so
      $(x-4)(x-1)<0$.  When $x\geq 4$ the product becomes nonnegative
      again.  Thus 
      \[ \{x\in\R\st x^2-5x+4\geq 0\} = (-\infty,1]\cup [4,\infty). \]
     \item[(c)] We have $x^4-5x^2+4=(x^2)^2-5(x^2)+4$, and part~(b)
      tells us that this is nonnegative if and only if
      $x^2\in(-\infty,1]\cup[4,\infty)$, which means $-1\leq x\leq 1$
      or $x\leq -2$ or $x\geq 2$.  Thus 
      \[ \{x\in\R\st x^4-5x^2+4\geq 0\} = 
         (-\infty,-2]\cup[-1,1]\cup[2,\infty).
      \]
    \end{itemize}
   \item[(iii)] omitted.
  \end{itemize}
\end{itemize}
\end{document}
