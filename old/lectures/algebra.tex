\documentclass[%
pdf,
neil,
colorBG,
slideColor,
]{prosper}
\usepackage{amsmath}
\usepackage[usenames,dvips]{color}

\newcommand{\GREENYELLOW}[1]{{\color{GreenYellow}#1}}
\newcommand{\YELLOW}[1]{{\color{Yellow}#1}}
\newcommand{\YLW}[1]{{\color{Yellow}#1}}
\newcommand{\GOLDENROD}[1]{{\color{Goldenrod}#1}}
\newcommand{\DANDELION}[1]{{\color{Dandelion}#1}}
\newcommand{\APRICOT}[1]{{\color{Apricot}#1}}
\newcommand{\PEACH}[1]{{\color{Peach}#1}}
\newcommand{\MELON}[1]{{\color{Melon}#1}}
\newcommand{\YELLOWORANGE}[1]{{\color{YellowOrange}#1}}
\newcommand{\ORANGE}[1]{{\color{Orange}#1}}
\newcommand{\BURNTORANGE}[1]{{\color{BurntOrange}#1}}
\newcommand{\BITTERSWEET}[1]{{\color{Bittersweet}#1}}
\newcommand{\REDORANGE}[1]{{\color{RedOrange}#1}}
\newcommand{\MAHOGANY}[1]{{\color{Mahogany}#1}}
\newcommand{\MAROON}[1]{{\color{Maroon}#1}}
\newcommand{\BRICKRED}[1]{{\color{BrickRed}#1}}
\newcommand{\RED}[1]{{\color{Red}#1}}
\newcommand{\ORANGERED}[1]{{\color{OrangeRed}#1}}
\newcommand{\RUBINERED}[1]{{\color{RubineRed}#1}}
\newcommand{\WILDSTRAWBERRY}[1]{{\color{WildStrawberry}#1}}
\newcommand{\SALMON}[1]{{\color{Salmon}#1}}
\newcommand{\CARNATIONPINK}[1]{{\color{CarnationPink}#1}}
\newcommand{\MAGENTA}[1]{{\color{Magenta}#1}}
\newcommand{\VIOLETRED}[1]{{\color{VioletRed}#1}}
\newcommand{\RHODAMINE}[1]{{\color{Rhodamine}#1}}
\newcommand{\MULBERRY}[1]{{\color{Mulberry}#1}}
\newcommand{\REDVIOLET}[1]{{\color{RedViolet}#1}}
\newcommand{\FUCHSIA}[1]{{\color{Fuchsia}#1}}
\newcommand{\LAVENDER}[1]{{\color{Lavender}#1}}
\newcommand{\THISTLE}[1]{{\color{Thistle}#1}}
\newcommand{\ORCHID}[1]{{\color{Orchid}#1}}
\newcommand{\DARKORCHID}[1]{{\color{DarkOrchid}#1}}
\newcommand{\PURPLE}[1]{{\color{Purple}#1}}
\newcommand{\PLUM}[1]{{\color{Plum}#1}}
\newcommand{\VIOLET}[1]{{\color{Violet}#1}}
\newcommand{\ROYALPURPLE}[1]{{\color{RoyalPurple}#1}}
\newcommand{\BLUEVIOLET}[1]{{\color{BlueViolet}#1}}
\newcommand{\PERIWINKLE}[1]{{\color{Periwinkle}#1}}
\newcommand{\CADETBLUE}[1]{{\color{CadetBlue}#1}}
\newcommand{\CORNFLOWERBLUE}[1]{{\color{CornflowerBlue}#1}}
\newcommand{\MIDNIGHTBLUE}[1]{{\color{MidnightBlue}#1}}
\newcommand{\NAVYBLUE}[1]{{\color{NavyBlue}#1}}
\newcommand{\ROYALBLUE}[1]{{\color{RoyalBlue}#1}}
\newcommand{\BLUE}[1]{{\color{Blue}#1}}
\newcommand{\CERULEAN}[1]{{\color{Cerulean}#1}}
\newcommand{\CYAN}[1]{{\color{Cyan}#1}}
\newcommand{\PROCESSBLUE}[1]{{\color{ProcessBlue}#1}}
\newcommand{\SKYBLUE}[1]{{\color{SkyBlue}#1}}
\newcommand{\TURQUOISE}[1]{{\color{Turquoise}#1}}
\newcommand{\TEALBLUE}[1]{{\color{TealBlue}#1}}
\newcommand{\AQUAMARINE}[1]{{\color{Aquamarine}#1}}
\newcommand{\BLUEGREEN}[1]{{\color{BlueGreen}#1}}
\newcommand{\EMERALD}[1]{{\color{Emerald}#1}}
\newcommand{\JUNGLEGREEN}[1]{{\color{JungleGreen}#1}}
\newcommand{\SEAGREEN}[1]{{\color{SeaGreen}#1}}
\newcommand{\GREEN}[1]{{\color{Green}#1}}
\newcommand{\FORESTGREEN}[1]{{\color{ForestGreen}#1}}
\newcommand{\PINEGREEN}[1]{{\color{PineGreen}#1}}
\newcommand{\LIMEGREEN}[1]{{\color{LimeGreen}#1}}
\newcommand{\YELLOWGREEN}[1]{{\color{YellowGreen}#1}}
\newcommand{\SPRINGGREEN}[1]{{\color{SpringGreen}#1}}
\newcommand{\OLIVEGREEN}[1]{{\color{OliveGreen}#1}}
\newcommand{\OLG}[1]{{\color{OliveGreen}#1}}
\newcommand{\RAWSIENNA}[1]{{\color{RawSienna}#1}}
\newcommand{\SEPIA}[1]{{\color{Sepia}#1}}
\newcommand{\BROWN}[1]{{\color{Brown}#1}}
\newcommand{\TAN}[1]{{\color{Tan}#1}}
\newcommand{\GRAY}[1]{{\color{Gray}#1}}
\newcommand{\WHITE}[1]{{\color{White}#1}}
\newcommand{\BLACK}[1]{{\color{Black}#1}}

\newcmykcolor{GreenYellow}{0.15 0 0.69 0}
\newcmykcolor{Yellow}{0 0 1 0}
\newcmykcolor{Goldenrod}{0 0.10 0.84 0}
\newcmykcolor{Dandelion}{0 0.29 0.84 0}
\newcmykcolor{Apricot}{0 0.32 0.52 0}
\newcmykcolor{Peach}{0 0.50 0.70 0}
\newcmykcolor{Melon}{0 0.46 0.50 0}
\newcmykcolor{YellowOrange}{0 0.42 1 0}
\newcmykcolor{Orange}{0 0.61 0.87 0}
\newcmykcolor{BurntOrange}{0 0.51 1 0}
\newcmykcolor{Bittersweet}{0 0.75 1 0.24}
\newcmykcolor{RedOrange}{0 0.77 0.87 0}
\newcmykcolor{Mahogany}{0 0.85 0.87 0.35}
\newcmykcolor{Maroon}{0 0.87 0.68 0.32}
\newcmykcolor{BrickRed}{0 0.89 0.94 0.28}
\newcmykcolor{Red}{0 1 1 0}
\newcmykcolor{OrangeRed}{0 1 0.50 0}
\newcmykcolor{RubineRed}{0 1 0.13 0}
\newcmykcolor{WildStrawberry}{0 0.96 0.39 0}
\newcmykcolor{Salmon}{0 0.53 0.38 0}
\newcmykcolor{CarnationPink}{0 0.63 0 0}
\newcmykcolor{Magenta}{0 1 0 0}
\newcmykcolor{VioletRed}{0 0.81 0 0}
\newcmykcolor{Rhodamine}{0 0.82 0 0}
\newcmykcolor{Mulberry}{0.34 0.90 0 0.02}
\newcmykcolor{RedViolet}{0.07 0.90 0 0.34}
\newcmykcolor{Fuchsia}{0.47 0.91 0 0.08}
\newcmykcolor{Lavender}{0 0.48 0 0}
\newcmykcolor{Thistle}{0.12 0.59 0 0}
\newcmykcolor{Orchid}{0.32 0.64 0 0}
\newcmykcolor{DarkOrchid}{0.40 0.80 0.20 0}
\newcmykcolor{Purple}{0.45 0.86 0 0}
\newcmykcolor{Plum}{0.50 1 0 0}
\newcmykcolor{Violet}{0.79 0.88 0 0}
\newcmykcolor{RoyalPurple}{0.75 0.90 0 0}
\newcmykcolor{BlueViolet}{0.86 0.91 0 0.04}
\newcmykcolor{Periwinkle}{0.57 0.55 0 0}
\newcmykcolor{CadetBlue}{0.62 0.57 0.23 0}
\newcmykcolor{CornflowerBlue}{0.65 0.13 0 0}
\newcmykcolor{MidnightBlue}{0.98 0.13 0 0.43}
\newcmykcolor{NavyBlue}{0.94 0.54 0 0}
\newcmykcolor{RoyalBlue}{1 0.50 0 0}
\newcmykcolor{Blue}{1 1 0 0}
\newcmykcolor{Cerulean}{0.94 0.11 0 0}
\newcmykcolor{Cyan}{1 0 0 0}
\newcmykcolor{ProcessBlue}{0.96 0 0 0}
\newcmykcolor{SkyBlue}{0.62 0 0.12 0}
\newcmykcolor{Turquoise}{0.85 0 0.20 0}
\newcmykcolor{TealBlue}{0.86 0 0.34 0.02}
\newcmykcolor{Aquamarine}{0.82 0 0.30 0}
\newcmykcolor{BlueGreen}{0.85 0 0.33 0}
\newcmykcolor{Emerald}{1 0 0.50 0}
\newcmykcolor{JungleGreen}{0.99 0 0.52 0}
\newcmykcolor{SeaGreen}{0.69 0 0.50 0}
\newcmykcolor{Green}{1 0 1 0}
\newcmykcolor{ForestGreen}{0.91 0 0.88 0.12}
\newcmykcolor{PineGreen}{0.92 0 0.59 0.25}
\newcmykcolor{LimeGreen}{0.50 0 1 0}
\newcmykcolor{YellowGreen}{0.44 0 0.74 0}
\newcmykcolor{SpringGreen}{0.26 0 0.76 0}
\newcmykcolor{OliveGreen}{0.64 0 0.95 0.40}
\newcmykcolor{RawSienna}{0 0.72 1 0.45}
\newcmykcolor{Sepia}{0 0.83 1 0.70}
\newcmykcolor{Brown}{0 0.81 1 0.60}
\newcmykcolor{Tan}{0.14 0.42 0.56 0}
\newcmykcolor{Gray}{0 0 0 0.50}
\newcmykcolor{Black}{0 0 0 1}
\newcmykcolor{White}{0 0 0 0}


\newcommand{\bbm}       {\left[\begin{matrix}}
\newcommand{\ebm}       {\end{matrix}\right]}
\newcommand{\bsm}       {\left[\begin{smallmatrix}}
\newcommand{\esm}       {\end{smallmatrix}\right]}
\newcommand{\bpm}       {\begin{pmatrix}}
\newcommand{\epm}       {\end{pmatrix}}
\newcommand{\bcf}[2]{\left(\begin{array}{c}{#1}\\{#2}\end{array}\right)}


\newcommand{\csch}     {\operatorname{csch}}
\newcommand{\sech}     {\operatorname{sech}}
\newcommand{\arcsinh}  {\operatorname{arcsinh}}
\newcommand{\arccosh}  {\operatorname{arccosh}}
\newcommand{\arctanh}  {\operatorname{arctanh}}

\newcommand{\range}     {\operatorname{range}}
\newcommand{\trans}     {\operatorname{trans}}
\newcommand{\trc}       {\operatorname{trace}}
\newcommand{\adj}       {\operatorname{adj}}

\newcommand{\tint}{\textstyle\int}
\newcommand{\tm}{\times}
\newcommand{\sse}{\subseteq}
\newcommand{\st}{\;|\;}
\newcommand{\sm}{\setminus}
\newcommand{\iffa}      {\Leftrightarrow}
\newcommand{\xra}{\xrightarrow}

\renewcommand{\:}{\colon}

\newcommand{\N}         {{\mathbb{N}}}
\newcommand{\Z}         {{\mathbb{Z}}}
\newcommand{\Q}         {{\mathbb{Q}}}
\renewcommand{\R}       {{\mathbb{R}}}
\newcommand{\C}         {{\mathbb{C}}}

\newcommand{\al}        {\alpha}
\newcommand{\bt}        {\beta} 
\newcommand{\gm}        {\gamma}
\newcommand{\dl}        {\delta}
\newcommand{\ep}        {\epsilon}
\newcommand{\zt}        {\zeta}
\newcommand{\et}        {\eta}
\newcommand{\tht}       {\theta}
\newcommand{\io}        {\iota}
\newcommand{\kp}        {\kappa}
\newcommand{\lm}        {\lambda}
\newcommand{\ph}        {\phi}
\newcommand{\ch}        {\chi}
\newcommand{\ps}        {\psi}
\newcommand{\rh}        {\rho}
\newcommand{\sg}        {\sigma}
\newcommand{\om}        {\omega}

\newcommand{\EMPH}[1]{\emph{\RED{#1}}}
\newcommand{\DEFN}[1]{\emph{\PURPLE{#1}}}
\newcommand{\VEC}[1]    {\mathbf{#1}}

\newcommand{\ghost}{{\tiny $\color[rgb]{1,1,1}.$}}


\title{Algebraic Manipulation}
\author{}

\begin{document}

%\slideCaption{\color{white}}

\begin{slide}{}
 {\Huge
  \vspace{6ex}
  \begin{center}
   Algebraic manipulation
  \end{center}
 }
\end{slide}

\overlays{9}{
\begin{slide}{Introduction}

\fromSlide{2}{\noindent{\bf Concepts to learn:}}
\begin{itemize}
 \fromSlide{3}{
  \item Rational functions and their poles}
\end{itemize}
\vspace{2ex}

\fromSlide{4}{\noindent{\bf Skills to learn or practice:}}
\begin{itemize}
 \fromSlide{5}{
  \item Expand out powers and products}
 \fromSlide{6}{
  \item Factorize simple expressions by inspection}
 \fromSlide{7}{
  \item Manipulate powers (using
   $a^na^m=a^{n+m}$, $(a^n)^m=a^{nm}$ and so on)}
 \fromSlide{8}{
  \item Manipulate and simplify algebraic fractions}
 \fromSlide{9}{
  \item Convert rational functions to partial fraction form}
\end{itemize}
\end{slide}
}

\overlays{12}{%
\begin{slide}{Expansion}
\begin{itemize}
 \fromSlide{2}{
  \item You should practice expanding out products and powers of
   algebraic expressions. 
 }
 \fromSlide{3}{
  \item You should check and remember the following identities:
   {\vspace{-2ex}\tiny\begin{align*}
    \fromSlide{4}{(a+b)(a-b)} & \fromSlide{4}{=a^2-b^2} 
     \hphantom{mmmmm} \\
    \fromSlide{5}{(a+b)^2} & \fromSlide{5}{=a^2+2ab+b^2} \\
    \fromSlide{6}{(a-b)^2} & \fromSlide{6}{=a^2-2ab+b^2}.
   \end{align*}}\vspace{-2ex}}
 \fromSlide{7}{
  \item Often you will need to use these when $a$ and $b$ are
   themselves complicated expressions.}
 \fromSlide{8}{
  \item {\bf Example:} To simplify
   {\tiny $(w+x+y+z)^2-(x+y+z)^2$}, \\
   put {\tiny $a=w+x+y+z$} and {\tiny $b=x+y+z$}.  Then
   {\tiny \begin{align*}
    \fromSlide{9}{(w+x+y+z)^2-(x+y+z)^2} & 
    \fromSlide{9}{=a^2-b^2}
    \onlySlide*{9}{\hphantom{=(a+b)(a-b)}}
    \fromSlide*{10}{=(a+b)(a-b)} 
    \fromSlide{9}{\hphantom{mmmm}}\\ &
    \fromSlide{11}{=(w+2x+2y+2z)w} \\ &
    \fromSlide{12}{=w^2+2xw+2yw+2zw.}
   \end{align*}}}
\end{itemize}
\end{slide}
}

\overlays{18}{
\begin{slide}{An example: Cauchy-Schwartz}
\begin{itemize}
 \fromSlide{2}{
  \item {\bf Problem:} Check the identity
   {\tiny \begin{eqnarray*}
    (x^2+y^2+z^2)(u^2+v^2+w^2) &=&
     (xu+yv+zw)^2 + \\ && (xv-yu)^2 + (yw-zv)^2 + (zu-xw)^2 \\
     & \fromSlide{18}{\geq} &
       \fromSlide{18}{(xu+yv+zw)^2}
   \end{eqnarray*}} \vspace{-4ex}}
 \fromSlide{3}{
  \item \ghost \vspace{-4ex}
   {\tiny
     \begin{eqnarray*}
      \onlySlide*{ 3}{(x^2+y^2+z^2)(u^2+v^2+w^2)}%
      \onlySlide*{ 4}{(\RED{x^2}+y^2+z^2)(\BLUE{u^2}+v^2+w^2)}%
      \onlySlide*{ 5}{(\RED{x^2}+y^2+z^2)(u^2+\BLUE{v^2}+w^2)}%
      \onlySlide*{ 6}{(\RED{x^2}+y^2+z^2)(u^2+v^2+\BLUE{w^2})}%
      \onlySlide*{ 7}{(x^2+\RED{y^2}+z^2)(\BLUE{u^2}+v^2+w^2)}%
      \onlySlide*{ 8}{(x^2+\RED{y^2}+z^2)(u^2+\BLUE{v^2}+w^2)}%
      \onlySlide*{ 9}{(x^2+\RED{y^2}+z^2)(u^2+v^2+\BLUE{w^2})}%
      \onlySlide*{10}{(x^2+y^2+\RED{z^2})(\BLUE{u^2}+v^2+w^2)}%
      \onlySlide*{11}{(x^2+y^2+\RED{z^2})(u^2+\BLUE{v^2}+w^2)}%
      \onlySlide*{12}{(x^2+y^2+\RED{z^2})(u^2+v^2+\BLUE{w^2})}%
      \fromSlide*{13}{(x^2+y^2+z^2)(u^2+v^2+w^2)} &
      \onlySlide*{3}{\hphantom{=}}
      \fromSlide*{4}{=} &
      \onlySlide*{ 4}{\RED{x^2}\BLUE{u^2} + \cdots}
      \onlySlide*{ 5}{x^2u^2 + \RED{x^2}\BLUE{v^2} + \cdots}
      \onlySlide*{ 6}{x^2u^2 + x^2v^2 + \RED{x^2}\BLUE{w^2} + \cdots}
      \onlySlide*{ 7}{x^2u^2 + x^2v^2 + x^2w^2 +}
      \onlySlide*{ 8}{x^2u^2 + x^2v^2 + x^2w^2 +}
      \onlySlide*{ 9}{x^2u^2 + x^2v^2 + x^2w^2 +}
      \onlySlide*{10}{x^2u^2 + x^2v^2 + x^2w^2 +}
      \onlySlide*{11}{x^2u^2 + x^2v^2 + x^2w^2 +}
      \onlySlide*{12}{x^2u^2 + x^2v^2 + x^2w^2 +}
      \onlySlide*{13}{x^2u^2 + x^2v^2 + x^2w^2 +}
      \fromSlide*{14}{\OLIVEGREEN{x^2u^2} + \RAWSIENNA{x^2v^2} + \PURPLE{x^2w^2} +} \\ &&
      \onlySlide*{ 7}{\RED{y^2}\BLUE{u^2} + \cdots}
      \onlySlide*{ 8}{y^2u^2 + \RED{y^2}\BLUE{v^2} + \cdots}
      \onlySlide*{ 9}{y^2u^2 + y^2v^2 + \RED{y^2}\BLUE{w^2} + \cdots}
      \onlySlide*{10}{y^2u^2 + y^2v^2 + y^2w^2 +}
      \onlySlide*{11}{y^2u^2 + y^2v^2 + y^2w^2 +}
      \onlySlide*{12}{y^2u^2 + y^2v^2 + y^2w^2 +}
      \onlySlide*{13}{y^2u^2 + y^2v^2 + y^2w^2 +}
      \fromSlide*{14}{\PURPLE{y^2u^2} + \OLIVEGREEN{y^2v^2} + \RAWSIENNA{y^2w^2} +} \\ &&
      \onlySlide*{10}{\RED{z^2}\BLUE{u^2} + \cdots}
      \onlySlide*{11}{z^2u^2 + \RED{z^2}\BLUE{v^2} + \cdots}
      \onlySlide*{12}{z^2u^2 + z^2v^2 + \RED{z^2}\BLUE{w^2}}
      \onlySlide*{13}{z^2u^2 + z^2v^2 + z^2w^2}
      \fromSlide*{14}{\RAWSIENNA{z^2u^2} + \PURPLE{z^2v^2} + \OLIVEGREEN{z^2w^2}} \\ &&
      \fromSlide*{3}{\hphantom{mmmmmmmmmmmmmmmmmmmmmmmmmmmmmmmmmm}}
     \end{eqnarray*}}\vspace{-6ex}}
 \fromSlide{14}{
  \item \ghost \vspace{-4ex}
   {\tiny
    \begin{eqnarray*}
     (xu+yv+zw)^2 &=&
      \onlySlide*{14}{\OLIVEGREEN{x^2u^2+y^2v^2+z^2w^2}+
                      2xyuv+2xzuw+2yzvw}%
      \onlySlide*{15}{\OLIVEGREEN{x^2u^2+y^2v^2+z^2w^2}+
                      \RED{2xyuv}+2xzuw+2yzvw}
      \onlySlide*{16}{\OLIVEGREEN{x^2u^2+y^2v^2+z^2w^2}+
                      \RED{2xyuv}+2xzuw+\BLUE{2yzvw}}
      \fromSlide*{17}{\OLIVEGREEN{x^2u^2+y^2v^2+z^2w^2}+
                      \RED{2xyuv}+\MAGENTA{2xzuw}+\BLUE{2yzvw}} \\
      \fromSlide{15}{+(xv-yu)^2} &&
      \fromSlide{15}{+\RAWSIENNA{x^2v^2} - \RED{2xyuv} + \PURPLE{y^2u^2}} \\
      \fromSlide{16}{+(yw-zv)^2} &&
      \fromSlide{16}{+\RAWSIENNA{y^2w^2} - \BLUE{2yzvw} + \PURPLE{z^2v^2}} \\
      \fromSlide{17}{+(zu-xw)^2} &&
      \fromSlide{17}{+\RAWSIENNA{z^2u^2} - \MAGENTA{2xzuw} + \PURPLE{x^2w^2}}
    \end{eqnarray*}}}
\end{itemize}
\end{slide}
}

\overlays{13}{
\begin{slide}{Factoring}
\begin{itemize}
 \fromSlide{2}{
  \item You should practice finding simple factorizations by
        inspection.}
 \fromSlide{3}{
  \item \ghost \vspace{-4ex}
   {\tiny
    \begin{align*}
     \fromSlide{3}{a^2-b^2} &
     \fromSlide{4}{=(a+b)(a-b)} \\
     \fromSlide{5}{a^3-b^3} & 
     \fromSlide{6}{=(a^2+ab+b^2)(a-b)} \\
     \fromSlide{7}{ax^2+bx^2+ay^2+by^2} & 
     \fromSlide{8}{=(a+b)(x^2+y^2)} \\
     \fromSlide{9}{1+t+t^2+t^3} & 
     \fromSlide{10}{=(1+t)(1+t^2)} \\
     \fromSlide{11}{u^2-5u+6} & 
     \fromSlide{12}{=(u-2)(u-3)} \\
     & \hphantom{mmmmmmmmmmmmmmmmmmmmmmmm}
    \end{align*}\vspace{-5ex}}
 }
 \fromSlide{13}{
  \item (There are systematic methods for factoring that can be
   carried out by computers, but they are too unwieldy to use by
   hand.)
 }
\end{itemize}
\end{slide}
}

\overlays{15}{
\begin{slide}{Powers}
\begin{itemize}
 \fromSlide{2}{
  \item You should practice using the basic rules for powers:
   {\tiny\[\begin{array}{rlcrl}
    \fromSlide{3}{a^na^m}  & \fromSlide{3}{=a^{n+m}} 
     & \hspace{2em} & 
    \fromSlide{4}{(a^n)^m} & \fromSlide{4}{=a^{nm}} \\ 
    \fromSlide{5}{a^nb^n}  & \fromSlide{5}{=(ab)^n} &&
    \fromSlide{6}{a^n/b^n} & \fromSlide{6}{=(a/b)^n=a^nb^{-n}} \\
    \fromSlide{7}{(a+b)^n} & \fromSlide{7}{\RED{\neq} a^n+b^n} &&
    \fromSlide{8}{(a+b)^n} &
     \fromSlide{8}{=\sum_{k=0}^n \frac{n!}{k!(n-k)!} a^k b^{n-k}} \\
    \hphantom{mmm} & \hphantom{mmmmm} &&
    \hphantom{mmm} & \hphantom{mmmmm} 
   \end{array}\]}\vspace{-4ex}}
 \fromSlide{9}{
  \item {\bf Example:}
   {\tiny\begin{align*}
    \fromSlide{10}{(2^{1/2}3^{1/3}4^{1/4})^3} &
    \fromSlide{11}{=2^{3/2}\RED{3^{3/3}}\BLUE{4^{3/4}}} \\ &
    \fromSlide{12}{=2^{3/2}\BLUE{(2^2)^{3/4}}\RED{3}} \\ &
    \fromSlide{13}{=2^{3/2}\BLUE{2^{3/2}}\RED{3}} \\ &
    \fromSlide{14}{=2^3 3}
    \fromSlide{15}{=24} \\
    & \hphantom{mmmmmmmmmmmmm}
   \end{align*}}}
\end{itemize}
\end{slide}
}

\overlays{9}{%
\begin{slide}{Algebraic fractions}
\begin{itemize}
 \fromSlide{2}{
  \item You should practice manipulating fractions of the form $a/b$,
   where $a$ and $b$ are themselves complicated algebraic expressions.
 }
 \fromSlide{3}{
  \item The rules are as follows:
   {\tiny\begin{align*}
    \fromSlide{4}{\frac{a}{b} + \frac{c}{d}} &
    \fromSlide{4}{= \frac{ad+bc}{bd}} \\
    \fromSlide{5}{\frac{a}{b} - \frac{c}{d}} &
    \fromSlide{5}{= \frac{ad-bc}{bd}} \\
    \fromSlide{6}{\frac{a}{b} \,.\, \frac{c}{d}} &
    \fromSlide{6}{= \frac{ac}{bd}} \\
    \fromSlide{7}{\frac{a}{b} / \frac{c}{d}} &
    \fromSlide{7}{= \frac{ad}{bc}} \\
    \fromSlide{8}{\left(\frac{a}{b}\right)^n} &
    \fromSlide{8}{= \frac{a^n}{b^n}} \\
    \fromSlide{9}{\left(\frac{a}{b}\right)^{-n}} &
    \fromSlide{9}{= \frac{b^n}{a^n}}
  \end{align*}}}
\end{itemize}
\end{slide}
}

\overlays{9}{%
\begin{slide}{An example: the cross-ratio}
 \begin{itemize}
 \fromSlide{2}{
   \item Put $\chi(a,b,c,d)=\frac{(d-a)(c-b)}{(d-b)(c-a)}$.}
 \fromSlide{3}{
   \item {\bf Problem:} Show that 
         $\chi(a,b,c,d)=\chi(a^{-1},b^{-1},c^{-1},d^{-1})$.}
 \fromSlide{4}{
  \item \ghost \vspace{-4ex}
   {\tiny
     \begin{align*}
      \chi(\frac{1}{a},\frac{1}{b},\frac{1}{c},\frac{1}{d})
       &= \frac{\left(\frac{1}{d}-\frac{1}{a}\right)
                \left(\frac{1}{c}-\frac{1}{b}\right)}
               {\left(\frac{1}{d}-\frac{1}{b}\right)
                \left(\frac{1}{c}-\frac{1}{a}\right)}
          \hphantom{mmmmm}  \\ &
 \fromSlide{5}{
        = \frac{\frac{a-d}{ad}\,\frac{b-c}{bc}}
               {\frac{b-d}{bd}\,\frac{a-c}{ac}}} \\ &
 \fromSlide{6}{
        = \frac{(a-d)(b-c)/(abcd)}{(b-d)(a-c)/(abcd)}} \\ &
 \fromSlide{7}{
        = \frac{--(d-a)(c-b)}{--(d-b)(c-a)}} \\ &
 \fromSlide{8}{
        = \frac{(d-a)(c-b)}{(d-b)(c-a)}} \\&
 \fromSlide{9}{
        = \chi(a,b,c,d).}
     \end{align*}}}
 \end{itemize}
\end{slide}
}

\overlays{5}{%
\begin{slide}{Rational functions}
\setcounter{item@step}{1}
\begin{itemstep}
\item A \DEFN{rational function} of $x$ is a function defined using
 only constants, addition, multiplication, division and integer
 powers.
\item No roots, fractional powers, logs, exponentials, trigonometric
 functions and so on can occur in a rational function.
\item {\bf Examples:}
{\tiny \begin{equation*}
   \frac{1+x+x^2}{1-x+x^2}
   \hspace{3em}
   \frac{1}{x} + \frac{\pi}{x-1} + \frac{\pi^2}{x-2}
   \hspace{3em}
   x^2+x+1+x^{-1}+x^{-2}
\end{equation*}}
\item{\bf Non-Examples:}
{\tiny \begin{equation*}
   e^{-x}\sin(x)
   \hspace{3em}
   \sqrt{1-x^2}
   \hspace{3em}
   \frac{\log(x)}{1+x}
   \hspace{3em}
   \frac{\arctan(x)}{2\pi}.
\end{equation*}}
\end{itemstep}
\end{slide}
}

\overlays{4}{%
\begin{slide}{Poles}
\setcounter{item@step}{1}
\begin{itemstep}
 \item The \DEFN{poles} of a rational function $f(x)$ are the
  values of $x$ where $f(x)$ becomes infinite due to division by
  zero. 
 \item The points $\RED{x=2}$ and $\OLIVEGREEN{x=-3}$ are poles of the
  function $\frac{1+x^2+x^4}{(\RED{x-2})^2(\OLIVEGREEN{x+3})^2}$.
 \item The point $\BLUE{x=1}$ is \EMPH{not} in fact a pole of the
  function $(x^4-1)/(\BLUE{x-1})$, because of the simplification 
  \[ (x^4-1)/(\BLUE{x-1}) = x^3 + x^2 + x + 1. \]
\end{itemstep}
\end{slide}
}

\overlays{9}{%
\begin{slide}{Partial fractions}
\setcounter{item@step}{1}
\begin{itemstep}
 \item Any rational function can be rewritten as a sum of multiples of
  terms $x^k$ (with $k\geq 0$) or $(x-a)^{-k}$ (with $k>0$). \\
  \fromSlide{7}{\RED{(provided we allow complex numbers)}}
 \item {\bf Example:}
  $\frac{1-x^3+x^4}{x^2-2x+1}=
   x^2 + x + 1 + \frac{1}{x-1} + \frac{1}{(x-1)^2}$
 \item This is called a \EMPH{partial fraction decomposition}.
 \item A function written in this form can be integrated easily.
 \item {\bf Example:}
  $\frac{360}{(x^2-25)(x^2-16)}
   = \frac{4}{x-5} - \frac{4}{x+5} - \frac{5}{x-4} + \frac{5}{x+4}$
 \item {\bf Example:}
  $\frac{2}{x^2+1} = \frac{\RED{i}}{x+\RED{i}} - 
                     \frac{\RED{i}}{x-\RED{i}}$.
 \item The \DEFN{order} of a pole at $x=a$ is the highest power of
  $1/(x-a)$ that occurs in the partial fraction decomposition.
 \item {\bf Example:} the function
  {\tiny $
   (x-\RED{1})^{-2}+(x-\RED{1})^{-\BLUE{3}}+(x-\MAGENTA{2})^{-\OLIVEGREEN{4}}
  $} \\
  has a pole of order $\BLUE{3}$ at $\RED{x=1}$ and a pole of order
  $\OLIVEGREEN{4}$ at $\MAGENTA{x=2}$.
\end{itemstep}
\end{slide}
}

\overlays{7}{%
\begin{slide}{Quadratic partial fractions}
\setcounter{item@step}{1}
\begin{itemstep}
 \item If we do not want to use complex numbers, we must allow terms
  like $1/(x^2+bx+c)^k$ \EMPH{and} $x/(x^2+bx+c)^k$ in the
  decomposition. 
 \item (It is a common mistake to omit the terms $x/(x^2+bx+c)^k$)
 \item These quadratic terms are only required when $x^2+bx+c$ does
  not have real roots, or in other words, when $b^2<4c$.
 \item {\bf Example:}
  $\frac{1}{1-x^4}=\frac{2}{x^2+1}+\frac{1}{x+1}-\frac{1}{x-1}$
 \item {\bf Example:}
  $\frac{9x^2}{(x^3-1)^2}=
    \frac{1}{(x-1)^2} - \frac{1}{(1+x+x^2)} 
     - \frac{3x}{(1+x+x^2)^2}$
 \item It is a little more complicated to integrate expressions in
  this form, but still possible (using
  {\tiny $\int\frac{dx}{x^2+1}=\arctan(x)$}).
\end{itemstep}
\end{slide}
}

% \overlays{9}{%
% \begin{slide}{To find a partial fraction decomposition}
% \begin{itemstep}
%  \item Start with a rational function $f(x)$.
%  \item Write $f(x)$ as $p(x)/q(x)$, cancelling any common factors.
%  \item Factorize $q(x)$ to find the poles of $f(x)$ and their orders.
%  \item Compare the degrees of $p(x)$ and $q(x)$, to see whether terms
%   of the form $x^k$ are needed.
%  \item Write down the general form of the decomposition, with
%   undetermined coefficients.
%  \item Express the decomposition in the form $r(x)/q(x)$ (where $q(x)$
%   is the same as before, and the coefficients must be chosen so that
%   $r(x)=p(x)$). 
%  \item Compare $r(x)$ and $p(x)$ to get a system of equations that
%   determine the coefficients.
%  \item Solve these equations.
%  \item Substitute in the values that you have found for the
%   coefficients. 
% \end{itemstep}
% \end{slide}
% }

\overlays{10}{%
\begin{slide}{Finding a decomposition -- I}
\begin{itemize}
 \fromSlide{1}{
  \item $f(x)=\frac{3x^2-6x+2}{x^3-3x^2+2x}$}
         \fromSlide{2}{$=\frac{3x^2-6x+2}{x(x^2-3x+2)}$}
         \fromSlide{3}{$=\frac{3x^2-6x+2}{(x-\RED{0})(x-\RED{1})(x-\RED{2})}$}
 \fromSlide{4}{
  \item Poles of order one at $x=\RED{0}$, $x=\RED{1}$ and $x=\RED{2}$
   (potentially).} 
 \fromSlide{5}{
  \item $\text{degree}(3x^{\RED{2}}-6x+2)=\RED{2}<
         \MAGENTA{3}=\text{degree}(x^\MAGENTA{3}-3x^2+2x)$, \\
        so no terms of the form $x^k$.}
 \fromSlide{6}{
  \item \ghost \vspace{-4ex}
   {\tiny
     \begin{align*}
      f(x) &=
         {\textstyle
           \frac{A}{x}+\frac{B}{x-1}+\frac{C}{x-2}} 
         \hphantom{mmmmmmmm}
  \\ &
 \fromSlide{7}{
       = {\textstyle
           \frac{A(x^2-3x+2)+B(x^2-2x)+C(x^2-x)}{x(x-1)(x-2)}}}
  \\ &
 \fromSlide{8}{
       = {\textstyle
           \frac{(\RED{A+B+C})x^2 + 
                 (\OLIVEGREEN{-3A-2B-C})x +
                 \BLUE{2A}}
                {x(x-1)(x-2)}}}
     \end{align*}
   }
 }
 \vspace{-2ex}
 \fromSlide{9}{
  \item This must be the same as
   {\tiny $(\RED{3}x^2\OLIVEGREEN{-6}x+\BLUE{2})/(x(x-1)(x-2))$}\\ 
   (for all $x$), so 
   {\tiny $\RED{A+B+C=3}$}, 
   {\tiny $\OLIVEGREEN{-3A-2B-C=-6}$} and 
   {\tiny $\BLUE{2A=2}$}.}
 \fromSlide{10}{
  \item These equations give $A=B=C=1$, so 
   {\tiny \[\textstyle
     f(x)=\frac{1}{x}+\frac{1}{x-1}+\frac{1}{x-2}.
   \]}}
\end{itemize}
\end{slide}
}

\overlays{8}{%
\begin{slide}{Finding a decomposition -- II}
\begin{itemize}
 \fromSlide{1}{
  \item $f(x)=
         \onlySlide*{1}{\frac{x^3+1}{(x-1)^2}}%
         \fromSlide*{2}{\frac{x^3+1}{(x-\RED{1})^\MAGENTA{2}}}
         \fromSlide{3}{=\frac{x^{\OLIVEGREEN{3}}+1}
                             {x^{\BLUE{2}}-2x+1}}$}
 \fromSlide{2}{
  \item Pole of order $\MAGENTA{2}$ at $x=\RED{1}$, giving terms in
   $(x-1)^{-1}$ and $(x-1)^{-2}$.}
 \fromSlide{3}{
  \item {\tiny 
          $\text{degree}(x^{\OLIVEGREEN{3}}+1)-
           \text{degree}(x^{\BLUE{2}}-2x+1)=
           \OLIVEGREEN{3}-\BLUE{2}=1$}, \\
   giving terms in $x^0$ and $x^1$.}
 \fromSlide{4}{
  \item \ghost \vspace{-4ex}
   {\tiny
     \begin{align*}
      f(x) &=
         {\textstyle
          Ax^1 + Bx^0 + \frac{C}{x-1} + \frac{D}{(x-1)^2}} 
         \hphantom{mmmmmmmm}
  \\ &
 \fromSlide{5}{
       = {\textstyle
           \frac{(Ax+B)(x^2-2x+1)+C(x-1)+D}{(x-1)^2}}}
  \\ &
 \fromSlide{6}{
       = {\textstyle
           \frac{\RED{A}x^3 + 
                 (\OLIVEGREEN{-2A+B})x^2 +
                 (\BLUE{A-2B+C})x +
                 (\MAGENTA{B-C+D})}
                {(x-1)^2}}}
     \end{align*}
   }
 }
 \vspace{-2ex}
 \fromSlide{7}{
  \item This must be the same as
   {\tiny $\frac{x^3+1}{(x-1)^2}=
    \frac{\RED{1}x^3+\OLIVEGREEN{0}x^2+\BLUE{0}x+\MAGENTA{1}}{(x-1)^2}$}
   (for all $x$), \\
   so 
   {\tiny $\RED{A=1}$}, 
   {\tiny $\OLIVEGREEN{-2A+B=0}$},
   {\tiny $\BLUE{A-2B+C=0}$} and 
   {\tiny $\MAGENTA{B-C+D=1}$}.}
 \fromSlide{8}{
  \item These equations give $A=1$, $B=2$, $C=3$ and $D=2$, so
   {\tiny \[\textstyle
     f(x)=x+2+\frac{3}{x-1} + \frac{2}{(x-1)^2}.
   \]}}
\end{itemize}
\end{slide}
}

\overlays{9}{%
\begin{slide}{Finding a decomposition -- III}
\begin{itemize}
 \fromSlide{1}{
  \item $f(x)=\frac{4}{1-x^4}
         \fromSlide{2}{=\frac{-4}{(x^2+1)(x^2-1)}
                       =\frac{-4}{(x^2+1)(x+1)(x-1)}}$}
 \fromSlide{3}{
  \item This gives terms 
   $\frac{1}{x^2+1}$, $\frac{x}{x^2+1}$,
   $\frac{1}{x+1}$ and $\frac{1}{x-1}$.}
 \fromSlide{4}{
  \item {\tiny 
          $\text{degree}(1) = 0 < 4 =
           \text{degree}(x^4-1)$}, \\
   so there are no terms of the form $x^k$.}
 \fromSlide{5}{
  \item \ghost \vspace{-4ex}
   {\tiny
     \begin{align*}
      f(x) &=
         {\textstyle
          \frac{Ax+B}{x^2+1} + \frac{C}{x+1} + \frac{D}{x-1}}
         \hphantom{mmmmmmmmmmmmmmmmm}
  \\ &
 \fromSlide{6}{
       = {\textstyle
           \frac{(Ax+B)(x^2-1)+C(x^3-x^2+x-1)+D(x^3+x^2+x+1)}{x^4-1}}}
  \\ &
 \fromSlide{7}{
       = {\textstyle
           \frac{(\RED{A+C+D})x^3 + 
                 (\OLIVEGREEN{B-C+D})x^2 +
                 (\BLUE{-A+C+D})x +
                 (\MAGENTA{-B-C+D})}
                {x^4-1}}}
     \end{align*}
   }
 }
 \vspace{-2ex}
 \fromSlide{8}{
  \item This must be the same as $\MAGENTA{-4}/(x^4-1)$, so \\
   {\tiny $\RED{A+C+D=0}$}, 
   {\tiny $\OLIVEGREEN{B-C+D=0}$},
   {\tiny $\BLUE{-A+C+D=0}$} and 
   {\tiny $\MAGENTA{-B-C+D=-4}$}.}
 \fromSlide{9}{
  \item These equations give $A=0$, $B=2$, $C=1$ and $D=-1$, so
   {\tiny \[\textstyle
     f(x)=\frac{2}{x^2+1}+\frac{1}{x+1} - \frac{1}{x-1}
   \]}}
\end{itemize}
\end{slide}
}

\end{document}

%%%%%%%%%%%%%%%%%%%%%%%%%%%%%%%%%%%%%%%%%%%%%%%%%%%%%%%%%%%%%%%%%%%%%%
%%%%%%%%%%%%%%%%%%%%%%%%%%%%%%%%%%%%%%%%%%%%%%%%%%%%%%%%%%%%%%%%%%%%%%
%%%%%%%%%%%%%%%%%%%%%%%%%%%%%%%%%%%%%%%%%%%%%%%%%%%%%%%%%%%%%%%%%%%%%%
