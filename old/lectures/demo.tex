\documentclass[%
pdf,
neil,
colorBG,
slideColor,
]{prosper}
\usepackage{amsmath}
\usepackage{verbatim}
\usepackage{pstricks,pst-node,pst-text,pst-3d,pst-plot,multido}
\usepackage[usenames,dvips]{color}
%\usepackage{graphicx}

\newcommand{\GREENYELLOW}[1]{{\color{GreenYellow}#1}}
\newcommand{\YELLOW}[1]{{\color{Yellow}#1}}
\newcommand{\YLW}[1]{{\color{Yellow}#1}}
\newcommand{\GOLDENROD}[1]{{\color{Goldenrod}#1}}
\newcommand{\DANDELION}[1]{{\color{Dandelion}#1}}
\newcommand{\APRICOT}[1]{{\color{Apricot}#1}}
\newcommand{\PEACH}[1]{{\color{Peach}#1}}
\newcommand{\MELON}[1]{{\color{Melon}#1}}
\newcommand{\YELLOWORANGE}[1]{{\color{YellowOrange}#1}}
\newcommand{\ORANGE}[1]{{\color{Orange}#1}}
\newcommand{\BURNTORANGE}[1]{{\color{BurntOrange}#1}}
\newcommand{\BITTERSWEET}[1]{{\color{Bittersweet}#1}}
\newcommand{\REDORANGE}[1]{{\color{RedOrange}#1}}
\newcommand{\MAHOGANY}[1]{{\color{Mahogany}#1}}
\newcommand{\MAROON}[1]{{\color{Maroon}#1}}
\newcommand{\BRICKRED}[1]{{\color{BrickRed}#1}}
\newcommand{\RED}[1]{{\color{Red}#1}}
\newcommand{\ORANGERED}[1]{{\color{OrangeRed}#1}}
\newcommand{\RUBINERED}[1]{{\color{RubineRed}#1}}
\newcommand{\WILDSTRAWBERRY}[1]{{\color{WildStrawberry}#1}}
\newcommand{\SALMON}[1]{{\color{Salmon}#1}}
\newcommand{\CARNATIONPINK}[1]{{\color{CarnationPink}#1}}
\newcommand{\MAGENTA}[1]{{\color{Magenta}#1}}
\newcommand{\VIOLETRED}[1]{{\color{VioletRed}#1}}
\newcommand{\RHODAMINE}[1]{{\color{Rhodamine}#1}}
\newcommand{\MULBERRY}[1]{{\color{Mulberry}#1}}
\newcommand{\REDVIOLET}[1]{{\color{RedViolet}#1}}
\newcommand{\FUCHSIA}[1]{{\color{Fuchsia}#1}}
\newcommand{\LAVENDER}[1]{{\color{Lavender}#1}}
\newcommand{\THISTLE}[1]{{\color{Thistle}#1}}
\newcommand{\ORCHID}[1]{{\color{Orchid}#1}}
\newcommand{\DARKORCHID}[1]{{\color{DarkOrchid}#1}}
\newcommand{\PURPLE}[1]{{\color{Purple}#1}}
\newcommand{\PLUM}[1]{{\color{Plum}#1}}
\newcommand{\VIOLET}[1]{{\color{Violet}#1}}
\newcommand{\ROYALPURPLE}[1]{{\color{RoyalPurple}#1}}
\newcommand{\BLUEVIOLET}[1]{{\color{BlueViolet}#1}}
\newcommand{\PERIWINKLE}[1]{{\color{Periwinkle}#1}}
\newcommand{\CADETBLUE}[1]{{\color{CadetBlue}#1}}
\newcommand{\CORNFLOWERBLUE}[1]{{\color{CornflowerBlue}#1}}
\newcommand{\MIDNIGHTBLUE}[1]{{\color{MidnightBlue}#1}}
\newcommand{\NAVYBLUE}[1]{{\color{NavyBlue}#1}}
\newcommand{\ROYALBLUE}[1]{{\color{RoyalBlue}#1}}
\newcommand{\BLUE}[1]{{\color{Blue}#1}}
\newcommand{\CERULEAN}[1]{{\color{Cerulean}#1}}
\newcommand{\CYAN}[1]{{\color{Cyan}#1}}
\newcommand{\PROCESSBLUE}[1]{{\color{ProcessBlue}#1}}
\newcommand{\SKYBLUE}[1]{{\color{SkyBlue}#1}}
\newcommand{\TURQUOISE}[1]{{\color{Turquoise}#1}}
\newcommand{\TEALBLUE}[1]{{\color{TealBlue}#1}}
\newcommand{\AQUAMARINE}[1]{{\color{Aquamarine}#1}}
\newcommand{\BLUEGREEN}[1]{{\color{BlueGreen}#1}}
\newcommand{\EMERALD}[1]{{\color{Emerald}#1}}
\newcommand{\JUNGLEGREEN}[1]{{\color{JungleGreen}#1}}
\newcommand{\SEAGREEN}[1]{{\color{SeaGreen}#1}}
\newcommand{\GREEN}[1]{{\color{Green}#1}}
\newcommand{\FORESTGREEN}[1]{{\color{ForestGreen}#1}}
\newcommand{\PINEGREEN}[1]{{\color{PineGreen}#1}}
\newcommand{\LIMEGREEN}[1]{{\color{LimeGreen}#1}}
\newcommand{\YELLOWGREEN}[1]{{\color{YellowGreen}#1}}
\newcommand{\SPRINGGREEN}[1]{{\color{SpringGreen}#1}}
\newcommand{\OLIVEGREEN}[1]{{\color{OliveGreen}#1}}
\newcommand{\OLG}[1]{{\color{OliveGreen}#1}}
\newcommand{\RAWSIENNA}[1]{{\color{RawSienna}#1}}
\newcommand{\SEPIA}[1]{{\color{Sepia}#1}}
\newcommand{\BROWN}[1]{{\color{Brown}#1}}
\newcommand{\TAN}[1]{{\color{Tan}#1}}
\newcommand{\GRAY}[1]{{\color{Gray}#1}}
\newcommand{\WHITE}[1]{{\color{White}#1}}
\newcommand{\BLACK}[1]{{\color{Black}#1}}

\newcmykcolor{GreenYellow}{0.15 0 0.69 0}
\newcmykcolor{Yellow}{0 0 1 0}
\newcmykcolor{Goldenrod}{0 0.10 0.84 0}
\newcmykcolor{Dandelion}{0 0.29 0.84 0}
\newcmykcolor{Apricot}{0 0.32 0.52 0}
\newcmykcolor{Peach}{0 0.50 0.70 0}
\newcmykcolor{Melon}{0 0.46 0.50 0}
\newcmykcolor{YellowOrange}{0 0.42 1 0}
\newcmykcolor{Orange}{0 0.61 0.87 0}
\newcmykcolor{BurntOrange}{0 0.51 1 0}
\newcmykcolor{Bittersweet}{0 0.75 1 0.24}
\newcmykcolor{RedOrange}{0 0.77 0.87 0}
\newcmykcolor{Mahogany}{0 0.85 0.87 0.35}
\newcmykcolor{Maroon}{0 0.87 0.68 0.32}
\newcmykcolor{BrickRed}{0 0.89 0.94 0.28}
\newcmykcolor{Red}{0 1 1 0}
\newcmykcolor{OrangeRed}{0 1 0.50 0}
\newcmykcolor{RubineRed}{0 1 0.13 0}
\newcmykcolor{WildStrawberry}{0 0.96 0.39 0}
\newcmykcolor{Salmon}{0 0.53 0.38 0}
\newcmykcolor{CarnationPink}{0 0.63 0 0}
\newcmykcolor{Magenta}{0 1 0 0}
\newcmykcolor{VioletRed}{0 0.81 0 0}
\newcmykcolor{Rhodamine}{0 0.82 0 0}
\newcmykcolor{Mulberry}{0.34 0.90 0 0.02}
\newcmykcolor{RedViolet}{0.07 0.90 0 0.34}
\newcmykcolor{Fuchsia}{0.47 0.91 0 0.08}
\newcmykcolor{Lavender}{0 0.48 0 0}
\newcmykcolor{Thistle}{0.12 0.59 0 0}
\newcmykcolor{Orchid}{0.32 0.64 0 0}
\newcmykcolor{DarkOrchid}{0.40 0.80 0.20 0}
\newcmykcolor{Purple}{0.45 0.86 0 0}
\newcmykcolor{Plum}{0.50 1 0 0}
\newcmykcolor{Violet}{0.79 0.88 0 0}
\newcmykcolor{RoyalPurple}{0.75 0.90 0 0}
\newcmykcolor{BlueViolet}{0.86 0.91 0 0.04}
\newcmykcolor{Periwinkle}{0.57 0.55 0 0}
\newcmykcolor{CadetBlue}{0.62 0.57 0.23 0}
\newcmykcolor{CornflowerBlue}{0.65 0.13 0 0}
\newcmykcolor{MidnightBlue}{0.98 0.13 0 0.43}
\newcmykcolor{NavyBlue}{0.94 0.54 0 0}
\newcmykcolor{RoyalBlue}{1 0.50 0 0}
\newcmykcolor{Blue}{1 1 0 0}
\newcmykcolor{Cerulean}{0.94 0.11 0 0}
\newcmykcolor{Cyan}{1 0 0 0}
\newcmykcolor{ProcessBlue}{0.96 0 0 0}
\newcmykcolor{SkyBlue}{0.62 0 0.12 0}
\newcmykcolor{Turquoise}{0.85 0 0.20 0}
\newcmykcolor{TealBlue}{0.86 0 0.34 0.02}
\newcmykcolor{Aquamarine}{0.82 0 0.30 0}
\newcmykcolor{BlueGreen}{0.85 0 0.33 0}
\newcmykcolor{Emerald}{1 0 0.50 0}
\newcmykcolor{JungleGreen}{0.99 0 0.52 0}
\newcmykcolor{SeaGreen}{0.69 0 0.50 0}
\newcmykcolor{Green}{1 0 1 0}
\newcmykcolor{ForestGreen}{0.91 0 0.88 0.12}
\newcmykcolor{PineGreen}{0.92 0 0.59 0.25}
\newcmykcolor{LimeGreen}{0.50 0 1 0}
\newcmykcolor{YellowGreen}{0.44 0 0.74 0}
\newcmykcolor{SpringGreen}{0.26 0 0.76 0}
\newcmykcolor{OliveGreen}{0.64 0 0.95 0.40}
\newcmykcolor{RawSienna}{0 0.72 1 0.45}
\newcmykcolor{Sepia}{0 0.83 1 0.70}
\newcmykcolor{Brown}{0 0.81 1 0.60}
\newcmykcolor{Tan}{0.14 0.42 0.56 0}
\newcmykcolor{Gray}{0 0 0 0.50}
\newcmykcolor{Black}{0 0 0 1}
\newcmykcolor{White}{0 0 0 0}


\newcommand{\bbm}       {\left[\begin{matrix}}
\newcommand{\ebm}       {\end{matrix}\right]}
\newcommand{\bsm}       {\left[\begin{smallmatrix}}
\newcommand{\esm}       {\end{smallmatrix}\right]}
\newcommand{\bpm}       {\begin{pmatrix}}
\newcommand{\epm}       {\end{pmatrix}}
\newcommand{\bcf}[2]{\left(\begin{array}{c}{#1}\\{#2}\end{array}\right)}


\newcommand{\csch}     {\operatorname{csch}}
\newcommand{\sech}     {\operatorname{sech}}
\newcommand{\arcsinh}  {\operatorname{arcsinh}}
\newcommand{\arccosh}  {\operatorname{arccosh}}
\newcommand{\arctanh}  {\operatorname{arctanh}}

\newcommand{\range}     {\operatorname{range}}
\newcommand{\trans}     {\operatorname{trans}}
\newcommand{\trc}       {\operatorname{trace}}
\newcommand{\adj}       {\operatorname{adj}}

\newcommand{\tint}{\textstyle\int}
\newcommand{\tm}{\times}
\newcommand{\sse}{\subseteq}
\newcommand{\st}{\;|\;}
\newcommand{\sm}{\setminus}
\newcommand{\iffa}      {\Leftrightarrow}
\newcommand{\xra}{\xrightarrow}

\renewcommand{\:}{\colon}

\newcommand{\N}         {{\mathbb{N}}}
\newcommand{\Z}         {{\mathbb{Z}}}
\newcommand{\Q}         {{\mathbb{Q}}}
\renewcommand{\R}       {{\mathbb{R}}}
\newcommand{\C}         {{\mathbb{C}}}

\newcommand{\al}        {\alpha}
\newcommand{\bt}        {\beta} 
\newcommand{\gm}        {\gamma}
\newcommand{\dl}        {\delta}
\newcommand{\ep}        {\epsilon}
\newcommand{\zt}        {\zeta}
\newcommand{\et}        {\eta}
\newcommand{\tht}       {\theta}
\newcommand{\io}        {\iota}
\newcommand{\kp}        {\kappa}
\newcommand{\lm}        {\lambda}
\newcommand{\ph}        {\phi}
\newcommand{\ch}        {\chi}
\newcommand{\ps}        {\psi}
\newcommand{\rh}        {\rho}
\newcommand{\sg}        {\sigma}
\newcommand{\om}        {\omega}

\newcommand{\EMPH}[1]{\emph{\RED{#1}}}
\newcommand{\DEFN}[1]{\emph{\PURPLE{#1}}}
\newcommand{\VEC}[1]    {\mathbf{#1}}

\newcommand{\ghost}{{\tiny $\color[rgb]{1,1,1}.$}}


\newcommand{\ff}{dup dup dup -0.6 mul 5.0 add exch mul -12.8 add exch mul 10.4 add exch mul 1 add}
\newcommand{\vl}[1]{\psdots(#1,0) \psline(#1,0)(! #1 #1 \ff)}
\newcommand{\hh}{0.2}
\newcommand{\bx}[1]{%
 \psframe*[linecolor=green](#1,0.0)(! #1 \hh\space add #1 \ff)%
 \psframe[linecolor=OliveGreen](#1,0.0)(! #1 \hh\space add #1 \ff)%
 \psdots(#1,0)(!#1 \hh\space add 0)}
\newcommand{\bxp}[1]{%
 \pscustom[linestyle=none,fillstyle=solid,fillcolor=green]{%
  \parametricplot{0}{1}{#1 \hh\space t mul add dup \ff}
  \psline(! #1 \hh\space add 0)(#1,0)}
 \pscustom[linestyle=none,fillstyle=solid,fillcolor=blue]{%
  \parametricplot{0}{1}{#1 \hh\space t mul add dup \ff}
  \psline(! #1 \hh\space add #1 \ff)(! #1 #1 \ff)}
 \psframe[linecolor=OliveGreen](#1,0.0)(! #1 \hh\space add #1 \ff)%
 \psdots(#1,0)(!#1 \hh\space add 0)}
\newcommand{\bxn}[1]{%
 \pscustom[linestyle=none,fillstyle=solid,fillcolor=green]{%
  \parametricplot{0}{1}{#1 \hh\space t mul add dup \ff}
  \psline(! #1 \hh\space add 0)(#1,0)}
 \pscustom[linestyle=none,fillstyle=solid,fillcolor=magenta]{%
  \parametricplot{0}{1}{#1 \hh\space t mul add dup \ff}
  \psline(! #1 \hh\space add #1 \ff)(! #1 #1 \ff)}
 \psframe[linecolor=OliveGreen](#1,0.0)(! #1 \hh\space add #1 \ff)%
 \psdots(#1,0)(!#1 \hh\space add 0)}

\begin{document}


\overlays{12}{%
\begin{slide}{The addition formula}
\begin{itemize}
 \fromSlide{2}{
  \item $\sin(a+b)=\sin(a)\cos(b)+\cos(a)\sin(b)$
 }
 \fromSlide{2}{
  \item 
    \psset{unit=5cm}
      \def\angleA{40}
      \def\angleB{30}
      \def\angleAB{70}
      \def\cosA{0.766}
      \def\sinA{0.643}
      \def\cosB{0.866}
      \def\sinB{0.500}
      \def\cosAB{0.342}
      \def\sinAB{0.940}
      \def\cosAcosB{0.663}
      \def\cosAsinB{0.383}
      \def\sinAcosB{0.557}
      \def\sinAsinB{0.321}
     \onlySlide*{3}{
      \begin{pspicture}[1](-0.2,-0.3)(1.4,1.0)
       \psset{linewidth=0.1pt}
       \psaxes[labels=none,ticksize=1pt]{->}(0,0)(-0.1,0)(1.3,1.0)
       \psarc[linecolor=red,linestyle=dotted](0,0){1}{0}{90}
       \psarc[linecolor=red](0,0){.2}{0}{\angleB}
       \rput(\cosB,0){\psline[linecolor=red](0,.1)(-.1,.1)(-.1,0)}
       \psline[linecolor=blue](0,0)(\cosB,\sinB)(\cosB,0)(0,0)
       \rput{15}(0,0){\rput{*0}(.25,0){$\scriptstyle b$}}
       \rput[t](.45,-.05){$\scriptstyle\cos(b)$}
       \rput[r](.83,0.25){$\scriptstyle\sin(b)$}
      \end{pspicture}}%
     \onlySlide*{4}{%
      \begin{pspicture}[1](-0.2,-0.3)(1.4,1.0)
       \psset{linewidth=0.1pt}
       \psaxes[labels=none,ticksize=1pt]{->}(0,0)(-0.1,0)(1.3,1.0)
       \psarc[linecolor=red,linestyle=dotted](0,0){1}{0}{90}
       \rput{\angleA}(0,0){
        \psarc[linecolor=red](0,0){.2}{0}{\angleB}
        \rput(\cosB,0){\psline[linecolor=red](0,.1)(-.1,.1)(-.1,0)}
        \psline[linecolor=blue](0,0)(\cosB,\sinB)(\cosB,0)(0,0)
        \psline[linecolor=blue,linestyle=dotted](\cosB,0)(1,0)
        \rput{15}(0,0){\rput{*0}(.25,0){$\scriptstyle b$}}
        \rput[t](.45,-.05){$\scriptstyle\cos(b)$}
        \rput[r](.83,0.25){$\scriptstyle\sin(b)$}
       }
       \psarc[linecolor=red](0,0){.2}{0}{\angleA}
      \end{pspicture}}%
     \onlySlide*{5}{%
      \begin{pspicture}[1](-0.2,-0.3)(1.4,1.0)
       \psset{linewidth=0.1pt}
       \psaxes[labels=none,ticksize=1pt]{->}(0,0)(-0.1,0)(1.3,1.0)
       \psarc[linecolor=red,linestyle=dotted](0,0){1}{0}{90}
       \rput{\angleA}(0,0){
        \psarc[linecolor=red](0,0){.2}{0}{\angleB}
        \rput(\cosB,0){\psline[linecolor=red](0,.1)(-.1,.1)(-.1,0)}
        \psline[linecolor=blue,linestyle=dotted](0,0)(\cosB,\sinB)(\cosB,0)(1,0)
        \rput{15}(0,0){\rput{*0}(.25,0){$\scriptstyle b$}}
        \rput[t](.45,-.05){$\scriptstyle\cos(b)$}
        \rput[r](.83,0.25){$\scriptstyle\sin(b)$}
       }
       \psarc[linecolor=red](0,0){.2}{0}{\angleA}
       \rput{20}(0,0){\rput{*0}(.25,0){$\scriptstyle a$}}
       \psline[linecolor=blue]%
        (0,0)(\cosAcosB,\sinAcosB)(\cosAcosB,0)(0,0)
       \rput(\cosAcosB,0){\psline[linecolor=red](0,.1)(-.1,.1)(-.1,0)}
       \rput[l](0.68,0.28){$\scriptstyle \sin(a)\cos(b)$}
      \end{pspicture}}%
     \onlySlide*{6}{%
      \begin{pspicture}[1](-0.2,-0.3)(1.4,1.0)
       \psset{linewidth=0.1pt}
       \psaxes[labels=none,ticksize=1pt]{->}(0,0)(-0.1,0)(1.3,1.0)
       \psarc[linecolor=red,linestyle=dotted](0,0){1}{0}{90}
       \rput{\angleA}(0,0){
        \psarc[linecolor=red](0,0){.2}{0}{\angleB}
        \rput(\cosB,0){\psline[linecolor=red](0,.1)(-.1,.1)(-.1,0)}
        \psline[linecolor=blue,linestyle=dotted](0,0)(\cosB,\sinB)(\cosB,0)(1,0)
        \rput{15}(0,0){\rput{*0}(.25,0){$\scriptstyle b$}}
        \rput[t](.45,-.05){$\scriptstyle\cos(b)$}
        \rput[r](.83,0.25){$\scriptstyle\sin(b)$}
       }
       \psarc[linecolor=red](0,0){.2}{0}{\angleA}
       \rput{20}(0,0){\rput{*0}(.25,0){$\scriptstyle a$}}
       \psline[linecolor=blue,linestyle=dotted]%
        (0,0)(\cosAcosB,\sinAcosB)(\cosAcosB,0)(0,0)
       \rput(\cosAcosB,0){\psline[linecolor=red](0,.1)(-.1,.1)(-.1,0)}
       \psline[linecolor=black](\cosAcosB,\sinAcosB)(1.2,\sinAcosB)
       \psline[linecolor=magenta,arrows=<->](1.1,0)(1.1,\sinAcosB)
       \rput[l](1.15,0.28){$\scriptstyle \sin(a)\cos(b)$}
      \end{pspicture}}%
     \onlySlide*{7}{%
      \begin{pspicture}[1](-0.2,-0.3)(1.4,1.0)
       \psset{linewidth=0.1pt}
       \psaxes[labels=none,ticksize=1pt]{->}(0,0)(-0.1,0)(1.3,1.0)
       \psarc[linecolor=red,linestyle=dotted](0,0){1}{0}{90}
       \rput{\angleA}(0,0){
        \psarc[linecolor=red](0,0){.2}{0}{\angleB}
        \rput(\cosB,0){\psline[linecolor=red](0,.1)(-.1,.1)(-.1,0)}
        \psline[linecolor=blue,linestyle=dotted](0,0)(\cosB,\sinB)(\cosB,0)(1,0)
        \rput{15}(0,0){\rput{*0}(.25,0){$\scriptstyle b$}}
        \rput[t](.45,-.05){$\scriptstyle\cos(b)$}
        \rput[r](.83,0.25){$\scriptstyle\sin(b)$}
       }
       \psarc[linecolor=red](0,0){.2}{0}{\angleA}
       \rput{20}(0,0){\rput{*0}(.25,0){$\scriptstyle a$}}
       \psline[linecolor=blue,linestyle=dotted]%
        (0,0)(\cosAcosB,\sinAcosB)(\cosAcosB,0)(0,0)
       \rput(\cosAcosB,0){\psline[linecolor=red](0,.1)(-.1,.1)(-.1,0)}
       \psline[linecolor=blue]%
        (\cosAB,\sinAB)(\cosAcosB,\sinAcosB)(\cosAcosB,\sinAB)(\cosAB,\sinAB)
       \rput{90}(\cosAcosB,\sinAcosB){
        \psarc[linecolor=red](0,0){.1}{0}{\angleA}
        \rput{20}(0,0){\rput{*0}(.15,0){$\scriptstyle a$}}
       }
       \rput[l](0.68,0.78){$\scriptstyle \cos(a)\sin(b)$}
       \psline[linecolor=black](\cosAcosB,\sinAcosB)(1.2,\sinAcosB)
       \psline[linecolor=magenta,arrows=<->](1.1,0)(1.1,\sinAcosB)
       \rput[l](1.15,0.28){$\scriptstyle \sin(a)\cos(b)$}
      \end{pspicture}}%
     \onlySlide*{8}{%
      \begin{pspicture}[1](-0.2,-0.3)(1.4,1.0)
       \psset{linewidth=0.1pt}
       \psaxes[labels=none,ticksize=1pt]{->}(0,0)(-0.1,0)(1.3,1.0)
       \psarc[linecolor=red,linestyle=dotted](0,0){1}{0}{90}
       \rput{\angleA}(0,0){
        \psarc[linecolor=red](0,0){.2}{0}{\angleB}
        \rput(\cosB,0){\psline[linecolor=red](0,.1)(-.1,.1)(-.1,0)}
        \psline[linecolor=blue,linestyle=dotted](0,0)(\cosB,\sinB)(\cosB,0)(1,0)
        \rput{15}(0,0){\rput{*0}(.25,0){$\scriptstyle b$}}
        \rput[t](.45,-.05){$\scriptstyle\cos(b)$}
        \rput[r](.83,0.25){$\scriptstyle\sin(b)$}
       }
       \psarc[linecolor=red](0,0){.2}{0}{\angleA}
       \rput{20}(0,0){\rput{*0}(.25,0){$\scriptstyle a$}}
       \psline[linecolor=blue,linestyle=dotted]%
        (0,0)(\cosAcosB,\sinAcosB)(\cosAcosB,0)(0,0)
       \rput(\cosAcosB,0){\psline[linecolor=red](0,.1)(-.1,.1)(-.1,0)}
       \psline[linecolor=blue,linestyle=dotted]%
        (\cosAB,\sinAB)(\cosAcosB,\sinAcosB)(\cosAcosB,\sinAB)(\cosAB,\sinAB)
       \rput{90}(\cosAcosB,\sinAcosB){
        \psarc[linecolor=red](0,0){.1}{0}{\angleA}
        \rput{20}(0,0){\rput{*0}(.15,0){$\scriptstyle a$}}
       }
       \psline[linecolor=black](\cosAB,\sinAB)(1.2,\sinAB)
       \psline[linecolor=magenta,arrows=<->](1.1,\sinAcosB)(1.1,\sinAB)
       \rput[l](1.15,0.78){$\scriptstyle \cos(a)\sin(b)$}
       \psline[linecolor=black](\cosAcosB,\sinAcosB)(1.2,\sinAcosB)
       \psline[linecolor=magenta,arrows=<->](1.1,0)(1.1,\sinAcosB)
       \rput[l](1.15,0.28){$\scriptstyle \sin(a)\cos(b)$}
      \end{pspicture}}%
     \fromSlide*{9}{%
      \begin{pspicture}[1](-0.2,-0.3)(1.4,1.0)
       \psset{linewidth=0.1pt}
       \psaxes[labels=none,ticksize=1pt]{->}(0,0)(-0.1,0)(1.3,1.0)
       \psarc[linecolor=red,linestyle=dotted](0,0){1}{0}{90}
       \psarc[linecolor=red](0,0){.2}{0}{\angleAB}
       \rput{35}(0,0){\rput{*0}(.25,0){$\scriptstyle a+b$}}
       \psline[linecolor=blue]%
        (0,0)(\cosAB,\sinAB)(\cosAB,0)(0,0)
       \rput[l](0.35,0.47){$\scriptstyle\sin(a+b)$}
       \psline[linecolor=black](\cosAB,\sinAB)(1.2,\sinAB)
       \psline[linecolor=magenta,arrows=<->](1.1,\sinAcosB)(1.1,\sinAB)
       \rput[l](1.15,0.78){$\scriptstyle \cos(a)\sin(b)$}
       \psline[linecolor=black](\cosAcosB,\sinAcosB)(1.2,\sinAcosB)
       \psline[linecolor=magenta,arrows=<->](1.1,0)(1.1,\sinAcosB)
       \rput[l](1.15,0.28){$\scriptstyle \sin(a)\cos(b)$}
      \end{pspicture}}%
 \vspace{-6ex}}
 \fromSlide{10}{
  \item \ghost\vspace{-4ex}{\tiny\begin{align*}
   \sin(a)\cos(b)+\cos(a)\sin(b) &=
     \frac{e^{ia}-e^{-ia}}{2i}\frac{e^{ib}+e^{-ib}}{2} + 
     \frac{e^{ia}+e^{-ia}}{2}\frac{e^{ib}-e^{-ib}}{2i} \\
     &\fromSlide{11}{=  \frac{e^{i(a+b)}-e^{-i(a+b)}}{2i}}
       \fromSlide{12}{=\sin(a+b)}
  \end{align*}}
 }
\end{itemize}
\end{slide}
}

\overlays{10}{%
\begin{slide}{Slopes}
   \psset{yunit=3cm,xunit=3cm}
\onlySlide*{1}{
   \begin{center}\begin{pspicture}[1](-0.2,-0.2)( 1.7, 2.00)
    \psset{linewidth=0.1pt}
    \psaxes[labels=none,ticksize=1pt]{->}(0,0)(-0.1,0)( 1.7, 1.81)
    \psplot[linecolor=red]{0}{ 1.7}{x -0.8 add dup mul 1 add}
    \rput[tl](1.6,1.6){${\scriptstyle y=f(x)}$}
  \end{pspicture}\end{center}
  Consider variables $x$ and $y$ related by $y=f(x)$.
}
\onlySlide*{2}{
   \begin{center}\begin{pspicture}[1](-0.2,-0.2)( 1.7, 2.00)
    \psset{linewidth=0.1pt}
    \psaxes[labels=none,ticksize=1pt]{->}(0,0)(-0.1,0)( 1.7, 1.81)
    \psplot[linecolor=red]{0}{ 1.7}{x -0.8 add dup mul 1 add}
    \psline[linecolor=blue](1,0)(1, 1.04)(0, 1.04)
    \rput[tr](1,-0.05){${\scriptstyle x}$}
    \rput[r](- 0.05, 1.04){${\scriptstyle y}$}
  \end{pspicture}\end{center}
  Consider variables $x$ and $y$ related by $y=f(x)$.
}
\onlySlide*{3}{
   \begin{center}\begin{pspicture}[1](-0.2,-0.2)( 1.7, 2.00)
    \psset{linewidth=0.1pt}
    \psaxes[labels=none,ticksize=1pt]{->}(0,0)(-0.1,0)( 1.7, 1.81)
    \psplot[linecolor=red]{0}{ 1.7}{x -0.8 add dup mul 1 add}
    \psline[linecolor=Orange](0, 0.64)( 1.7, 1.32)
    \psline[linecolor=blue](1,0)(1, 1.04)(0, 1.04)
    \rput[tr](1,-0.05){${\scriptstyle x}$}
    \rput[r](- 0.05, 1.04){${\scriptstyle y}$}
    \rput[tl]( 1.75, 1.32){slope $\scriptstyle dy/dx$}
  \end{pspicture}\end{center}
  $dy/dx$ is the slope of the tangent line to the graph.
}
\onlySlide*{4}{
   \begin{center}\begin{pspicture}[1](-0.2,-0.2)( 1.7, 2.00)
    \psset{linewidth=0.1pt}
    \psaxes[labels=none,ticksize=1pt]{->}(0,0)(-0.1,0)( 1.7, 1.81)
    \psplot[linecolor=red]{0}{ 1.7}{x -0.8 add dup mul 1 add}
    \psline[linecolor=Orange](0, 0.64)( 1.7, 1.32)
    \psline[linecolor=blue](1,0)(1, 1.04)(0, 1.04)
    \psline[linecolor=blue]( 1.5,0)( 1.5, 1.49)(0, 1.49)
    \rput[tr](1,-0.05){${\scriptstyle x}$}
    \rput[tl]( 1.5,-0.05){${\scriptstyle x+\dl x}$}
    \rput[r](- 0.05, 1.04){${\scriptstyle y}$}
    \rput[r](- 0.05, 1.49){${\scriptstyle y+\dl y}$}
    \rput[tl]( 1.75, 1.32){slope $\scriptstyle dy/dx$}
  \end{pspicture}\end{center}
  If $x$ changes by a small amount $\dl x$, then $y$ will change by a
  small amount $\dl y$.
}
\onlySlide*{5}{
   \begin{center}\begin{pspicture}[1](-0.2,-0.2)( 1.7, 2.00)
    \psset{linewidth=0.1pt}
    \psaxes[labels=none,ticksize=1pt]{->}(0,0)(-0.1,0)( 1.7, 1.81)
    \psplot[linecolor=red]{0}{ 1.7}{x -0.8 add dup mul 1 add}
    \psline[linecolor=Orange](0, 0.64)( 1.7, 1.32)
    \psline[linecolor=blue](1,0)(1, 1.04)(0, 1.04)
    \psline[linecolor=blue]( 1.5,0)( 1.5, 1.49)(0, 1.49)
    \rput[tr](1,-0.05){${\scriptstyle x}$}
    \rput[tl]( 1.5,-0.05){${\scriptstyle x+\dl x}$}
    \rput[r](- 0.05, 1.04){${\scriptstyle y}$}
    \rput[r](- 0.05, 1.49){${\scriptstyle y+\dl y}$}
    \psline[linecolor=OliveGreen,arrows=<->](1, 1.04)( 1.5, 1.04)
    \psline[linecolor=OliveGreen,arrows=<->](1, 1.04)(1, 1.49)
    \rput[t]( 1.250000000, 0.94){${\scriptstyle \dl x}$} 
    \rput[r]( 0.9, 1.265000000){${\scriptstyle \dl y}$}
    \rput[tl]( 1.75, 1.32){slope $\scriptstyle dy/dx$}
  \end{pspicture}\end{center}
  If $x$ changes by a small amount $\dl x$, then $y$ will change by a
  small amount $\dl y$.
}
\onlySlide*{6}{
   \begin{center}\begin{pspicture}[1](-0.2,-0.2)( 1.7, 2.00)
    \psset{linewidth=0.1pt}
    \psaxes[labels=none,ticksize=1pt]{->}(0,0)(-0.1,0)( 1.7, 1.81)
    \psplot[linecolor=red]{0}{ 1.7}{x -0.8 add dup mul 1 add}
    \psline[linecolor=OliveGreen](0, 0.1400000000)( 1.7, 1.670000000)
    \psline[linecolor=Orange](0, 0.64)( 1.7, 1.32)
    \psline[linecolor=blue](1,0)(1, 1.04)(0, 1.04)
    \psline[linecolor=blue]( 1.5,0)( 1.5, 1.49)(0, 1.49)
    \rput[tr](1,-0.05){${\scriptstyle x}$}
    \rput[tl]( 1.5,-0.05){${\scriptstyle x+\dl x}$}
    \rput[r](- 0.05, 1.04){${\scriptstyle y}$}
    \rput[r](- 0.05, 1.49){${\scriptstyle y+\dl y}$}
    \psline[linecolor=OliveGreen,arrows=<->](1, 1.04)( 1.5, 1.04)
    \psline[linecolor=OliveGreen,arrows=<->](1, 1.04)(1, 1.49)
    \rput[t]( 1.250000000, 0.94){${\scriptstyle \dl x}$} 
    \rput[r]( 0.9, 1.265000000){${\scriptstyle \dl y}$}
    \rput[bl]( 1.75, 1.670000000){slope $\scriptstyle \dl y/\dl x$}
    \rput[tl]( 1.75, 1.32){slope $\scriptstyle dy/dx$}
  \end{pspicture}\end{center}
  The ratio $\dl y/\dl x$ is the slope of a chord cutting across the
  graph.
}
\onlySlide*{7}{
  \begin{center}\begin{pspicture}[1](-0.2,-0.2)( 1.7, 2.00)
    \psset{linewidth=0.1pt}
    \psaxes[labels=none,ticksize=1pt]{->}(0,0)(-0.1,0)( 1.7, 1.81)
    \psplot[linecolor=red]{0}{ 1.7}{x -0.8 add dup mul 1 add}
    \psline[linecolor=OliveGreen](0, 0.3900000000)( 1.7, 1.495000000)
    \psline[linecolor=Orange](0, 0.64)( 1.7, 1.32)
    \psline[linecolor=blue](1,0)(1, 1.04)(0, 1.04)
    \psline[linecolor=blue]( 1.250000000,0)( 1.250000000, 1.202500000)(0, 1.202500000)
    \rput[tr](1,-0.05){${\scriptstyle x}$}
    \rput[tl]( 1.250000000,-0.05){${\scriptstyle x+\dl x}$}
    \rput[r](- 0.05, 1.04){${\scriptstyle y}$}
    \rput[r](- 0.05, 1.202500000){${\scriptstyle y+\dl y}$}
    \psline[linecolor=OliveGreen,arrows=<->](1, 1.04)( 1.250000000, 1.04)
    \psline[linecolor=OliveGreen,arrows=<->](1, 1.04)(1, 1.202500000)
    \rput[t]( 1.125000000, 0.94){${\scriptstyle \dl x}$} 
    \rput[r]( 0.9, 1.121250000){${\scriptstyle \dl y}$}
    \rput[bl]( 1.75, 1.495000000){slope $\scriptstyle \dl y/\dl x$}
    \rput[tl]( 1.75, 1.32){slope $\scriptstyle dy/dx$}
  \end{pspicture}\end{center}
  The slope of the chord changes slightly as $\dl x$ decreases.
}
\onlySlide*{8}{
  \begin{center}\begin{pspicture}[1](-0.2,-0.2)( 1.7, 2.00)
    \psset{linewidth=0.1pt}
    \psaxes[labels=none,ticksize=1pt]{->}(0,0)(-0.1,0)( 1.7, 1.81)
    \psplot[linecolor=red]{0}{ 1.7}{x -0.8 add dup mul 1 add}
    \psline[linecolor=OliveGreen](0, 0.5150000000)( 1.7, 1.407500000)
    \psline[linecolor=Orange](0, 0.64)( 1.7, 1.32)
    \psline[linecolor=blue](1,0)(1, 1.04)(0, 1.04)
    \psline[linecolor=blue]( 1.125000000,0)( 1.125000000, 1.105625000)(0, 1.105625000)
    \rput[tr](1,-0.05){${\scriptstyle x}$}
    \rput[tl]( 1.125000000,-0.05){${\scriptstyle x+\dl x}$}
    \rput[r](- 0.05, 1.04){${\scriptstyle y}$}
    \rput[r](- 0.05, 1.105625000){${\scriptstyle y+\dl y}$}
    \psline[linecolor=OliveGreen,arrows=<->](1, 1.04)( 1.125000000, 1.04)
    \psline[linecolor=OliveGreen,arrows=<->](1, 1.04)(1, 1.105625000)
    \rput[t]( 1.062500000, 0.94){${\scriptstyle \dl x}$} 
    \rput[r]( 0.9, 1.072812500){${\scriptstyle \dl y}$}
    \rput[bl]( 1.75, 1.407500000){slope $\scriptstyle \dl y/\dl x$}
    \rput[tl]( 1.75, 1.32){slope $\scriptstyle dy/dx$}
  \end{pspicture}\end{center}
  As $\dl x$ approaches zero, the chord approaches the tangent, and
  $\dl y/\dl x$ approaches $dy/dx$.
}
\onlySlide*{9}{
  \begin{center}\begin{pspicture}[1](-0.2,-0.2)( 1.7, 2.00)
    \psset{linewidth=0.1pt}
    \psaxes[labels=none,ticksize=1pt]{->}(0,0)(-0.1,0)( 1.7, 1.81)
    \psplot[linecolor=red]{0}{ 1.7}{x -0.8 add dup mul 1 add}
    \psline[linecolor=OliveGreen](0, 0.5775000000)( 1.7, 1.363750000)
    \psline[linecolor=Orange](0, 0.64)( 1.7, 1.32)
    \psline[linecolor=blue](1,0)(1, 1.04)(0, 1.04)
    \psline[linecolor=blue]( 1.062500000,0)( 1.062500000, 1.068906250)(0, 1.068906250)
    \rput[tr](1,-0.05){${\scriptstyle x}$}
    \rput[tl]( 1.062500000,-0.05){${\scriptstyle x+\dl x}$}
    \rput[r](- 0.05, 1.04){${\scriptstyle y}$}
    \rput[r](- 0.05, 1.068906250){${\scriptstyle y+\dl y}$}
    \psline[linecolor=OliveGreen,arrows=<->](1, 1.04)( 1.062500000, 1.04)
    \psline[linecolor=OliveGreen,arrows=<->](1, 1.04)(1, 1.068906250)
    \rput[t]( 1.031250000, 0.94){${\scriptstyle \dl x}$} 
    \rput[r]( 0.9, 1.054453125){${\scriptstyle \dl y}$}
    \rput[bl]( 1.75, 1.363750000){slope $\scriptstyle \dl y/\dl x$}
    \rput[tl]( 1.75, 1.32){slope $\scriptstyle dy/dx$}
  \end{pspicture}\end{center}
  As $\dl x$ approaches zero, the chord approaches the tangent, and
  $\dl y/\dl x$ approaches $dy/dx$.
}
\fromSlide{10}{
  \begin{center}\begin{pspicture}[1](-0.2,-0.2)( 1.7, 2.00)
    \psset{linewidth=0.1pt}
    \psaxes[labels=none,ticksize=1pt]{->}(0,0)(-0.1,0)( 1.7, 1.81)
    \psplot[linecolor=red]{0}{ 1.7}{x -0.8 add dup mul 1 add}
    \psline[linecolor=OliveGreen](0, 0.64)( 1.7, 1.32)
    \psline[linecolor=blue](1,0)(1, 1.04)(0, 1.04)
    \rput[tr](1,-0.05){${\scriptstyle x}$}
    \rput[r](- 0.05, 1.04){${\scriptstyle y}$}
    \rput[tl]( 1.75, 1.32){slope $\scriptstyle dy/dx$}
  \end{pspicture}\end{center}
  As $\dl x$ approaches zero, the chord approaches the tangent, and
  $\dl y/\dl x$ approaches $dy/dx$.
}
\end{slide}
}

\overlays{10}{%
\begin{slide}{Areas}

 \psset{xunit=2.5cm,yunit=1.25cm}
 \onlySlide*{1}{
  \begin{center}\begin{pspicture}[1](-0.3,-0.3)(3.8,5.3)
   \SpecialCoor
   \psset{linewidth=0.1pt}
   \psaxes[labels=none,ticksize=0pt]{->}(0,0)(-0.3,-0.3)( 3.8, 5.3)
   \psplot[linecolor=red]{0}{3.5}{x \ff}
   \vl{1.0}
   \vl{3.0}
   \rput[B](1.0,-0.2){$\scriptstyle a$}
   \rput[B](3.0,-0.2){$\scriptstyle b$}
   \rput(3.0,4.7){$\scriptstyle y=f(x)$}
  \end{pspicture}\end{center}
  Consider the integral $\int_a^b fx)\,dx$.
 }
 \onlySlide*{2}{
  \begin{center}\begin{pspicture}[1](-0.3,-0.3)(3.8,5.3)
   \SpecialCoor
   \psset{linewidth=0.1pt}
   \bx{1.4}
   \psaxes[labels=none,ticksize=0pt]{->}(0,0)(-0.3,-0.3)( 3.8, 5.3)
   \psplot[linecolor=red]{0}{3.5}{x \ff}
   \vl{1.0}
   \vl{3.0}
   \psline[linecolor=magenta,arrows=<->](1.35,0)(! 1.35 1.40 \ff)
   \rput[r](! 1.32 1.40 \ff\space 0.5 mul){$\scriptstyle f(x)$}
   \psline[linecolor=magenta,arrows=<->]%
    (! 1.40 1.40 \ff\space 0.1 add)(! 1.60 1.40 \ff\space 0.1 add)
   \rput[b](! 1.50 1.40 \ff\space 0.2 add){$\scriptstyle h$}
   \psline[arrows=->](1.8,0.5)(1.5,0.5)
   \rput[l](1.82,0.5){$\scriptstyle \mbox{Area} = f(x)h$}
   \rput[B](1.0,-0.2){$\scriptstyle a$}
   \rput[Br](1.4,-0.2){$\scriptstyle x$}
   \rput[Bl](1.6,-0.2){$\scriptstyle x+h$}
   \rput[B](3.0,-0.2){$\scriptstyle b$}
   \rput(3.0,4.7){$\scriptstyle y=f(x)$}
  \end{pspicture}\end{center}
  For each short interval $[x,x+h]\subset [a,b]$, we have a contribution
  $f(x)h$.  This is the area of the green rectangle.
 }
 \onlySlide*{3}{
  \begin{center}\begin{pspicture}[1](-0.3,-0.3)(3.8,5.3)
   \SpecialCoor
   \psset{linewidth=0.1pt}
   \bx{1.4}
   \psaxes[labels=none,ticksize=0pt]{->}(0,0)(-0.3,-0.3)( 3.8, 5.3)
   \psplot[linecolor=red]{0}{3.5}{x \ff}
   \vl{1.0}
   \vl{3.0}
   \rput[B](1.0,-0.2){$\scriptstyle a$}
   \rput[B](3.0,-0.2){$\scriptstyle b$}
   \rput(3.0,4.7){$\scriptstyle y=f(x)$}
  \end{pspicture}\end{center}
  This is the contribution from one short interval, but we need to add
  together the contributions from many short intervals.
 }
 \onlySlide*{4}{
  \begin{center}\begin{pspicture}[1](-0.3,-0.3)(3.8,5.3)
   \SpecialCoor
   \psset{linewidth=0.1pt}
   \bx{1.2} \bx{1.4} \bx{1.6}
   \psaxes[labels=none,ticksize=0pt]{->}(0,0)(-0.3,-0.3)( 3.8, 5.3)
   \psplot[linecolor=red]{0}{3.5}{x \ff}
   \vl{1.0}
   \vl{3.0}
   \rput[B](1.0,-0.2){$\scriptstyle a$}
   \rput[B](3.0,-0.2){$\scriptstyle b$}
   \rput(3.0,4.7){$\scriptstyle y=f(x)$}
  \end{pspicture}\end{center}
  Here we have added in two more intervals
 }
 \onlySlide*{5}{
  \begin{center}\begin{pspicture}[1](-0.3,-0.3)(3.8,5.3)
   \SpecialCoor
   \psset{linewidth=0.1pt}
   \bx{1.0} \bx{1.2} \bx{1.4} \bx{1.6} \bx{1.8}
   \psaxes[labels=none,ticksize=0pt]{->}(0,0)(-0.3,-0.3)( 3.8, 5.3)
   \psplot[linecolor=red]{0}{3.5}{x \ff}
   \vl{1.0}
   \vl{3.0}
   \rput[B](1.0,-0.2){$\scriptstyle a$}
   \rput[B](3.0,-0.2){$\scriptstyle b$}
   \rput(3.0,4.7){$\scriptstyle y=f(x)$}
  \end{pspicture}\end{center}
  Here we have added in two more intervals
  --- and two more
 }
 \onlySlide*{6}{
  \begin{center}\begin{pspicture}[1](-0.3,-0.3)(3.8,5.3)
   \SpecialCoor
   \psset{linewidth=0.1pt}
   \bx{1.0} \bx{1.2} \bx{1.4} \bx{1.6} \bx{1.8} 
   \bx{2.0} \bx{2.2}
   \psaxes[labels=none,ticksize=0pt]{->}(0,0)(-0.3,-0.3)( 3.8, 5.3)
   \psplot[linecolor=red]{0}{3.5}{x \ff}
   \vl{1.0}
   \vl{3.0}
   \rput[B](1.0,-0.2){$\scriptstyle a$}
   \rput[B](3.0,-0.2){$\scriptstyle b$}
   \rput(3.0,4.7){$\scriptstyle y=f(x)$}
  \end{pspicture}\end{center}
  Here we have added in two more intervals
  --- and two more --- and two more
 }
 \onlySlide*{7}{
  \begin{center}\begin{pspicture}[1](-0.3,-0.3)(3.8,5.3)
   \SpecialCoor
   \psset{linewidth=0.1pt}
   \bx{1.0} \bx{1.2} \bx{1.4} \bx{1.6} \bx{1.8} 
   \bx{2.0} \bx{2.2} \bx{2.4} \bx{2.6} \bx{2.8}
   \psaxes[labels=none,ticksize=0pt]{->}(0,0)(-0.3,-0.3)( 3.8, 5.3)
   \psplot[linecolor=red]{0}{3.5}{x \ff}
   \vl{1.0}
   \vl{3.0}
   \rput[B](1.0,-0.2){$\scriptstyle a$}
   \rput[B](3.0,-0.2){$\scriptstyle b$}
   \rput(3.0,4.7){$\scriptstyle y=f(x)$}
  \end{pspicture}\end{center}
  Now we have divided the whole interval $[a,b]$ into subintervals of
  length $h$.  The sum of the terms $f(x)h$ is the area of the green
  region.
 }
 \onlySlide*{8}{
  \begin{center}\begin{pspicture}[1](-0.3,-0.3)(3.8,5.3)
   \SpecialCoor
   \psset{linewidth=0.1pt}
   \bxp{1.0} \bxp{1.2} \bxp{1.4} \bxp{1.6} \bxp{1.8} 
   \bxn{2.0} \bxn{2.2} \bxn{2.4} \bxn{2.6} \bxn{2.8}
   \psaxes[labels=none,ticksize=0pt]{->}(0,0)(-0.3,-0.3)( 3.8, 5.3)
   \psplot[linecolor=red]{0}{3.5}{x \ff}
   \vl{1.0}
   \vl{3.0}
   \rput[B](1.0,-0.2){$\scriptstyle a$}
   \rput[B](3.0,-0.2){$\scriptstyle b$}
   \rput(3.0,4.7){$\scriptstyle y=f(x)$}
  \end{pspicture}\end{center}
  This is not exactly the same as the area under the curve, because of
  the regions marked in blue and pink.
 }
 \onlySlide*{9}{
  \renewcommand{\hh}{0.1}
  \begin{center}\begin{pspicture}[1](-0.3,-0.3)(3.8,5.3)
   \SpecialCoor
   \psset{linewidth=0.1pt}
   \bxp{1.0} \bxp{1.1} \bxp{1.2} \bxp{1.3} \bxp{1.4} 
   \bxp{1.5} \bxp{1.6} \bxp{1.7} \bxp{1.8} \bxp{1.9}
   \bxn{2.0} \bxn{2.1} \bxn{2.2} \bxn{2.3} \bxn{2.4} 
   \bxn{2.5} \bxn{2.6} \bxn{2.7} \bxn{2.8} \bxn{2.9}
   \psaxes[labels=none,ticksize=0pt]{->}(0,0)(-0.3,-0.3)( 3.8, 5.3)
   \psplot[linecolor=red]{0}{3.5}{x \ff}
   \vl{1.0}
   \vl{3.0}
   \rput[B](1.0,-0.2){$\scriptstyle a$}
   \rput[B](3.0,-0.2){$\scriptstyle b$}
   \rput(3.0,4.7){$\scriptstyle y=f(x)$}
  \end{pspicture}\end{center}
  However, the error decreases if we make $h$ smaller
 }
 \onlySlide*{10}{
  \begin{center}\begin{pspicture}[1](-0.3,-0.3)(3.8,5.3)
   \SpecialCoor
   \psset{linewidth=0.1pt}
   \pscustom[linestyle=none,fillstyle=solid,fillcolor=green]{%
    \psplot{1.0}{3.0}{x \ff}
    \psline(3.0,0)(1.0,0)}
   \psaxes[labels=none,ticksize=0pt]{->}(0,0)(-0.3,-0.3)( 3.8, 5.3)
   \psplot[linecolor=red]{0}{3.5}{x \ff}
   \vl{1.0}
   \vl{3.0}
   \rput[Bc](1.0,-0.2){$\scriptstyle a$}
   \rput[Bc](3.0,-0.2){$\scriptstyle b$}
   \rput[cc](3.0,4.7){$\scriptstyle y=f(x)$}
  \end{pspicture}\end{center}
  However, the error decreases if we make $h$ smaller, and tends to
  zero in the limit.
 }
\end{slide}
}

\end{document}

%%%%%%%%%%%%%%%%%%%%%%%%%%%%%%%%%%%%%%%%%%%%%%%%%%%%%%%%%%%%%%%%%%%%%%

