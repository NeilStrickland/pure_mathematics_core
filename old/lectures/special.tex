\documentclass[%
pdf,
neil,
colorBG,
slideColor,
]{prosper}
\usepackage{amsmath}
\usepackage{pstricks,pst-node,pst-text,pst-3d,pst-plot}
\usepackage[usenames,dvips]{color}

\newcommand{\GREENYELLOW}[1]{{\color{GreenYellow}#1}}
\newcommand{\YELLOW}[1]{{\color{Yellow}#1}}
\newcommand{\YLW}[1]{{\color{Yellow}#1}}
\newcommand{\GOLDENROD}[1]{{\color{Goldenrod}#1}}
\newcommand{\DANDELION}[1]{{\color{Dandelion}#1}}
\newcommand{\APRICOT}[1]{{\color{Apricot}#1}}
\newcommand{\PEACH}[1]{{\color{Peach}#1}}
\newcommand{\MELON}[1]{{\color{Melon}#1}}
\newcommand{\YELLOWORANGE}[1]{{\color{YellowOrange}#1}}
\newcommand{\ORANGE}[1]{{\color{Orange}#1}}
\newcommand{\BURNTORANGE}[1]{{\color{BurntOrange}#1}}
\newcommand{\BITTERSWEET}[1]{{\color{Bittersweet}#1}}
\newcommand{\REDORANGE}[1]{{\color{RedOrange}#1}}
\newcommand{\MAHOGANY}[1]{{\color{Mahogany}#1}}
\newcommand{\MAROON}[1]{{\color{Maroon}#1}}
\newcommand{\BRICKRED}[1]{{\color{BrickRed}#1}}
\newcommand{\RED}[1]{{\color{Red}#1}}
\newcommand{\ORANGERED}[1]{{\color{OrangeRed}#1}}
\newcommand{\RUBINERED}[1]{{\color{RubineRed}#1}}
\newcommand{\WILDSTRAWBERRY}[1]{{\color{WildStrawberry}#1}}
\newcommand{\SALMON}[1]{{\color{Salmon}#1}}
\newcommand{\CARNATIONPINK}[1]{{\color{CarnationPink}#1}}
\newcommand{\MAGENTA}[1]{{\color{Magenta}#1}}
\newcommand{\VIOLETRED}[1]{{\color{VioletRed}#1}}
\newcommand{\RHODAMINE}[1]{{\color{Rhodamine}#1}}
\newcommand{\MULBERRY}[1]{{\color{Mulberry}#1}}
\newcommand{\REDVIOLET}[1]{{\color{RedViolet}#1}}
\newcommand{\FUCHSIA}[1]{{\color{Fuchsia}#1}}
\newcommand{\LAVENDER}[1]{{\color{Lavender}#1}}
\newcommand{\THISTLE}[1]{{\color{Thistle}#1}}
\newcommand{\ORCHID}[1]{{\color{Orchid}#1}}
\newcommand{\DARKORCHID}[1]{{\color{DarkOrchid}#1}}
\newcommand{\PURPLE}[1]{{\color{Purple}#1}}
\newcommand{\PLUM}[1]{{\color{Plum}#1}}
\newcommand{\VIOLET}[1]{{\color{Violet}#1}}
\newcommand{\ROYALPURPLE}[1]{{\color{RoyalPurple}#1}}
\newcommand{\BLUEVIOLET}[1]{{\color{BlueViolet}#1}}
\newcommand{\PERIWINKLE}[1]{{\color{Periwinkle}#1}}
\newcommand{\CADETBLUE}[1]{{\color{CadetBlue}#1}}
\newcommand{\CORNFLOWERBLUE}[1]{{\color{CornflowerBlue}#1}}
\newcommand{\MIDNIGHTBLUE}[1]{{\color{MidnightBlue}#1}}
\newcommand{\NAVYBLUE}[1]{{\color{NavyBlue}#1}}
\newcommand{\ROYALBLUE}[1]{{\color{RoyalBlue}#1}}
\newcommand{\BLUE}[1]{{\color{Blue}#1}}
\newcommand{\CERULEAN}[1]{{\color{Cerulean}#1}}
\newcommand{\CYAN}[1]{{\color{Cyan}#1}}
\newcommand{\PROCESSBLUE}[1]{{\color{ProcessBlue}#1}}
\newcommand{\SKYBLUE}[1]{{\color{SkyBlue}#1}}
\newcommand{\TURQUOISE}[1]{{\color{Turquoise}#1}}
\newcommand{\TEALBLUE}[1]{{\color{TealBlue}#1}}
\newcommand{\AQUAMARINE}[1]{{\color{Aquamarine}#1}}
\newcommand{\BLUEGREEN}[1]{{\color{BlueGreen}#1}}
\newcommand{\EMERALD}[1]{{\color{Emerald}#1}}
\newcommand{\JUNGLEGREEN}[1]{{\color{JungleGreen}#1}}
\newcommand{\SEAGREEN}[1]{{\color{SeaGreen}#1}}
\newcommand{\GREEN}[1]{{\color{Green}#1}}
\newcommand{\FORESTGREEN}[1]{{\color{ForestGreen}#1}}
\newcommand{\PINEGREEN}[1]{{\color{PineGreen}#1}}
\newcommand{\LIMEGREEN}[1]{{\color{LimeGreen}#1}}
\newcommand{\YELLOWGREEN}[1]{{\color{YellowGreen}#1}}
\newcommand{\SPRINGGREEN}[1]{{\color{SpringGreen}#1}}
\newcommand{\OLIVEGREEN}[1]{{\color{OliveGreen}#1}}
\newcommand{\OLG}[1]{{\color{OliveGreen}#1}}
\newcommand{\RAWSIENNA}[1]{{\color{RawSienna}#1}}
\newcommand{\SEPIA}[1]{{\color{Sepia}#1}}
\newcommand{\BROWN}[1]{{\color{Brown}#1}}
\newcommand{\TAN}[1]{{\color{Tan}#1}}
\newcommand{\GRAY}[1]{{\color{Gray}#1}}
\newcommand{\WHITE}[1]{{\color{White}#1}}
\newcommand{\BLACK}[1]{{\color{Black}#1}}

\newcmykcolor{GreenYellow}{0.15 0 0.69 0}
\newcmykcolor{Yellow}{0 0 1 0}
\newcmykcolor{Goldenrod}{0 0.10 0.84 0}
\newcmykcolor{Dandelion}{0 0.29 0.84 0}
\newcmykcolor{Apricot}{0 0.32 0.52 0}
\newcmykcolor{Peach}{0 0.50 0.70 0}
\newcmykcolor{Melon}{0 0.46 0.50 0}
\newcmykcolor{YellowOrange}{0 0.42 1 0}
\newcmykcolor{Orange}{0 0.61 0.87 0}
\newcmykcolor{BurntOrange}{0 0.51 1 0}
\newcmykcolor{Bittersweet}{0 0.75 1 0.24}
\newcmykcolor{RedOrange}{0 0.77 0.87 0}
\newcmykcolor{Mahogany}{0 0.85 0.87 0.35}
\newcmykcolor{Maroon}{0 0.87 0.68 0.32}
\newcmykcolor{BrickRed}{0 0.89 0.94 0.28}
\newcmykcolor{Red}{0 1 1 0}
\newcmykcolor{OrangeRed}{0 1 0.50 0}
\newcmykcolor{RubineRed}{0 1 0.13 0}
\newcmykcolor{WildStrawberry}{0 0.96 0.39 0}
\newcmykcolor{Salmon}{0 0.53 0.38 0}
\newcmykcolor{CarnationPink}{0 0.63 0 0}
\newcmykcolor{Magenta}{0 1 0 0}
\newcmykcolor{VioletRed}{0 0.81 0 0}
\newcmykcolor{Rhodamine}{0 0.82 0 0}
\newcmykcolor{Mulberry}{0.34 0.90 0 0.02}
\newcmykcolor{RedViolet}{0.07 0.90 0 0.34}
\newcmykcolor{Fuchsia}{0.47 0.91 0 0.08}
\newcmykcolor{Lavender}{0 0.48 0 0}
\newcmykcolor{Thistle}{0.12 0.59 0 0}
\newcmykcolor{Orchid}{0.32 0.64 0 0}
\newcmykcolor{DarkOrchid}{0.40 0.80 0.20 0}
\newcmykcolor{Purple}{0.45 0.86 0 0}
\newcmykcolor{Plum}{0.50 1 0 0}
\newcmykcolor{Violet}{0.79 0.88 0 0}
\newcmykcolor{RoyalPurple}{0.75 0.90 0 0}
\newcmykcolor{BlueViolet}{0.86 0.91 0 0.04}
\newcmykcolor{Periwinkle}{0.57 0.55 0 0}
\newcmykcolor{CadetBlue}{0.62 0.57 0.23 0}
\newcmykcolor{CornflowerBlue}{0.65 0.13 0 0}
\newcmykcolor{MidnightBlue}{0.98 0.13 0 0.43}
\newcmykcolor{NavyBlue}{0.94 0.54 0 0}
\newcmykcolor{RoyalBlue}{1 0.50 0 0}
\newcmykcolor{Blue}{1 1 0 0}
\newcmykcolor{Cerulean}{0.94 0.11 0 0}
\newcmykcolor{Cyan}{1 0 0 0}
\newcmykcolor{ProcessBlue}{0.96 0 0 0}
\newcmykcolor{SkyBlue}{0.62 0 0.12 0}
\newcmykcolor{Turquoise}{0.85 0 0.20 0}
\newcmykcolor{TealBlue}{0.86 0 0.34 0.02}
\newcmykcolor{Aquamarine}{0.82 0 0.30 0}
\newcmykcolor{BlueGreen}{0.85 0 0.33 0}
\newcmykcolor{Emerald}{1 0 0.50 0}
\newcmykcolor{JungleGreen}{0.99 0 0.52 0}
\newcmykcolor{SeaGreen}{0.69 0 0.50 0}
\newcmykcolor{Green}{1 0 1 0}
\newcmykcolor{ForestGreen}{0.91 0 0.88 0.12}
\newcmykcolor{PineGreen}{0.92 0 0.59 0.25}
\newcmykcolor{LimeGreen}{0.50 0 1 0}
\newcmykcolor{YellowGreen}{0.44 0 0.74 0}
\newcmykcolor{SpringGreen}{0.26 0 0.76 0}
\newcmykcolor{OliveGreen}{0.64 0 0.95 0.40}
\newcmykcolor{RawSienna}{0 0.72 1 0.45}
\newcmykcolor{Sepia}{0 0.83 1 0.70}
\newcmykcolor{Brown}{0 0.81 1 0.60}
\newcmykcolor{Tan}{0.14 0.42 0.56 0}
\newcmykcolor{Gray}{0 0 0 0.50}
\newcmykcolor{Black}{0 0 0 1}
\newcmykcolor{White}{0 0 0 0}


\newcommand{\bbm}       {\left[\begin{matrix}}
\newcommand{\ebm}       {\end{matrix}\right]}
\newcommand{\bsm}       {\left[\begin{smallmatrix}}
\newcommand{\esm}       {\end{smallmatrix}\right]}
\newcommand{\bpm}       {\begin{pmatrix}}
\newcommand{\epm}       {\end{pmatrix}}
\newcommand{\bcf}[2]{\left(\begin{array}{c}{#1}\\{#2}\end{array}\right)}


\newcommand{\csch}     {\operatorname{csch}}
\newcommand{\sech}     {\operatorname{sech}}
\newcommand{\arcsinh}  {\operatorname{arcsinh}}
\newcommand{\arccosh}  {\operatorname{arccosh}}
\newcommand{\arctanh}  {\operatorname{arctanh}}

\newcommand{\range}     {\operatorname{range}}
\newcommand{\trans}     {\operatorname{trans}}
\newcommand{\trc}       {\operatorname{trace}}
\newcommand{\adj}       {\operatorname{adj}}

\newcommand{\tint}{\textstyle\int}
\newcommand{\tm}{\times}
\newcommand{\sse}{\subseteq}
\newcommand{\st}{\;|\;}
\newcommand{\sm}{\setminus}
\newcommand{\iffa}      {\Leftrightarrow}
\newcommand{\xra}{\xrightarrow}

\renewcommand{\:}{\colon}

\newcommand{\N}         {{\mathbb{N}}}
\newcommand{\Z}         {{\mathbb{Z}}}
\newcommand{\Q}         {{\mathbb{Q}}}
\renewcommand{\R}       {{\mathbb{R}}}
\newcommand{\C}         {{\mathbb{C}}}

\newcommand{\al}        {\alpha}
\newcommand{\bt}        {\beta} 
\newcommand{\gm}        {\gamma}
\newcommand{\dl}        {\delta}
\newcommand{\ep}        {\epsilon}
\newcommand{\zt}        {\zeta}
\newcommand{\et}        {\eta}
\newcommand{\tht}       {\theta}
\newcommand{\io}        {\iota}
\newcommand{\kp}        {\kappa}
\newcommand{\lm}        {\lambda}
\newcommand{\ph}        {\phi}
\newcommand{\ch}        {\chi}
\newcommand{\ps}        {\psi}
\newcommand{\rh}        {\rho}
\newcommand{\sg}        {\sigma}
\newcommand{\om}        {\omega}

\newcommand{\EMPH}[1]{\emph{\RED{#1}}}
\newcommand{\DEFN}[1]{\emph{\PURPLE{#1}}}
\newcommand{\VEC}[1]    {\mathbf{#1}}

\newcommand{\ghost}{{\tiny $\color[rgb]{1,1,1}.$}}


\title{Special functions}
\author{}

\begin{document}

%\slideCaption{\color{white}}

\begin{slide}{}
 {\Huge
  \vspace{6ex}
  \begin{center}
   Special functions
  \end{center}
 }
\end{slide}

%\maketitle

\overlays{12}{
\begin{slide}{Introduction}
\fromSlide{2}{
 The \DEFN{primary special functions} are
 \begin{center}
  $\exp$, $\log$, $\sin$, $\cos$, $\tan$,
  $\arcsin$, $\arccos$, $\arctan$.
 \end{center} }
\fromSlide{3}{\noindent{\bf Things you should know:}}
\begin{itemize}
\fromSlide{4}{
 \item The detailed shape of the graphs}
\fromSlide{5}{
 \item Domains, ranges and inverses}
\fromSlide{6}{
 \item Properties such as $\sin(x+y)=\sin(x)\cos(y)+\cos(x)\sin(y)$}
\fromSlide{7}{
 \item Derivatives and integrals (covered in later lectures).
}
\end{itemize}

\fromSlide{8}{
 The \DEFN{secondary special functions} are
 \begin{center}
  $\sec$, $\csc$, $\cot$, $\sinh$, $\cosh$, $\tanh$, \\
  $\sech$, $\csch$, $\coth$, $\arcsinh$, $\arccosh$, $\arctanh$.
 \end{center}}
\begin{itemize}
 \fromSlide{9}{
  \item You should know how these are defined in terms of the primary
   functions \\
   \fromSlide{10}{
    (for example, $\sinh(x)=(\exp(x)-\exp(-x))/2$, and
                  $\sec(x)=1/\cos(x)$)}}  
  \fromSlide{11}{
   \item You should either remember the properties of the secondary
    functions, or be able to derive them from the properties of the
    primary functions}
\end{itemize}
\end{slide}
}


\overlays{6}{%
\begin{slide}{The exponential function}
\begin{itemize}
 \fromSlide{2}{
  \item $\PURPLE{\exp(x)}=1+x+\frac{x^2}{2!}+\frac{x^3}{3!}+\frac{x^4}{4!}+\cdots$\\
   \fromSlide{3}{
    \ORANGE{\bf Warning:} infinite sums are subtle.}}
 \fromSlide{4}{
  \item $\PURPLE{e}=\exp(1)=1+1+\frac{1}{2!}+\frac{1}{3!}+\cdots
    \simeq 2.71828.$}
 \fromSlide{5}{
  \item \psframebox[framearc=.3,linecolor=magenta]{\tiny
   $\begin{array}{rlcrl}
    \exp(x+y) &= \exp(x)\exp(y) &\qquad& \exp(x-y) &=\exp(x)/\exp(y) \\
    \exp(0)   &= 1              &      & \exp(-x)  &=1/\exp(x) \\
    \exp(nx)  &= \exp(x)^n      &      & \exp(x)   &= e^x
   \end{array}$}}
 \fromSlide{6}{
  \item %
   \psset{xunit=1cm,yunit=.4cm}
    \begin{center}\begin{pspicture}[1](-2.3,-0.5)(2.5,8)
    \psaxes[labels=none,linewidth=0.1pt,ticksize=1pt]{->}(0,0)(-2,0)(2,7.5)
    \psplot[linewidth=0.1pt,linecolor=red]{-2}{2.1}{2.72 x exp}
    \psline[linewidth=0.1pt,linecolor=green](1,0)(1,2.72)(0,2.72)
    \psline[linewidth=0.1pt,linecolor=green](2,0)(2,7.39)(0,7.39)
    \rput[r](-.2,1){$1$}
    \rput[r](-.2,2.72){$e$}
    \rput[r](-.2,7.39){$e^2$}
    \rput[t](-2,-.3){$-2$}
    \rput[t](-1,-.3){$-1$}
    \rput[t]( 0,-.3){$0$}
    \rput[t]( 1,-.3){$1$}
    \rput[t]( 2,-.3){$2$}
   \end{pspicture}\end{center}
 }
\end{itemize}
\end{slide}
}

\overlays{7}{%
\begin{slide}{The logarithm}
\begin{itemize}
 \fromSlide{2}{
  \item The natural log function $\log(y)=\ln(y)$ is the inverse of the
   exponential.  }
 \fromSlide{3}{
  \item $\log(y)$ is defined only when $y>0$ (unless we use complex
   numbers).}
 \fromSlide{4}{
  \item We have $\log(\exp(x))=\log(e^x)=x$ for all $x$, and
        $\exp(\log(y))=e^{\log(y)}=y$ when $y>0$ 
        \fromSlide{5}{(\EMPH{\underline{NOT}} $\log(x)=1/\exp(x)$)}.}
 \fromSlide{6}{
  \item \psframebox[framearc=.3,linecolor=magenta]{\tiny
   $\begin{array}{rlcrl}
    \log(xy) &= \log(x)+\log(y) &\qquad& \log(x/y) &= \log(x)-\log(y) \\
    \log(1)  &= 0               &      & \log(1/y) &= -\log(y) \\
    \log(y^n)&= n\log(y)        &      & \log(e)   &= 1.
   \end{array}$}}
 \fromSlide{7}{
  \item \ghost \vspace{-5ex}
   \psset{yunit=.8cm,xunit=.4cm}
   \begin{center}\begin{pspicture}[1](-.5,-2.3)(8,2.5)
   \psaxes[labels=none,linewidth=0.1pt,ticksize=1pt]{->}(0,0)(0,-2)(7.5,2)
   \parametricplot[linewidth=0.1pt,linecolor=red]{-2}{2.1}{2.72 t exp t}
   \psline[linewidth=0.1pt,linecolor=green](0,1)(2.72,1)(2.72,0)
   \psline[linewidth=0.1pt,linecolor=green](0,2)(7.39,2)(7.39,0)
   \rput[t](1,-.2){$1$}
   \rput[t](2.72,-.2){$e$}
   \rput[t](7.39,-.2){$e^2$}
   \rput[r](-.3,-2){$-2$}
   \rput[r](-.3,-1){$-1$}
   \rput[r](-.3, 0){$0$}
   \rput[r](-.3, 1){$1$}
   \rput[r](-.3, 2){$2$}
   \rput(5,1){$\log(x)$}
   \end{pspicture}\end{center}}
\end{itemize} 
\end{slide}
}

\overlays{10}{%
\begin{slide}{Logs to other bases}
\begin{itemize}
 \fromSlide{2}{
  \item $\PURPLE{\log_a(y)}$ is the number $t$ such that $y=a^t$
   (defined for $a,y>0$).
 }
 \fromSlide{3}{
  \item \ghost \vspace{-4ex}
   {\tiny \begin{align*}
    \log_{10}(1000) &= \log_{10}(10^3) = 3 \\
    \log_2(1024) &= \log_2(2^{10}) = 10 \\
%    \log_{1024}(2) &= \log_{1024}(1024^{1/10}) = 1/10 \\
    \log_3(1/9) &= \log_3(3^{-2}) = -2
   \end{align*}}
 }
 \fromSlide{4}{
  \item $\log_a(y)=\log(y)/\log(a)$
 }
 \fromSlide{5}{
  \item {\ORANGE{Check:}}
   $a^{\log(y)/\log(a)}=(e^{\log(a)})^{\log(y)/\log(a)}=e^{\log(y)}=y$.
 }
 \fromSlide{6}{
  \item $\log_{10}(y) = \text{the number $t$ such that $10^t=y$}$ \\
        \hspace{4em} $\simeq$ the number of digits in $y$ left of the decimal point.}
 \fromSlide{7}{
  \item This is mostly of historical importance.
 }
 \fromSlide{8}{
  \item $\log_{2}(y) = \text{the number $t$ such that $2^t=y$}$ \\
        \hspace{4em} $\simeq$ the number of bits in $y$.}
 \fromSlide{9}{
  \item This is of some use in computer science and information
   theory. 
 }
 \fromSlide{10}{
  \item $\log_e(y)=\text{(the number $t$ such that $e^t=y$)}=\log(y)$.
 }
\end{itemize}
\end{slide}
}


\overlays{11}{%
\begin{slide}{Hyperbolic functions}
\begin{itemize}
 \fromSlide{2}{
  \item The hyperbolic functions are defined as follows:
   {\tiny\[\begin{array}{rlcrlcrl}
    \fromSlide{3}{\PURPLE{\sinh(x)}} &
     \fromSlide{3}{=\frac{e^x - e^{-x}}{2}} &\quad&
    \fromSlide{5}{\PURPLE{\tanh(x)}} &
     \fromSlide{5}{=\frac{\sinh(x)}{\cosh(x)}} &\quad&
    \fromSlide{8}{\PURPLE{\csch(x)}} &
     \fromSlide{8}{=\frac{1}{\sinh(x)}} \\
    \fromSlide{4}{\PURPLE{\cosh(x)}} &
     \fromSlide{4}{=\frac{e^x + e^{-x}}{2}} &&
    \fromSlide{6}{\PURPLE{\coth(x)}} &
     \fromSlide{6}{=\frac{\cosh(x)}{\sinh(x)}} &&
    \fromSlide{7}{\PURPLE{\sech(x)}} &
     \fromSlide{7}{=\frac{1}{\cosh(x)}}
   \end{array} \]}
 }
 \fromSlide{9}{
  \item Properties are easily deduced from those of $\exp$.
 }
 \fromSlide{10}{
  \item These are related to trig functions using complex numbers, eg
   $\sin(x)=\sinh(ix)/i$.
 }
 \fromSlide{11}{
  \item \ghost \vspace{-5ex}
   \psset{xunit=1cm,yunit=.4cm}
   \begin{center}\begin{pspicture}[1](-2.3,-4.3)(2.5,4.5)
    \psaxes[labels=none,linewidth=0.1pt,ticksize=1pt]{->}(0,0)(-2.2,-4.2)(2.4,4.4)
    \psline[linewidth=.1pt,linestyle=dotted](-2.3,+1)(2.3,+1)
    \psline[linewidth=.1pt,linestyle=dotted](-2.3,-1)(2.3,-1)
    \psplot[linewidth=0.1pt,linecolor=red]{-2}{2.1}{%
     2.72 x exp 2.72 x neg exp sub 2 div} % sinh
    \psplot[linewidth=0.1pt,linecolor=OliveGreen]{-2}{2.1}{%
     2.72 x exp 2.72 x neg exp add 2 div} % cosh
    \psplot[linewidth=0.1pt,linecolor=blue]{-2}{2.1}{%
     2.72 x exp 2.72 x neg exp sub
     2.72 x exp 2.72 x neg exp add div} % tanh
    \rput( 1.9,1.77){\RED{$\scriptstyle\sinh(x)$}}
    \rput(-1.0,2.36){\OLIVEGREEN{$\scriptstyle\cosh(x)$}}
    \rput( 1.5,0.54){\BLUE{$\scriptstyle\tanh(x)$}}
   \end{pspicture}\end{center}
 }
\end{itemize}
\end{slide}
}

\overlays{13}{%
\begin{slide}{Hyperbolic identities}
\begin{itemize}
 \fromSlide{2}{
  \item \ghost \vspace{-5ex}
   {\tiny \begin{align*}
    \fromSlide{2}{\cosh(x)^2 - \sinh(x)^2} &
     \fromSlide{2}{= 1 \hphantom{mmmmmmmmmmmmmmmmmmmmmmmmm}}\\
    \fromSlide{3}{\sech(x)^2 + \tanh(x)^2} &
     \fromSlide{3}{= 1 }\\
    \fromSlide{4}{\sinh(x+y)} &
     \fromSlide{4}{= \sinh(x)\cosh(y) + \cosh(x)\sinh(y) }\\
    \fromSlide{5}{\cosh(x+y)} &
     \fromSlide{5}{= \cosh(x)\cosh(y) + \sinh(x)\sinh(y) }\\
    \fromSlide{6}{\sinh(2x) } &
     \fromSlide{6}{= 2\sinh(x)\cosh(x) }\\
    \fromSlide{7}{\cosh(2x) } &
     \fromSlide{7}{= 2\cosh(x)^2 - 1. }
   \end{align*}} \vspace{-4ex}}
 \fromSlide{8}{
  \item To check these, put $u=e^x$, so $\sinh(x)=\frac{u-u^{-1}}{2}$ and
   $\cosh(x)=\frac{u+u^{-1}}{2}$. 
 }
 \fromSlide{9}{
  \item \ghost \vspace{-5ex}
   {\tiny \begin{align*}
    \fromSlide{9}{ \cosh(x)^2 - \sinh(x)^2} &
     \fromSlide{10}{= \frac{(u+u^{-1})^2}{4} - \frac{(u-u^{-1})^2}{4}
                    \hphantom{mmmmmmm}} \\&
     \fromSlide{11}{= \frac{(\RED{u^2}+2+\OLIVEGREEN{u^{-2}}) -
           (\RED{u^2}-2+\OLIVEGREEN{u^{-2}})}{4}} \\ &
     \fromSlide{12}{= (2-(-2))/4 = 1.}
   \end{align*}}}
 \fromSlide{13}{
  \item The other identities can be checked in the same way.
 }
\end{itemize}
\end{slide}
}

\overlays{9}{%
\begin{slide}{Inverse hyperbolic functions}
\begin{itemize}
 \fromSlide{2}{
  \item \parbox[t]{8cm}{
   The graph of $y=\sinh(x)$ crosses each horizontal line precisely
   once, which means that there is an inverse function 
   $x=\sinh^{-1}(y)=\arcsinh(y)$, defined for all $y\in\R$.
   } \hfill
   \psset{xunit=.6cm,yunit=.25cm}
   \begin{pspicture}[.9](-2.3,-4.3)(2.5,4)
    \psaxes[labels=none,linewidth=0.1pt,ticksize=0pt]{->}(0,0)(-2.2,-4.2)(2.4,4.4)
    \multips(-2,-4)(0,1){9}{%
     \psline[linewidth=.1pt,linestyle=dotted](0,0)(4,0)}
    \psplot[linewidth=0.1pt,linecolor=red]{-2}{2.1}{%
     2.72 x exp 2.72 x neg exp sub 2 div} % sinh
   \end{pspicture}}
 \fromSlide{3}{
  \item This can be written in terms of $\log$: \qquad
   $\arcsinh(y) = \log(y + \sqrt{1+y^2}).$
 }
 \fromSlide{4}{
  \item {}\ORANGE{\bf Check:} Suppose $y=\sinh(x)$; we must show that
   $x=\log(y+\sqrt{1+y^2})$.  
   \begin{itemize}
    \fromSlide{5}{
     \item We have $1+y^2=1+\sinh(x)^2=\cosh(x)^2$ (and $\cosh(y)>0$),
      so $\sqrt{1+y^2}=\cosh(x)$.}
    \fromSlide{6}{
     \item Thus $y + \sqrt{1+y^2} = \sinh(x) + \cosh(x) =
      \frac{e^x-e^{-x}}{2} + \frac{e^x+e^{-x}}{2} = e^x$}
    \fromSlide{7}{
     \item so $\log(y+\sqrt{1+y^2})=\log(e^x)=x$ as required.}
   \end{itemize}
 }
 \fromSlide{8}{
  \item Similarly, $\arccosh(y)=\log(y+\sqrt{y^2-1})$, defined for
   $y\geq 1$
 }
 \fromSlide{9}{
  \item and $\arctanh(y)=\frac{1}{2}\log\left(\frac{1+y}{1-y}\right)$,
   defined when $-1<y<1$.
 }
\end{itemize}
\end{slide}
}

\begin{slide}{Graphs}
\vspace{2ex}
\[ \begin{array}{ccc}
   \psset{xunit=.6cm,yunit=.3cm}
   \begin{pspicture}[.9](-2.3,-4.3)(2.5,4)
    \psaxes[labels=none,linewidth=0.1pt,ticksize=1pt]{->}(0,0)(-2.2,-4.2)(2.4,4.4)
    \psplot[linewidth=0.1pt,linecolor=red]{-2}{2.1}{%
     2.72 x exp 2.72 x neg exp sub 2 div} % sinh
   \end{pspicture} &
   \psset{xunit=.6cm,yunit=.3cm}
   \begin{pspicture}[.9](-2.3,-4.3)(2.5,4)
    \psaxes[labels=none,linewidth=0.1pt,ticksize=1pt]{->}(0,0)(0,0)(2.4,4.4)
    \psline[linewidth=0.1pt,linestyle=dotted](-2.2,0)(0,0)
    \psline[linewidth=0.1pt,linestyle=dotted](0,-4.2)(0,0)
    \psplot[linewidth=0.1pt,linestyle=dotted]{-2.1}{0}{%
     2.72 x exp 2.72 x neg exp add 2 div} % cosh
    \psplot[linewidth=0.1pt,linecolor=red]{0}{2.1}{%
     2.72 x exp 2.72 x neg exp add 2 div} % cosh
   \end{pspicture} &
   \psset{xunit=.6cm,yunit=.3cm}
   \begin{pspicture}[.9](-2.3,-4.3)(2.5,4)
    \psaxes[labels=none,linewidth=0.1pt,ticksize=1pt]{->}(0,0)(-2.2,-1)(2.4,1)
    \psline[linewidth=0.1pt,linestyle=dotted](0,-4.2)(0,4.2)
    \psplot[linewidth=0.1pt,linecolor=red]{-2}{2.1}{%
     2.72 x exp 2.72 x neg exp sub
     2.72 x exp 2.72 x neg exp add div} % tanh
   \end{pspicture} \\
   \sinh(x) & \cosh(x) & \tanh(x) \\
   && \\
   \psset{xunit=.3cm,yunit=.6cm}
   \begin{pspicture}[.9](-4.3,-2.3)(4,2.5)
    \psaxes[labels=none,linewidth=0.1pt,ticksize=1pt]{->}(0,0)(-4.2,-2.2)(4.4,2.4)
    \parametricplot[linewidth=0.1pt,linecolor=red]{-2}{2.1}{%
     2.72 t exp 2.72 t neg exp sub 2 div t} % arcsinh
   \end{pspicture} &
   \psset{xunit=.3cm,yunit=.6cm}
   \begin{pspicture}[.9](-4.3,-2.3)(4,2.5)
    \psaxes[labels=none,linewidth=0.1pt,ticksize=1pt]{->}(0,0)(0,0)(4.4,2.4)
    \psline[linewidth=0.1pt,linestyle=dotted](0,-2.2)(0,0)
    \psline[linewidth=0.1pt,linestyle=dotted](-4.2,0)(0,0)
    \parametricplot[linewidth=0.1pt,linecolor=red]{0}{2.1}{%
     2.72 t exp 2.72 t neg exp add 2 div t} % arccosh
   \end{pspicture} &
   \psset{xunit=.3cm,yunit=.6cm}
   \begin{pspicture}[.9](-4.3,-2.3)(4,2.5)
    \psaxes[labels=none,linewidth=0.1pt,ticksize=1pt]{->}(0,0)(-1,-2.2)(1,2.4)
    \psline[linewidth=0.1pt,linestyle=dotted](-4.2,0)(4.2,0)
    \parametricplot[linewidth=0.1pt,linecolor=red]{-2}{2.1}{%
     2.72 t exp 2.72 t neg exp sub
     2.72 t exp 2.72 t neg exp add div t} % tanh
   \end{pspicture} \\
   \arcsinh(x) & \arccosh(x) & \arctanh(x) \\
\end{array} \]
\end{slide}


\overlays{8}{%
\begin{slide}{Trigonometric functions}
\begin{itemize}
 \fromSlide{2}{
  \item Let $P$ be one unit away from the origin, at an angle of
   $\tht$ measured anticlockwise from the point $A=(1,0)$. 
   \begin{center}
    \psset{unit=2.5cm}
     \begin{pspicture}[1](-1.2,-0.3)(1.2,1.2)
      \psset{linewidth=0.1pt}
      \psaxes[labels=none,ticksize=1pt]{->}(0,0)(-1.2,0)(1.2,1.2)
      \psdots(1,0)(.5,.87)
      \psarc[linecolor=red](0,0){1}{0}{180}
      \psarc[linecolor=red](0,0){.2}{0}{60}
      \psline[linecolor=blue](0,0)(.5,.87)
      \psline[linecolor=magenta](-.3,.87)(.5,.87)(.5,-.3)
      \psline[arrows=->](-.15,0)(-.15,.87)
      \psline[arrows=->](-.15,.87)(-.15,0)
      \psline[arrows=->](0,-.15)(.5,-.15)
      \psline[arrows=->](.5,-.15)(0,-.15)
      \rput[t](.25,-.2){$\scriptstyle\cos(\theta)$}
      \rput[r](-.2,.43){$\scriptstyle\sin(\theta)$}
      \rput[lb](.55,.91){$\scriptstyle P=(\cos(\theta),\sin(\theta))$}
      \rput[lt](1.05,-.05){$\scriptstyle A=(1,0)$}
      \rput(.26,.15){$\scriptstyle\theta$}
     \end{pspicture}
    \end{center}}
 \fromSlide{3}{
  \item (We measure $\tht$ in radians, so the length of the arc $AP$
   is $\tht$.)
 }
 \fromSlide{4}{
  \item The numbers \PURPLE{$\cos(\tht)$} and \PURPLE{$\sin(\tht)$}
   are \EMPH{defined} to be the $x$ and $y$ coordinates of $P$.
 }
 \fromSlide{5}{
  \item We also put 
   {\vspace{-3ex} \tiny\[\begin{array}{rlcrl}
    \fromSlide{5}{\PURPLE{\tan(x)}} &
     \fromSlide{5}{=\frac{\sin(x)}{\cos(x)}} &\quad&
    \fromSlide{8}{\PURPLE{\csc(x)}} &
     \fromSlide{8}{=\frac{1}{\sin(x)}} \\
    \fromSlide{6}{\PURPLE{\cot(x)}} &
     \fromSlide{6}{=\frac{\cos(x)}{\sin(x)}} &&
    \fromSlide{7}{\PURPLE{\sec(x)}} &
     \fromSlide{7}{=\frac{1}{\cos(x)}}
   \end{array} \]}
 }
\end{itemize}
\end{slide}
}

\begin{slide}{Graphs}
\begin{center}
 \psset{xunit=1.2cm,yunit=0.9cm}
 \begin{pspicture}[1](-2,-1.2)(2.4,1.9)
  \psaxes[labels=none,linewidth=0.1pt,ticksize=1pt]{->}(0,0)(-2,-1.1)(2.2,1.1)
  \psplot[linewidth=0.1pt,linecolor=red]{-2}{2.1}{x 360 mul sin}
  \rput[B]( -2,-.3){$\scriptstyle -4\pi$}
  \rput[B]( -1,-.3){$\scriptstyle -2\pi$} 
  \rput[B](0.5,-.3){$\scriptstyle \pi$}
  \rput[B](  1,-.3){$\scriptstyle 2\pi$}
  \rput[B](1.5,-.3){$\scriptstyle 3\pi$}
  \rput[B](  2,-.3){$\scriptstyle 4\pi$}
  \rput[b](1.5,1.1){$\scriptstyle \sin(\theta)$}
 \end{pspicture} 
\hfill
 \begin{pspicture}[1](-2,-1.2)(2.4,1.9)
  \psaxes[labels=none,linewidth=0.1pt,ticksize=1pt]{->}(0,0)(-2,-1.1)(2.2,1.1)
  \psplot[linewidth=0.1pt,linecolor=red]{-2}{2.1}{x 360 mul cos}
  \rput[B]( -2,-.3){$\scriptstyle -4\pi$}
  \rput[B]( -1,-.3){$\scriptstyle -2\pi$} 
  \rput[B](0.5,-.3){$\scriptstyle \pi$}
  \rput[B](  1,-.3){$\scriptstyle 2\pi$}
  \rput[B](1.5,-.3){$\scriptstyle 3\pi$}
  \rput[B](  2,-.3){$\scriptstyle 4\pi$}
  \rput[b](1.5,1.1){$\scriptstyle \cos(\theta)$}
 \end{pspicture} 
\\[4ex]
 \psset{xunit=1.2cm,yunit=0.25cm}
 \begin{pspicture}(-2.2,-4.2)(2.4,4.5)
  \psset{linewidth=0.1pt}
  \psaxes[labels=none,ticksize=1pt]{->}(0,0)(-2,-4.1)(2.2,4.3)
  \psset{linecolor=red}
  \psplot{-2.00}{-1.58}{%
   x 180 mul sin x 180 mul cos div}
  \psplot{-1.42}{-0.58}{%
   x 180 mul sin x 180 mul cos div}
  \psplot{-0.42}{ 0.42}{%
   x 180 mul sin x 180 mul cos div}
  \psplot{ 0.58}{ 1.42}{%
   x 180 mul sin x 180 mul cos div}
  \psplot{ 1.58}{ 2.00}{%
   x 180 mul sin x 180 mul cos div}
  \psset{linecolor=OliveGreen,linestyle=dotted}
  \psline(-1.5,-4)(-1.5,4)
  \psline(-0.5,-4)(-0.5,4)
  \psline( 0.5,-4)( 0.5,4)
  \psline( 1.5,-4)( 1.5,4)
  \rput[B]( -2,-1){$\scriptstyle -2\pi$}
  \rput[B]( -1,-1){$\scriptstyle -\pi$} 
  \rput[B](0.5,-1){$\scriptstyle \frac{\pi}{2}$}
  \rput[B](  1,-1){$\scriptstyle \pi$}
  \rput[B](1.5,-1){$\scriptstyle \frac{3\pi}{2}$}
  \rput[B](  2,-1){$\scriptstyle 2\pi$}
  \rput[b](1.5,4.5){$\scriptstyle \tan(\theta)$}
 \end{pspicture}
\hfill
 \begin{pspicture}(-2.2,-4.2)(2.4,4.5)
  \psset{linewidth=0.1pt}
  \psaxes[labels=none,ticksize=1pt]{->}(0,0)(-2,-4.1)(2.2,4.3)
  \psset{linecolor=red}
  \psplot{-1.92}{-1.08}{%
   x 180 mul cos x 180 mul sin div}
  \psplot{-0.92}{-0.08}{%
   x 180 mul cos x 180 mul sin div}
  \psplot{ 0.08}{ 0.92}{%
   x 180 mul cos x 180 mul sin div}
  \psplot{ 1.08}{ 1.92}{%
   x 180 mul cos x 180 mul sin div}
  \psset{linecolor=OliveGreen,linestyle=dotted}
  \psline(-2.0,-4)(-2.0,4)
  \psline(-1.0,-4)(-1.0,4)
  \psline( 0.0,-4)( 0.0,4)
  \psline( 1.0,-4)( 1.0,4)
  \psline( 2.0,-4)( 2.0,4)
  \rput[B]( -2,-1){$\scriptstyle -2\pi$}
  \rput[B]( -1,-1){$\scriptstyle -\pi$} 
  \rput[B](0.5,-1){$\scriptstyle \frac{\pi}{2}$}
  \rput[B](  1,-1){$\scriptstyle \pi$}
  \rput[B](1.5,-1){$\scriptstyle \frac{3\pi}{2}$}
  \rput[B](  2,-1){$\scriptstyle 2\pi$}
  \rput[b](1.5,4.5){$\scriptstyle \cot(\theta)$}
 \end{pspicture}
\end{center}

\begin{center}
\psframebox[framearc=.3,linecolor=magenta]{\tiny$\begin{array}{rllrl}
 \sin(\pi/2+x) &= \cos(x)  &\hspace{4em}& \cos(\pi/2+x) &= -\sin(x) \\
 \sin(\pi+x)   &= -\sin(x) &            & \cos(\pi+x)   &= -\cos(x) \\
 \sin(2\pi+x)  &= \sin(x)  &            & \cos(2\pi+x)  &= \cos(x) \\
 \sin(-x)      &= -\sin(x) &            & \cos(-x)      &= \cos(x).
\end{array}$}
\end{center}
\end{slide}


\overlays{16}{%
\begin{slide}{De Moivre's theorem}
\begin{itemize}
 \fromSlide{2}{
  \item {\hfill \psframebox[framearc=.3,linecolor=magenta]{
   $e^{i\tht} = \exp(i\tht) = \cos(\tht) + \sin(\tht)i$} \hfill}}
 \fromSlide{3}{
  \item \ghost\vspace{-4ex}{\tiny \begin{align*}
   \fromSlide{3}{e^{i\tht}} &
    \fromSlide{3}{= \cos(\tht) + \sin(\tht)i \hphantom{mmmmmmmmmmm}} \\
   \fromSlide{3}{e^{-i\tht} = \exp(-i\tht)} &
    \fromSlide{3}{= \cos(\tht) - \sin(\tht)i} \\
   \fromSlide{4}{\sin(\tht)} &
    \onlySlide*{4}{=\frac{e^{i\tht}-e^{-i\tht}}{2i}
                   \hphantom{=\sinh(i\tht)/i}} 
    \fromSlide*{5}{=\frac{e^{i\tht}-e^{-i\tht}}{2i}=\sinh(i\tht)/i} \\ 
   \fromSlide{6}{\cos(\tht)} &
    \onlySlide*{6}{=\frac{e^{i\tht}+e^{-i\tht}}{2}
                   \hphantom{=\cosh(\tht)}} 
    \fromSlide*{7}{=\frac{e^{i\tht}+e^{-i\tht}}{2}=\cosh(i\tht)} \\ 
   \fromSlide{8}{\tan(\tht)} &
    \fromSlide{8}{=\frac{\sin(\tht)}{\cos(\tht)}}
    \fromSlide{9}{=\frac{\sinh(i\tht)/i}{\cosh(i\tht)}}
    \fromSlide{10}{=\tanh(i\tht)/i.}
  \end{align*}
 }}
 \fromSlide{11}{
  \item \ghost \vspace{-5ex}
   {\hfill \psframebox[framearc=.3,linecolor=magenta]{
   $ \begin{array}{rl}
    \fromSlide{11}{\cos(a)^2 + \sin(a)^2} &
     \fromSlide{11}{= 1 \hphantom{mmmmmmmmmmm}}\\
    \fromSlide{12}{\sec(a)^2} &
     \fromSlide{12}{= 1  + \tan(a)^2 }\\
    \fromSlide{13}{\sin(a+b)} &
     \fromSlide{13}{= \sin(a)\cos(b) + \cos(a)\sin(b) }\\
    \fromSlide{14}{\cos(a+b)} &
     \fromSlide{14}{= \cos(a)\cos(b) - \sin(a)\sin(b) }\\
    \fromSlide{15}{\sin(2a) } &
     \fromSlide{15}{= 2\sin(a)\cos(a) }\\
    \fromSlide{16}{\cos(2a) } &
     \fromSlide{16}{= 2\cos(a)^2 - 1 = 1-2\sin(a)^2. }
   \end{array} $
  }
 \hfill}}
\end{itemize}
\end{slide}
}

\overlays{12}{%
\begin{slide}{The addition formula}
\begin{itemize}
 \fromSlide{2}{
  \item $\sin(a+b)=\sin(a)\cos(b)+\cos(a)\sin(b)$
 }
 \fromSlide{2}{
  \item 
    \psset{unit=5cm}
      \def\angleA{40}
      \def\angleB{30}
      \def\angleAB{70}
      \def\cosA{0.766}
      \def\sinA{0.643}
      \def\cosB{0.866}
      \def\sinB{0.500}
      \def\cosAB{0.342}
      \def\sinAB{0.940}
      \def\cosAcosB{0.663}
      \def\cosAsinB{0.383}
      \def\sinAcosB{0.557}
      \def\sinAsinB{0.321}
     \onlySlide*{3}{
      \begin{pspicture}[1](-0.2,-0.3)(1.4,1.0)
       \psset{linewidth=0.1pt}
       \psaxes[labels=none,ticksize=1pt]{->}(0,0)(-0.1,0)(1.3,1.0)
       \psarc[linecolor=red,linestyle=dotted](0,0){1}{0}{90}
       \psarc[linecolor=red](0,0){.2}{0}{\angleB}
       \rput(\cosB,0){\psline[linecolor=red](0,.1)(-.1,.1)(-.1,0)}
       \psline[linecolor=blue](0,0)(\cosB,\sinB)(\cosB,0)(0,0)
       \rput{15}(0,0){\rput{*0}(.25,0){$\scriptstyle b$}}
       \rput[t](.45,-.05){$\scriptstyle\cos(b)$}
       \rput[r](.83,0.25){$\scriptstyle\sin(b)$}
      \end{pspicture}}%
     \onlySlide*{4}{%
      \begin{pspicture}[1](-0.2,-0.3)(1.4,1.0)
       \psset{linewidth=0.1pt}
       \psaxes[labels=none,ticksize=1pt]{->}(0,0)(-0.1,0)(1.3,1.0)
       \psarc[linecolor=red,linestyle=dotted](0,0){1}{0}{90}
       \rput{\angleA}(0,0){
        \psarc[linecolor=red](0,0){.2}{0}{\angleB}
        \rput(\cosB,0){\psline[linecolor=red](0,.1)(-.1,.1)(-.1,0)}
        \psline[linecolor=blue](0,0)(\cosB,\sinB)(\cosB,0)(0,0)
        \psline[linecolor=blue,linestyle=dotted](\cosB,0)(1,0)
        \rput{15}(0,0){\rput{*0}(.25,0){$\scriptstyle b$}}
        \rput[t](.45,-.05){$\scriptstyle\cos(b)$}
        \rput[r](.83,0.25){$\scriptstyle\sin(b)$}
       }
       \psarc[linecolor=red](0,0){.2}{0}{\angleA}
      \end{pspicture}}%
     \onlySlide*{5}{%
      \begin{pspicture}[1](-0.2,-0.3)(1.4,1.0)
       \psset{linewidth=0.1pt}
       \psaxes[labels=none,ticksize=1pt]{->}(0,0)(-0.1,0)(1.3,1.0)
       \psarc[linecolor=red,linestyle=dotted](0,0){1}{0}{90}
       \rput{\angleA}(0,0){
        \psarc[linecolor=red](0,0){.2}{0}{\angleB}
        \rput(\cosB,0){\psline[linecolor=red](0,.1)(-.1,.1)(-.1,0)}
        \psline[linecolor=blue,linestyle=dotted](0,0)(\cosB,\sinB)(\cosB,0)(1,0)
        \rput{15}(0,0){\rput{*0}(.25,0){$\scriptstyle b$}}
        \rput[t](.45,-.05){$\scriptstyle\cos(b)$}
        \rput[r](.83,0.25){$\scriptstyle\sin(b)$}
       }
       \psarc[linecolor=red](0,0){.2}{0}{\angleA}
       \rput{20}(0,0){\rput{*0}(.25,0){$\scriptstyle a$}}
       \psline[linecolor=blue]%
        (0,0)(\cosAcosB,\sinAcosB)(\cosAcosB,0)(0,0)
       \rput(\cosAcosB,0){\psline[linecolor=red](0,.1)(-.1,.1)(-.1,0)}
       \rput[l](0.68,0.28){$\scriptstyle \sin(a)\cos(b)$}
      \end{pspicture}}%
     \onlySlide*{6}{%
      \begin{pspicture}[1](-0.2,-0.3)(1.4,1.0)
       \psset{linewidth=0.1pt}
       \psaxes[labels=none,ticksize=1pt]{->}(0,0)(-0.1,0)(1.3,1.0)
       \psarc[linecolor=red,linestyle=dotted](0,0){1}{0}{90}
       \rput{\angleA}(0,0){
        \psarc[linecolor=red](0,0){.2}{0}{\angleB}
        \rput(\cosB,0){\psline[linecolor=red](0,.1)(-.1,.1)(-.1,0)}
        \psline[linecolor=blue,linestyle=dotted](0,0)(\cosB,\sinB)(\cosB,0)(1,0)
        \rput{15}(0,0){\rput{*0}(.25,0){$\scriptstyle b$}}
        \rput[t](.45,-.05){$\scriptstyle\cos(b)$}
        \rput[r](.83,0.25){$\scriptstyle\sin(b)$}
       }
       \psarc[linecolor=red](0,0){.2}{0}{\angleA}
       \rput{20}(0,0){\rput{*0}(.25,0){$\scriptstyle a$}}
       \psline[linecolor=blue,linestyle=dotted]%
        (0,0)(\cosAcosB,\sinAcosB)(\cosAcosB,0)(0,0)
       \rput(\cosAcosB,0){\psline[linecolor=red](0,.1)(-.1,.1)(-.1,0)}
       \psline[linecolor=black](\cosAcosB,\sinAcosB)(1.2,\sinAcosB)
       \psline[linecolor=magenta,arrows=<->](1.1,0)(1.1,\sinAcosB)
       \rput[l](1.15,0.28){$\scriptstyle \sin(a)\cos(b)$}
      \end{pspicture}}%
     \onlySlide*{7}{%
      \begin{pspicture}[1](-0.2,-0.3)(1.4,1.0)
       \psset{linewidth=0.1pt}
       \psaxes[labels=none,ticksize=1pt]{->}(0,0)(-0.1,0)(1.3,1.0)
       \psarc[linecolor=red,linestyle=dotted](0,0){1}{0}{90}
       \rput{\angleA}(0,0){
        \psarc[linecolor=red](0,0){.2}{0}{\angleB}
        \rput(\cosB,0){\psline[linecolor=red](0,.1)(-.1,.1)(-.1,0)}
        \psline[linecolor=blue,linestyle=dotted](0,0)(\cosB,\sinB)(\cosB,0)(1,0)
        \rput{15}(0,0){\rput{*0}(.25,0){$\scriptstyle b$}}
        \rput[t](.45,-.05){$\scriptstyle\cos(b)$}
        \rput[r](.83,0.25){$\scriptstyle\sin(b)$}
       }
       \psarc[linecolor=red](0,0){.2}{0}{\angleA}
       \rput{20}(0,0){\rput{*0}(.25,0){$\scriptstyle a$}}
       \psline[linecolor=blue,linestyle=dotted]%
        (0,0)(\cosAcosB,\sinAcosB)(\cosAcosB,0)(0,0)
       \rput(\cosAcosB,0){\psline[linecolor=red](0,.1)(-.1,.1)(-.1,0)}
       \psline[linecolor=blue]%
        (\cosAB,\sinAB)(\cosAcosB,\sinAcosB)(\cosAcosB,\sinAB)(\cosAB,\sinAB)
       \rput{90}(\cosAcosB,\sinAcosB){
        \psarc[linecolor=red](0,0){.1}{0}{\angleA}
        \rput{20}(0,0){\rput{*0}(.15,0){$\scriptstyle a$}}
       }
       \rput[l](0.68,0.78){$\scriptstyle \cos(a)\sin(b)$}
       \psline[linecolor=black](\cosAcosB,\sinAcosB)(1.2,\sinAcosB)
       \psline[linecolor=magenta,arrows=<->](1.1,0)(1.1,\sinAcosB)
       \rput[l](1.15,0.28){$\scriptstyle \sin(a)\cos(b)$}
      \end{pspicture}}%
     \onlySlide*{8}{%
      \begin{pspicture}[1](-0.2,-0.3)(1.4,1.0)
       \psset{linewidth=0.1pt}
       \psaxes[labels=none,ticksize=1pt]{->}(0,0)(-0.1,0)(1.3,1.0)
       \psarc[linecolor=red,linestyle=dotted](0,0){1}{0}{90}
       \rput{\angleA}(0,0){
        \psarc[linecolor=red](0,0){.2}{0}{\angleB}
        \rput(\cosB,0){\psline[linecolor=red](0,.1)(-.1,.1)(-.1,0)}
        \psline[linecolor=blue,linestyle=dotted](0,0)(\cosB,\sinB)(\cosB,0)(1,0)
        \rput{15}(0,0){\rput{*0}(.25,0){$\scriptstyle b$}}
        \rput[t](.45,-.05){$\scriptstyle\cos(b)$}
        \rput[r](.83,0.25){$\scriptstyle\sin(b)$}
       }
       \psarc[linecolor=red](0,0){.2}{0}{\angleA}
       \rput{20}(0,0){\rput{*0}(.25,0){$\scriptstyle a$}}
       \psline[linecolor=blue,linestyle=dotted]%
        (0,0)(\cosAcosB,\sinAcosB)(\cosAcosB,0)(0,0)
       \rput(\cosAcosB,0){\psline[linecolor=red](0,.1)(-.1,.1)(-.1,0)}
       \psline[linecolor=blue,linestyle=dotted]%
        (\cosAB,\sinAB)(\cosAcosB,\sinAcosB)(\cosAcosB,\sinAB)(\cosAB,\sinAB)
       \rput{90}(\cosAcosB,\sinAcosB){
        \psarc[linecolor=red](0,0){.1}{0}{\angleA}
        \rput{20}(0,0){\rput{*0}(.15,0){$\scriptstyle a$}}
       }
       \psline[linecolor=black](\cosAB,\sinAB)(1.2,\sinAB)
       \psline[linecolor=magenta,arrows=<->](1.1,\sinAcosB)(1.1,\sinAB)
       \rput[l](1.15,0.78){$\scriptstyle \cos(a)\sin(b)$}
       \psline[linecolor=black](\cosAcosB,\sinAcosB)(1.2,\sinAcosB)
       \psline[linecolor=magenta,arrows=<->](1.1,0)(1.1,\sinAcosB)
       \rput[l](1.15,0.28){$\scriptstyle \sin(a)\cos(b)$}
      \end{pspicture}}%
     \fromSlide*{9}{%
      \begin{pspicture}[1](-0.2,-0.3)(1.4,1.0)
       \psset{linewidth=0.1pt}
       \psaxes[labels=none,ticksize=1pt]{->}(0,0)(-0.1,0)(1.3,1.0)
       \psarc[linecolor=red,linestyle=dotted](0,0){1}{0}{90}
       \psarc[linecolor=red](0,0){.2}{0}{\angleAB}
       \rput{35}(0,0){\rput{*0}(.25,0){$\scriptstyle a+b$}}
       \psline[linecolor=blue]%
        (0,0)(\cosAB,\sinAB)(\cosAB,0)(0,0)
       \rput[l](0.35,0.47){$\scriptstyle\sin(a+b)$}
       \psline[linecolor=black](\cosAB,\sinAB)(1.2,\sinAB)
       \psline[linecolor=magenta,arrows=<->](1.1,\sinAcosB)(1.1,\sinAB)
       \rput[l](1.15,0.78){$\scriptstyle \cos(a)\sin(b)$}
       \psline[linecolor=black](\cosAcosB,\sinAcosB)(1.2,\sinAcosB)
       \psline[linecolor=magenta,arrows=<->](1.1,0)(1.1,\sinAcosB)
       \rput[l](1.15,0.28){$\scriptstyle \sin(a)\cos(b)$}
      \end{pspicture}}%
 \vspace{-6ex}}
 \fromSlide{10}{
  \item \ghost\vspace{-4ex}{\tiny\begin{align*}
   \sin(a)\cos(b)+\cos(a)\sin(b) &=
     \frac{e^{ia}-e^{-ia}}{2i}\frac{e^{ib}+e^{-ib}}{2} + 
     \frac{e^{ia}+e^{-ia}}{2}\frac{e^{ib}-e^{-ib}}{2i} \\
     &\fromSlide{11}{=  \frac{e^{i(a+b)}-e^{-i(a+b)}}{2i}}
       \fromSlide{12}{=\sin(a+b)}
  \end{align*}}
 }
\end{itemize}
\end{slide}
}

\overlays{11}{%
\begin{slide}{Finite Fourier series}
\begin{itemize}
 \fromSlide{2}{
  \item A \DEFN{finite Fourier series} is a sum of constant multiples
   of functions of the form $\RED{\sin(nx)}$ or $\BLUE{\cos(mx)}$ (with
   $n,m\in\Z$).  Note that the constant function $f(x)=a=a\cos(0x)$ is
   included.  
 }
 \fromSlide{3}{
  \item The phrase \DEFN{trigonometric polynomial} means the same
   thing. 
 }
 \fromSlide{4}{
  \item Many functions can be rewritten as finite Fourier series:
   {\vspace{-1ex}\tiny \begin{align*}
    \fromSlide{5}{\sin(x)^2} &
     \fromSlide{5}{={\scriptstyle \frac{1}{2}} -
                    {\scriptstyle \frac{1}{2}}\BLUE{\cos(2x)}
                    \hphantom{mmmmmmmmm}
                  } \\
    \fromSlide{6}{\sin(x)^3} &
     \fromSlide{6}{= {\scriptstyle \frac{3}{4}}\RED{\sin(x)} -
                     {\scriptstyle \frac{1}{4}}\BLUE{\sin(3x)}} \\
    \fromSlide{7}{\sin(x)\sin(2x)\sin(4x)} &
     \fromSlide{7}{= -\RED{\sin(x)}/4 + \RED{\sin(3x)}/4 +
                      \RED{\sin(5x)}/4 - \RED{\sin(7x)}/4} \\
    \fromSlide{8}{\sin(x)^4 + \cos(x)^4} &
     \fromSlide{8}{= {\scriptstyle \frac{3}{4}} +
                     {\scriptstyle \frac{1}{4}}\BLUE{\cos(4x)}} \\
    \fromSlide{9}{\sin(nx)\sin(mx)} &
     \fromSlide{9}{= {\scriptstyle \frac{1}{2}}\BLUE{\cos((n-m)x)} -
                     {\scriptstyle \frac{1}{2}}\BLUE{\cos((n+m)x)}.}
   \end{align*} \vspace{-4ex}}
 }
 \fromSlide{10}{
  \item {\bf Method:} Rewrite using
   $\cos(n\tht)=(e^{in\tht}+e^{-in\tht})/2$ and 
   $\sin(n\tht)=(e^{in\tht}+e^{-in\tht})/2i$, expand out,
   then rewrite using $e^{im\tht}=\BLUE{\cos(m\tht)}+\RED{\sin(m\tht)}i$.
 }
 \fromSlide{11}{
  \item Once a function has been rewritten in this form, it is very
   easy to differentiate it or integrate it.
 }
\end{itemize}
\end{slide}
}

\overlays{4}{%
\begin{slide}{Special values}
\begin{itemize}
 \fromSlide{2}{
  \item You should know the following values of $\sin(\tht)$ and
   $\cos(\tht)$: 
   {\vspace{-2ex}\tiny \[ \begin{array}{|r|ccc|}
    \hline
    \tht  & \sin(\tht)  & \cos(\tht) & \tan(\tht) \\
    \hline
    \pi/2 & 1           & 0           & \infty    \\
    \pi/3 & \sqrt{3}/2  & 1/2         & \sqrt{3}  \\
    \pi/4 & \sqrt{2}/2  & \sqrt{2}/2  & 1         \\
    \pi/6 & 1/2         & \sqrt{3}/2  & \sqrt{3}/3\\
    \hline
  \end{array}\] \vspace{-4ex}}
 }
 \fromSlide{3}{
  \item 
   Proved by considering these triangles:\\[2ex]
   {\tiny 
   \hspace{4em}
   \psset{unit=2cm}
   \begin{pspicture}[0.25](-.3,-.3)(1.3,0.9)
    \psset{linewidth=0.1pt}
    \psline[linecolor=blue](0,0)(1,0)(0,1)(0,0)
    \psline[linecolor=red](.1,0)(.1,.1)(0,.1)
    \psarc[linecolor=red](1,0){.2}{135}{180}
    \psarc[linecolor=red](0,1){.2}{270}{315}
    \rput[t](.5,-.1){$\scriptstyle 1$}
    \rput[r](-.1,.5){$\scriptstyle 1$}
    \rput[bl](.54,.54){$\scriptstyle\sqrt{2}$}
    \rput(.10,.75){$\scriptstyle\pi/4$}
    \rput(.70,.06){$\scriptstyle\pi/4$}
    \rput[bl](.11,.11){$\scriptstyle\pi/2$}
   \end{pspicture}
   \hfill
   \psset{unit=2.3cm}
   \begin{pspicture}[.3](-.8,-.3)(.8,0.7)
    \psset{linewidth=0.1pt}
    \psline[linecolor=blue](0,0)(0,.866)(-.5,0)(.5,0)(0,.866)
    \psline[linecolor=red](-.1,0)(-.1,.1)(.1,.1)(.1,0)
    \psarc[linecolor=red](-.5,0){.14}{0}{60}
    \psarc[linecolor=red]( .5,0){.14}{120}{180}
    \psarc[linecolor=red](0,.866){.14}{240}{300}
    \rput[br](-.4,.4){$\scriptstyle 1$}
    \rput[bl]( .4,.4){$\scriptstyle 1$}
    \rput[t](-.25,-.07){$\scriptstyle 1\over 2$}
    \rput[t]( .25,-.07){$\scriptstyle 1\over 2$}
    \rput[t](-.07,.4){$\scriptstyle \sqrt{3}\over 2$}
    \rput(-.31,.12){$\scriptstyle \pi\over 3$}
    \rput( .31,.12){$\scriptstyle \pi\over 3$}
    \rput(-.05,.65){$\scriptstyle \pi\over 6$}
    \rput( .05,.65){$\scriptstyle \pi\over 6$}
   \end{pspicture}
  \hspace{4em}}
 }
 \fromSlide{4}{
  \item You should also be able to deduce things like
   $\cos(5\pi/6)=-\sqrt{3}/2$. 
 }
\end{itemize}
\end{slide}
}

\begin{slide}{Inverse trigonometric functions}
\[
 \begin{array}{ccc}
  \psset{unit=1.2cm,linewidth=.1pt}
  \begin{pspicture}[.5](-1.3,-1.3)(1.3,1.1)
   \psaxes[labels=none,ticksize=1pt](0,0)(-1.1,-1.1)(1.1,1.1)
   \psplot[linecolor=red]{-1}{1}{x 90 mul sin}
   \rput[r](-.1,-1){$\scriptstyle -1$}
   \rput[r](-.1,+1){$\scriptstyle 1$}
   \rput[t](-1,-.1){$\scriptstyle -\frac{\pi}{2}$}
   \rput[t](+1,-.1){$\scriptstyle \frac{\pi}{2}$}
  \end{pspicture} & 
  \psset{unit=1.2cm,linewidth=.1pt}
  \begin{pspicture}[.5](-0.3,-1.3)(2.3,1.1)
   \psaxes[labels=none,ticksize=1pt](0,0)(-0.1,-1.1)(2.1,1.1)
   \psplot[linecolor=red]{0}{2}{x 90 mul cos}
   \rput[r](-.1,-1){$\scriptstyle -1$}
   \rput[r](-.1,+1){$\scriptstyle 1$}
   \rput[t](+2,-.1){$\scriptstyle \pi$}
  \end{pspicture} & 
  \psset{xunit=1.2cm,yunit=.4cm,linewidth=.1pt}
  \begin{pspicture}[.5](-1.3,-3.9)(1.3,3.3)
   \psaxes[labels=none,ticksize=1pt](0,0)(-1.1,-3.3)(1.1,3.3)
   \psplot[linecolor=red]{-.8}{.8}{x 90 mul sin x 90 mul cos div}
   \psline[linecolor=OliveGreen,linestyle=dotted](-1,-3)(-1,3)
   \psline[linecolor=OliveGreen,linestyle=dotted](+1,-3)(+1,3)
   \rput[tr](-1,-.1){$\scriptstyle -\frac{\pi}{2}$}
   \rput[tl](+1,-.1){$\scriptstyle \frac{\pi}{2}$}
  \end{pspicture} \\
   {\scriptstyle \sin\:[-\frac{\pi}{2},\frac{\pi}{2}]\xra{}[-1,1]} &
   {\scriptstyle \cos\:[0,\pi]\xra{}[-1,1]} &
   {\scriptstyle \tan\:[-\frac{\pi}{2},\frac{\pi}{2}]\xra{}\R} \\
   && \\
  \psset{unit=1.2cm,linewidth=.1pt}
  \begin{pspicture}[.5](-1.3,-1.3)(1.3,1.3)
   \psaxes[labels=none,ticksize=1pt](0,0)(-1.1,-1.1)(1.1,1.1)
   \parametricplot[linecolor=red]{-1}{1}{t 90 mul sin t}
   \rput[t](-1,-.1){$\scriptstyle -1$}
   \rput[t](+1,-.1){$\scriptstyle 1$}
   \rput[r](-.1,-1){$\scriptstyle -\frac{\pi}{2}$}
   \rput[r](-.1,+1){$\scriptstyle \frac{\pi}{2}$}
  \end{pspicture} & 
  \psset{unit=1.2cm,linewidth=.1pt}
  \begin{pspicture}[.5](-1.3,-0.3)(1.3,2.3)
   \psaxes[labels=none,ticksize=1pt](0,0)(-1.1,-0.1)(1.1,2.1)
   \parametricplot[linecolor=red]{0}{2}{t 90 mul cos t}
   \rput[t](-1,-.1){$\scriptstyle -1$}
   \rput[t](+1,-.1){$\scriptstyle 1$}
   \rput[r](-.1,+2){$\scriptstyle \pi$}
  \end{pspicture} & 
  \psset{yunit=1.2cm,xunit=.4cm,linewidth=.1pt}
  \begin{pspicture}[.5](-3.9,-1.3)(3.3,1.3)
   \psaxes[labels=none,ticksize=1pt](0,0)(-3.3,-1.1)(3.3,1.1)
   \parametricplot[linecolor=red]{-.8}{.8}%
    {t 90 mul sin t 90 mul cos div t}
   \psline[linecolor=OliveGreen,linestyle=dotted](-3,-1)(3,-1)
   \psline[linecolor=OliveGreen,linestyle=dotted](-3,+1)(3,+1)
   \rput[tr](-.1,-1){$\scriptstyle -\frac{\pi}{2}$}
   \rput[br](-.1,+1){$\scriptstyle \frac{\pi}{2}$}
  \end{pspicture} \\
   {\scriptstyle \arcsin\:[-1,1]\xra{}[-\frac{\pi}{2},\frac{\pi}{2}]} &
   {\scriptstyle \arccos\:[-1,1]\xra{}[0,\pi]} &
   {\scriptstyle \arctan\:\R\xra{}[-\frac{\pi}{2},\frac{\pi}{2}]}
\end{array} \]
\end{slide}

\end{document}
