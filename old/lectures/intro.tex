\documentclass[%
pdf,
neil,
colorBG,
slideColor,
]{prosper}
\usepackage{amsmath}
\usepackage{verbatim}
\usepackage[usenames,dvips]{color}

\newcommand{\GREENYELLOW}[1]{{\color{GreenYellow}#1}}
\newcommand{\YELLOW}[1]{{\color{Yellow}#1}}
\newcommand{\YLW}[1]{{\color{Yellow}#1}}
\newcommand{\GOLDENROD}[1]{{\color{Goldenrod}#1}}
\newcommand{\DANDELION}[1]{{\color{Dandelion}#1}}
\newcommand{\APRICOT}[1]{{\color{Apricot}#1}}
\newcommand{\PEACH}[1]{{\color{Peach}#1}}
\newcommand{\MELON}[1]{{\color{Melon}#1}}
\newcommand{\YELLOWORANGE}[1]{{\color{YellowOrange}#1}}
\newcommand{\ORANGE}[1]{{\color{Orange}#1}}
\newcommand{\BURNTORANGE}[1]{{\color{BurntOrange}#1}}
\newcommand{\BITTERSWEET}[1]{{\color{Bittersweet}#1}}
\newcommand{\REDORANGE}[1]{{\color{RedOrange}#1}}
\newcommand{\MAHOGANY}[1]{{\color{Mahogany}#1}}
\newcommand{\MAROON}[1]{{\color{Maroon}#1}}
\newcommand{\BRICKRED}[1]{{\color{BrickRed}#1}}
\newcommand{\RED}[1]{{\color{Red}#1}}
\newcommand{\ORANGERED}[1]{{\color{OrangeRed}#1}}
\newcommand{\RUBINERED}[1]{{\color{RubineRed}#1}}
\newcommand{\WILDSTRAWBERRY}[1]{{\color{WildStrawberry}#1}}
\newcommand{\SALMON}[1]{{\color{Salmon}#1}}
\newcommand{\CARNATIONPINK}[1]{{\color{CarnationPink}#1}}
\newcommand{\MAGENTA}[1]{{\color{Magenta}#1}}
\newcommand{\VIOLETRED}[1]{{\color{VioletRed}#1}}
\newcommand{\RHODAMINE}[1]{{\color{Rhodamine}#1}}
\newcommand{\MULBERRY}[1]{{\color{Mulberry}#1}}
\newcommand{\REDVIOLET}[1]{{\color{RedViolet}#1}}
\newcommand{\FUCHSIA}[1]{{\color{Fuchsia}#1}}
\newcommand{\LAVENDER}[1]{{\color{Lavender}#1}}
\newcommand{\THISTLE}[1]{{\color{Thistle}#1}}
\newcommand{\ORCHID}[1]{{\color{Orchid}#1}}
\newcommand{\DARKORCHID}[1]{{\color{DarkOrchid}#1}}
\newcommand{\PURPLE}[1]{{\color{Purple}#1}}
\newcommand{\PLUM}[1]{{\color{Plum}#1}}
\newcommand{\VIOLET}[1]{{\color{Violet}#1}}
\newcommand{\ROYALPURPLE}[1]{{\color{RoyalPurple}#1}}
\newcommand{\BLUEVIOLET}[1]{{\color{BlueViolet}#1}}
\newcommand{\PERIWINKLE}[1]{{\color{Periwinkle}#1}}
\newcommand{\CADETBLUE}[1]{{\color{CadetBlue}#1}}
\newcommand{\CORNFLOWERBLUE}[1]{{\color{CornflowerBlue}#1}}
\newcommand{\MIDNIGHTBLUE}[1]{{\color{MidnightBlue}#1}}
\newcommand{\NAVYBLUE}[1]{{\color{NavyBlue}#1}}
\newcommand{\ROYALBLUE}[1]{{\color{RoyalBlue}#1}}
\newcommand{\BLUE}[1]{{\color{Blue}#1}}
\newcommand{\CERULEAN}[1]{{\color{Cerulean}#1}}
\newcommand{\CYAN}[1]{{\color{Cyan}#1}}
\newcommand{\PROCESSBLUE}[1]{{\color{ProcessBlue}#1}}
\newcommand{\SKYBLUE}[1]{{\color{SkyBlue}#1}}
\newcommand{\TURQUOISE}[1]{{\color{Turquoise}#1}}
\newcommand{\TEALBLUE}[1]{{\color{TealBlue}#1}}
\newcommand{\AQUAMARINE}[1]{{\color{Aquamarine}#1}}
\newcommand{\BLUEGREEN}[1]{{\color{BlueGreen}#1}}
\newcommand{\EMERALD}[1]{{\color{Emerald}#1}}
\newcommand{\JUNGLEGREEN}[1]{{\color{JungleGreen}#1}}
\newcommand{\SEAGREEN}[1]{{\color{SeaGreen}#1}}
\newcommand{\GREEN}[1]{{\color{Green}#1}}
\newcommand{\FORESTGREEN}[1]{{\color{ForestGreen}#1}}
\newcommand{\PINEGREEN}[1]{{\color{PineGreen}#1}}
\newcommand{\LIMEGREEN}[1]{{\color{LimeGreen}#1}}
\newcommand{\YELLOWGREEN}[1]{{\color{YellowGreen}#1}}
\newcommand{\SPRINGGREEN}[1]{{\color{SpringGreen}#1}}
\newcommand{\OLIVEGREEN}[1]{{\color{OliveGreen}#1}}
\newcommand{\OLG}[1]{{\color{OliveGreen}#1}}
\newcommand{\RAWSIENNA}[1]{{\color{RawSienna}#1}}
\newcommand{\SEPIA}[1]{{\color{Sepia}#1}}
\newcommand{\BROWN}[1]{{\color{Brown}#1}}
\newcommand{\TAN}[1]{{\color{Tan}#1}}
\newcommand{\GRAY}[1]{{\color{Gray}#1}}
\newcommand{\WHITE}[1]{{\color{White}#1}}
\newcommand{\BLACK}[1]{{\color{Black}#1}}

\newcmykcolor{GreenYellow}{0.15 0 0.69 0}
\newcmykcolor{Yellow}{0 0 1 0}
\newcmykcolor{Goldenrod}{0 0.10 0.84 0}
\newcmykcolor{Dandelion}{0 0.29 0.84 0}
\newcmykcolor{Apricot}{0 0.32 0.52 0}
\newcmykcolor{Peach}{0 0.50 0.70 0}
\newcmykcolor{Melon}{0 0.46 0.50 0}
\newcmykcolor{YellowOrange}{0 0.42 1 0}
\newcmykcolor{Orange}{0 0.61 0.87 0}
\newcmykcolor{BurntOrange}{0 0.51 1 0}
\newcmykcolor{Bittersweet}{0 0.75 1 0.24}
\newcmykcolor{RedOrange}{0 0.77 0.87 0}
\newcmykcolor{Mahogany}{0 0.85 0.87 0.35}
\newcmykcolor{Maroon}{0 0.87 0.68 0.32}
\newcmykcolor{BrickRed}{0 0.89 0.94 0.28}
\newcmykcolor{Red}{0 1 1 0}
\newcmykcolor{OrangeRed}{0 1 0.50 0}
\newcmykcolor{RubineRed}{0 1 0.13 0}
\newcmykcolor{WildStrawberry}{0 0.96 0.39 0}
\newcmykcolor{Salmon}{0 0.53 0.38 0}
\newcmykcolor{CarnationPink}{0 0.63 0 0}
\newcmykcolor{Magenta}{0 1 0 0}
\newcmykcolor{VioletRed}{0 0.81 0 0}
\newcmykcolor{Rhodamine}{0 0.82 0 0}
\newcmykcolor{Mulberry}{0.34 0.90 0 0.02}
\newcmykcolor{RedViolet}{0.07 0.90 0 0.34}
\newcmykcolor{Fuchsia}{0.47 0.91 0 0.08}
\newcmykcolor{Lavender}{0 0.48 0 0}
\newcmykcolor{Thistle}{0.12 0.59 0 0}
\newcmykcolor{Orchid}{0.32 0.64 0 0}
\newcmykcolor{DarkOrchid}{0.40 0.80 0.20 0}
\newcmykcolor{Purple}{0.45 0.86 0 0}
\newcmykcolor{Plum}{0.50 1 0 0}
\newcmykcolor{Violet}{0.79 0.88 0 0}
\newcmykcolor{RoyalPurple}{0.75 0.90 0 0}
\newcmykcolor{BlueViolet}{0.86 0.91 0 0.04}
\newcmykcolor{Periwinkle}{0.57 0.55 0 0}
\newcmykcolor{CadetBlue}{0.62 0.57 0.23 0}
\newcmykcolor{CornflowerBlue}{0.65 0.13 0 0}
\newcmykcolor{MidnightBlue}{0.98 0.13 0 0.43}
\newcmykcolor{NavyBlue}{0.94 0.54 0 0}
\newcmykcolor{RoyalBlue}{1 0.50 0 0}
\newcmykcolor{Blue}{1 1 0 0}
\newcmykcolor{Cerulean}{0.94 0.11 0 0}
\newcmykcolor{Cyan}{1 0 0 0}
\newcmykcolor{ProcessBlue}{0.96 0 0 0}
\newcmykcolor{SkyBlue}{0.62 0 0.12 0}
\newcmykcolor{Turquoise}{0.85 0 0.20 0}
\newcmykcolor{TealBlue}{0.86 0 0.34 0.02}
\newcmykcolor{Aquamarine}{0.82 0 0.30 0}
\newcmykcolor{BlueGreen}{0.85 0 0.33 0}
\newcmykcolor{Emerald}{1 0 0.50 0}
\newcmykcolor{JungleGreen}{0.99 0 0.52 0}
\newcmykcolor{SeaGreen}{0.69 0 0.50 0}
\newcmykcolor{Green}{1 0 1 0}
\newcmykcolor{ForestGreen}{0.91 0 0.88 0.12}
\newcmykcolor{PineGreen}{0.92 0 0.59 0.25}
\newcmykcolor{LimeGreen}{0.50 0 1 0}
\newcmykcolor{YellowGreen}{0.44 0 0.74 0}
\newcmykcolor{SpringGreen}{0.26 0 0.76 0}
\newcmykcolor{OliveGreen}{0.64 0 0.95 0.40}
\newcmykcolor{RawSienna}{0 0.72 1 0.45}
\newcmykcolor{Sepia}{0 0.83 1 0.70}
\newcmykcolor{Brown}{0 0.81 1 0.60}
\newcmykcolor{Tan}{0.14 0.42 0.56 0}
\newcmykcolor{Gray}{0 0 0 0.50}
\newcmykcolor{Black}{0 0 0 1}
\newcmykcolor{White}{0 0 0 0}


\newcommand{\bbm}       {\left[\begin{matrix}}
\newcommand{\ebm}       {\end{matrix}\right]}
\newcommand{\bsm}       {\left[\begin{smallmatrix}}
\newcommand{\esm}       {\end{smallmatrix}\right]}
\newcommand{\bpm}       {\begin{pmatrix}}
\newcommand{\epm}       {\end{pmatrix}}
\newcommand{\bcf}[2]{\left(\begin{array}{c}{#1}\\{#2}\end{array}\right)}


\newcommand{\csch}     {\operatorname{csch}}
\newcommand{\sech}     {\operatorname{sech}}
\newcommand{\arcsinh}  {\operatorname{arcsinh}}
\newcommand{\arccosh}  {\operatorname{arccosh}}
\newcommand{\arctanh}  {\operatorname{arctanh}}

\newcommand{\range}     {\operatorname{range}}
\newcommand{\trans}     {\operatorname{trans}}
\newcommand{\trc}       {\operatorname{trace}}
\newcommand{\adj}       {\operatorname{adj}}

\newcommand{\tint}{\textstyle\int}
\newcommand{\tm}{\times}
\newcommand{\sse}{\subseteq}
\newcommand{\st}{\;|\;}
\newcommand{\sm}{\setminus}
\newcommand{\iffa}      {\Leftrightarrow}
\newcommand{\xra}{\xrightarrow}

\renewcommand{\:}{\colon}

\newcommand{\N}         {{\mathbb{N}}}
\newcommand{\Z}         {{\mathbb{Z}}}
\newcommand{\Q}         {{\mathbb{Q}}}
\renewcommand{\R}       {{\mathbb{R}}}
\newcommand{\C}         {{\mathbb{C}}}

\newcommand{\al}        {\alpha}
\newcommand{\bt}        {\beta} 
\newcommand{\gm}        {\gamma}
\newcommand{\dl}        {\delta}
\newcommand{\ep}        {\epsilon}
\newcommand{\zt}        {\zeta}
\newcommand{\et}        {\eta}
\newcommand{\tht}       {\theta}
\newcommand{\io}        {\iota}
\newcommand{\kp}        {\kappa}
\newcommand{\lm}        {\lambda}
\newcommand{\ph}        {\phi}
\newcommand{\ch}        {\chi}
\newcommand{\ps}        {\psi}
\newcommand{\rh}        {\rho}
\newcommand{\sg}        {\sigma}
\newcommand{\om}        {\omega}

\newcommand{\EMPH}[1]{\emph{\RED{#1}}}
\newcommand{\DEFN}[1]{\emph{\PURPLE{#1}}}
\newcommand{\VEC}[1]    {\mathbf{#1}}

\newcommand{\ghost}{{\tiny $\color[rgb]{1,1,1}.$}}


\title{Introduction}
\author{}

\begin{document}

%\slideCaption{\color{white}}

\begin{slide}{}
 {\Huge
  \vspace{6ex}
  \begin{center}
   Pure Mathematics Core\\
   (PMA101)
  \end{center}
 }
\end{slide}

\begin{slide}{Staff}
 \begin{tabular}{ll}
  \parbox{8cm}{
   The lecturer is Professor Neil Strickland.\\
   {\tt N.P.Strickland@sheffield.ac.uk}
  } &
  \raisebox{-2cm}{\includegraphics{photos/NPS.eps}}
 \end{tabular}

 \noindent The tutors are: \\
 {\setlength{\tabcolsep}{-1mm}
 \begin{tabular}{llllll}
  \includegraphics{photos/PGD.eps} &
  \includegraphics{photos/NPD.eps} &
  \includegraphics{photos/OH.eps} &
  \includegraphics{photos/JM.eps} &
  \includegraphics{photos/NN.eps} \\
  {\parbox{2cm}{\tiny Dr Peter \\ Dixon}} &
  {\parbox{2cm}{\tiny Dr Neil \\ Dummigan}} &
  {\parbox{2cm}{\tiny Mr Owen \\ Hinchcliffe}} &
  {\parbox{2.4cm}{\tiny Dr Jayanta \\ Manoharmayum}} &
  {\parbox{2cm}{\tiny Dr Nicole \\ Nossem}} 
 \end{tabular}}
\end{slide}

\overlays{9}{%
\begin{slide}{The web page}
\begin{itemize}
 \fromSlide{2}{
  \item The course web page is
   \BLUE{\url{http://www.shef.ac.uk/~puremath/PMA101}}. 
 }
 \fromSlide{3}{
  \item Visit the web page for
   \begin{itemize}
    \fromSlide{4}{ 
     \item Administrative arrangements }
    \fromSlide{5}{ 
     \item Course material in electronic form\\
      (notes, lecture presentations, tutorial problems, solutions,
      \ldots)}
    \fromSlide{6}{
     \item Links to other mathematical resources}
    \fromSlide{7}{
     \item Information about the exam}
    \fromSlide{8}{
     \item Information about computer-aided assessment}
   \end{itemize}
 }
 \fromSlide{9}{
  \item Announcements will be sent out by email, and will also appear
   on the web page.
 }
\end{itemize}
\end{slide}
}

\overlays{10}{%
\begin{slide}{Arrangements}
\begin{itemize}
 \fromSlide{2}{
  \item Eleven lectures, one per week until 15/12, except 
   10/11 (Reading Week).}
 \fromSlide{3}{
  \item One tutorial per week; you will be assigned to one of eight
   groups. }
 \fromSlide{4}{
  \item Problem surgeries Tuesday 12:10-13:00 and Thursday
   13:10-14:00, Hicks computer room G11.  (Will move after three
   weeks.) }
 \fromSlide{5}{
  \item Four hours private study per week, working through notes and
   doing problems with AiM (computer aided assessment system).}
 \fromSlide{6}{
  \item Routine problems will be handled by AiM; help given in
   tutorials and/or the problem surgery.}
 \fromSlide{7}{
  \item Less routine problems assigned for you to prepare for
   discussion in tutorials.  You are encouraged to work on these with
   your friends.}
 \fromSlide{8}{
  \item \ghost\vspace{-3ex}
   \begin{itemize}
    \fromSlide{8}{
     \item Five of the AiM tests will together count for 10\% of
      your final mark.  }
    \fromSlide{9}{
     \item 60\% for Section~A of the final exam; similar to the AiM
      questions. }
    \fromSlide{10}{
     \item 30\% for Section~B of the final exam; similar to the more
      challenging problems assigned for tutorials.}
   \end{itemize}
 }
\end{itemize}
\end{slide}
}

\overlays{9}{%
\begin{slide}{Syllabus}
\begin{itemize}
 \fromSlide{2}{
  \item Manipulation of algebraic expressions and inequalities.}
 \fromSlide{3}{
  \item Sets of numbers, vectors and other mathematical objects;
   geometric figures as sets of points; sets of solutions to equations.}
 \fromSlide{4}{
  \item General theory of functions between sets.} 
 \fromSlide{5}{
  \item Special functions, such as $\sin$, $\cos$, $\exp$ and $\log$.}
 \fromSlide{6}{
  \item Special classes of functions, such as polynomials, rational
   functions, periodic functions, bell curves, \ldots}
 \fromSlide{7}{
  \item Differentiation: the meaning of derivatives, and general
   techniques for calculating them.  Derivatives of particular
   special functions, and of special classes of functions.}
 \fromSlide{8}{
  \item Integration: the meaning of integrals, and general
   techniques for calculating them.  Integrals of particular
   special functions, and of special classes of functions.}
 \fromSlide{9}{
  \item Vectors and matrices, emphasising the link with systems of
   linear equations.}
\end{itemize}
\end{slide}
}


\overlays{6}{%
\begin{slide}{Approach}
\begin{itemize}
 \fromSlide{2}{
  \item Many topics may be familiar from A-level; but we will take a
   higher level approach.}
 \fromSlide{3}{
  \item We will stand back from the detailed calculations, look for
   patterns and common features, and understand more deeply why things
   work the way they do.}
 \fromSlide{4}{
  \item We will provide pointers to other areas of mathematics that
   you may study in the next few years.}
 \fromSlide{5}{
  \item All examples are carefully chosen, and most have a story
   behind them.}
 \fromSlide{6}{
  \item Computer systems such as
   \href{file:///C:/Documents\%20and\%20Settings/Neil\%20Strickland/My\%20Documents/Teach/core/lectures/mapleintro.mws}{\BLUE{Maple}}
   can do algebra and calculus.  This makes life easier but provides
   new challenges.
 }
\end{itemize}
\end{slide}
}

\end{document}

%%%%%%%%%%%%%%%%%%%%%%%%%%%%%%%%%%%%%%%%%%%%%%%%%%%%%%%%%%%%%%%%%%%%%%
%%%%%%%%%%%%%%%%%%%%%%%%%%%%%%%%%%%%%%%%%%%%%%%%%%%%%%%%%%%%%%%%%%%%%%
%%%%%%%%%%%%%%%%%%%%%%%%%%%%%%%%%%%%%%%%%%%%%%%%%%%%%%%%%%%%%%%%%%%%%%
