\documentclass[%
pdf,
neil,
colorBG,
slideColor,
]{prosper}
\usepackage{amsmath}
\usepackage{pstricks,pst-node,pst-text,pst-3d,pst-plot}
\usepackage[usenames,dvips]{color}

\newcommand{\GREENYELLOW}[1]{{\color{GreenYellow}#1}}
\newcommand{\YELLOW}[1]{{\color{Yellow}#1}}
\newcommand{\YLW}[1]{{\color{Yellow}#1}}
\newcommand{\GOLDENROD}[1]{{\color{Goldenrod}#1}}
\newcommand{\DANDELION}[1]{{\color{Dandelion}#1}}
\newcommand{\APRICOT}[1]{{\color{Apricot}#1}}
\newcommand{\PEACH}[1]{{\color{Peach}#1}}
\newcommand{\MELON}[1]{{\color{Melon}#1}}
\newcommand{\YELLOWORANGE}[1]{{\color{YellowOrange}#1}}
\newcommand{\ORANGE}[1]{{\color{Orange}#1}}
\newcommand{\BURNTORANGE}[1]{{\color{BurntOrange}#1}}
\newcommand{\BITTERSWEET}[1]{{\color{Bittersweet}#1}}
\newcommand{\REDORANGE}[1]{{\color{RedOrange}#1}}
\newcommand{\MAHOGANY}[1]{{\color{Mahogany}#1}}
\newcommand{\MAROON}[1]{{\color{Maroon}#1}}
\newcommand{\BRICKRED}[1]{{\color{BrickRed}#1}}
\newcommand{\RED}[1]{{\color{Red}#1}}
\newcommand{\ORANGERED}[1]{{\color{OrangeRed}#1}}
\newcommand{\RUBINERED}[1]{{\color{RubineRed}#1}}
\newcommand{\WILDSTRAWBERRY}[1]{{\color{WildStrawberry}#1}}
\newcommand{\SALMON}[1]{{\color{Salmon}#1}}
\newcommand{\CARNATIONPINK}[1]{{\color{CarnationPink}#1}}
\newcommand{\MAGENTA}[1]{{\color{Magenta}#1}}
\newcommand{\VIOLETRED}[1]{{\color{VioletRed}#1}}
\newcommand{\RHODAMINE}[1]{{\color{Rhodamine}#1}}
\newcommand{\MULBERRY}[1]{{\color{Mulberry}#1}}
\newcommand{\REDVIOLET}[1]{{\color{RedViolet}#1}}
\newcommand{\FUCHSIA}[1]{{\color{Fuchsia}#1}}
\newcommand{\LAVENDER}[1]{{\color{Lavender}#1}}
\newcommand{\THISTLE}[1]{{\color{Thistle}#1}}
\newcommand{\ORCHID}[1]{{\color{Orchid}#1}}
\newcommand{\DARKORCHID}[1]{{\color{DarkOrchid}#1}}
\newcommand{\PURPLE}[1]{{\color{Purple}#1}}
\newcommand{\PLUM}[1]{{\color{Plum}#1}}
\newcommand{\VIOLET}[1]{{\color{Violet}#1}}
\newcommand{\ROYALPURPLE}[1]{{\color{RoyalPurple}#1}}
\newcommand{\BLUEVIOLET}[1]{{\color{BlueViolet}#1}}
\newcommand{\PERIWINKLE}[1]{{\color{Periwinkle}#1}}
\newcommand{\CADETBLUE}[1]{{\color{CadetBlue}#1}}
\newcommand{\CORNFLOWERBLUE}[1]{{\color{CornflowerBlue}#1}}
\newcommand{\MIDNIGHTBLUE}[1]{{\color{MidnightBlue}#1}}
\newcommand{\NAVYBLUE}[1]{{\color{NavyBlue}#1}}
\newcommand{\ROYALBLUE}[1]{{\color{RoyalBlue}#1}}
\newcommand{\BLUE}[1]{{\color{Blue}#1}}
\newcommand{\CERULEAN}[1]{{\color{Cerulean}#1}}
\newcommand{\CYAN}[1]{{\color{Cyan}#1}}
\newcommand{\PROCESSBLUE}[1]{{\color{ProcessBlue}#1}}
\newcommand{\SKYBLUE}[1]{{\color{SkyBlue}#1}}
\newcommand{\TURQUOISE}[1]{{\color{Turquoise}#1}}
\newcommand{\TEALBLUE}[1]{{\color{TealBlue}#1}}
\newcommand{\AQUAMARINE}[1]{{\color{Aquamarine}#1}}
\newcommand{\BLUEGREEN}[1]{{\color{BlueGreen}#1}}
\newcommand{\EMERALD}[1]{{\color{Emerald}#1}}
\newcommand{\JUNGLEGREEN}[1]{{\color{JungleGreen}#1}}
\newcommand{\SEAGREEN}[1]{{\color{SeaGreen}#1}}
\newcommand{\GREEN}[1]{{\color{Green}#1}}
\newcommand{\FORESTGREEN}[1]{{\color{ForestGreen}#1}}
\newcommand{\PINEGREEN}[1]{{\color{PineGreen}#1}}
\newcommand{\LIMEGREEN}[1]{{\color{LimeGreen}#1}}
\newcommand{\YELLOWGREEN}[1]{{\color{YellowGreen}#1}}
\newcommand{\SPRINGGREEN}[1]{{\color{SpringGreen}#1}}
\newcommand{\OLIVEGREEN}[1]{{\color{OliveGreen}#1}}
\newcommand{\OLG}[1]{{\color{OliveGreen}#1}}
\newcommand{\RAWSIENNA}[1]{{\color{RawSienna}#1}}
\newcommand{\SEPIA}[1]{{\color{Sepia}#1}}
\newcommand{\BROWN}[1]{{\color{Brown}#1}}
\newcommand{\TAN}[1]{{\color{Tan}#1}}
\newcommand{\GRAY}[1]{{\color{Gray}#1}}
\newcommand{\WHITE}[1]{{\color{White}#1}}
\newcommand{\BLACK}[1]{{\color{Black}#1}}

\newcmykcolor{GreenYellow}{0.15 0 0.69 0}
\newcmykcolor{Yellow}{0 0 1 0}
\newcmykcolor{Goldenrod}{0 0.10 0.84 0}
\newcmykcolor{Dandelion}{0 0.29 0.84 0}
\newcmykcolor{Apricot}{0 0.32 0.52 0}
\newcmykcolor{Peach}{0 0.50 0.70 0}
\newcmykcolor{Melon}{0 0.46 0.50 0}
\newcmykcolor{YellowOrange}{0 0.42 1 0}
\newcmykcolor{Orange}{0 0.61 0.87 0}
\newcmykcolor{BurntOrange}{0 0.51 1 0}
\newcmykcolor{Bittersweet}{0 0.75 1 0.24}
\newcmykcolor{RedOrange}{0 0.77 0.87 0}
\newcmykcolor{Mahogany}{0 0.85 0.87 0.35}
\newcmykcolor{Maroon}{0 0.87 0.68 0.32}
\newcmykcolor{BrickRed}{0 0.89 0.94 0.28}
\newcmykcolor{Red}{0 1 1 0}
\newcmykcolor{OrangeRed}{0 1 0.50 0}
\newcmykcolor{RubineRed}{0 1 0.13 0}
\newcmykcolor{WildStrawberry}{0 0.96 0.39 0}
\newcmykcolor{Salmon}{0 0.53 0.38 0}
\newcmykcolor{CarnationPink}{0 0.63 0 0}
\newcmykcolor{Magenta}{0 1 0 0}
\newcmykcolor{VioletRed}{0 0.81 0 0}
\newcmykcolor{Rhodamine}{0 0.82 0 0}
\newcmykcolor{Mulberry}{0.34 0.90 0 0.02}
\newcmykcolor{RedViolet}{0.07 0.90 0 0.34}
\newcmykcolor{Fuchsia}{0.47 0.91 0 0.08}
\newcmykcolor{Lavender}{0 0.48 0 0}
\newcmykcolor{Thistle}{0.12 0.59 0 0}
\newcmykcolor{Orchid}{0.32 0.64 0 0}
\newcmykcolor{DarkOrchid}{0.40 0.80 0.20 0}
\newcmykcolor{Purple}{0.45 0.86 0 0}
\newcmykcolor{Plum}{0.50 1 0 0}
\newcmykcolor{Violet}{0.79 0.88 0 0}
\newcmykcolor{RoyalPurple}{0.75 0.90 0 0}
\newcmykcolor{BlueViolet}{0.86 0.91 0 0.04}
\newcmykcolor{Periwinkle}{0.57 0.55 0 0}
\newcmykcolor{CadetBlue}{0.62 0.57 0.23 0}
\newcmykcolor{CornflowerBlue}{0.65 0.13 0 0}
\newcmykcolor{MidnightBlue}{0.98 0.13 0 0.43}
\newcmykcolor{NavyBlue}{0.94 0.54 0 0}
\newcmykcolor{RoyalBlue}{1 0.50 0 0}
\newcmykcolor{Blue}{1 1 0 0}
\newcmykcolor{Cerulean}{0.94 0.11 0 0}
\newcmykcolor{Cyan}{1 0 0 0}
\newcmykcolor{ProcessBlue}{0.96 0 0 0}
\newcmykcolor{SkyBlue}{0.62 0 0.12 0}
\newcmykcolor{Turquoise}{0.85 0 0.20 0}
\newcmykcolor{TealBlue}{0.86 0 0.34 0.02}
\newcmykcolor{Aquamarine}{0.82 0 0.30 0}
\newcmykcolor{BlueGreen}{0.85 0 0.33 0}
\newcmykcolor{Emerald}{1 0 0.50 0}
\newcmykcolor{JungleGreen}{0.99 0 0.52 0}
\newcmykcolor{SeaGreen}{0.69 0 0.50 0}
\newcmykcolor{Green}{1 0 1 0}
\newcmykcolor{ForestGreen}{0.91 0 0.88 0.12}
\newcmykcolor{PineGreen}{0.92 0 0.59 0.25}
\newcmykcolor{LimeGreen}{0.50 0 1 0}
\newcmykcolor{YellowGreen}{0.44 0 0.74 0}
\newcmykcolor{SpringGreen}{0.26 0 0.76 0}
\newcmykcolor{OliveGreen}{0.64 0 0.95 0.40}
\newcmykcolor{RawSienna}{0 0.72 1 0.45}
\newcmykcolor{Sepia}{0 0.83 1 0.70}
\newcmykcolor{Brown}{0 0.81 1 0.60}
\newcmykcolor{Tan}{0.14 0.42 0.56 0}
\newcmykcolor{Gray}{0 0 0 0.50}
\newcmykcolor{Black}{0 0 0 1}
\newcmykcolor{White}{0 0 0 0}


\newcommand{\bbm}       {\left[\begin{matrix}}
\newcommand{\ebm}       {\end{matrix}\right]}
\newcommand{\bsm}       {\left[\begin{smallmatrix}}
\newcommand{\esm}       {\end{smallmatrix}\right]}
\newcommand{\bpm}       {\begin{pmatrix}}
\newcommand{\epm}       {\end{pmatrix}}
\newcommand{\bcf}[2]{\left(\begin{array}{c}{#1}\\{#2}\end{array}\right)}


\newcommand{\csch}     {\operatorname{csch}}
\newcommand{\sech}     {\operatorname{sech}}
\newcommand{\arcsinh}  {\operatorname{arcsinh}}
\newcommand{\arccosh}  {\operatorname{arccosh}}
\newcommand{\arctanh}  {\operatorname{arctanh}}

\newcommand{\range}     {\operatorname{range}}
\newcommand{\trans}     {\operatorname{trans}}
\newcommand{\trc}       {\operatorname{trace}}
\newcommand{\adj}       {\operatorname{adj}}

\newcommand{\tint}{\textstyle\int}
\newcommand{\tm}{\times}
\newcommand{\sse}{\subseteq}
\newcommand{\st}{\;|\;}
\newcommand{\sm}{\setminus}
\newcommand{\iffa}      {\Leftrightarrow}
\newcommand{\xra}{\xrightarrow}

\renewcommand{\:}{\colon}

\newcommand{\N}         {{\mathbb{N}}}
\newcommand{\Z}         {{\mathbb{Z}}}
\newcommand{\Q}         {{\mathbb{Q}}}
\renewcommand{\R}       {{\mathbb{R}}}
\newcommand{\C}         {{\mathbb{C}}}

\newcommand{\al}        {\alpha}
\newcommand{\bt}        {\beta} 
\newcommand{\gm}        {\gamma}
\newcommand{\dl}        {\delta}
\newcommand{\ep}        {\epsilon}
\newcommand{\zt}        {\zeta}
\newcommand{\et}        {\eta}
\newcommand{\tht}       {\theta}
\newcommand{\io}        {\iota}
\newcommand{\kp}        {\kappa}
\newcommand{\lm}        {\lambda}
\newcommand{\ph}        {\phi}
\newcommand{\ch}        {\chi}
\newcommand{\ps}        {\psi}
\newcommand{\rh}        {\rho}
\newcommand{\sg}        {\sigma}
\newcommand{\om}        {\omega}

\newcommand{\EMPH}[1]{\emph{\RED{#1}}}
\newcommand{\DEFN}[1]{\emph{\PURPLE{#1}}}
\newcommand{\VEC}[1]    {\mathbf{#1}}

\newcommand{\ghost}{{\tiny $\color[rgb]{1,1,1}.$}}


\title{Differentiation}
\author{}

\begin{document}

\slideCaption{\color{white}}

\begin{slide}{}
 {\Huge
  \vspace{6ex}
  \begin{center}
   Differentiation
  \end{center}
 }
\end{slide}

%\maketitle

\overlays{13}{
\begin{slide}{Introduction}
\fromSlide{2}{\noindent{\bf Things you should know:}}
\begin{itemize}
\fromSlide{3}{
 \item The meaning of differentiation 
  (slopes of graphs, time-dependent and space-dependent variables,
  etc)}
\fromSlide{4}{
 \item Some derivatives from first principles: $x^2$, $1/x$, $e^x$.}
\fromSlide{5}{
 \item Rules for finding derivatives:
  \begin{itemize}
   \fromSlide{6}{ \item The product rule ($(uv)'=u'v+uv'$) }
   \fromSlide{7}{ \item The quotient rule ($(u/v)'=(u'v-uv')/v^2$) }
   \fromSlide{8}{ \item The chain rule ($\frac{dz}{dx}=\frac{dz}{dy}\frac{dy}{dx}$)}
   \fromSlide{9}{ \item The power rule ($(u^n)'=n u^{n-1}u'$) }
   \fromSlide{10}{ \item The logarithmic rule ($\log(u)'=u'/u$) }
   \fromSlide{11}{ \item The inverse function rule ($\frac{dx}{dy} = 1/\frac{dy}{dx}$)}
  \end{itemize}
} 
\fromSlide{12}{
 \item Derivatives of various classes of functions (eg the derivative
  of a rational function is another rational function.)
}
\end{itemize}
\fromSlide{13}{
 You must learn to find derivatives quickly and accurately.
}
\end{slide}
}

\overlays{12}{%
\begin{slide}{Meaning}
\begin{itemize}
 \fromSlide{2}{
  \item Consider related variables $x$ and $y$;
   so whenever $x$ changes, so does $y$.
 }
 \fromSlide{3}{
  \item Examples:
   \begin{itemize}
    \fromSlide{4}{ \item $p=\text{ price of chocolate }$;
                         $d=\text{ demand for chocolate }$. }
    \fromSlide{5}{ \item $t=\text{ time }$;
                         $d=\text{ atmospheric $CO_2$ concentration }$. }
    \fromSlide{6}{ \item $r=\text{ distance from sun }$;
                         $g=\text{ strength of solar gravity }$. }
   \end{itemize}
 }

 \fromSlide{7}{
  \item If $x$ changes to $x+\dl x$, then $y$ changes to $y+\dl y$.
   \fromSlide{8}{
    \[ \frac{dy}{dx} = \lim_{\dl x\xra{} 0} \frac{\dl y}{\dl x}
        = \text{ derivative of $y$ with respect to $x$. }
    \]}
 }
 \fromSlide{9}{
  \item If $y=f(x)$, then $\dl y=f(x+\dl x)-f(x)$, so
   \fromSlide{10}{\[ f'(x) = \frac{dy}{dx} =
       \lim_{\dl x\xra{}0} \frac{f(x+\dl x)-f(x)}{\dl x} 
       \fromSlide{11}{= \lim_{h\xra{}0} \frac{f(x+h)-f(x)}{h}.}
   \]}
 }
 \fromSlide{12}{
  \item We sometimes write $y'$ for $dy/dx$ ({\Red care needed}).
 }
\end{itemize}
\end{slide}
}

\overlays{10}{%
\begin{slide}{Slopes}
   \psset{yunit=3cm,xunit=3cm}
\onlySlide*{1}{
   \begin{center}\begin{pspicture}[1](-0.2,-0.2)( 1.7, 2.00)
    \psset{linewidth=0.1pt}
    \psaxes[labels=none,ticksize=1pt]{->}(0,0)(-0.1,0)( 1.7, 1.81)
    \psplot[linecolor=red]{0}{ 1.7}{x -0.8 add dup mul 1 add}
    \rput[tl](1.6,1.6){${\scriptstyle y=f(x)}$}
  \end{pspicture}\end{center}
  Consider variables $x$ and $y$ related by $y=f(x)$.
}
\onlySlide*{2}{
   \begin{center}\begin{pspicture}[1](-0.2,-0.2)( 1.7, 2.00)
    \psset{linewidth=0.1pt}
    \psaxes[labels=none,ticksize=1pt]{->}(0,0)(-0.1,0)( 1.7, 1.81)
    \psplot[linecolor=red]{0}{ 1.7}{x -0.8 add dup mul 1 add}
    \psline[linecolor=blue](1,0)(1, 1.04)(0, 1.04)
    \rput[tr](1,-0.05){${\scriptstyle x}$}
    \rput[r](- 0.05, 1.04){${\scriptstyle y}$}
  \end{pspicture}\end{center}
  Consider variables $x$ and $y$ related by $y=f(x)$.
}
\onlySlide*{3}{
   \begin{center}\begin{pspicture}[1](-0.2,-0.2)( 1.7, 2.00)
    \psset{linewidth=0.1pt}
    \psaxes[labels=none,ticksize=1pt]{->}(0,0)(-0.1,0)( 1.7, 1.81)
    \psplot[linecolor=red]{0}{ 1.7}{x -0.8 add dup mul 1 add}
    \psline[linecolor=Orange](0, 0.64)( 1.7, 1.32)
    \psline[linecolor=blue](1,0)(1, 1.04)(0, 1.04)
    \rput[tr](1,-0.05){${\scriptstyle x}$}
    \rput[r](- 0.05, 1.04){${\scriptstyle y}$}
    \rput[tl]( 1.75, 1.32){slope $\scriptstyle dy/dx$}
  \end{pspicture}\end{center}
  $dy/dx$ is the slope of the tangent line to the graph.
}
\onlySlide*{4}{
   \begin{center}\begin{pspicture}[1](-0.2,-0.2)( 1.7, 2.00)
    \psset{linewidth=0.1pt}
    \psaxes[labels=none,ticksize=1pt]{->}(0,0)(-0.1,0)( 1.7, 1.81)
    \psplot[linecolor=red]{0}{ 1.7}{x -0.8 add dup mul 1 add}
    \psline[linecolor=Orange](0, 0.64)( 1.7, 1.32)
    \psline[linecolor=blue](1,0)(1, 1.04)(0, 1.04)
    \psline[linecolor=blue]( 1.5,0)( 1.5, 1.49)(0, 1.49)
    \rput[tr](1,-0.05){${\scriptstyle x}$}
    \rput[tl]( 1.5,-0.05){${\scriptstyle x+\dl x}$}
    \rput[r](- 0.05, 1.04){${\scriptstyle y}$}
    \rput[r](- 0.05, 1.49){${\scriptstyle y+\dl y}$}
    \rput[tl]( 1.75, 1.32){slope $\scriptstyle dy/dx$}
  \end{pspicture}\end{center}
  If $x$ changes by a small amount $\dl x$, then $y$ will change by a
  small amount $\dl y$.
}
\onlySlide*{5}{
   \begin{center}\begin{pspicture}[1](-0.2,-0.2)( 1.7, 2.00)
    \psset{linewidth=0.1pt}
    \psaxes[labels=none,ticksize=1pt]{->}(0,0)(-0.1,0)( 1.7, 1.81)
    \psplot[linecolor=red]{0}{ 1.7}{x -0.8 add dup mul 1 add}
    \psline[linecolor=Orange](0, 0.64)( 1.7, 1.32)
    \psline[linecolor=blue](1,0)(1, 1.04)(0, 1.04)
    \psline[linecolor=blue]( 1.5,0)( 1.5, 1.49)(0, 1.49)
    \rput[tr](1,-0.05){${\scriptstyle x}$}
    \rput[tl]( 1.5,-0.05){${\scriptstyle x+\dl x}$}
    \rput[r](- 0.05, 1.04){${\scriptstyle y}$}
    \rput[r](- 0.05, 1.49){${\scriptstyle y+\dl y}$}
    \psline[linecolor=OliveGreen,arrows=<->](1, 1.04)( 1.5, 1.04)
    \psline[linecolor=OliveGreen,arrows=<->](1, 1.04)(1, 1.49)
    \rput[t]( 1.250000000, 0.94){${\scriptstyle \dl x}$} 
    \rput[r]( 0.9, 1.265000000){${\scriptstyle \dl y}$}
    \rput[tl]( 1.75, 1.32){slope $\scriptstyle dy/dx$}
  \end{pspicture}\end{center}
  If $x$ changes by a small amount $\dl x$, then $y$ will change by a
  small amount $\dl y$.
}
\onlySlide*{6}{
   \begin{center}\begin{pspicture}[1](-0.2,-0.2)( 1.7, 2.00)
    \psset{linewidth=0.1pt}
    \psaxes[labels=none,ticksize=1pt]{->}(0,0)(-0.1,0)( 1.7, 1.81)
    \psplot[linecolor=red]{0}{ 1.7}{x -0.8 add dup mul 1 add}
    \psline[linecolor=OliveGreen](0, 0.1400000000)( 1.7, 1.670000000)
    \psline[linecolor=Orange](0, 0.64)( 1.7, 1.32)
    \psline[linecolor=blue](1,0)(1, 1.04)(0, 1.04)
    \psline[linecolor=blue]( 1.5,0)( 1.5, 1.49)(0, 1.49)
    \rput[tr](1,-0.05){${\scriptstyle x}$}
    \rput[tl]( 1.5,-0.05){${\scriptstyle x+\dl x}$}
    \rput[r](- 0.05, 1.04){${\scriptstyle y}$}
    \rput[r](- 0.05, 1.49){${\scriptstyle y+\dl y}$}
    \psline[linecolor=OliveGreen,arrows=<->](1, 1.04)( 1.5, 1.04)
    \psline[linecolor=OliveGreen,arrows=<->](1, 1.04)(1, 1.49)
    \rput[t]( 1.250000000, 0.94){${\scriptstyle \dl x}$} 
    \rput[r]( 0.9, 1.265000000){${\scriptstyle \dl y}$}
    \rput[bl]( 1.75, 1.670000000){slope $\scriptstyle \dl y/\dl x$}
    \rput[tl]( 1.75, 1.32){slope $\scriptstyle dy/dx$}
  \end{pspicture}\end{center}
  The ratio $\dl y/\dl x$ is the slope of a chord cutting across the
  graph.
}
\onlySlide*{7}{
  \begin{center}\begin{pspicture}[1](-0.2,-0.2)( 1.7, 2.00)
    \psset{linewidth=0.1pt}
    \psaxes[labels=none,ticksize=1pt]{->}(0,0)(-0.1,0)( 1.7, 1.81)
    \psplot[linecolor=red]{0}{ 1.7}{x -0.8 add dup mul 1 add}
    \psline[linecolor=OliveGreen](0, 0.3900000000)( 1.7, 1.495000000)
    \psline[linecolor=Orange](0, 0.64)( 1.7, 1.32)
    \psline[linecolor=blue](1,0)(1, 1.04)(0, 1.04)
    \psline[linecolor=blue]( 1.250000000,0)( 1.250000000, 1.202500000)(0, 1.202500000)
    \rput[tr](1,-0.05){${\scriptstyle x}$}
    \rput[tl]( 1.250000000,-0.05){${\scriptstyle x+\dl x}$}
    \rput[r](- 0.05, 1.04){${\scriptstyle y}$}
    \rput[r](- 0.05, 1.202500000){${\scriptstyle y+\dl y}$}
    \psline[linecolor=OliveGreen,arrows=<->](1, 1.04)( 1.250000000, 1.04)
    \psline[linecolor=OliveGreen,arrows=<->](1, 1.04)(1, 1.202500000)
    \rput[t]( 1.125000000, 0.94){${\scriptstyle \dl x}$} 
    \rput[r]( 0.9, 1.121250000){${\scriptstyle \dl y}$}
    \rput[bl]( 1.75, 1.495000000){slope $\scriptstyle \dl y/\dl x$}
    \rput[tl]( 1.75, 1.32){slope $\scriptstyle dy/dx$}
  \end{pspicture}\end{center}
  The slope of the chord changes slightly as $\dl x$ decreases.
}
\onlySlide*{8}{
  \begin{center}\begin{pspicture}[1](-0.2,-0.2)( 1.7, 2.00)
    \psset{linewidth=0.1pt}
    \psaxes[labels=none,ticksize=1pt]{->}(0,0)(-0.1,0)( 1.7, 1.81)
    \psplot[linecolor=red]{0}{ 1.7}{x -0.8 add dup mul 1 add}
    \psline[linecolor=OliveGreen](0, 0.5150000000)( 1.7, 1.407500000)
    \psline[linecolor=Orange](0, 0.64)( 1.7, 1.32)
    \psline[linecolor=blue](1,0)(1, 1.04)(0, 1.04)
    \psline[linecolor=blue]( 1.125000000,0)( 1.125000000, 1.105625000)(0, 1.105625000)
    \rput[tr](1,-0.05){${\scriptstyle x}$}
    \rput[tl]( 1.125000000,-0.05){${\scriptstyle x+\dl x}$}
    \rput[r](- 0.05, 1.04){${\scriptstyle y}$}
    \rput[r](- 0.05, 1.105625000){${\scriptstyle y+\dl y}$}
    \psline[linecolor=OliveGreen,arrows=<->](1, 1.04)( 1.125000000, 1.04)
    \psline[linecolor=OliveGreen,arrows=<->](1, 1.04)(1, 1.105625000)
    \rput[t]( 1.062500000, 0.94){${\scriptstyle \dl x}$} 
    \rput[r]( 0.9, 1.072812500){${\scriptstyle \dl y}$}
    \rput[bl]( 1.75, 1.407500000){slope $\scriptstyle \dl y/\dl x$}
    \rput[tl]( 1.75, 1.32){slope $\scriptstyle dy/dx$}
  \end{pspicture}\end{center}
  As $\dl x$ approaches zero, the chord approaches the tangent, and
  $\dl y/\dl x$ approaches $dy/dx$.
}
\onlySlide*{9}{
  \begin{center}\begin{pspicture}[1](-0.2,-0.2)( 1.7, 2.00)
    \psset{linewidth=0.1pt}
    \psaxes[labels=none,ticksize=1pt]{->}(0,0)(-0.1,0)( 1.7, 1.81)
    \psplot[linecolor=red]{0}{ 1.7}{x -0.8 add dup mul 1 add}
    \psline[linecolor=OliveGreen](0, 0.5775000000)( 1.7, 1.363750000)
    \psline[linecolor=Orange](0, 0.64)( 1.7, 1.32)
    \psline[linecolor=blue](1,0)(1, 1.04)(0, 1.04)
    \psline[linecolor=blue]( 1.062500000,0)( 1.062500000, 1.068906250)(0, 1.068906250)
    \rput[tr](1,-0.05){${\scriptstyle x}$}
    \rput[tl]( 1.062500000,-0.05){${\scriptstyle x+\dl x}$}
    \rput[r](- 0.05, 1.04){${\scriptstyle y}$}
    \rput[r](- 0.05, 1.068906250){${\scriptstyle y+\dl y}$}
    \psline[linecolor=OliveGreen,arrows=<->](1, 1.04)( 1.062500000, 1.04)
    \psline[linecolor=OliveGreen,arrows=<->](1, 1.04)(1, 1.068906250)
    \rput[t]( 1.031250000, 0.94){${\scriptstyle \dl x}$} 
    \rput[r]( 0.9, 1.054453125){${\scriptstyle \dl y}$}
    \rput[bl]( 1.75, 1.363750000){slope $\scriptstyle \dl y/\dl x$}
    \rput[tl]( 1.75, 1.32){slope $\scriptstyle dy/dx$}
  \end{pspicture}\end{center}
  As $\dl x$ approaches zero, the chord approaches the tangent, and
  $\dl y/\dl x$ approaches $dy/dx$.
}
\fromSlide{10}{
  \begin{center}\begin{pspicture}[1](-0.2,-0.2)( 1.7, 2.00)
    \psset{linewidth=0.1pt}
    \psaxes[labels=none,ticksize=1pt]{->}(0,0)(-0.1,0)( 1.7, 1.81)
    \psplot[linecolor=red]{0}{ 1.7}{x -0.8 add dup mul 1 add}
    \psline[linecolor=OliveGreen](0, 0.64)( 1.7, 1.32)
    \psline[linecolor=blue](1,0)(1, 1.04)(0, 1.04)
    \rput[tr](1,-0.05){${\scriptstyle x}$}
    \rput[r](- 0.05, 1.04){${\scriptstyle y}$}
    \rput[tl]( 1.75, 1.32){slope $\scriptstyle dy/dx$}
  \end{pspicture}\end{center}
  As $\dl x$ approaches zero, the chord approaches the tangent, and
  $\dl y/\dl x$ approaches $dy/dx$.
}
\end{slide}
}

\overlays{11}{%
\begin{slide}{The function $f(x)=x^2$}
\begin{itemize}
 \fromSlide{2}{
  \item Consider the function $f(x)=x^2$.
 }
 \fromSlide{3}{
  \item Then $f(x+h)=(x+h)^2=x^2+2xh+h^2$
   \fromSlide{4}{
    , so 
    \begin{align*}
     \frac{f(x+h) - f(x)}{h} &= \frac{(x+h)^2 - x^2}{h}
      \hphantom{mmmmmmmm}\\
     & \fromSlide{5}{
      = \frac{\RED{x^2} + 2xh + h^2 - \RED{x^2}}{h}
     } \\
     & \fromSlide{6}{
      = \frac{2x\OLIVEGREEN{h} + \OLIVEGREEN{h}^2}{\OLIVEGREEN{h}}
     } \\
     & \fromSlide{7}{
      = 2x+h
     }
    \end{align*} 
   }
 }
 \fromSlide{8}{
  \item Thus
   \[ f'(x) = \lim_{h\xra{}0} \frac{f(x+h) - f(x)}{h}
       \fromSlide{9}{= \lim_{\RED{h}\xra{}0} (2x+\RED{h})} 
       \fromSlide{10}{= 2x.}
   \]
 }
 \fromSlide{11}{
  \item Similarly: 
   \psframebox[framearc=.3,linecolor=magenta]{
    $\frac{d}{dx}(x^n)=nx^{n-1}$ for all $n$.} 
 }

\end{itemize}
\end{slide}
}

\overlays{9}{%
\begin{slide}{The function $f(x)=1/x$}
\begin{itemize}
 \fromSlide{2}{
  \item Consider the function $f(x)=1/x$.
 }
 \fromSlide{3}{
  \item \ghost\vspace{-3ex}
   \[ f(x+h)-f(x)
       = \frac{1}{x+h} - \frac{1}{x} 
%   \untilSlide*{3}{\hphantom{= \frac{\RED{x}-(\RED{x}+h)}{x(x+h)}}}
   \fromSlide{4}{= \frac{\RED{x}-(\RED{x}+h)}{x(x+h)}}
%   \untilSlide*{4}{\hphantom{= \frac{-h}{x(x+h)}}}
   \fromSlide{5}{= \frac{-h}{x(x+h)}}
   \]
  \fromSlide{6}{
   so 
   \[ \frac{f(x+h) - f(x)}{h} = 
      \frac{-1}{x(x+h)}
   \]
  }
  \fromSlide{7}{
   so
   \[ f'(x) = 
       \lim_{h\xra{}0} \frac{f(x+h) - f(x)}{h} 
      \fromSlide{8}{= \lim_{\RED{h}\xra{}0} \frac{-1}{x(x+\RED{h})}}
      \fromSlide{9}{= \frac{-1}{x^2}}
   \]
  }
 }
\end{itemize}
\end{slide}
}

\overlays{11}{%
\begin{slide}{The exponential function}
\begin{itemize}
 \fromSlide{2}{
  \item Consider the function
   $f(x)=e^x=1+x+\frac{x^2}{2!}+\frac{x^3}{3!}+\dotsb$.
 }
 \fromSlide{3}{
  \item \ghost\vspace{-3ex}
   \[ f(x+h)-f(x)
       = e^{x+h} - e^x 
   \fromSlide{4}{= e^x(e^h - 1)}
   \fromSlide{5}{= e^x\left(h + \tfrac{h^2}{2!} + \tfrac{h^3}{3!}+\dotsb\right)}
   \]
  \fromSlide{6}{
   so 
   \[ \frac{f(x+h) - f(x)}{h} = 
      e^x\left(1 + \tfrac{h}{2!} + \tfrac{h^2}{3!} + \dotsb\right) 
   \]
  }
  \fromSlide{7}{
   so
   \begin{align*} f'(x) &= 
       \lim_{h\xra{}0} \frac{f(x+h) - f(x)}{h} \hphantom{mmmmmmm} \\ &
      \fromSlide{8}{= \lim_{\RED{h}\xra{}0} 
       e^x\left(1 + \tfrac{\RED{h}}{2!} + \tfrac{\RED{h}^2}{3!} + \dotsb\right)} \\ & 
      \fromSlide{9}{= e^x(1+0+0+\dotsb)} \\ &
      \fromSlide{10}{= e^x.}
   \end{align*}
  }
}
 \fromSlide{11}{
  \item Conclusion: $\exp'(x)=\exp(x)$.
 }

\end{itemize}
\end{slide}
}

\overlays{6}{%
\begin{slide}{Special functions}
\[ \begin{array}{rlcrl}
 \fromSlide{2}{\exp'(x)}     & \fromSlide{2}{=\exp(x)} &\hspace{3em}&
 \fromSlide{5}{\log'(x)}     & \fromSlide{5}{=1/x} \\
 \fromSlide{3}{\sinh'(x)}    & \fromSlide{3}{=\cosh(x)} &&
 \fromSlide{6}{\arcsinh'(x)} & \fromSlide{6}{=(1+x^2)^{-1/2}} \\
 \fromSlide{3}{\cosh'(x)}    & \fromSlide{3}{=\sinh(x)} &&
 \fromSlide{6}{\arccosh'(x)} & \fromSlide{6}{=(x^2-1)^{-1/2}} \\
 \fromSlide{3}{\tanh'(x)}    & \fromSlide{3}{=\sech(x)^2 = 1 - \tanh(x)^2} &&
 \fromSlide{6}{\arctanh'(x)} & \fromSlide{6}{=(1-x^2)^{-1}} \\
 \fromSlide{4}{\sin'(x)}     & \fromSlide{4}{=\cos(x)} &&
 \fromSlide{6}{\arcsin'(x)}  & \fromSlide{6}{=(1-x^2)^{-1/2}} \\
 \fromSlide{4}{\cos'(x)}     & \fromSlide{4}{=-\sin(x)} &&
 \fromSlide{6}{\arccos'(x)}  & \fromSlide{6}{=-(1-x^2)^{-1/2}} \\
 \fromSlide{4}{\tan'(x)}     & \fromSlide{4}{=\sec(x)^2 = 1 + \tan(x)^2} &&
 \fromSlide{6}{\arctan'(x)}  & \fromSlide{6}{=(1+x^2)^{-1}}
\end{array} \]
\begin{itemize}
 \fromSlide{2}{\item We showed earlier that $\exp'(x)=\exp(x)$}
 \fromSlide{3}{\item We deduce $\sinh'(x)$ using the identity
 $\sinh(x)=(e^x-e^{-x})/2$.  Similarly for $\cosh$ and $\tanh$.}
 \fromSlide{4}{\item Using $\cos(x)=\cosh(ix)$ etc, we find
 $\sin'(x)$, $\cos'(x)$ and $\tan'(x)$.}
 \fromSlide{5}{\item Using $\exp'(x)=\exp(x)$ and the inverse function
 rule, we find that $\log'(x)=1/x$}
 \fromSlide{6}{\item The inverse function rule also gives the
 remaining derivatives.}
\end{itemize}
\end{slide}
}

\overlays{11}{%
\begin{slide}{The product rule}
\begin{itemize}
 \fromSlide{2}{
  \item Consider variables $u$ and $v$ depending on $x$, and put
   $w=uv$.  Then
   \[ \psframebox[framearc=.3,linecolor=magenta]{
        w' = (uv)' = u'v + u v' \vphantom{\frac{dw}{dx}} } 
     \fromSlide{3}{
      \hspace{1em} \text{ or } \hspace{1 em}
     \psframebox[framearc=.3,linecolor=magenta]{
       \frac{dw}{dx} = \frac{d}{dx}(uv) =
        \frac{du}{dx} v + u \frac{dv}{dx}.
   }} \]
  }
 \fromSlide{4}{
  \item If $x$ changes to $x+\RED{\dl x}$, then $u$ changes to
   $u+\RED{\dl u}$ \& $v$ changes to $v+\RED{\dl v}$
   \fromSlide{5}{ so $w$ changes to
    \begin{align*} w + \RED{\dl w}  &= (u+\RED{\dl u})(v+\RED{\dl v}) 
     \fromSlide{6}{ = uv + \RED{(\dl u)\,v + u\,(\dl v) + (\dl u)(\dl v)}}
     \\
     \fromSlide{7}{ \dl w} &
     \fromSlide{7}{ = (\dl u)\,v + u\,(\dl v) + (\dl u)(\dl v)}
     \\
     \fromSlide{8}{\frac{\dl w}{\dl x}} &
     \fromSlide{8}{
       =
        \frac{\dl u}{\dl x} v + u\frac{\dl v}{\dl x} + 
        \frac{\dl u}{\dl x} \frac{\dl v}{\dl x} \dl x}
     \\ &
     \fromSlide{9}{
      \simeq
        \frac{du}{dx} v + u\frac{dv}{dx} + 
        \frac{du}{dx} \frac{dv}{dx} \dl x
     } \\ &
     \fromSlide{10}{
      \simeq
        \frac{du}{dx} v + u\frac{dv}{dx}
     }
    \end{align*}}
    \fromSlide{11}{
     (The approximations become exact in the limit as $\dl x\xra{}0$.)
    }
   }
\end{itemize}
\end{slide}
}

\overlays{13}{%
\begin{slide}{Examples of the product rule}
\begin{align*}
 \fromSlide{2}{(\RED{u}\BLUE{v})'} & 
 \fromSlide{2}{=\RED{u}'\BLUE{v}+\RED{u}\BLUE{v'}
               \hphantom{mmmmmmmmmmmmmmmmmmmmmm}} \\
 \fromSlide{3}{\frac{d}{dx}(\RED{\sin(x)}\BLUE{\cos(x)})} &
 \fromSlide{4}{=\RED{\sin'(x)}\BLUE{\cos(x)} +
                \RED{\sin(x)}\BLUE{\cos'(x)}}  \\ &
 \fromSlide{5}{=\RED{\cos(x)}\BLUE{\cos(x)} + 
                \RED{\sin(x)}(\BLUE{-\sin(x)}) } \\ &
 \fromSlide{6}{=\cos(x)^2 - \sin(x)^2 } \\
 \fromSlide{7}{\frac{d}{dx}(\RED{x^3}\BLUE{\log(x)})} &
 \fromSlide{8}{=\RED{3x^2}\BLUE{\log(x)} + 
                \RED{x^3}\BLUE{\log'(x)} } \\ &
 \fromSlide{9}{=\RED{3x^2}\BLUE{\log(x)} + 
                \RED{x^3}(\BLUE{x^{-1}})} \\ &
 \fromSlide{10}{=(3\log(x) - 1)x^2 } \\ 
 \fromSlide{11}{\frac{d}{dx}(\RED{e^{ax}}\BLUE{\sin(bx)})} &
 \fromSlide{12}{=\RED{a\,e^{ax}}\BLUE{\sin(bx)} +
                 \RED{e^{ax}}\BLUE{b\cos(bx)}}  \\ &
 \fromSlide{13}{=e^{ax}(a\sin(bx)+b\cos(bx))}
\end{align*}
\end{slide}
}

\overlays{8}{%
\begin{slide}{The quotient rule}
\begin{itemize}
 \fromSlide{2}{
  \item Consider variables $u$ and $v$ depending on $x$, and put
   $w=u/v$.  Then
   \[ \psframebox[framearc=.3,linecolor=magenta]{
        w' = \left(\frac{u}{v}\right)' = \frac{u'v - u v'}{v^2} }
   \]
  }
 \fromSlide{3}{
  \item Indeed: $u=vw$
   \fromSlide{4}{, so $u'=v'w+vw'$ (product rule)}
   \fromSlide{5}{, so
    \[ w' = \frac{u' - v'w}{v}
      \fromSlide{6}{ = \frac{u'}{v} - \frac{v'.(u/v)}{v} }
      \fromSlide{7}{ = \frac{u'}{v} - \frac{uv'}{v^2} }
      \fromSlide{8}{ = \frac{u'v - u v'}{v^2}. }
    \]
   }
 }
\end{itemize}
\end{slide}
}

\overlays{17}{%
\begin{slide}{Examples of the quotient rule}
\begin{align*}
% \fromSlide{2}{(\RED{u}/\BLUE{v})'} & 
% \fromSlide{2}{=(\RED{u}'\BLUE{v}-\RED{u}\BLUE{v'})/\BLUE{v^2}
%               \hphantom{mmmmmmmmmmmmmmmmmmmmmm}} \\
 \fromSlide{2}{\frac{d}{dx}\left(\frac{\RED{x}}{\BLUE{\log(x)}}\right)} &
 \fromSlide{3}{=\frac{\RED{1}.\BLUE{\log(x)} - 
                      \RED{x}\BLUE{x^{-1}}}{\BLUE{\log(x)}^2}} 
 \fromSlide{4}{=\frac{\log(x)-1}{\log(x)^2}}
 \fromSlide{5}{=\log(x)^{-1}-\log(x)^{-2}} \\
\intertext{\fromSlide{6}{
 \hspace{3em} (Aside: $x/\log(x) \simeq ($ number of primes $\leq x)$)
}}
 \fromSlide{7}{\frac{d}{dx}\left(\frac{\RED{x}}{\BLUE{1-x^2}}\right)} &
 \fromSlide{8}{= \frac{\RED{1}.(\BLUE{1-x^2}) - \RED{x}.(\BLUE{-2x})}
                      {\BLUE{(1-x^2)^2}} } 
 \fromSlide{9}{= \frac{1-x^2+2x^2}{(1-x^2)^2}}
 \fromSlide{10}{= \frac{1+x^2}{(1-x^2)^2}} \\
\intertext{
 \fromSlide{11}{Now consider $\tan'(x)$, remembering that $\tan(x)=\sin(x)/\cos(x)$.}}
 \fromSlide{12}{\frac{d}{dx}\left(\frac{\RED{\sin(x)}}{\BLUE{\cos(x)}}\right)} & 
 \fromSlide{13}{=\frac{\RED{\sin'(x)}\BLUE{\cos(x)}-\RED{\sin(x)}\BLUE{\cos'(x)}}
                      {\BLUE{\cos(x)}^2}} \\ &
 \fromSlide{14}{=\frac{\RED{\cos(x)}\BLUE{\cos(x)}-\RED{\sin(x)}(\BLUE{-\sin(x)})}
                      {\BLUE{\cos(x)}^2} } \\ &
 \fromSlide{15}{=\frac{\cos(x)^2+\sin(x)^2}{\cos(x)^2}}
 \fromSlide{16}{=\frac{1}{\cos(x)^2}}
 \fromSlide{17}{=\sec(x)^2}
\end{align*}
\end{slide}
}


\overlays{6}{%
\begin{slide}{The chain rule}
\begin{itemize}
 \fromSlide{2}{
  \item Suppose that $y$ depends on $u$, and $u$ depends on $x$.  Then
   \[ \psframebox[framearc=.3,linecolor=magenta]{
       \frac{dy}{dx} = \frac{dy}{du} \frac{du}{dx} }
   \]
 }
 \fromSlide{3}{
  \item If $x$ changes to $x+\dl x$, then $u$ changes to $u+\dl u$ and $y$
   changes to $y+\dl y$.
   \fromSlide{4}{Clearly
    \[ \frac{\dl y}{\dl x} = \frac{\dl y}{\dl u} \frac{\dl u}{\dl x}. \]
   }
   \fromSlide{5}{In the limit, $\dl x$, $\dl u$ and $\dl y$ all approach
    zero, and we get
    \[ \frac{dy}{dx} = \frac{dy}{du} \frac{du}{dx}. \]
   }
 }
 \fromSlide{6}{
  \item Alternative notation: suppose that $f(x)=g(h(x))$.  Then
   \[ \psframebox[framearc=.3,linecolor=magenta]{
       f'(x) = g'(h(x)) h'(x) }
   \]
 }
 \end{itemize}
\end{slide}
}


\overlays{19}{%
\begin{slide}{Examples of the chain rule}
\begin{itemize}
 \fromSlide{2}{
  \item Consider $y=\cos(x^2)$.
  \fromSlide{3}{This is $y=\cos(u)$, where $u=x^2$.}
  \[
   \fromSlide{4}{\frac{du}{dx} = 2x \hspace{3em}}
   \fromSlide{5}{\frac{dy}{du} = -\sin(u)}
   \fromSlide{6}{= -\sin(x^2)}
  \] \[
   \fromSlide{7}{\frac{dy}{dx} = \frac{dy}{du}\frac{du}{dx}}
   \fromSlide{8}{= -\sin(x^2).2x}
   \fromSlide{9}{= -2x\sin(x^2).}
  \]}
 \fromSlide{10}{
  \item Consider $f(x)=\exp(\sin(x))$.  
   \[ \fromSlide{11}{f'(x) = \exp'(\sin(x)).\sin'(x)}
      \fromSlide{12}{= \exp(\sin(x))\cos(x).}
   \]
 }
 \fromSlide{13}{
  \item Consider $y=a\sin(b x+c)$.  
   \fromSlide{14}{Put $u=bx+c$, so $y=a\sin(u)$. \par}
   \fromSlide{15}{Then $\frac{du}{dx}=b$}
   \fromSlide{16}{and $\frac{dy}{du}=a\cos(u)$}
   \fromSlide{17}{so }
    \[ \fromSlide{17}{\frac{dy}{dx} = \frac{dy}{du}\frac{du}{dx}}
       \fromSlide{18}{= a\cos(u).b = ab\cos(u)}
       \fromSlide{19}{= ab\cos(bx+c). }
    \]
   }

\end{itemize}
\end{slide}
}

\overlays{8}{%
\begin{slide}{The power rule}
\begin{itemize}
 \fromSlide{2}{
  \item If $u$ depends on $x$ and $n$ does not, then
   \[ \psframebox[framearc=.3,linecolor=magenta]{
       \frac{d}{dx}(u^n)=
        \RED{nu^{n-1}}\BLUE{\frac{du}{dx}}}
   \]}
 \fromSlide{3}{
  \item Reason: If $y=u^n$ then $\frac{dy}{du}=nu^{n-1}$ so
   $\frac{dy}{dx}=\frac{dy}{du}\frac{du}{dx}=nu^{n-1}\frac{du}{dx}$
 }
 \fromSlide{4}{
  \item Consider $y=\sqrt{1+x^2}$.  This is $y=u^{1/2}$, where
   $u=1+x^2$.  Then 
   \[ \frac{dy}{du}=\RED{\frac{1}{2}u^{-1/2}}
                   = \RED{\frac{1}{2\sqrt{1+x^2}}}
      \hspace{3em}
      \fromSlide{5}{\frac{du}{dx}=\BLUE{2x}}
   \]
   \fromSlide{6}{
   \[ \frac{dy}{dx} = \RED{\frac{1}{2\sqrt{1+x^2}}}\BLUE{2x}
                    = \frac{x}{\sqrt{1+x^2}}.
   \]}
 }
 \fromSlide{7}{
  \item $\frac{d}{dx}\left(\sin(x)^5\right)=
         \RED{5\sin(x)^4}\BLUE{\cos(x)}$
 }
 \fromSlide{8}{
  \item $\frac{d}{dx}\left(\log(x)^3\right)=
         \RED{3\log(x)^2}\BLUE{x^{-1}}=3\log(x)^2/x$
 } 
\end{itemize}
\end{slide}
}



\overlays{17}{%
\begin{slide}{The logarithmic rule}
\begin{itemize}
 \fromSlide{2}{
  \item \ghost \vspace{-3ex}
   \[ 
    \psframebox[framearc=.3,linecolor=magenta]{
     \frac{d}{dx}\log(u) = \frac{1}{u}\,\frac{du}{dx}}
    \hspace{3em}
    \fromSlide{3}{\psframebox[framearc=.3,linecolor=magenta]{
     \frac{du}{dx} = u\,\frac{d}{dx}\log(u)}}
   \]
 }
 \fromSlide{4}{
  \item \ghost \vspace{-3ex}
   \[ \frac{d}{dx}\log(\cos(x)) 
       \fromSlide{5}{= \frac{1}{\cos(x)}\cos'(x) }
       \fromSlide{6}{= \frac{-\sin(x)}{\cos(x)} }
       \fromSlide{7}{= -\tan(x)}
   \]
 }
 \fromSlide{8}{
  \item \ghost \vspace{-3ex}
   \[ \frac{d}{dx}\log(1+x^2) 
       \fromSlide{9}{= \frac{\frac{d}{dx}(1+x^2)}{1+x^2}}
       \fromSlide{10}{= \frac{2x}{1+x^2} }
   \]
 }
 \fromSlide{11}{
  \item Consider $y=x^x$%
   \fromSlide{12}{, so $\log(y)=x\log(x)$.}
   \fromSlide{13}{Then }
   \begin{align*}
    \fromSlide{13}{\frac{d}{dx}\log(y)} &
    \fromSlide{13}{= \frac{d}{dx}(x\log(x))} \\ &
    \fromSlide{14}{= 1.\log(x) + x.x^{-1}}
    \fromSlide{15}{= \log(x) + 1} \\
    \fromSlide{16}{\frac{dy}{dx}} &
    \fromSlide{16}{= y\frac{d}{dx}\log(y)} \\ &
    \fromSlide{17}{= x^x(\log(x) + 1).}
   \end{align*}
 }
\end{itemize}
\end{slide}
}

\overlays{10}{%
\begin{slide}{The inverse function rule}
\begin{itemize}
 \fromSlide{2}{
  \item If $x$ and $y$ are interdependent variables, then
   \[ \psframebox[framearc=.3,linecolor=magenta]{
       \frac{dx}{dy} = 1/\frac{dy}{dx}}
   \]
 }
 \fromSlide{3}{
  \item (Take limits in the obvious relation 
   $\frac{\dl x}{\dl y} = 1/\frac{\dl y}{\dl x}$.)
 }
 \fromSlide{4}{
  \item Consider $y=\log(x)$\fromSlide{5}{, so $x=e^y$.} 
   \[
    \fromSlide{6}{\frac{dx}{dy}} 
    \fromSlide{6}{=e^y=x} \hspace{4em}
    \fromSlide{7}{\frac{dy}{dx}} 
    \fromSlide{7}{=1/\frac{dx}{dy}}
    \fromSlide{8}{=\frac{1}{x}}
   \]
 }
 \fromSlide{9}{
  \item Alternative notation: if $y=g(x)$ then $x=f(y)$, where
   $f=g^{-1}$ and $g=f^{-1}$.  Then
   \[ \psframebox[framearc=.3,linecolor=magenta]{
       g'(x) = 1/f'(g(x))}
   \]
 }
 \fromSlide{10}{
  \item $\log'(x)=1/\exp'(\log(x))=1/\exp(\log(x))=1/x$.
 }
 \end{itemize}
\end{slide}
}

\overlays{11}{%
\begin{slide}{The arcsin function}
\begin{itemize}
 \fromSlide{2}{
  \item Consider $y=\arcsin(x)$\fromSlide{3}{, so $x=\sin(y)$.}
   \begin{align*}
    \fromSlide{4}{\frac{dx}{dy}} &
    \fromSlide{5}{= \sin'(y) = \cos(y) \hphantom{mmmmm}} \\
    \fromSlide{6}{\frac{dy}{dx}} &
    \fromSlide{6}{= 1/\frac{dx}{dy} = \cos(y)^{-1}.}
   \end{align*}
 }
 \fromSlide{7}{
  \item Also $\sin(y)^2+\cos(y)^2=1$\fromSlide{8}{, so}
   \begin{align*}
    \fromSlide{8}{\cos(y)} &
    \fromSlide{8}{=\sqrt{1-\sin(y)^2}} 
    \fromSlide{9}{=\sqrt{1-x^2}} \\
    \fromSlide{10}{\cos(y)^{-1}} &
    \fromSlide{10}{=(1-x^2)^{-1/2}\hphantom{mmmmmmmmm}}
   \end{align*}
 }
 \fromSlide{11}{
  \item So $\arcsin'(x)=\frac{dy}{dx}=(1-x^2)^{-1/2}$.
 }
\end{itemize}
\end{slide}
}


\overlays{10}{%
\begin{slide}{The arctanh function}
\begin{itemize}
 \fromSlide{2}{
  \item Consider $y=\arctanh(x)$\fromSlide{3}{, so
    $x=\tanh(y)=\frac{\sinh(y)}{\cosh(y)}$.}
   \begin{align*}
    \fromSlide{4}{\frac{dx}{dy}} &
    \fromSlide{4}{= \tanh'(y) \hphantom{mmmmmmmmmmmmmmmmm}} \\ & 
    \fromSlide{5}{= \frac{\sinh'(y)\cosh(y)-\sinh(y)\cosh'(y)}
                         {\cosh(y)^2}} \\ &
    \fromSlide{6}{= \frac{\cosh(y)^2-\sinh(y)^2}{\cosh(y)^2}} \\ &
    \fromSlide{7}{= 1 - \tanh(y)^2}
    \fromSlide{8}{= 1-x^2} \\
    \fromSlide{9}{\frac{dy}{dx}} &
    \fromSlide{9}{= 1/\frac{dx}{dy} = \frac{1}{1-x^2}.}
   \end{align*}
 }
 \fromSlide{10}{
  \item So $\arctanh'(x)=\frac{dy}{dx}=(1-x^2)^{-1}$.
 }
\end{itemize}
\end{slide}
}


\overlays{13}{%
\begin{slide}{Classes of functions}
\begin{itemize}
 \fromSlide{2}{
  \item If $f(x)$ is a polynomial, then so is $f'(x)$.
   \begin{itemize}
    \item \fromSlide{3}{
     Eg $f(x)=x+x^{10}+x^{100}$;\qquad $f'(x)=1+10x^9+100x^{99}$}
    \item \fromSlide{4}{
     Eg $f(x)=(x-1)^4+(x+1)^4$;\qquad $f'(x)=4(x-1)^3+4(x+1)^3$}
   \end{itemize}
 }
 \fromSlide{5}{
  \item If $f(x)$ is a rational function, then so is $f'(x)$.
   \begin{itemize}
    \item \fromSlide{6}{
     Eg $f(x)=\frac{x^2-1}{x^2+1}$;\qquad $f'(x)=\frac{4x}{(x^2+1)^2}$}
    \item \fromSlide{7}{
     Eg $f(x)=\frac{1}{x}+\frac{1}{x+1}+\frac{1}{x+2}$;\qquad
        $f'(x)=-\frac{1}{x^2}-\frac{1}{(x+1)^2}-\frac{1}{(x+2)^2}$}
   \end{itemize}
 }
 \fromSlide{8}{
  \item If $f(x)$ is a trigonomatric polynomial, so is $f'(x)$.
   \begin{itemize}
    \item \fromSlide{9}{
     Eg $f(x)=\sin(x)+\sin(3x)/3+\sin(5x)/5$;\qquad
        $f'(x)=\cos(x)+\cos(3x)+\cos(5x)$.}
    \item \fromSlide{10}{
     Eg $f(x)=\sin(3x)+\cos(3x)$;\qquad $f'(x)=3\cos(3x)-3\sin(3x)$.}
   \end{itemize}
 }
 \fromSlide{11}{
  \item If $f(x)$ is a polynomial times $e^x$, so is $f'(x)$.
   \begin{itemize}
    \item \fromSlide{12}{
     Eg $f(x)=(x+x^2)e^x$;\qquad $f'(x)=(1+3x+x^2)e^x$.}
    \item \fromSlide{13}{
     Eg $f(x)=(x^4-4x^3+12x^2-24x+24)e^x$;\qquad $f'(x)=x^4e^x$.}
   \end{itemize}
 }
\end{itemize}
\end{slide}
}

\end{document}
