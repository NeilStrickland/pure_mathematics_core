\documentclass[%
pdf,
neil,
colorBG,
slideColor,
]{prosper}
\usepackage{amsmath}
\usepackage{pstricks,pst-node,pst-text,pst-3d,pst-plot}
\usepackage[usenames,dvips]{color}

\newcommand{\GREENYELLOW}[1]{{\color{GreenYellow}#1}}
\newcommand{\YELLOW}[1]{{\color{Yellow}#1}}
\newcommand{\YLW}[1]{{\color{Yellow}#1}}
\newcommand{\GOLDENROD}[1]{{\color{Goldenrod}#1}}
\newcommand{\DANDELION}[1]{{\color{Dandelion}#1}}
\newcommand{\APRICOT}[1]{{\color{Apricot}#1}}
\newcommand{\PEACH}[1]{{\color{Peach}#1}}
\newcommand{\MELON}[1]{{\color{Melon}#1}}
\newcommand{\YELLOWORANGE}[1]{{\color{YellowOrange}#1}}
\newcommand{\ORANGE}[1]{{\color{Orange}#1}}
\newcommand{\BURNTORANGE}[1]{{\color{BurntOrange}#1}}
\newcommand{\BITTERSWEET}[1]{{\color{Bittersweet}#1}}
\newcommand{\REDORANGE}[1]{{\color{RedOrange}#1}}
\newcommand{\MAHOGANY}[1]{{\color{Mahogany}#1}}
\newcommand{\MAROON}[1]{{\color{Maroon}#1}}
\newcommand{\BRICKRED}[1]{{\color{BrickRed}#1}}
\newcommand{\RED}[1]{{\color{Red}#1}}
\newcommand{\ORANGERED}[1]{{\color{OrangeRed}#1}}
\newcommand{\RUBINERED}[1]{{\color{RubineRed}#1}}
\newcommand{\WILDSTRAWBERRY}[1]{{\color{WildStrawberry}#1}}
\newcommand{\SALMON}[1]{{\color{Salmon}#1}}
\newcommand{\CARNATIONPINK}[1]{{\color{CarnationPink}#1}}
\newcommand{\MAGENTA}[1]{{\color{Magenta}#1}}
\newcommand{\VIOLETRED}[1]{{\color{VioletRed}#1}}
\newcommand{\RHODAMINE}[1]{{\color{Rhodamine}#1}}
\newcommand{\MULBERRY}[1]{{\color{Mulberry}#1}}
\newcommand{\REDVIOLET}[1]{{\color{RedViolet}#1}}
\newcommand{\FUCHSIA}[1]{{\color{Fuchsia}#1}}
\newcommand{\LAVENDER}[1]{{\color{Lavender}#1}}
\newcommand{\THISTLE}[1]{{\color{Thistle}#1}}
\newcommand{\ORCHID}[1]{{\color{Orchid}#1}}
\newcommand{\DARKORCHID}[1]{{\color{DarkOrchid}#1}}
\newcommand{\PURPLE}[1]{{\color{Purple}#1}}
\newcommand{\PLUM}[1]{{\color{Plum}#1}}
\newcommand{\VIOLET}[1]{{\color{Violet}#1}}
\newcommand{\ROYALPURPLE}[1]{{\color{RoyalPurple}#1}}
\newcommand{\BLUEVIOLET}[1]{{\color{BlueViolet}#1}}
\newcommand{\PERIWINKLE}[1]{{\color{Periwinkle}#1}}
\newcommand{\CADETBLUE}[1]{{\color{CadetBlue}#1}}
\newcommand{\CORNFLOWERBLUE}[1]{{\color{CornflowerBlue}#1}}
\newcommand{\MIDNIGHTBLUE}[1]{{\color{MidnightBlue}#1}}
\newcommand{\NAVYBLUE}[1]{{\color{NavyBlue}#1}}
\newcommand{\ROYALBLUE}[1]{{\color{RoyalBlue}#1}}
\newcommand{\BLUE}[1]{{\color{Blue}#1}}
\newcommand{\CERULEAN}[1]{{\color{Cerulean}#1}}
\newcommand{\CYAN}[1]{{\color{Cyan}#1}}
\newcommand{\PROCESSBLUE}[1]{{\color{ProcessBlue}#1}}
\newcommand{\SKYBLUE}[1]{{\color{SkyBlue}#1}}
\newcommand{\TURQUOISE}[1]{{\color{Turquoise}#1}}
\newcommand{\TEALBLUE}[1]{{\color{TealBlue}#1}}
\newcommand{\AQUAMARINE}[1]{{\color{Aquamarine}#1}}
\newcommand{\BLUEGREEN}[1]{{\color{BlueGreen}#1}}
\newcommand{\EMERALD}[1]{{\color{Emerald}#1}}
\newcommand{\JUNGLEGREEN}[1]{{\color{JungleGreen}#1}}
\newcommand{\SEAGREEN}[1]{{\color{SeaGreen}#1}}
\newcommand{\GREEN}[1]{{\color{Green}#1}}
\newcommand{\FORESTGREEN}[1]{{\color{ForestGreen}#1}}
\newcommand{\PINEGREEN}[1]{{\color{PineGreen}#1}}
\newcommand{\LIMEGREEN}[1]{{\color{LimeGreen}#1}}
\newcommand{\YELLOWGREEN}[1]{{\color{YellowGreen}#1}}
\newcommand{\SPRINGGREEN}[1]{{\color{SpringGreen}#1}}
\newcommand{\OLIVEGREEN}[1]{{\color{OliveGreen}#1}}
\newcommand{\OLG}[1]{{\color{OliveGreen}#1}}
\newcommand{\RAWSIENNA}[1]{{\color{RawSienna}#1}}
\newcommand{\SEPIA}[1]{{\color{Sepia}#1}}
\newcommand{\BROWN}[1]{{\color{Brown}#1}}
\newcommand{\TAN}[1]{{\color{Tan}#1}}
\newcommand{\GRAY}[1]{{\color{Gray}#1}}
\newcommand{\WHITE}[1]{{\color{White}#1}}
\newcommand{\BLACK}[1]{{\color{Black}#1}}

\newcmykcolor{GreenYellow}{0.15 0 0.69 0}
\newcmykcolor{Yellow}{0 0 1 0}
\newcmykcolor{Goldenrod}{0 0.10 0.84 0}
\newcmykcolor{Dandelion}{0 0.29 0.84 0}
\newcmykcolor{Apricot}{0 0.32 0.52 0}
\newcmykcolor{Peach}{0 0.50 0.70 0}
\newcmykcolor{Melon}{0 0.46 0.50 0}
\newcmykcolor{YellowOrange}{0 0.42 1 0}
\newcmykcolor{Orange}{0 0.61 0.87 0}
\newcmykcolor{BurntOrange}{0 0.51 1 0}
\newcmykcolor{Bittersweet}{0 0.75 1 0.24}
\newcmykcolor{RedOrange}{0 0.77 0.87 0}
\newcmykcolor{Mahogany}{0 0.85 0.87 0.35}
\newcmykcolor{Maroon}{0 0.87 0.68 0.32}
\newcmykcolor{BrickRed}{0 0.89 0.94 0.28}
\newcmykcolor{Red}{0 1 1 0}
\newcmykcolor{OrangeRed}{0 1 0.50 0}
\newcmykcolor{RubineRed}{0 1 0.13 0}
\newcmykcolor{WildStrawberry}{0 0.96 0.39 0}
\newcmykcolor{Salmon}{0 0.53 0.38 0}
\newcmykcolor{CarnationPink}{0 0.63 0 0}
\newcmykcolor{Magenta}{0 1 0 0}
\newcmykcolor{VioletRed}{0 0.81 0 0}
\newcmykcolor{Rhodamine}{0 0.82 0 0}
\newcmykcolor{Mulberry}{0.34 0.90 0 0.02}
\newcmykcolor{RedViolet}{0.07 0.90 0 0.34}
\newcmykcolor{Fuchsia}{0.47 0.91 0 0.08}
\newcmykcolor{Lavender}{0 0.48 0 0}
\newcmykcolor{Thistle}{0.12 0.59 0 0}
\newcmykcolor{Orchid}{0.32 0.64 0 0}
\newcmykcolor{DarkOrchid}{0.40 0.80 0.20 0}
\newcmykcolor{Purple}{0.45 0.86 0 0}
\newcmykcolor{Plum}{0.50 1 0 0}
\newcmykcolor{Violet}{0.79 0.88 0 0}
\newcmykcolor{RoyalPurple}{0.75 0.90 0 0}
\newcmykcolor{BlueViolet}{0.86 0.91 0 0.04}
\newcmykcolor{Periwinkle}{0.57 0.55 0 0}
\newcmykcolor{CadetBlue}{0.62 0.57 0.23 0}
\newcmykcolor{CornflowerBlue}{0.65 0.13 0 0}
\newcmykcolor{MidnightBlue}{0.98 0.13 0 0.43}
\newcmykcolor{NavyBlue}{0.94 0.54 0 0}
\newcmykcolor{RoyalBlue}{1 0.50 0 0}
\newcmykcolor{Blue}{1 1 0 0}
\newcmykcolor{Cerulean}{0.94 0.11 0 0}
\newcmykcolor{Cyan}{1 0 0 0}
\newcmykcolor{ProcessBlue}{0.96 0 0 0}
\newcmykcolor{SkyBlue}{0.62 0 0.12 0}
\newcmykcolor{Turquoise}{0.85 0 0.20 0}
\newcmykcolor{TealBlue}{0.86 0 0.34 0.02}
\newcmykcolor{Aquamarine}{0.82 0 0.30 0}
\newcmykcolor{BlueGreen}{0.85 0 0.33 0}
\newcmykcolor{Emerald}{1 0 0.50 0}
\newcmykcolor{JungleGreen}{0.99 0 0.52 0}
\newcmykcolor{SeaGreen}{0.69 0 0.50 0}
\newcmykcolor{Green}{1 0 1 0}
\newcmykcolor{ForestGreen}{0.91 0 0.88 0.12}
\newcmykcolor{PineGreen}{0.92 0 0.59 0.25}
\newcmykcolor{LimeGreen}{0.50 0 1 0}
\newcmykcolor{YellowGreen}{0.44 0 0.74 0}
\newcmykcolor{SpringGreen}{0.26 0 0.76 0}
\newcmykcolor{OliveGreen}{0.64 0 0.95 0.40}
\newcmykcolor{RawSienna}{0 0.72 1 0.45}
\newcmykcolor{Sepia}{0 0.83 1 0.70}
\newcmykcolor{Brown}{0 0.81 1 0.60}
\newcmykcolor{Tan}{0.14 0.42 0.56 0}
\newcmykcolor{Gray}{0 0 0 0.50}
\newcmykcolor{Black}{0 0 0 1}
\newcmykcolor{White}{0 0 0 0}


\newcommand{\bbm}       {\left[\begin{matrix}}
\newcommand{\ebm}       {\end{matrix}\right]}
\newcommand{\bsm}       {\left[\begin{smallmatrix}}
\newcommand{\esm}       {\end{smallmatrix}\right]}
\newcommand{\bpm}       {\begin{pmatrix}}
\newcommand{\epm}       {\end{pmatrix}}
\newcommand{\bcf}[2]{\left(\begin{array}{c}{#1}\\{#2}\end{array}\right)}


\newcommand{\csch}     {\operatorname{csch}}
\newcommand{\sech}     {\operatorname{sech}}
\newcommand{\arcsinh}  {\operatorname{arcsinh}}
\newcommand{\arccosh}  {\operatorname{arccosh}}
\newcommand{\arctanh}  {\operatorname{arctanh}}

\newcommand{\range}     {\operatorname{range}}
\newcommand{\trans}     {\operatorname{trans}}
\newcommand{\trc}       {\operatorname{trace}}
\newcommand{\adj}       {\operatorname{adj}}

\newcommand{\tint}{\textstyle\int}
\newcommand{\tm}{\times}
\newcommand{\sse}{\subseteq}
\newcommand{\st}{\;|\;}
\newcommand{\sm}{\setminus}
\newcommand{\iffa}      {\Leftrightarrow}
\newcommand{\xra}{\xrightarrow}

\renewcommand{\:}{\colon}

\newcommand{\N}         {{\mathbb{N}}}
\newcommand{\Z}         {{\mathbb{Z}}}
\newcommand{\Q}         {{\mathbb{Q}}}
\renewcommand{\R}       {{\mathbb{R}}}
\newcommand{\C}         {{\mathbb{C}}}

\newcommand{\al}        {\alpha}
\newcommand{\bt}        {\beta} 
\newcommand{\gm}        {\gamma}
\newcommand{\dl}        {\delta}
\newcommand{\ep}        {\epsilon}
\newcommand{\zt}        {\zeta}
\newcommand{\et}        {\eta}
\newcommand{\tht}       {\theta}
\newcommand{\io}        {\iota}
\newcommand{\kp}        {\kappa}
\newcommand{\lm}        {\lambda}
\newcommand{\ph}        {\phi}
\newcommand{\ch}        {\chi}
\newcommand{\ps}        {\psi}
\newcommand{\rh}        {\rho}
\newcommand{\sg}        {\sigma}
\newcommand{\om}        {\omega}

\newcommand{\EMPH}[1]{\emph{\RED{#1}}}
\newcommand{\DEFN}[1]{\emph{\PURPLE{#1}}}
\newcommand{\VEC}[1]    {\mathbf{#1}}

\newcommand{\ghost}{{\tiny $\color[rgb]{1,1,1}.$}}



\title{Sets}
\author{}
\begin{document}

\begin{slide}{}
 {\Huge
  \vspace{6ex}
  \begin{center}
   Sets
  \end{center}
 }
\end{slide}

\overlays{12}{
\begin{slide}{Introduction}
\fromSlide{2}{\noindent{\bf Key ideas:}
\setcounter{item@step}{2}
\begin{itemstep}
 \item Mathematical objects should be gathered into sets:
  the set of all real numbers, the set of all $3\tm 3$ matrices, \dots
 \item Geometric figures (such as lines, rectangles or circles) are
  subsets of $\R^2$ or $\R^3$.
 \item ``Solving'' a system of equations means finding the solution
  set.
\end{itemstep}
}

\fromSlide{6}{
\noindent{\bf Skills to learn or practice:}
\setcounter{item@step}{6}
\begin{itemstep}
 \item Understand and use the basic notation of set theory
 \item Show that one set is a subset of another
 \item Show that two sets are the same
 \item Find the intersection or union of two sets
 \item Find the complement of one set in another
 \item Manipulate inequalities.
\end{itemstep}
}
\end{slide}
}

\overlays{6}{%
\begin{slide}{Terminology}
\setcounter{item@step}{1}
\begin{itemstep}
 \item A \DEFN{set} is an (unordered) collection of mathematical
  objects.  For example, $A=\{-2,-1,0,1,2\}$ is a set of numbers.
 \item The members of the collection are called the \DEFN{elements} of
  the set.  They could be numbers (as in the example) or vectors,
  matrices, or functions, or \ldots.
 \item The notation $\RED{x\in X}$ means that $x$ is an element of the
  set $X$.  For example, $2\in A$ but $3\not\in A$.
 \item The number of different elements in $X$ is called the
  \DEFN{order} of $X$, written $\RED{|X|}$.  For example $|A|=5$.
 \item Given sets $X$ and $Y$, we say that $Y$ is a \DEFN{subset} of
  $X$ (written $\RED{Y\sse X}$) if every element of $Y$ is also an
  element of $X$.  For example, the set $B=\{0,1,2\}$ is a subset of
  $A$. 
\end{itemstep}
\end{slide}
}

\overlays{8}{%
\begin{slide}{Describing sets}
 \fromSlide{2}{A set can be described by:}
 \setcounter{item@step}{2}
 \begin{itemstep}
  \item Listing the elements (eg $A=\{-2,-1,0,1,2\}$)
  \item Describing the elements 
   (eg $B=\{\text{ all prime numbers }\}$)
  \item Cutting down a larger set
   (eg $C=\{n\in A\st n<0\}=\{-1,-2\}$)
  \item Applying an operation to the elements of another set \\
   (eg $D=\{n^2+2\st n\in A\}=\{(-2)^2+2,(-1)^2+2,0^2+2,1^2+2,2^2+2\}=\{2,3,6\}$)
  \item Using standard names for standard sets (see next slide)
  \item Using intersections, unions and complements (see later slide).
 \end{itemstep}
\end{slide}
}


\overlays{13}{%
\begin{slide}{Standard sets of numbers}
\begin{itemize}
 \fromSlide{2}{
  \item \ghost \vspace{-5ex}
   {\tiny \begin{align*}
    \fromSlide{2}{\RED{\N}} &
    \fromSlide{2}{= \{ \text{natural numbers} \}
                  = \{1,2,3,\ldots\}} \hphantom{mmmmmm} \\
    \fromSlide{3}{\RED{\Z}} &
    \fromSlide{3}{= \{ \text{integers} \}=\{ \ldots,-2,-1,0,1,2,\ldots \}} \\
    \fromSlide{4}{\RED{\R}} &
    \fromSlide{4}{= \{ \text{real numbers} \} 
                    \supseteq\{5,\sqrt{3},\pi,e,10^{100}\} } \\
    \fromSlide{5}{\RED{\C}} &
    \fromSlide{5}{= \{ \text{complex numbers} \}
                    \supseteq\{5,\pi,\sqrt{-7},e^{i\pi/6}\}}
   \end{align*} \vspace{-5ex}}}
 \fromSlide{6}{
  \item $\infty$ is \EMPH{not} an element of $\N$, $\Z$, $\R$ or $\C$.}
  \fromSlide{7}{
   \item These sets are called (\RED{open}, \OLIVEGREEN{closed}, or 
    \PURPLE{half-open}) intervals:
    {\vspace{-1ex}\tiny \[ \begin{array}{rlcrl}
     \fromSlide{7}{\RED{(a,b)}} &
     \fromSlide{7}{= \{x\in\R\st a<x<b\}}
     &\hspace{2em}&
     \fromSlide{7}{\OLIVEGREEN{[a,b]}} &
     \fromSlide{7}{= \{x\in\R\st a\leq x\leq b\}} \\
     \fromSlide{8}{\PURPLE{[a,b)}} &
     \fromSlide{8}{= \{x\in\R\st a\leq x<b\}} &&
     \fromSlide{8}{\PURPLE{(a,b]}} &
     \fromSlide{8}{= \{x\in\R\st a<x\leq b\}} \\
     \fromSlide{9}{\RED{(a,\infty)}} &
     \fromSlide{9}{= \{x\in\R\st a<x\}} &&
     \fromSlide{9}{\RED{(-\infty,b)}} &
     \fromSlide{9}{= \{x\in\R\st x< b\}} \\
     \fromSlide{10}{\PURPLE{[a,\infty)}} &
     \fromSlide{10}{= \{x\in\R\st a\leq x\}} &&
     \fromSlide{10}{\PURPLE{(-\infty,b]}} &
     \fromSlide{10}{= \{x\in\R\st x\leq b\}}
    \end{array} \] \vspace{-3ex}}}
 \fromSlide{11}{
  \item $(-\infty,\infty)$ is the same as $\R$.}
 \fromSlide{12}{
  \item Notation like $[a,\infty\RED{]}$ is usually a mistake, because
   $\infty\not\in\R$.}
 \fromSlide{13}{
  \item $\emptyset$ or $\{\}$ denotes the empty set, which has no
   elements.  It is useful for the same sort of reason that the number
   zero is useful.}
\end{itemize}
\end{slide}
}

\overlays{8}{%
\begin{slide}{Sets in the plane}
\begin{itemize}
 \fromSlide{2}{
  \item $\RED{\R^2}=\{\text{points in the plane}\}
                   =\{\text{2D vectors}\}
                   =\{(x,y)\st x,y\in\R\}$.}
 \fromSlide{3}{
  \item A geometric figure in the plane is a subset of $\R^2$.}
 \fromSlide{4}{
  \item
   \psset{unit=.7cm}
   \fromSlide{4}{\begin{pspicture}[1](-2,-3)(2,2)
    \psaxes[labels=none,ticks=none]{->}(0,0)(-1.7,-1.7)(1.7,1.7)
    \pscircle[linecolor=red](0,0){1}
    \rput(0,-2.3){$\scriptstyle\{(x,y)\in\R^2\st x^2+y^2=1\}$}
   \end{pspicture}}
   \hspace{2em}
   \fromSlide{5}{\begin{pspicture}[1](-2,-3)(2,2)
    \psaxes[labels=none,ticks=none]{->}(0,0)(-1.7,-1.7)(1.7,1.7)
    \psdots[linecolor=red](1,1)(1,-1)(-1,1)(-1,-1)
    \rput(0,-2.3){$\scriptstyle\{(x,y)\in\R^2\st x^2=y^2=1\}$}
   \end{pspicture}}
   \hspace{2em}
   \fromSlide{6}{\begin{pspicture}[1](-2,-3)(2,2)
    \psframe*[linecolor=red](0,0)(1,1)
    \psaxes[labels=none,ticks=none]{->}(0,0)(-1.7,-1.7)(1.7,1.7)
    \rput(0,-2.3){$\scriptstyle\{(x,y)\in\R^2\st 0\leq x,y\leq 1\}$}
   \end{pspicture}}}
 \fromSlide{7}{
  \item $\RED{\R^3}=\{\text{3D vectors}\}=
         \{(x,y,z)\st x,y,z\in\R\}$}
 \fromSlide{8}{
  \item 3D geometric figures are subsets of $\R^3$.}
\end{itemize} 
\end{slide}
}

\overlays{9}{%
\begin{slide}{Solution sets}
\begin{itemize}
 \fromSlide{2}{
  \item An equation (or system of equations) may have no solutions,
   one solution, or many solutions.}
 \fromSlide{3}{
  \item Instead of talking about ``the solution'', we should consider
   the \EMPH{set} of \EMPH{all} solutions.}
 \fromSlide{4}{
  \item The roots of $x^2-5x+6$ are $x=2$ and $x=3$, \\
   but the function $x^2-5x+7$ has no real roots.}
 \fromSlide{5}{
  \item {\tiny $\{x\in\R\st x^2-5x+6=0\}=\{2,3\}$}, but 
        {\tiny $\{x\in\R\st x^2-5x+7=0\}=\emptyset$}.}
 \fromSlide{6}{
  \item The general solution of the equations $x+y=2=y+z$ is \\
   $x=1+t$, $y=1-t$ and $z=1+t$ (where $t$ is arbitrary).}
 \fromSlide{7}{
  \item $\{(x,y,z)\in\R^3\st x+y=2=y+z\}=
         \{(1+t,1-t,1+t)\st t\in\R\}$.}
 \fromSlide{8}{
  \item The equation $x^2+y^2=1$ describes the unit circle, which is
   given parametrically by the equations $x=\cos(\tht)$ and
   $y=\sin(\tht)$, where $\tht$ is arbitrary.} 
 \fromSlide{9}{
  \item $\{(x,y)\in\R^2\st x^2+y^2=1\}=
         \{(\cos(\tht),\sin(\tht))\st\tht\in\R\}$.}
\end{itemize}
\end{slide}
}

\overlays{9}{%
\begin{slide}{Boolean operations}
\begin{itemize}
 \fromSlide{2}{
  \item The \EMPH{intersection} of $A$ and $B$ is\\
   $\RED{A\cap B}=\{x\st x\in A \text{ and also } x\in B\}$}
 \fromSlide{3}{
  \item The \EMPH{union} of $A$ and $B$ is\\
   $\RED{A\cup B}=\{x\st x\in A \text{ or } x\in B \text{ (or both) }\}$}
 \fromSlide{4}{
  \psset{unit=.8cm}
  \psset{dotsize=0.1pt 0}
  \item
   \fromSlide{4}{\begin{pspicture}[1](-1.2,-2)(1.2,1.2)
    \psframe*[linecolor=blue](-1,-1)(1,1)
    \psframe*[linecolor=white](-0.5,-0.5)(0.5,0.5)
    \psframe[linestyle=dashed,dash=2pt 2pt,linewidth=0.1pt](-0.8,-0.2)(0.8,0.2)
    \rput(0,-1.4){$\BLUE{A}$}
   \end{pspicture}}
   \fromSlide{5}{\begin{pspicture}[1](-1.2,-2)(1.2,1.2)
    \psframe*[linecolor=OliveGreen](-0.8,-0.2)(0.8,0.2)
    \psframe[linestyle=dashed,dash=2pt 2pt,linewidth=0.1pt](-1,-1)(1,1)
    \psframe[linestyle=dashed,dash=2pt 2pt,linewidth=0.1pt](-0.5,-0.5)(0.5,0.5)
    \rput(0,-1.4){$\OLIVEGREEN{B}$}
   \end{pspicture}}
   \fromSlide{6}{\begin{pspicture}[1](-1.2,-2)(1.2,1.2)
    \psframe*[linecolor=Red](-0.8,-0.2)(-0.5,0.2)
    \psframe*[linecolor=Red](0.5,-0.2)(0.8,0.2)
    \psframe[linestyle=dashed,dash=2pt 2pt,linewidth=0.1pt](-0.8,-0.2)(0.8,0.2)
    \psframe[linestyle=dashed,dash=2pt 2pt,linewidth=0.1pt](-1,-1)(1,1)
    \psframe[linestyle=dashed,dash=2pt 2pt,linewidth=0.1pt](-0.5,-0.5)(0.5,0.5)
    \rput(0,-1.4){$\RED{A\cap B}$}
   \end{pspicture}}
   \fromSlide{7}{\begin{pspicture}[1](-1.2,-2)(1.2,1.2)
    \psframe*[linecolor=Purple](-1,-1)(1,1)
    \psframe*[linecolor=white](-0.5,0.2)(0.5,0.5)
    \psframe*[linecolor=white](-0.5,-0.5)(0.5,-0.2)
    \rput(0,-1.4){$\PURPLE{A\cup B}$}
   \end{pspicture}}
 }
 \fromSlide{8}{
  \item The \EMPH{relative complement} of $B$ in $A$ is\\
   $\RED{A\sm B}=\{x\in A \st x\not\in B\}$}
 \fromSlide{9}{
  \item Often all sets under consideration are subsets of some fixed
   set $E$ (typically $E=\R$, $\R^2$ or $\R^3$).  We then write
   $\RED{A^c}$ for $E\sm A$, and call this the \EMPH{complement} of
   $A$.  Note that $A^{cc}=A$.}
\end{itemize} 
\end{slide}
}

\overlays{8}{%
\begin{slide}{De Morgan's laws}
\[ (A\cap B)^c = A^c\cup B^c \hspace{5em} (A\cup B)^c = A^c\cap B^c \]
\[ \begin{array}{ccc}
 \fromSlide{2}{
  \begin{pspicture}(-1.3,-1.3)(1.3,1.3)
   \psframe*[linecolor=Red](-1,-1)(1,0)
   \psframe[linestyle=dashed,dash=2pt 2pt,linewidth=0.1pt](-1,-1)( 0, 0)
   \psframe[linestyle=dashed,dash=2pt 2pt,linewidth=0.1pt]( 0,-1)( 1, 0)
   \psframe[linestyle=dashed,dash=2pt 2pt,linewidth=0.1pt](-1, 0)( 0, 1)
   \psframe[linestyle=dashed,dash=2pt 2pt,linewidth=0.1pt]( 0, 0)( 1, 1)
  \end{pspicture}} &
 \onlySlide*{6}{
  \begin{pspicture}(-1.3,-1.3)(1.3,1.3)
   \psframe*[linecolor=OliveGreen](-1,-1)(0,0)
   \psframe[linestyle=dashed,dash=2pt 2pt,linewidth=0.1pt](-1,-1)( 0, 0)
   \psframe[linestyle=dashed,dash=2pt 2pt,linewidth=0.1pt]( 0,-1)( 1, 0)
   \psframe[linestyle=dashed,dash=2pt 2pt,linewidth=0.1pt](-1, 0)( 0, 1)
   \psframe[linestyle=dashed,dash=2pt 2pt,linewidth=0.1pt]( 0, 0)( 1, 1)
  \end{pspicture}}
 \onlySlide*{7}{
  \begin{pspicture}(-1.3,-1.3)(1.3,1.3)
   \psframe*[linecolor=OliveGreen](-1,-1)(0,0)
   \psframe[linestyle=dashed,dash=2pt 2pt,linewidth=0.1pt](-1,-1)( 0, 0)
   \psframe[linestyle=dashed,dash=2pt 2pt,linewidth=0.1pt]( 0,-1)( 1, 0)
   \psframe[linestyle=dashed,dash=2pt 2pt,linewidth=0.1pt](-1, 0)( 0, 1)
   \psframe[linestyle=dashed,dash=2pt 2pt,linewidth=0.1pt]( 0, 0)( 1, 1)
  \end{pspicture}}
 \fromSlide*{8}{
  \begin{pspicture}(-1.3,-1.3)(1.3,1.3)
   \psframe*[linecolor=OliveGreen](-1,-1)(0,1)
   \psframe*[linecolor=OliveGreen](-1,-1)(1,0)
   \psframe[linestyle=dashed,dash=2pt 2pt,linewidth=0.1pt](-1,-1)( 0, 0)
   \psframe[linestyle=dashed,dash=2pt 2pt,linewidth=0.1pt]( 0,-1)( 1, 0)
   \psframe[linestyle=dashed,dash=2pt 2pt,linewidth=0.1pt](-1, 0)( 0, 1)
   \psframe[linestyle=dashed,dash=2pt 2pt,linewidth=0.1pt]( 0, 0)( 1, 1)
  \end{pspicture}}
 &
 \fromSlide{3}{
  \begin{pspicture}(-1.3,-1.3)(1.3,1.3)
   \psframe*[linecolor=Blue](-1,-1)(0,1)
   \psframe[linestyle=dashed,dash=2pt 2pt,linewidth=0.1pt](-1,-1)( 0, 0)
   \psframe[linestyle=dashed,dash=2pt 2pt,linewidth=0.1pt]( 0,-1)( 1, 0)
   \psframe[linestyle=dashed,dash=2pt 2pt,linewidth=0.1pt](-1, 0)( 0, 1)
   \psframe[linestyle=dashed,dash=2pt 2pt,linewidth=0.1pt]( 0, 0)( 1, 1)
  \end{pspicture}} \\
 \fromSlide{2}{\RED{A}} &
 \onlySlide*{6}{\OLIVEGREEN{A\cap B}}
 \onlySlide*{7}{\OLIVEGREEN{A\cap B}}
 \fromSlide*{8}{\OLIVEGREEN{A\cup B}} &
 \fromSlide{3}{\BLUE{B}} \\
 \fromSlide{4}{
  \begin{pspicture}(-1.3,-1.3)(1.3,1.3)
   \psframe*[linecolor=Emerald](-1,0)(1,1)
   \psframe[linestyle=dashed,dash=2pt 2pt,linewidth=0.1pt](-1,-1)( 0, 0)
   \psframe[linestyle=dashed,dash=2pt 2pt,linewidth=0.1pt]( 0,-1)( 1, 0)
   \psframe[linestyle=dashed,dash=2pt 2pt,linewidth=0.1pt](-1, 0)( 0, 1)
   \psframe[linestyle=dashed,dash=2pt 2pt,linewidth=0.1pt]( 0, 0)( 1, 1)
  \end{pspicture}} &
 \onlySlide*{7}{
  \begin{pspicture}(-1.3,-1.3)(1.3,1.3)
   \psframe*[linecolor=Orange](0,-1)(1,1)
   \psframe*[linecolor=Orange](-1,0)(1,1)
   \psframe[linestyle=dashed,dash=2pt 2pt,linewidth=0.1pt](-1,-1)( 0, 0)
   \psframe[linestyle=dashed,dash=2pt 2pt,linewidth=0.1pt]( 0,-1)( 1, 0)
   \psframe[linestyle=dashed,dash=2pt 2pt,linewidth=0.1pt](-1, 0)( 0, 1)
   \psframe[linestyle=dashed,dash=2pt 2pt,linewidth=0.1pt]( 0, 0)( 1, 1)
  \end{pspicture}}
 \fromSlide*{8}{
  \begin{pspicture}(-1.3,-1.3)(1.3,1.3)
   \psframe*[linecolor=Orange](0,0)(1,1)
   \psframe[linestyle=dashed,dash=2pt 2pt,linewidth=0.1pt](-1,-1)( 0, 0)
   \psframe[linestyle=dashed,dash=2pt 2pt,linewidth=0.1pt]( 0,-1)( 1, 0)
   \psframe[linestyle=dashed,dash=2pt 2pt,linewidth=0.1pt](-1, 0)( 0, 1)
   \psframe[linestyle=dashed,dash=2pt 2pt,linewidth=0.1pt]( 0, 0)( 1, 1)
  \end{pspicture}}
 &
 \fromSlide{5}{
  \begin{pspicture}(-1.3,-1.3)(1.3,1.3)
   \psframe*[linecolor=Magenta](0,-1)(1,1)
   \psframe[linestyle=dashed,dash=2pt 2pt,linewidth=0.1pt](-1,-1)( 0, 0)
   \psframe[linestyle=dashed,dash=2pt 2pt,linewidth=0.1pt]( 0,-1)( 1, 0)
   \psframe[linestyle=dashed,dash=2pt 2pt,linewidth=0.1pt](-1, 0)( 0, 1)
   \psframe[linestyle=dashed,dash=2pt 2pt,linewidth=0.1pt]( 0, 0)( 1, 1)
  \end{pspicture}} \\
 \fromSlide{4}{\EMERALD{A^c}} &
 \untilSlide*{6}{\hphantom{A^c\cap B^c=(A\cup B)^c}}
 \onlySlide*{7}{\ORANGE{A^c\cup B^c=(A\cap B)^c}}
 \fromSlide*{8}{\ORANGE{A^c\cap B^c=(A\cup B)^c}} &
 \fromSlide{5}{\MAGENTA{B^c}}
\end{array} \]
\end{slide}
}

\overlays{4}{%
\begin{slide}{Inequalities}
\begin{itemize}
 \fromSlide{2}{
  \item Sets are often defined by inequalities 
   (intervals are the most basic example).}
 \fromSlide{3}{
  \item
   \psset{unit=.6cm}
   \fromSlide{3}{\begin{pspicture}[1](-1,-2)(4,3.5)
    \psaxes[labels=none,linewidth=0.1pt,ticksize=1pt]{->}(0,0)(0,0)(3.5,3.5)
    \pscustom[linestyle=none,fillstyle=solid,fillcolor=red]{%
     \psplot{.3}{3.33}{1 x div} \psline(3.33,.3)(3.33,3.33)(.3,3.33)}
    \rput(2,-1.5){$\scriptstyle\{(x,y)\st x,y>0\;,\;xy>1\}$}
   \end{pspicture}}
   \hspace{2em}
   \fromSlide{4}{\begin{pspicture}[1](-3,-2)(4,3.5)
    \psaxes[labels=none,linewidth=0.1pt,ticksize=1pt]{->}(0,2.5)(-2.5,1.5)(3.5,3.5)
    \psplot[linewidth=0.1pt]{-2}{3}{x 180 mul sin 2.5 add}
    \psline[linestyle=dotted,linewidth=0.1pt](-2.6,0)(3.6,0)
    \psline[linewidth=3pt,linecolor=red](-2,0)(-1,0)
    \psline[linewidth=3pt,linecolor=red]( 0,0)( 1,0)
    \psline[linewidth=3pt,linecolor=red]( 2,0)( 3,0)
    \psline[linewidth=0.1pt](-2,-.05)(-2,.05)
    \psline[linewidth=0.1pt](-1,-.05)(-1,.05)
    \psline[linewidth=0.1pt]( 0,-.05)( 0,.05)
    \psline[linewidth=0.1pt]( 1,-.05)( 1,.05)
    \psline[linewidth=0.1pt]( 2,-.05)( 2,.05)
    \psline[linewidth=0.1pt]( 3,-.05)( 3,.05)
    \rput(-2,-.4){$\scriptstyle -2\pi$}
    \rput(-1, .4){$\scriptstyle -\pi$}
    \rput( 0,-.4){$\scriptstyle 0$}
    \rput( 1, .4){$\scriptstyle \pi$}
    \rput( 2,-.4){$\scriptstyle 2\pi$}
    \rput( 3, .4){$\scriptstyle 3\pi$}
    \rput(0.5,-1.5){$\scriptstyle\{x\in\R\st\sin(x)\geq 0\}$}
   \end{pspicture}}}
\end{itemize}
\end{slide}
}

\overlays{7}{%
\begin{slide}{Terminology}
\begin{itemize}
 \fromSlide{2}{
  \item A real number $x$ is \DEFN{strictly positive} if $x>0$}
 \fromSlide{3}{
  \item A real number $x$ is \DEFN{weakly positive} or
   \DEFN{nonnegative} if $x\geq 0$.} 
 \fromSlide{4}{
  \item The \DEFN{absolute value} of a real number $x$ is 
   \[ \PURPLE{\text{abs}(x)} = \PURPLE{|x|} = 
       \text{the distance between $0$ and $x$},
   \]
   so if $t\geq 0$ then $|t|=|-t|=t$.}
 \fromSlide{5}{
  \item
   \begin{itemize}
    \fromSlide{5}{
     \item If $x$ and $y$ have the same sign, then $|x+y|=|x|+|y|$\\
      (eg $|(-4)+(-5)|=|-9|=9=|-4|+|-5|$).}
    \fromSlide{6}{
     \item If $x$ and $y$ have opposite signs then they partially
      cancel, and $|x+y|<|x|+|y|$\\
      (eg $|2+(-5)|=|-3|=3<7=|2|+|-5|$).}
    \fromSlide{7}{
     \item In all cases, 
      \psframebox[framearc=.3,linecolor=magenta]{$|x+y|\leq|x|+|y|$} \\
      This is called the \DEFN{Triangle Inequality}.}
   \end{itemize}}
\end{itemize}
\end{slide}
}


\overlays{10}{%
\begin{slide}{Algebraic rules}
\begin{itemize}
 \fromSlide{2}{
  \item[\pscirclebox{\tiny 1}]
   For any real number $x$ we have $x^2\geq 0$.}
 \fromSlide{3}{
  \item[\pscirclebox{\tiny 2}]
   For any $c$, the inequalities $x<y$ and $x+c<y+c$ are equivalent.}
 \fromSlide{4}{
  \item[\pscirclebox{\tiny 3}]
   If $u<v$ and $x<y$ then $u+x<v+y$.}
 \fromSlide{5}{
  \item[\pscirclebox{\tiny 4}]
   For any $\RED{c>0}$, the inequalities $x<y$ and $cx<cy$ are equivalent.}
 \fromSlide{6}{
  \item[\pscirclebox{\tiny 5}]
   If $\RED{c<0}$ then the inequality $x<y$ is equivalent to $cy<cx$ \\
   (\ORANGE{note order reversed}).  In particular, $x<y\;\;\iffa\;\;-y<-x$.}
\end{itemize}
\fromSlide{7}{
 Now suppose that $u$, $v$, $x$ and $y$ are all \RED{strictly positive}.
 \begin{itemize}
  \fromSlide{8}{
   \item[\pscirclebox{\tiny 6}] 
    If $u<v$ and $x<y$ then $ux<vy$.}
  \fromSlide{9}{
   \item[\pscirclebox{\tiny 7}] 
    If $x<y$ then $1/y<1/x$ (\ORANGE{note order reversed}).}
  \fromSlide{10}{
   \item[\pscirclebox{\tiny 8}] 
    If $a>0$ then the inequality $x<y$ is equivalent to $x^a<y^a$. \\
    In particular,
    $x<y \;\;\iffa\;\; x^2<y^2 \;\;\iffa \sqrt{x}<\sqrt{y}$.}
 \end{itemize}
}
\end{slide}
}

\overlays{11}{%
\begin{slide}{Proving inclusions}
\begin{itemize}
\fromSlide{2}{
 \item To show $A\sse B$, check that 
  \RED{every element of $A$ is an element of $B$}.}
\fromSlide{3}{
 \item If $x$ is any real number, then $1+x^2$ is a real number,
  and $1+x^2>1/2$. \\
  \onlySlide*{4}{
   Thus $\{1+x^2\st x\in\R\}\sse(1/2,\infty)$.}
  \fromSlide*{5}{
   Thus $\{1+x^2\st x\in\R\}\OLIVEGREEN{=[1,\infty)}\sse(1/2,\infty)$.}}
\fromSlide{6}{
 \item If $n$ is any integer, then $n+\frac{1}{2}$ is a real number
  that is not an integer.  \\
  \fromSlide{7}{
   Thus $\{n+\frac{1}{2}\st n\in\Z\}\sse\R\sm\Z$.}
  \onlySlide*{8}{
   \psset{unit=.7cm}
   \begin{center}\begin{pspicture}[1](-8,-.5)(4,1.5)
    \rput(-7,1){$\scriptstyle\{n+\frac{1}{2}\st n\in\Z\}$}
    \psdots[linecolor=red,dotsize=2pt](-3.5,1)(-2.5,1)(-1.5,1)(-0.5,1)
    \psdots[linecolor=red,dotsize=2pt](+3.5,1)(+2.5,1)(+1.5,1)(+0.5,1)
    \rput(-7,0){$\scriptstyle \R\sm\Z$}
    \multips(-4,0)(1,0){8}{%
     \psline[linecolor=blue,linewidth=1pt,arrows=c-c](.1,0)(.9,0)}
   \end{pspicture}\end{center}}
}
\fromSlide{9}{
 \item If $x$ and $y$ are \EMPH{integers} and $x^2+y^2=1$ then either
  $x=0$ or $y=0$ (otherwise $x^2+y^2\geq 2$).  
  This means that $xy$ must be zero.  \\
  \fromSlide{10}{
   Thus $\{(x,y)\in\Z^2\st x^2+y^2=1\}\sse\{(x,y)\in\R^2\st xy=0\}$.}}
   \fromSlide{11}{
    \begin{center}
     \psset{unit=0.5cm}
     \begin{pspicture}[1](-2,-3)(2,2)
      \multips(-2,-2)(1,0){5}{%
       \psline[linestyle=dotted,linewidth=0.1pt](0,0)(0,4)}
      \multips(-2,-2)(0,1){5}{%
       \psline[linestyle=dotted,linewidth=0.1pt](0,0)(4,0)}
      \pscircle[linewidth=0.1pt,linestyle=dotted](0,0){1}
      \psdots[linecolor=red,dotsize=2pt](-1,0)(1,0)(0,-1)(0,1)
      \rput(0,-2.5){$\scriptstyle\{(x,y)\in\Z^2\st x^2+y^2=1\}$}
     \end{pspicture}
     \hspace{4em}
     \begin{pspicture}[1](-2,-2.5)(2,2)
      \multips(-2,-2)(1,0){5}{%
       \psline[linestyle=dotted,linewidth=0.1pt](0,0)(0,4)}
      \multips(-2,-2)(0,1){5}{%
       \psline[linestyle=dotted,linewidth=0.1pt](0,0)(4,0)}
      \psline[linecolor=blue,linewidth=1pt](-2,0)(2,0)
      \psline[linecolor=blue,linewidth=1pt](0,-2)(0,2)
      \rput(0,-2.5){$\scriptstyle\{(x,y)\in\R^2\st xy=0\}$}
     \end{pspicture}
    \end{center}
   }
\end{itemize}
\end{slide}
}

\overlays{12}{%
\begin{slide}{Proving equality}
\begin{itemize}
\fromSlide{2}{
 \item To show that $A=B$, you must check that \\
  \fromSlide{3}{
   \RED{every element of $A$ is in $B$,
        \underline{\bf AND}
        every element of $A$ is in $B$}} \\
  \fromSlide{4}{
   or equivalently $A\sse B$ and $B\sse A$.}}
\fromSlide{5}{
 \item Consider $A=\{x+y\st x,y\in[0,1]\}$ and $B=[0,2]$.
  \fromSlide{6}{
   \begin{itemize}
    \fromSlide{6}{
     \item If $z\in A$ then $z=x+y$ for some $x,y$ with
      $0\leq x,y\leq 1$,\\ so $0\leq z\leq 2$, so $z\in B$.
      \PURPLE{This means that $A\sse B$.}}
    \fromSlide{7}{
     \item If $z\in B$, we can take $x=y=z/2$; then $x,y\in[0,1]$ and
      $z=x+y$, so $z\in A$.
      \PURPLE{This means that $B\sse A$, so $A=B$. }}
   \end{itemize}}}
\fromSlide{8}{
 \item Alternatively: check that
  \RED{every element in $A$ is in $B$, \underline{\bf AND} \\
       every element not in $A$ is not in $B$}.}
\fromSlide{9}{
 \item Consider $A=\{-1,0,1\}$ and $B=\{x\in\R\st x^3=x\}$.
  \fromSlide{10}{
   \begin{itemize}
    \fromSlide{10}{
     \item $(-1)^3=-1$, $0^3=0$ and $1^3=1$, so
           every element of $A$ is in $B$.}
    \fromSlide{11}{
     \item If $x\not\in A$, then $x+1$, $x$ and $x-1$ are nonzero, so
      $x^3-x=(x+1)x(x-1)\neq 0$, so $x^3\neq x$, so $x\not\in B$}
    \fromSlide{12}{
     \item Thus $A=B$.}
   \end{itemize}
  }  
}
\end{itemize}
\end{slide}
}

\end{document}

%%%%%%%%%%%%%%%%%%%%%%%%%%%%%%%%%%%%%%%%%%%%%%%%%%%%%%%%%%%%%%%%%%%%%%
%%%%%%%%%%%%%%%%%%%%%%%%%%%%%%%%%%%%%%%%%%%%%%%%%%%%%%%%%%%%%%%%%%%%%%
%%%%%%%%%%%%%%%%%%%%%%%%%%%%%%%%%%%%%%%%%%%%%%%%%%%%%%%%%%%%%%%%%%%%%%


