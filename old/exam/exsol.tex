\documentclass{amsart}
\usepackage{a4wide}
\usepackage[pdftex]{graphicx}


\newcommand{\half}      {{\textstyle\frac{1}{2}}}
\newcommand{\om}{\omega}
\newcommand{\xra}{\xrightarrow}
\renewcommand{\:}       {\colon}

\newcommand{\seen}      {}
\newcommand{\unseen}    {}
\newcommand{\simseen}   {}
\newcommand{\bwk}       {}
\newcommand{\mkrem}[1]  {}

\newcommand{\mks}[1]    {}
\newcommand{\mk}        {}

\renewcommand{\theenumii}{\roman{enumii}}
\renewcommand{\theenumiii}{\alph{enumiii}}

\begin{document}

\begin{center}
{\Large Pure Mathematics Core --- Exam solutions}
\end{center}

\renewcommand{\theenumi}{A\arabic{enumi}}

\begin{enumerate}
 \item % pfrac 3
  Consider the rational function $f(x)=\frac{x^2+x+1}{(x+1)^2}$.  On
  the bottom we have $(x+1)^2$, indicating a pole of order $2$ at
  $x=-1$.  We therefore need multiples of $(x+1)^{-1}$ and
  $(x+1)^{-2}$ in the partial fraction form.  The top and bottom of
  $f(x)$ are both quadratics, so they both have degree $2$.  When the
  top and the bottom have the same degree, we need to add in an extra
  constant.   The general form is therefore 
  \[ \frac{x^2+x+1}{(x+1)^2} =
      A + \frac{B}{x+1} + \frac{C}{(x+1)^2}.\mks{2}
  \]
  The rules for finding the general form are explained in Section 2.4
  of the booklet, and further examples are given in the solutions to
  the AiM test on partial fractions and the solutions to past exams.
  Some mistakes to avoid:
  \begin{itemize}
   \item You should never have the same term repeated twice but with
    different arbitrary coefficients.  Things like $A/(x+1)+B/(x+1)$
    are never correct.
   \item In some cases one has terms like $Ax/(x^2+x+1)$ or
    $Cx/(x^2+1)^3$ with an $x$ in the numerator.  These only occur
    when the denominator is a quadratic with no real roots, or a power
    of such a quadratic.  If the denominator is of the form $(x-a)$ or
    $(x-a)^n$ then the numerator should just be an arbitrary constant
    with no $x$.  Terms like $A/(x+1)$ and $B/(x-3)^2$ are OK, but
    terms like $Cx/(x+1)$ or $(Dx+E)/(x-2)^3$ are not.
  \end{itemize}

  Multiplying the above equation by $(x+1)^2$ gives
  \begin{align*}
   x^2+x+1 &= A(x+1)^2 + B(x+1) + C  = Ax^2 +2Ax + A + Bx + B + C \\
           &= Ax^2 + (2A+B)x + (A+B+C), \mk
  \end{align*}
  so $A=1$ and $2A+B=1$ and $A+B+C=1$, which gives $B=-1$ and $C=1$. \mk
  This means that
  \begin{align*}
   \frac{x^2+x+1}{(x+1)^2} &= 1 - \frac{1}{x+1} + \frac{1}{(x+1)^2} \\
   \int \frac{x^2+x+1}{(x+1)^2}\,dx &= x - \log(x+1) - \frac{1}{(x+1)}.\mks{2}
  \end{align*}

 \item % inverse 1
  If $x=f(y)=1/(1-e^{-y})$ \mk then $1/x=1-e^{-y}$ \mk, so
  $e^{-y}=1-1/x=(x-1)/x$, so $e^y=x/(x-1)$ \mk, so
  $f^{-1}(x)=y=\log(x/(x-1))$ \mk. 

  Some mistakes to avoid:
  \begin{itemize}
   \item $\log(1-e^{-y})$ is \emph{not} the same as
    $\log(1)-\log(e^{-y})$.  Similarly, $\log(y-ye^{-x})$ is not the
    same as $\log(y)-\log(ye^{-x})$.  More generally, $\log(a+b)$ is
    almost never the same as $\log(a)+\log(b)$, so taking logs of sums
    is often not helpful.  Yous hould manipulate your equations
    algebraically until you have $e^y$ on its own first, and then take
    logs. 
   \item Again, $\log(-e^{-y})$ is not the same as $\log(e^y)=y$; the
    signs do not ``cancel out''.  In fact, $-e^{-y}$ is always
    negative (for real $y$) and so $\log(-e^{-y})$ is not even
    defined.  If we allow complex numbers then 
    \[ \log(-e^{-y}) = \log((-1).e^{-y}) = 
        \log(-1) + \log(e^{-y}) = i\pi - y.
    \]
  \end{itemize}

 \item % composite 2
  \[ (\exp\circ f\circ\log)(x) = \exp(f(\log(x))) \mk = 
      \exp(2\log(x)+2) \mk = (e^{\log(x)})^2 e^2 \mk = e^2x^2 \mk.
  \]
  Here $f(t)=2t+2$, so to apply $f$ to something we double it and add
  $2$.  We start with $x$ and work from right to left.  We apply
  $\log$ to get $\log(x)$, then apply $f$ to get $2\log(x)+2$, then
  apply $\exp$ to get $\exp(2\log(x)+2)$.  
  \begin{itemize}
   \item The expression $\log(2x+2)$ is $\log(f(x))$ or
    $(\log\circ f)(x)$, not $(f\circ\log)(x)$.  Thus, $\log(2x+2)$
    should not appear in your calculation at all.  Similarly, the
    expressions $(f\circ\exp)(x)=2e^x+2$ and
    $\exp(\log(2x+2))=(\exp\circ\log\circ f)(x)$ should not appear.
   \item Even if you think that $\log(2x+2)$ is relevant, you should
    not write equations like $\log(x)=\log(2x+2)$ to indicate this.
    The symbol ``='' means equality, and $\log(x)$ is obviously not
    the same function as $\log(2x+2)$.
   \item The expression $(2x+2)\log(x)$ is $f(x)\log(x)$, not
    $f(\log(x))$.  Thus, $(2x+2)\log(x)$ should not appear in your
    calculation at all.
   \item The rule is that $\exp(a+b)=\exp(a)\exp(b)$, not
    $\exp(a)+\exp(b)$ or $\exp(a)+b$.  Thus $\exp(2\log(x)+2)$ is
    $\exp(2\log(x)).\exp(2)$ or $e^{2\log(x)}.e^2$, which simplifies
    further to $x^2e^2$.  We do not have
    $\exp(2\log(x)+2)=\exp(2\log(x))+\exp(2)$ (which would simplify to
    $x^2+e^2$) or $\exp(2\log(x)+2)=\exp(2\log(x))+2$ (which would
    simplify to $x^2+2$).
   \item The rule is that $\exp(2a)=\exp(a)^2$, not
    $\exp(2a)=2\exp(a)$.  Thus $\exp(2\log(x))$ is $\exp(\log(x))^2$
    (which simplifies to $x^2$) not $2\exp(\log(x))$ (which simplifies
    to $2x$).
  \end{itemize}

 \item % log 1
  We must find $\log_{16}(1/2)$, which is the number $t$ such that
  $16^t=1/2$.  We note that $16=2^4$ \mk, so $2=16^{1/4}$, so
  $1/2=16^{-1/4}$, so $\log_{16}(1/2)=-1/4$ \mk.

 \item % trig value
  Note that $\sin(x)$ repeats with period $2\pi$, so
  \[ \sin(-7\pi/3) = \sin(-7\pi/3+2\pi) = \sin(-\pi/3) \mk
      = -\sin(\pi/3) = -\sqrt{3}/2 \mk.
  \]

 \item % identity 1
  Put $u=e^x$.  Then 
  \[ \tanh(x) = \frac{\sinh(x)}{\cosh(x)} = 
      \frac{(u-u^{-1})/2}{(u+u^{-1})/2} = \frac{u-u^{-1}}{u+u^{-1}}, \mks{2}
  \]
  so
  \begin{align*}
   1+\tanh(x)^2 &= 
    1 + \left(\frac{u-u^{-1}}{u+u^{-1}}\right)^2 = 
    1 + \frac{u^2-2+u^{-2}}{u^2+2+u^{-2}} \\
    &= \frac{(u^2+2+u^{-2})+(u^2-2+u^{-2})}{u^2+2+u^{-2}} = 
       \frac{2u^2+2u^{-2}}{u^2+2+u^{-2}} \mks{2} \\
   1-\tanh(x)^2 &= 
    1 - \left(\frac{u-u^{-1}}{u+u^{-1}}\right)^2 = 
    1 - \frac{u^2-2+u^{-2}}{u^2+2+u^{-2}} \\
    &= \frac{(u^2+2+u^{-2})-(u^2-2+u^{-2})}{u^2+2+u^{-2}} = 
       \frac{4}{u^2+2+u^{-2}} \mk \\
\intertext{ so }
   \frac{1+\tanh(x)^2}{1-\tanh(x)^2} &= 
    \frac{2u^2+2u^{-2}}{4} = \frac{e^{2x}+e^{-2x}}{2} = \cosh(2x) \mk.
  \end{align*}
  Alternatively,
  \[ \frac{1+\tanh(x)^2}{1-\tanh(x)^2} = 
      \frac{1+\frac{\sinh(x)^2}{\cosh(x)^2}}
           {1-\frac{\sinh(x)^2}{\cosh(x)^2}} = 
      \frac{\cosh(x)^2+\sinh(x)^2}
           {\cosh(x)^2-\sinh(x)^2} = 
      \frac{\cosh(2x)}{1}.
  \]
  This is shorter, but you need to remember the identities
  $\cosh(2x)=\cosh(x)^2+\sinh(x)^2$ and $\cosh(x)^2-\sinh(x)^2=1$, and
  you need to get the signs right.  The first method is more
  systematic and will work for all possible identities.

 \item % diff rat 2
  Put $u=x^n+a$ and $y=f(x)=u^m$.  Then $du/dx=nx^{n-1}$ and
  $dy/du=mu^{m-1}$, so 
  \[ f'(x) = \frac{dy}{dx} = mu^{m-1}\frac{du}{dx} = 
      mn(x^n+a)^{m-1}x^{n-1}. \mks{2}
  \]
 
 \item % diff log 
  Put $u=1+x+x^2+x^3$ and $y=\log(u)$, so $dy/du=1/u$ and
  \[ \frac{dy}{dx} = 
     \frac{du}{dx}\,\frac{dy}{du} = 
     (1+2x+3x^2)u^{-1} = \frac{1+2x+3x^2}{1+x+x^2+x^3} \mks{2}.
  \]
  The most common wrong answers were $(1+x+x^2+x^3)^{-1}$ and
  $(1+2x+3x^2)\log(1+x+x^2+x^3).x^{-1}$.  Make sure you understand why
  these are not correct.

 \item % diff quotient
  The quotient rule gives
  \begin{align*}
   \frac{d}{dx}\left(\frac{x}{\log(x)}\right) 
    &= \frac{1.\log(x) - x.\log'(x)}{\log(x)^2} \mk 
     = \frac{\log(x) - x.x^{-1}}{\log(x)^2}  \mk \\
    &= \frac{1}{\log(x)} - \frac{1}{\log(x)^2} \mk.
  \end{align*}
  Note here that $\log(x)^2$ is not the same as $\log(x^2)$.  In fact
  we have $\log(x^2)=2\log(x)$, but $\log(x)^2\neq 2\log(x)$.

 \item % diff mobius
  \begin{align*}
   \frac{d}{dx}\left(\frac{3x+2}{4x+3}\right) 
    &= \frac{3(4x+3)-4(3x+2)}{(4x+3)^2} \mk \\
    &= \frac{12x+9-12x-8}{(4x+3)^2} = (4x+3)^{-2} \mk   
  \end{align*}

 \item % diff packet
  First put $u=-(x-a)^2/b$, so $du/dx=-2(x-a)/b$.  Then put
  $v=\exp(u)=e^{-(x-a)^2/b}$, so the chain rule gives
  \[ \frac{dv}{dx} = -2(x-a)b^{-1}e^{-(x-a)^2/b}. \mks{2} \]
  Now put $w=\sin(\omega x)$, so $dw/dx=\om\cos(\om x)$.  
  Finally, put $y=vw=e^{-(x-a)^2/b}\sin(\om x)$ and apply the product
  rule: 
  \begin{align*}
   \frac{dy}{dx} &= \frac{dv}{dx}w + v\frac{dw}{dx} \\
   &= -2(x-a)b^{-1}e^{-(x-a)^2/b} \sin(\om x) + 
      e^{-(x-a)^2/b} \om\cos(\om x) \\
   &= e^{-(x-a)^2/b}(\om\cos(\om x) - 2(x-a)b^{-1}\sin(\om x)) \mks{2}.
  \end{align*}
  With practice you can leave out some of these steps, but it is
  always safest to write them all out carefully.

 \item % int poly exp
  We know that
  \[ \int x^2 e^x\, dx = (ax^2+bx+c) e^x \]
  for some constants $a$, $b$ and $c$ \mks{2}.  To find these, we
  differentiate to get 
  \begin{align*}
   x^2 e^x &= \frac{d}{dx}((ax^2+bx+c) e^x) 
    = (2ax+b)e^x + (ax^2+bx+c) e^x \mk \\
    &= (ax^2+(2a+b)x+(b+c))e^x.
  \end{align*}
  We equate coefficients to see that $a=1$ and $2a+b=b+c=0$ \mk, which
  gives $b=-2$ and $c=2$.  We conclude that
  \[ \int x^2 e^x\, dx = (x^2-2x+2) e^x. \mk \]

  Alternatively, we can integrate twice by parts.  For the first step,
  put $u=x^2$ (so $du/dx=2x$) and $dv/dx=e^x$ (so $v=e^x$ as well).
  We then have
  \[ \int x^2e^x\,dx = \int u\frac{dv}{dx}\,dx = 
      uv - \int \frac{du}{dx} v\,dx = 
      x^2e^x - 2\int x\,e^x\,dx.
  \]
  For the second step, put $u=x$ (so $du/dx=1$) and $dv/dx=e^x$ (so
  $v=e^x$ as well).  We then have
  \[ \int x\,e^x\,dx = \int u\frac{dv}{dx}\,dx = 
      uv - \int \frac{du}{dx} v\,dx = 
      x e^x - \int e^x\,dx = x e^x - e^x.
  \]
  Putting this together, we get
  \[ \int x^2e^x\,dx = x^2e^x - 2(xe^x - e^x) = (x^2-2x+2)e^x. \] 

 \item % int exp trig 2
  We know that 
  \[ \int e^{3x}\sin(4x)\,dx = e^{3x}(A\cos(4x) + B\sin(4x)) \]
  for some $A$ and $B$ \mks{2}.  To find these, we differentiate and
  equate coefficients: 
  \begin{align*}
   e^{3x}\sin(4x) &=
    \frac{d}{dx}\left(e^{3x}(A\cos(4x) + B\sin(4x))\right) \\
    &= 3e^{3x}(A\cos(4x) + B\sin(4x)) +
        e^{3x}(-4A\sin(4x) + 4B\cos(4x)) \\
    &= e^{3x}((3A+4B)\cos(4x) + (3B-4A)\sin(4x)) \mk,
  \end{align*}
  so $3A+4B=0$ and $3B-4A=1$ \mk.  This gives $A=-4B/3$ so
  $1=3B-4A=3B+16B/3=25B/3$, so $B=3/25$, so $A=-4B/3=-4/25$.  The
  conclusion is that
  \[ \int e^{3x}\sin(4x)\,dx = e^{3x}(3\sin(4x) - 4\cos(4x))/25. \mk \]
  Alternatively, we can integrate twice by parts.  Put
  $I=\int e^{3x}\sin(4x)\,dx$ and $J=\int e^{3x}\cos(4x)\,dx$.  Put
  $du/dx=e^{3x}$ (so $u=e^{3x}/3$) and $v=\sin(4x)$ (so
  $dv/dx=4\cos(4x)$).  This gives
  \[ I = \int \frac{du}{dx}v \,dx = 
      uv - \int u\frac{dv}{dx}\,dx = 
      e^{3x}\sin(4x)/3 - \int 4e^{3x}\cos(4x)/3\, dx = 
      e^{3x}\sin(4x)/3 - 4J/3.
  \]
  Now put $w=\cos(4x)$, so $dw/dx=-4\sin(4x)$.  This gives
  \[ J = \int \frac{du}{dx}w \,dx = 
      uw - \int u\frac{dw}{dx}\,dx = 
      e^{3x}\cos(4x)/3 - \int 4e^{3x}(-\sin(4x))/3\, dx = 
      e^{3x}\cos(4x)/3 + 4I/3.
  \]
  Putting this together gives
  \begin{align*}
   I &= \frac{1}{3}e^{3x}\sin(4x) -
        \frac{4}{3}\left(\frac{1}{3}e^{3x}\cos(4x) + \frac{4}{3}I\right) \\
     &= e^{3x}\left(\frac{1}{3}\sin(4x) - \frac{4}{9}\cos(4x)\right) - 
        \frac{16}{9}I.
  \end{align*}
  We can rearrange this to get
  \begin{align*}
   \frac{25}{9}I &=
    e^{3x}\left(\frac{1}{3}\sin(4x) - \frac{4}{9}\cos(4x)\right) \\
   I &= \frac{9}{25}e^{3x}
         \left(\frac{1}{3}\sin(4x) - \frac{4}{9}\cos(4x)\right) \\
     &= e^{3x}(3\sin(4x) - 4\cos(4x))/25.
  \end{align*}

 \item % linear 1
  The matrix of coefficients is
  \[ \left[\begin{array}{cccc}
      1 & 1 & 1 & 1 \\ 1 & 1 & -1 & -1 \\ 1 & -1 & 1 & -1
     \end{array}\right] \mk
  \]
  This can be row-reduced as follows:
  \[ \left[\begin{array}{cccc}
      1 & 1 & 1 & 1 \\ 1 & 1 & -1 & -1 \\ 1 & -1 & 1 & -1
     \end{array}\right] \xra{}
     \left[\begin{array}{cccc}
      1 & 1 & 1 & 1 \\ 0 & 0 & -2 & -2 \\ 0 & -2 & 0 & -2
     \end{array}\right] \xra{}
     \left[\begin{array}{cccc}
      1 & 1 & 1 & 1 \\ 0 & 0 & 1 & 1 \\ 0 & 1 & 0 & 1
     \end{array}\right] \xra{}
  \]
  \[ \left[\begin{array}{cccc}
      1 & 1 & 1 & 1 \\ 0 & 1 & 0 & 1 \\ 0 & 0 & 1 & 1
     \end{array}\right] \xra{}
     \left[\begin{array}{cccc}
      1 & 0 & 1 & 0 \\ 0 & 1 & 0 & 1 \\ 0 & 0 & 1 & 1
     \end{array}\right] \xra{}
     \left[\begin{array}{cccc}
      1 & 0 & 0 & -1 \\ 0 & 1 & 0 & 1 \\ 0 & 0 & 1 & 1
     \end{array}\right] \mks{2M 2A}
  \]
  There is no pivot in the last column, so the variable $z$ is
  independent \mk(you need to say this explicitly).  The final matrix
  corresponds to the equations $w-z=x+z=y+z=0$, so
  $(w,x,y,z)=(z,-z,-z,z)$ \mk. 

 \item % inverse matrix
  We write down the augmented matrix and row-reduce it as follows:
  \[ \left[\begin{array}{ccc|ccc}
      1 & a & b & 1 & 0 & 0 \\ 
      0 & 1 & c & 0 & 1 & 0 \\ 
      0 & 0 & 1 & 0 & 0 & 1 
     \end{array}\right] \mk \xra{}
     \left[\begin{array}{ccc|ccc}
      1 & 0 & b-ac & 1 & -a & 0 \\ 
      0 & 1 & c & 0 & 1 & 0 \\ 
      0 & 0 & 1 & 0 & 0 & 1 
     \end{array}\right] \xra{}
  \] \[
     \left[\begin{array}{ccc|ccc}
      1 & 0 & 0 & 1 & -a & ac-b \\ 
      0 & 1 & c & 0 & 1 & 0 \\ 
      0 & 0 & 1 & 0 & 0 & 1 
     \end{array}\right] \xra{}
     \left[\begin{array}{ccc|ccc}
      1 & 0 & 0 & 1 & -a & ac-b \\ 
      0 & 1 & 0 & 0 & 1 & -c \\ 
      0 & 0 & 1 & 0 & 0 & 1 
     \end{array}\right] \mks{2M 2A}
  \]
  At the final stage, the left hand block is the identity, so
  the right hand block is the inverse of the original matrix, ie
  \[ \left[\begin{array}{ccc}
      1 & a & b \\ 0 & 1 & c \\ 0 & 0 & 1 
     \end{array}\right]^{-1} = 
     \left[\begin{array}{ccc}
      1 & -a & ac-b \\ 0 & 1 & -c \\ 0 & 0 & 1 
     \end{array}\right] \mk.
  \]
  Alternatively, we can use the cofactor method.  The determinants of
  the minors are as follows:
  \[ \left[\begin{array}{ccc}
      \det\left[\begin{array}{cc} 1&c\\0&1 \end{array}\right] &
      \det\left[\begin{array}{cc} 0&c\\0&1 \end{array}\right] &
      \det\left[\begin{array}{cc} 0&1\\0&0 \end{array}\right] \\
      \det\left[\begin{array}{cc} a&b\\0&1 \end{array}\right] &
      \det\left[\begin{array}{cc} 1&b\\0&1 \end{array}\right] &
      \det\left[\begin{array}{cc} 1&a\\0&0 \end{array}\right] \\
      \det\left[\begin{array}{cc} a&b\\1&c \end{array}\right] &
      \det\left[\begin{array}{cc} 1&b\\0&c \end{array}\right] &
      \det\left[\begin{array}{cc} 1&a\\0&1 \end{array}\right]
     \end{array} \right] = 
     \left[\begin{array}{ccc}
      1 & 0 & 0 \\ a & 1 & 0 \\ ac-b & c & 1 
     \end{array}\right].
  \]
  We multiply by the associated signs and take the transpose to get
  \[ \text{adj}(A) = 
      \left[\begin{array}{ccc}
       1 & 0 & 0 \\ -a & 1 & 0 \\ ac-b & -c & 1 
      \end{array}\right]^T = 
      \left[\begin{array}{ccc}
       1 & -a & ac-b \\ 0 & 1 & -c \\ 0 & 0 & 1 
      \end{array}\right]^T.
  \]
  The determinant is the dot product of the first row of $A$ (which is
  $(1,a,b)$) with the first column of $\text{adj}(A)$ (which is
  $(1,0,0)$).  Thus $\det(A)=1.1+0.a+0.b=1$ and
  $A^{-1}=\text{adj}(A)/\det(A)=\text{adj}(A)$.

\end{enumerate}

\renewcommand{\theenumi}{B\arabic{enumi}}

\begin{enumerate}
 \item % range 4
  Observe that $f(x)=(x+1)^2+2$ \mk.  As
  $x$ runs from $-1$ to $1$ (excluding the endpoints), $x+1$ increases
  from $0$ to $2$ and so $(x+1)^2+2$ increases from $0^2+2=2$ to
  $2^2+2=6$ \mks{2}.  In all cases the endpoints are excluded, so the
  range of $f$ is $(2,6)$ \mk.

 \item % int trig
  The general method that works for all trigonometric functions is to
  rewrite them as sums of terms like $\sin(nx)$ or $\cos(mx)$.  (In
  special cases other methods may work or may be easier, but this is
  not one of them.)  

  Note that $\sin(x)\cos(x)=\sin(2x)/2$ \mk, so
  \[ \sin(x)^2\cos(x)^2 = \sin(2x)^2/4 \mk = (1-\cos(4x))/8 \mks{2}.
  \]
  Thus
  \begin{align*}
   \int \sin(x)^2\cos(x)^2\, dx &=
    \tfrac{1}{8} \int 1-\cos(4x)\, dx \mk \\
    &= \frac{x}{8} - \frac{\sin(4x)}{32} = \frac{4x-\sin(4x)}{32} \mks{2}.
  \end{align*}

 \item % int subs 2
  Put $u=\cos(x)$, so $du=-\sin(x)\,dx$ \mks{2}.  Then
  \begin{align*}
   \int\sin(x)\log(\cos(x))\,dx &= 
    -\int \log(u)\,du \mk = -(u\log(u)-u) \mks{2} = u (1 - \log(u)) \\
    &= \cos(x)(1 - \log(\cos(x))) \mk.
  \end{align*}

 \item % int misc
  We first note that
  \begin{align*}
   \frac{d}{dx}\left(x^3(a\log(x)^2 + b\log(x) + c)\right) 
    &= 3x^2(a\log(x)^2 + b\log(x) + c) + 
       x^3(2a\log(x)/x+b/x)  \mk \\
    &= x^2(3a\log(x)^2 + (3b+2a)\log(x) + (3c+b)). \mk
  \end{align*}
  This must also be equal to $x^2\log(x)^2$ for all $x$, \mk so we must
  have  
  \begin{align*}
   3a &= 1 \\
   3b+2a &= 0 \\
   3c+b &= 0, \mk
  \end{align*}
  so $a=1/3$ and $b=-2/9$ and $c=2/27$,  \mk giving
  \[ \int x^2\log(x)^2\,dx = x^3(\log(x)^2/3-2\log(x)/9+2/27). \]
  It follows that 
  \begin{align*}
   \int_1^e x^2\log(x)^2\, dx &= 
    \left[ x^3(\log(x)^2/3-2\log(x)/9+2/27) \right]_1^e  \mk\\
   &= e^3(1/3-2/9+2/27) - 1^3(0/3-0/9+2/27) \\
   &= (5e^3 - 2)/27. \mk
  \end{align*}

 \item % determinant
  Put
  \[ A = \left[\begin{array}{cccc}
      1 & a & 0 & 0 \\
      a & 1 & b & 0 \\
      0 & b & 1 & c \\
      0 & 0 & c & 1
     \end{array}\right].
  \]
  The direct approach is as follows:
  \begin{align*}
   \det(A) &= 
    \det\left[\begin{array}{ccc} 1&b&0 \\ b&1&c \\ 0&c&1 \end{array}\right] -
    a \det\left[\begin{array}{ccc} a&b&0 \\ 0&1&c \\ 0&c&1 \end{array}\right]
     \mks{2} \\
    &= \left(\det\left[\begin{array}{cc} 1&c\\ c&1\end{array}\right] - 
             b \det\left[\begin{array}{cc} b&c\\ 0&1\end{array}\right]\right) - 
       a\left(a \det\left[\begin{array}{cc} 1&c\\ c&1\end{array}\right] - 
              b \det\left[\begin{array}{cc} 0&c\\ 0&1\end{array}\right]\right)
     \mks{2} \\
    &= (1 - c^2 - b(b-0)) - a(a(1-c^2) - b . 0)  \mk \\
    &= 1-a^2-b^2-c^2+a^2c^2. \mk
  \end{align*}
  Alternatively, if we subtract $a$ times the first row from the second
  row, and subtract $c$ times the fourth row from the third row, we
  obtain the matrix 
  \[ B = \left[\begin{array}{cccc}
      1 & a     & 0     & 0 \\
      0 & 1-a^2 & b     & 0 \\
      0 & b     & 1-c^2 & 0 \\
      0 & 0     & c     & 1
     \end{array}\right]
  \]
  with $\det(A)=\det(B)$.  We can expand down the first column to see
  that
  \[ \det(A)=\det(B)=\det\left[\begin{array}{ccc} 
      1-a^2 & b     & 0 \\
      b     & 1-c^2 & 0 \\
      0     & c     & 1
     \end{array}\right],
  \]
  and then expand this down the last column to get
  \[ \det(A) =
      \det\left[\begin{array}{cc}
       1-a^2 & b \\ b & 1-c^2
      \end{array}\right] =
      (1-a^2)(1-c^2)-b^2 = 1-a^2-b^2-c^2+a^2c^2.
  \]

\end{enumerate} 

\end{document}
