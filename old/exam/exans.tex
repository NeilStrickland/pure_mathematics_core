\documentclass{amsart}
\usepackage{a4wide}
\usepackage[metapost,mplabels,truebbox]{mfpic}
\usepackage[pdftex]{graphicx}


\newcommand{\half}      {{\textstyle\frac{1}{2}}}
\newcommand{\om}{\omega}
\newcommand{\xra}{\xrightarrow}
\renewcommand{\:}       {\colon}

\newcommand{\seen}      {\mbox{\bf [seen]}}
\newcommand{\unseen}    {\mbox{\bf [unseen]}}
\newcommand{\simseen}   {\mbox{\bf [similar examples seen]}}
\newcommand{\bwk}       {\mbox{\bf [bookwork]}}
\newcommand{\mkrem}[1]  {\mbox{\bf [#1]}}

\newcommand{\mks}[1]    {\mbox{\bf [#1]}}
\newcommand{\mk}        {\mbox{\bf [1]}}

\renewcommand{\theenumii}{\roman{enumii}}
\renewcommand{\theenumiii}{\alph{enumiii}}

\begin{document}

\opengraphsfile{exans_mfpic}

\begin{center}
{\Large Pure Mathematics Core --- Exam solutions}
\end{center}

\renewcommand{\theenumi}{A\arabic{enumi}}

\begin{enumerate}
 \item % pfrac 3
  The general form is
  \[ \frac{x^2+x+1}{(x+1)^2} = A + \frac{B}{x+1} + \frac{C}{(x+1)^2}.\mks{2}
  \]
  Multiplying by $(x+1)^2$ gives
  \begin{align*}
   x^2+x+1 &= A(x+1)^2 + B(x+1) + C  = Ax^2 +2Ax + A + Bx + B + C \\
           &= Ax^2 + (2A+B)x + (A+B+C), \mk
  \end{align*}
  so $A=1$ and $2A+B=1$ and $A+B+C=1$, which gives $B=-1$ and $C=1$. \mk
  This means that
  \begin{align*}
   \frac{x^2+x+1}{(x+1)^2} &= 1 - \frac{1}{x+1} + \frac{1}{(x+1)^2} \\
   \int \frac{x^2+x+1}{(x+1)^2}\,dx &= x - \log(x+1) - \frac{1}{(x+1)}.\mks{2}
  \end{align*}

 \item % inverse 1
  If $x=f(y)=1/(1-e^{-y})$ \mk then $1/x=1-e^{-y}$ \mk, so
  $e^{-y}=1-1/x=(x-1)/x$, so $e^y=x/(x-1)$ \mk, so
  $f^{-1}(x)=y=\log(x/(x-1))$ \mk. 

 \item % composite 2
  \[ (\exp\circ f\circ\log)(x) = \exp(f(\log(x))) \mk = 
      \exp(2\log(x)+2) \mk = (e^{\log(x)})^2 e^2 \mk = e^2x^2 \mk.
  \]

 \item % log 1
  We note that $16=2^4$ \mk, so $2=16^{1/4}$, so $1/2=16^{-1/4}$, so
  $\log_{16}(1/2)=-1/4$ \mk. 

 \item % trig value
  Note that $\sin(x)$ repeats with period $2\pi$, so
  \[ \sin(-7\pi/3) = \sin(-7\pi/3+2\pi) = \sin(-\pi/3) \mk
      = -\sin(\pi/3) = -\sqrt{3}/2 \mk.
  \]

 \item % identity 1
  Put $u=e^x$.  Then 
  \[ \tanh(x) = \frac{\sinh(x)}{\cosh(x)} = 
      \frac{(u-u^{-1})/2}{(u+u^{-1})/2} = \frac{u-u^{-1}}{u+u^{-1}}, \mks{2}
  \]
  so
  \begin{align*}
   1+\tanh(x)^2 &= 
    1 + \left(\frac{u-u^{-1}}{u+u^{-1}}\right)^2 = 
    1 + \frac{u^2-2+u^{-2}}{u^2+2+u^{-2}} \\
    &= \frac{(u^2+2+u^{-2})+(u^2-2+u^{-2})}{u^2+2+u^{-2}} = 
       \frac{2u^2+2u^{-2}}{u^2+2+u^{-2}} \mks{2} \\
   1-\tanh(x)^2 &= 
    1 - \left(\frac{u-u^{-1}}{u+u^{-1}}\right)^2 = 
    1 - \frac{u^2-2+u^{-2}}{u^2+2+u^{-2}} \\
    &= \frac{(u^2+2+u^{-2})-(u^2-2+u^{-2})}{u^2+2+u^{-2}} = 
       \frac{4}{u^2+2+u^{-2}} \mk \\
\intertext{ so }
   \frac{1+\tanh(x)^2}{1-\tanh(x)^2} &= 
    \frac{2u^2+2u^{-2}}{4} = \frac{e^{2x}+e^{-2x}}{2} = \cosh(2x) \mk.
  \end{align*}

 \item % diff rat 2
  Put $u=x^n+a$ and $y=f(x)=u^m$.  Then $du/dx=nx^{n-1}$ and
  $dy/du=mu^{m-1}$, so 
  \[ f'(x) = \frac{dy}{dx} = mu^{m-1}\frac{du}{dx} = 
      mn(x^n+a)^{m-1}x^{n-1}. \mks{2}
  \]
 
 \item % diff log 
  Put $u=1+x+x^2+x^3$ and $y=\log(u)$, so 
  \[ y' = \frac{u'}{u} = \frac{1+2x+3x^2}{1+x+x^2+x^3} \mks{2}. \]

 \item % diff quotient
  The quotient rule gives
  \begin{align*}
   \frac{d}{dx}\left(\frac{x}{\log(x)}\right) 
    &= \frac{1.\log(x) - x.\log'(x)}{\log(x)^2} \mk 
     = \frac{\log(x) - x.x^{-1}}{\log(x)^2}  \mk \\
    &= \frac{1}{\log(x)} - \frac{1}{\log(x)^2} \mk.
  \end{align*}

 \item % diff mobius
  \begin{align*}
   \frac{d}{dx}\left(\frac{3x+2}{4x+3}\right) 
    &= \frac{3(4x+3)-4(3x+2)}{(4x+3)^2} \mk \\
    &= \frac{12x+9-12x-8}{(4x+3)^2} = (4x+3)^{-2} \mk   
  \end{align*}

 \item % diff packet
  First put $u=-(x-a)^2/b$, so $du/dx=-2(x-a)/b$.  Then put
  $v=\exp(u)=e^{-(x-a)^2/b}$, so the chain rule gives
  \[ \frac{dv}{dx} = -2(x-a)b^{-1}e^{-(x-a)^2/b}. \mks{2} \]
  Finally, we apply the product rule:
  \begin{align*}
   \frac{d}{dx}\left(e^{-(x-a)^2/b}\sin(\omega x)\right) &= 
    -2(x-a)b^{-1}e^{-(x-a)^2/b} \sin(\om x) + 
    e^{-(x-a)^2/b} \om\cos(\om x) \\
   &= e^{-(x-a)^2/b}(\om\cos(\om x) - 2(x-a)b^{-1}\sin(\om x)) \mks{2}.
  \end{align*}

 \item % int poly exp
  We know that
  \[ \int x^2 e^x\, dx = (ax^2+bx+c) e^x \]
  for some constants $a$, $b$ and $c$ \mks{2}.  To find these, we
  differentiate to get 
  \begin{align*}
   x^2 e^x &= \frac{d}{dx}((ax^2+bx+c) e^x) 
    = (2ax+b)e^x + (ax^2+bx+c) e^x \mk \\
    &= (ax^2+(2a+b)x+(b+c))e^x.
  \end{align*}
  We equate coefficients to see that $a=1$ and $2a+b=b+c=0$ \mk, which
  gives $b=-2$ and $c=2$.  We conclude that
  \[ \int x^2 e^x\, dx = (x^2-2x+2) e^x. \mk \]

 \item % int exp trig 2
  We know that 
  \[ \int e^{3x}\sin(4x)\,dx = e^{3x}(A\cos(4x) + B\sin(4x)) \]
  for some $A$ and $B$ \mks{2}.  To find these, we differentiate and
  equate coefficients: 
  \begin{align*}
   e^{3x}\sin(4x) &=
    \frac{d}{dx}\left(e^{3x}(A\cos(4x) + B\sin(4x))\right) \\
    &= 3e^{3x}(A\cos(4x) + B\sin(4x)) +
        e^{3x}(-4A\sin(4x) + 4B\cos(4x)) \\
    &= e^{3x}((3A+4B)\cos(4x) + (3B-4A)\sin(4x)) \mk,
  \end{align*}
  so $3A+4B=0$ and $3B-4A=1$ \mk.  This gives $A=-4B/3$ so
  $1=3B-4A=3B+16B/3=25B/3$, so $B=3/25$, so $A=-4B/3=-4/25$.  The
  conclusion is that
  \[ \int e^{3x}\sin(4x)\,dx = e^{3x}(3\sin(4x) - 4\cos(4x))/25. \mk \]

 \item % linear 1
  The matrix of coefficients is
  \[ \left[\begin{array}{cccc}
      1 & 1 & 1 & 1 \\ 1 & 1 & -1 & -1 \\ 1 & -1 & 1 & -1
     \end{array}\right] \mk
  \]
  This can be row-reduced as follows:
  \[ \left[\begin{array}{cccc}
      1 & 1 & 1 & 1 \\ 1 & 1 & -1 & -1 \\ 1 & -1 & 1 & -1
     \end{array}\right] \xra{}
     \left[\begin{array}{cccc}
      1 & 1 & 1 & 1 \\ 0 & 0 & -2 & -2 \\ 0 & -2 & 0 & -2
     \end{array}\right] \xra{}
     \left[\begin{array}{cccc}
      1 & 1 & 1 & 1 \\ 0 & 0 & 1 & 1 \\ 0 & 1 & 0 & 1
     \end{array}\right] \xra{}
  \]
  \[ \left[\begin{array}{cccc}
      1 & 1 & 1 & 1 \\ 0 & 1 & 0 & 1 \\ 0 & 0 & 1 & 1
     \end{array}\right] \xra{}
     \left[\begin{array}{cccc}
      1 & 0 & 1 & 0 \\ 0 & 1 & 0 & 1 \\ 0 & 0 & 1 & 1
     \end{array}\right] \xra{}
     \left[\begin{array}{cccc}
      1 & 0 & 0 & -1 \\ 0 & 1 & 0 & 1 \\ 0 & 0 & 1 & 1
     \end{array}\right] \mks{2M 2A}
  \]
  There is no pivot in the last column, so the variable $z$ is
  independent \mk.  The final matrix corresponds to the equations
  $w-z=x+z=y+z=0$, so $(w,x,y,z)=(z,-z,-z,z)$ \mk.

 \item % inverse matrix
  We write down the augmented matrix and row-reduce it as follows:
  \[ \left[\begin{array}{ccc|ccc}
      1 & a & b & 1 & 0 & 0 \\ 
      0 & 1 & c & 0 & 1 & 0 \\ 
      0 & 0 & 1 & 0 & 0 & 1 
     \end{array}\right] \mk \xra{}
     \left[\begin{array}{ccc|ccc}
      1 & 0 & b-ac & 1 & -a & 0 \\ 
      0 & 1 & c & 0 & 1 & 0 \\ 
      0 & 0 & 1 & 0 & 0 & 1 
     \end{array}\right] \xra{}
  \] \[
     \left[\begin{array}{ccc|ccc}
      1 & 0 & 0 & 1 & -a & ac-b \\ 
      0 & 1 & c & 0 & 1 & 0 \\ 
      0 & 0 & 1 & 0 & 0 & 1 
     \end{array}\right] \xra{}
     \left[\begin{array}{ccc|ccc}
      1 & 0 & 0 & 1 & -a & ac-b \\ 
      0 & 1 & 0 & 0 & 1 & -c \\ 
      0 & 0 & 1 & 0 & 0 & 1 
     \end{array}\right] \mks{2M 2A}
  \]
  At the final stage, the left hand block is the identity, so
  the right hand block is the inverse of the original matrix, ie
  \[ \left[\begin{array}{ccc}
      1 & a & b \\ 0 & 1 & c \\ 0 & 0 & 1 
     \end{array}\right]^{-1} = 
     \left[\begin{array}{ccc}
      1 & -a & ac-b \\ 0 & 1 & -c \\ 0 & 0 & 1 
     \end{array}\right] \mk.
  \]

\end{enumerate}

\renewcommand{\theenumi}{B\arabic{enumi}}

\begin{enumerate}
 \item % range 4
  Observe that $f(x)=(x+1)^2+2$ \mk.  As
  $x$ runs from $-1$ to $1$ (excluding the endpoints), $x+1$ increases
  from $0$ to $2$ and so $(x+1)^2+2$ increases from $0^2+2=2$ to
  $2^2+2=6$ \mks{2}.  In all cases the endpoints are excluded, so the
  range of $f$ is $(2,6)$ \mk.

 \item % int trig
  Note that $\sin(x)\cos(x)=\sin(2x)/2$ \mk, so
  \[ \sin(x)^2\cos(x)^2 = \sin(2x)^2/4 \mk = (1-\cos(4x))/8 \mks{2}.
  \]
  Thus
  \begin{align*}
   \int \sin(x)^2\cos(x)^2\, dx &=
    \tfrac{1}{8} \int 1-\cos(4x)\, dx \mk \\
    &= \frac{x}{8} - \frac{\sin(4x)}{32} = \frac{4x-\sin(4x)}{32} \mks{2}.
  \end{align*}

 \item % int subs 2
  Put $u=\cos(x)$, so $du=-\sin(x)\,dx$ \mks{2}.  Then
  \begin{align*}
   \int\sin(x)\log(\cos(x))\,dx &= 
    -\int \log(u)\,du \mk = -(u\log(u)-u) \mks{2} = u (1 - \log(u)) \\
    &= \cos(x)(1 - \log(\cos(x))) \mk.
  \end{align*}

 \item % int misc
  We first note that
  \begin{align*}
   \frac{d}{dx}\left(x^3(a\log(x)^2 + b\log(x) + c)\right) 
    &= 3x^2(a\log(x)^2 + b\log(x) + c) + 
       x^3(2a\log(x)/x+b/x)  \mk \\
    &= x^2(3a\log(x)^2 + (3b+2a)\log(x) + (3c+b)). \mk
  \end{align*}
  This must also be equal to $x^2\log(x)^2$ for all $x$, \mk so we must
  have  
  \begin{align*}
   3a &= 1 \\
   3b+2a &= 0 \\
   3c+b &= 0, \mk
  \end{align*}
  so $a=1/3$ and $b=-2/9$ and $c=2/27$,  \mk giving
  \[ \int x^2\log(x)^2\,dx = x^3(\log(x)^2/3-2\log(x)/9+2/27). \]
  It follows that 
  \begin{align*}
   \int_1^e x^2\log(x)^2\, dx &= 
    \left[ x^3(\log(x)^2/3-2\log(x)/9+2/27) \right]_1^e  \mk\\
   &= e^3(1/3-2/9+2/27) - 1^3(0/3-0/9+2/27) \\
   &= (5e^3 - 2)/27. \mk
  \end{align*}

 \item % determinant
  Put
  \[ A = \left[\begin{array}{cccc}
      1 & a & 0 & 0 \\
      a & 1 & b & 0 \\
      0 & b & 1 & c \\
      0 & 0 & c & 1
     \end{array}\right].
  \]
  The direct approach is as follows:
  \begin{align*}
   \det(A) &= 
    \det\left[\begin{array}{ccc} 1&b&0 \\ b&1&c \\ 0&c&1 \end{array}\right] -
    a \det\left[\begin{array}{ccc} a&b&0 \\ 0&1&c \\ 0&c&1 \end{array}\right]
     \mks{2} \\
    &= \left(\det\left[\begin{array}{cc} 1&c\\ c&1\end{array}\right] - 
             b \det\left[\begin{array}{cc} b&c\\ 0&1\end{array}\right]\right) - 
       a\left(a \det\left[\begin{array}{cc} 1&c\\ c&1\end{array}\right] - 
              b \det\left[\begin{array}{cc} 0&c\\ 0&1\end{array}\right]\right)
     \mks{2} \\
    &= (1 - c^2 - b(b-0)) - a(a(1-c^2) - b . 0)  \mk \\
    &= 1-a^2-b^2-c^2+a^2c^2. \mk
  \end{align*}
  Alternatively, if we subtract $a$ times the first row from the second
  row, and subtract $c$ times the fourth row from the third row, we
  obtain the matrix 
  \[ B = \left[\begin{array}{cccc}
      1 & a     & 0     & 0 \\
      0 & 1-a^2 & b     & 0 \\
      0 & b     & 1-c^2 & 0 \\
      0 & 0     & c     & 1
     \end{array}\right]
  \]
  with $\det(A)=\det(B)$.  We can expand down the first column to see
  that
  \[ \det(A)=\det(B)=\det\left[\begin{array}{ccc} 
      1-a^2 & b     & 0 \\
      b     & 1-c^2 & 0 \\
      0     & c     & 1
     \end{array}\right],
  \]
  and then expand this down the last column to get
  \[ \det(A) =
      \det\left[\begin{array}{cc}
       1-a^2 & b \\ b & 1-c^2
      \end{array}\right] =
      (1-a^2)(1-c^2)-b^2 = 1-a^2-b^2-c^2+a^2c^2.
  \]

\end{enumerate} 

\closegraphsfile
\end{document}
