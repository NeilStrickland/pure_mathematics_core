\documentclass{amsart}
\usepackage{a4wide}

\begin{document}
\title{Report on PMA101}
\author{N.~P.~Strickland}
\bibliographystyle{abbrv}

\vspace{-14ex}
\begin{center}
 {\Huge Report on PMA101}
\end{center}
\vspace{2ex}

The results for this year's PMA101 exam were very poor.  This was in
sharp contrast to the online continuous assessment marks, both on
average and in individual cases; an astonishing number of students did
well in the online tests, and abysmally on the exam.  The students'
acceptable performance on the tests seemed consistent with other kinds
of evidence indicating that they found the course tough but
manageable, which led me to believe that my general approach was
broadly correct.  Apparently, this conclusion was not justified.
Results for other first-year courses were also unusually bad, so
general weakness of the students must form a significant part of the
explanation.  Of course, when we have a weak class we should detect
the situation and adjust to it, so this explanation still leaves many
issues to resolve.  The rest of this document discusses the problem in
more detail and attempts to identify other causes.

\section*{Background}

I taught PMA101 for the first time this year (never having taught a
course below M-level, or with more than about 35 students, in the
past).  There were 183 students registered for the exam.  I gave one
lecture per week, using the data projector in the new Chemistry
auditorium.  I used a special \LaTeX package (called \verb+prosper+)
to produce PDF files that can be used to give a PowerPoint-like
presentation.  These files were placed on the course web page
(\verb+http://www.shef.ac.uk/~pm1nps/teach/core+) so that students
could review them after the lecture.  I also wrote a new booklet for
the course, which was distributed on paper in the first lecture, and
also made available on the course web page.

As usual, each student was also scheduled to attend one tutorial per
week (overall attendance rate was 74\%).  In tutorials, they worked on
specified problems from the booklet, or from additional problem sheets
that I prepared for chapters that did not have enough problems in the
first place.  Solutions for some problems were provided in the
booklet.  Solutions for other problems were given on the website after
the end of the week in which the students were asked to attempt them.
All solutions were fully detailed.  Jayanta Manoharmayum ran weekly
help sessions, which very few students attended.

The other main plank of the course consisted of online tests using the
AIM system.  There was one test per week, and five of the tests
counted 2\% each to the final result for the course.  On average,
students attempted $4.3$ of these $5$ tests, and the overall average
score (with missed tests counted as zero) was 66\%.  The tests were
set up so that students could correct wrong answers with a fairly
small penalty.  Complete solutions for all tests were released online
after the deadline for completion had passed.

The format of the final exam was slightly different to previous years,
but not significantly so.  I do not think that it was any more
difficult, and I believe that Peter Dixon (who checked it) and David
Jordan (who marked half of it) will concur in this judgement.  I
provided a mock exam in the new format, which was distributed on paper
in the final lecture.  Complete solutions were made available online,
but the students were exhorted not read them before attempting the
questions themselves.  I also provided brief solutions for the exams
from the previous two years, with the same exhortation.

I monitored the progress of the course by reviewing the results of the
online tests, talking to my personal tutees, the students in my
tutorial group, and the other staff members who were taking tutorials.
My overall impression was that the students found the course tough but
manageable.  I regarded the AIM evidence as being most reliable, as it
was comprehensive and objective.  Unfortunately, the generally
satisfactory results of these tests were not repeated in the exam.

\section*{Exam results}

Raw marks out of 90 for the exam were as follows:
\[ \small \begin{array}{|c|c|c|c|c|c|c|c|c|c|}
    \hline
    \text{Absent} & 0's & 10's & 20's & 30's & 40's &
     50's & 60's & 70's & 80's \\
    \hline
     5 & 1 & 13 & 30 & 23 & 38 & 29 & 19 & 17 & 8 \\
    \hline
   \end{array}
\]
The average was $46.1$ with absent students excluded, or $44.8$ if
they are included.

Raw marks out of 100 for the exam plus AIM tests were as follows:
\[ \small \begin{array}{|c|c|c|c|c|c|c|c|c|c|c|}
    \hline
    \text{Absent} & 0's & 10's & 20's & 30's & 40's &
     50's & 60's & 70's & 80's & 90's \\
    \hline
     5 & 0 & 8 & 17 & 27 & 29 & 35 & 20 & 20 & 15 & 7 \\
    \hline
   \end{array}
\]
The average was $52.8$ with absent students excluded, or $51.4$ if
they are included.

The correlation between AIM scores and exam marks, for students who
sat the exam, was $0.51$.  There were a remarkable number of students
who did well in the online tests, but very badly in the exam.  Here
are the figures for all students with raw totals between 30 and 40.
\[ \begin{array}{|c|c|c|c|c|c|c|c|c|c|c|} \hline
\text{Exam} & \text{AIM} & \text{Total} & \hspace{1em} &
\text{Exam} & \text{AIM} & \text{Total} & \hspace{1em} &
\text{Exam} & \text{AIM} & \text{Total} \\ \hline
27 & 3 & 30 & \hspace{1em} &
29 & 1 & 30 & \hspace{1em} &
26 & 4 & 30 \\ \hline
25 & 5 & 30 & \hspace{1em} &
24 & 7 & 31 & \hspace{1em} &
26 & 5 & 31 \\ \hline
24 & 8 & 32 & \hspace{1em} &
24 & 8 & 32 & \hspace{1em} &
26 & 7 & 33 \\ \hline
26 & 7 & 33 & \hspace{1em} &
27 & 6 & 33 & \hspace{1em} &
25 & 8 & 33 \\ \hline
24 & 9 & 33 & \hspace{1em} &
27 & 7 & 34 & \hspace{1em} &
26 & 8 & 34 \\ \hline
28 & 6 & 34 & \hspace{1em} &
30 & 4 & 34 & \hspace{1em} &
34 & 0 & 34 \\ \hline
28 & 7 & 35 & \hspace{1em} &
29 & 6 & 35 & \hspace{1em} &
30 & 6 & 36 \\ \hline
33 & 4 & 37 & \hspace{1em} &
31 & 6 & 37 & \hspace{1em} &
29 & 8 & 37 \\ \hline
35 & 3 & 38 & \hspace{1em} &
30 & 8 & 38 & \hspace{1em} &
31 & 8 & 39 \\ \hline
36 & 4 & 40 & \hspace{1em} &
35 & 5 & 40 & \hspace{1em} &
   &   &    \\ \hline
\end{array}
\]

%\newpage


Here is a breakdown of marks by question.  This is taken from a sample
of 50 scripts, which seem to be slightly worse than average overall.
\[ \renewcommand{\arraystretch}{1.5}\begin{array}{|c|l|c|c|}\hline
& \text{Question} & \text{Value} & \text{Average} \\ \hline
\text{A1 } & \int (x^2+x+1)/(x+1)^2\,dx         & 6 & 3.65 \\ \hline
\text{A2 } & \text{ inverse of } 1/(1-e^{-x})   & 4 & 2.67 \\ \hline
\text{A3 } & f(x)=2x+2; \exp\circ f\circ\log(x) & 4 & 2.20 \\ \hline
\text{A4 } & \log_{16}(1/2)                     & 2 & 1.25 \\ \hline
\text{A5 } & \sin(-7\pi/3)                      & 2 & 1.61 \\ \hline
\text{A6 } & \text{ Prove }
             \frac{1+\tanh(x)^2}{1-\tanh(x)^2} 
             = \cosh(2x)                        & 6 & 2.90 \\ \hline
\text{A7 } & \frac{d}{dx}(x^n+a)^m              & 2 & 1.53 \\ \hline
\text{A8 } & \frac{d}{dx}\log(1+x+x^2+x^3)      & 2 & 1.39 \\ \hline
\text{A9 } & \frac{d}{dx}(x/\log(x))            & 3 & 2.29 \\ \hline
\text{A10} & \frac{d}{dx}((3x+2)/(4x+3))        & 2 & 1.69 \\ \hline
\text{A11} & \frac{d}{dx}e^{-(x-a)^2/b}\sin(\omega x)& 4 & 2.30 \\ \hline
\text{A12} & \int x^2e^x\,dx                    & 5 & 3.64 \\ \hline
\text{A13} & \int e^{3x}\sin(4x)\,dx            & 5 & 1.76 \\ \hline
\text{A14} & \text{ 3 linear eqs, 4 unknowns }  & 7 & 4.14 \\ \hline
\text{A15} & \text{ inverse of triangular matrix }& 6 & 3.56 \\ \hline
\text{B1}  & \text{ range of quadratic }        & 4 & 2.00 \\ \hline
\text{B2}  & \int\sin(x)^2\cos(x)^2\,dx         & 7 & 2.15 \\ \hline
\text{B3}  & \int\sin(x)\log(\cos(x))\,dx       & 6 & 3.15 \\ \hline
\text{B4}  & \int x^2\log(x)^2\,dx \text{ (with hint) }& 7 & 2.10 \\ \hline
\text{B5}  & \text{ 4 by 4 sparse determinant } & 6 & 3.02 \\ \hline
\end{array}
\]

\section*{What went wrong?}

\subsection*{The booklet:}

The booklet was substantially longer than the total of the booklets
used in previous years (while still being much shorter than the
textbooks commonly used for one-semester calculus courses in the US).
A very substantial part of the additional length consisted of extra
examples and detailed solutions (where the old booklets just gave
answers).  The contents are listed below; the notation $i+j+k$ means
$i$ pages of notes, plus $j$ pages of solutions in the booklet, plus
$k$ pages of solutions released later.  

\begin{itemize}
 \item \textbf{Introduction:} 2 pages.
 \item \textbf{Algebraic manipulation:} 8 + 7 + 8 pages.  More
  elaborate examples than in previous years, with various exercises
  asking students to explore and find patterns.  The discussion of
  partial fractions tried to discuss the concepts more than in
  previous years (including informal definitions of the terms ``pole''
  and ``order'').  This does not seem to have worked well.  I did not
  find a good way to talk about the situation where the poles are
  non-real. 
 \item \textbf{Sets:} 12 + 3 + 3 pages.  This chapter contained
  material from the old booklet on Inequalities, as well as the one on
  Sets.  The contents and the approach were fairly standard.  The
  material involved was not seriously covered in the exam.
 \item \textbf{General theory of functions:} 8 + 4 + 4 pages.  This
  chapter covered the concept of a function, (co)domains and ranges,
  composition, and inverses.  All of this was in the old booklet on
  Functions and Differentiation; I covered it at greater length and
  with more care, which was probably not appropriate, especially given
  the weakness of the year.
 \item \textbf{Special functions:} 10 + 1 + 6 pages.  This covered the
  exponential and the logarithm, and hyperbolic and trigonometric
  functions.  I talked about proving hyperbolic identities and
  assigned problems on this.  I described the connection with
  trigonometric identities via de Moivre's theorem, but did not
  emphasise or test this.
 \item \textbf{Differentiation:} 7 + 4 + 4 pages.  The explanations
  were fairly standard, but the examples in the text were less
  obvious. 
 \item \textbf{Integration:} 16 + 2 + 5 pages.  I modified the usual
  approach by using the method of undetermined coefficients for
  various classes of integrands that are more usually done by repeated
  integration by parts.  This is, in my opinion, more explicit and
  illuminating, and often quicker.  However, students generally
  preferred to use the traditional method in the exam.  There were too
  few examples (for substitution in particular) in the booklet, but I
  distributed more for the tutorials, and covered further cases in the
  lecture. 
 \item \textbf{Vectors and Matrices:} 20 + 7 + 7 pages.  This was
  fairly standard.  Examples were carefully chosen to minimise
  arithmetic problems; indeed, I overdid this somewhat and gave some
  students the impression that row-reductions leading to non-integer
  entries were not permitted.  There were some problems with typos in
  this chapter.
\end{itemize}

\subsection*{The lectures:}

Due to technical problems in the first lecture, I was forced to cover
Sets before Algebraic Manipulation.  I do not think that this caused
any real problems.

I had some problems with timing in the first few weeks, as the
material was refusing to fit neatly into 50 minute slots.  This lead
me to drag out the discussion of inverse functions to fill almost a
whole lecture, in order to get back to a natural rhythm.

My first lecture on Vectors and Matrices was disrupted by the failure
of the data projector.  As this was the penultimate lecture, I was not
able to recover the lost time, so instead I instructed the students to
fill the gap by reading the lecture slides that I placed on the web.
I do not know how many did so.

I had some complaints that students did not know what they needed to
write down during lectures.  Part of the reason was that they did not
immediately know the relationship between the lecture slides and the
booklet; I tried to give better guidance thereafter, but there was
still room for improvement.  Many slides had dynamic effects that were
(I believe) very useful for students who were awake and following the
argument.  However, it was not possible to make useful notes of these
dynamic effects in real time.  They were made available on the web
after the lecture.

On the first occasion when I wrote anything significant on the
whiteboard (in Week 4) I used a colour that was not legible.  On other
occasions I avoided this problem, but I think there were still slight
problems as the whiteboard is really too low.  It is a defect of the
technical setup that one cannot superimpose scribblings on the
projected image.

\subsection*{The tutorials:}

The early tutorials involved exploratory problems which students were
asked to attempt before the class, discuss with each other, and come
perpared to explain.  I felt that this worked well, but I did not have
time to design good questions of this type for later weeks, so
tutorials drifted back to their traditional mode.  Attendance was
about 75\% 
over the whole term.  Students tackled the assigned problems more
slowly than I would have liked, but they did not seem unhappy with
them.  My policy was to put the most standard questions in the online
tests, and assign more interesting questions for tutorials.  Most
students did succeed in doing the standard questions in the online
tests, so this approach was not obviously wrong.  Nonetheless, the
policy should be reexamined.

\subsection*{The online tests}

There were 12 online tests, of which one was only a trivial
demonstration, and 5 others contributed 2\% each to the final course
grade.  Full solutions were given for all tests, and this was an
important source of worked examples for some topics.  I made little
use of randomisation, so cheating by copying was certainly possible,
as was unauthorized use of Maple.  The questions on the tests were of
very similar type to those in the exam.  However, students could work
on each test for a whole week at their leisure, and were allowed to
make repeated attempts with only a small penalty.  This may have given
some students an unjustified feeling of complacency.  Of course, they
all have plenty of experience of doing problems under exam conditions,
so this should not be overemphasised.

\subsection*{The exam}

The exam consisted of 20 questions, all of which were to be
attempted.  I distributed a mock exam in the same style, with full
solutions, as well as brief solutions to the exams from the previous
two years.  I felt that students who had studied these diligently, and
who had revised the online tests, should have found the exam fairly
straightforward. 

David Jordan (who did half the marking) expressed some concern that
students were being confused by too wide a choice of possible methods.
\begin{itemize}
 \item A6 asked for a proof that
  $(1+\tanh(x)^2))/(1-\tanh(x)^2)=\cosh(2x)$.  In lectures and
  solutions (including the solution to a similar problem on the mock
  exam) I always rewrote everything in terms of $u=e^x$ and proceeded
  from there.  It is quicker to use various hyperbolic identities, but
  you run the risk of misrememnbering the signs, as many students
  did, often using the signs for trigonometric functions without
  modification. 
 \item A12 asked for $\int x^2e^x\,dx$.  My preferred method for this
  was to write the answer as $(Ax^2+Bx+C)e^x$ and find the
  coefficients by differentiating.  Students generally preferred to
  integrate by parts twice instead.
 \item A13 asked for $\int e^{3x}\sin(4x)\,dx$.  My preferred method for this
  was to write the answer as $e^{3x}(A\cos(4x)+B\sin(4x))$ and find the
  coefficients by differentiating.  Students generally preferred to
  integrate by parts instead.  Some of them understood that they
  needed to do this twice and substitute back, but many did not.
 \item B2 asked for $\int\sin(x)^2\cos(x)^2\,dx$.  This was the main
  example of a question where there were reasonable-looking methods
  that could not be used successfully.  It was a deliberate decision
  to include a question like this.
 \item B3 asked for $\int\sin(x)\log(\cos(x))\,dx$, ``by using a
  suitable substitution''.  There is a certain amount of scope to
  choose the wrong one, but not too much.
\end{itemize}

\end{document}
