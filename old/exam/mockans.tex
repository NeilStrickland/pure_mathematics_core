\documentclass{amsart}
\usepackage{a4wide}
\usepackage[metapost,mplabels,truebbox]{mfpic}
\usepackage[pdftex]{graphicx}


\newcommand{\half}      {{\textstyle\frac{1}{2}}}
\newcommand{\om}{\omega}
\newcommand{\xra}{\xrightarrow}
\renewcommand{\:}       {\colon}
\newcommand{\bbm}{\begin{pmatrix}}
\newcommand{\ebm}{\end{pmatrix}}

\newcommand{\seen}      {}
\newcommand{\unseen}    {}
\newcommand{\simseen}   {}
\newcommand{\bwk}       {}
\newcommand{\mkrem}[1]  {}

\newcommand{\mks}[1]    {}
\newcommand{\mk}        {}

\renewcommand{\theenumii}{\roman{enumii}}
\renewcommand{\theenumiii}{\alph{enumiii}}

\begin{document}

\opengraphsfile{mockans_mfpic}

\begin{center}
{\Large Pure Mathematics Core --- Mock exam solutions}
\end{center}

\renewcommand{\theenumi}{A\arabic{enumi}}
\begin{enumerate}
 \item % pfrac 1
  The denominator factorises as $(x+1)^2(x-1)$, so the general form is 
  \[ \frac{4}{(x+1)^2(x-1)} = 
     \frac{A}{x+1} + \frac{B}{(x+1)^2} + \frac{C}{x-1}.
  \]
  Multiplying by $(x+1)^2(x-1)$ gives
  \begin{align*}
    4 &= A(x+1)(x-1) + B(x-1) + C(x+1)^2 \\
      &= Ax^2 - A + Bx - B + Cx^2 +2Cx + 2C \\
      &= (A+C)x^2 + (B+2C)x + (-A-B+C). 
  \end{align*}
  We can now equate coefficients to see that
  \begin{align*}
   A+C  &= 0 \\
   B+2C &= 0 \\
   -A-B+C &= 4.
  \end{align*}
  The first two equations give $A=-C$ and $B=-2C$; substituting these
  into the third equation gives
  \[ 4 = -(-C) - (-2C) + C = 4C, \]
  so $C=1$, $B=-2$ and $A=-1$.  We conclude that
  \[ f(x) = \frac{1}{x-1} - \frac{1}{x+1} - \frac{2}{(x+1)^2} \]
  so
  \[ \int f(x)\,dx = \log(x-1) - \log(x+1) + 2(x+1)^{-1}. \]

 \item % inverse 3
  Put $y=f(x)=(5x+4)/(4x+5)$.  Then $4xy+5y=5x+4$, so $4xy-5x=4-5y$,
  so $(4y-5)x=4-5y$, so $f^{-1}(y)=x=(4-5y)/(4y-5)$.  Changing
  notation, we find that $f^{-1}(x)=(4-5x)/(4x-5)$.

 \item % composite 1
  If $x=f(y)=a+y$ then $f^{-1}(x)=y=x-a$.  Similarly, if $x=g(y)=b-y$
  then $g^{-1}(x)=y=b-x$.  Finally, we have
  \[ (g\circ f\circ g)(x) = g(f(g(x))) =
      g(f(b-x)) = g(a+b-x) = b-(a+b-x) = x-a.
  \]

 \item % log 2
  Note that $e^ae^b=e^{a+b}$ and $(e^c)^d=e^{cd}$, so
  $e^a e^b/((e^c)^d)=e^{a+b-cd}$.  It follows that
  $\log(e^a e^b/(e^c)^d)=\log(e^{a+b-cd})=a+b-cd$.

 \item % arctan
  \[ \mbox{
   \begin{mfpic}[25][13]{-3.2}{3.4}{-4.2}{4.5}
    \drawcolor{red}
    \function{-3.00,-2.58,.042}{sind(180*x)/cosd(180*x)}
    \function{-2.42,-1.58,.042}{sind(180*x)/cosd(180*x)}
    \function{-1.42,-0.58,.042}{sind(180*x)/cosd(180*x)}
    \function{-0.42, 0.42,.042}{sind(180*x)/cosd(180*x)}
    \function{ 0.58, 1.42,.042}{sind(180*x)/cosd(180*x)}
    \function{ 1.58, 2.42,.042}{sind(180*x)/cosd(180*x)}
    \function{ 2.58, 3.00,.042}{sind(180*x)/cosd(180*x)}
    \drawcolor{blue}
    \function{-3,3,.1}{sind(180*x)}
    \drawcolor{green}
    \dotted\lines{(-2.5,-4),(-2.5,4)}
    \dotted\lines{(-1.5,-4),(-1.5,4)}
    \dotted\lines{(-0.5,-4),(-0.5,4)}
    \dotted\lines{( 0.5,-4),( 0.5,4)}
    \dotted\lines{( 1.5,-4),( 1.5,4)}
    \dotted\lines{( 2.5,-4),( 2.5,4)}
    \tlabel[cc](-3,-.5){$-3\pi$}
    \tlabel[cc](-2,-.5){$-2\pi$}
    \tlabel[cc](-1,-.5){$-\pi$} 
    \tlabel[cc]( 1,-.5){$\pi$}
    \tlabel[cc]( 2,-.5){$2\pi$}
    \tlabel[cc]( 3,-.5){$3\pi$} 
    \tlabel[cc]( 1.5, 4.5){$\tan(\theta)$}
    \drawcolor{black}
    \doaxes{xy}
    \xmarks{-3 upto 3}
    \ymarks{-4 upto 4}
   \end{mfpic}  
  } \]
  We see from this that $\tan(5\pi/4)=1$ and
  $\sin(5\pi/4)=-2^{-1/2}<0$.

 \item % identity 2 
  Put $u=e^x$, so $\cosh(x)=(u+u^{-1})/2$.  Then 
  \begin{align*}
   8\cosh(x)^4 &= 8(u+u^{-1})^4/16 \\
    &= (u^4 + 4u^2 + 6 + 4u^{-2}+u^{-4})/2 \\
   -8\cosh(u)^2 &= -8(u+u^{-1})^2/4 \\
    &= (-4u^2-8-4u^{-2})/2,
  \end{align*}
  so
  \[ 8\cosh(x)^4 - 8 \cosh(x)^2 + 1 = 
     (u^4+u^{-4})/2 = \cosh(4x).
  \]

 \item % diff rat 1
  Put $u=(x-a)/(x-b)$ and $y=f(x)=u^n$.  Then 
  \[ \frac{du}{dx} = \frac{1.(x-b) - (x-a).1}{(x-b)^2} = 
      \frac{a-b}{(x-b)^2}, 
  \] 
  so
  \[ f'(x) = \frac{dy}{dx} = nu^{n-1}\frac{du}{dx} = 
      n(a-b)\left(\frac{x-a}{x-b}\right)^{n-1}(x-b)^{-2} = 
      n(a-b)(x-a)^{n-1}(x-b)^{-n-1}.
  \]

 \item % diff chain
  By the chain rule, we have
  \[ \frac{d}{dx}\cos\left(\left(\frac{x+1}{2}\right)^2\right) = 
      -\sin\left(\left(\frac{x+1}{2}\right)^2\right). 
      \frac{d}{dx}\left(\frac{x+1}{2}\right)^2 = 
      -\sin\left(\left(\frac{x+1}{2}\right)^2\right).
       \frac{x+1}{2}.
  \]

 \item % diff log
  By the logarithmic rule, we have 
  \[ \frac{d}{dx}\log(\cos(x)) = \frac{\cos'(x)}{\cos(x)} = 
      -\frac{\sin(x)}{\cos(x)} = -\tan(x).
  \]

 \item % diff rat 3
  Put $u=ax+b/x$ and $v=cx+d/x$ and $y=u/v$; we must find $y'$.  Note
  that
  \begin{align*}
   u' &= a-b/x^2 \\
   v' &= c-d/x^2 \\
   u'v - uv' &= (a-b/x^2)(cx+d/x) - (ax+b/x)(c-d/x^2) \\
             &= acx + ad/x -bc/x -bd/x^3 
                -acx + ad/x -bc/x + bd/x^3 \\
             &= 2(ad-bc)/x, \\
   y' &= \frac{u'v-uv'}{v^2} = \frac{2(ad-bc)}{x(cx+d/x)^2}.
  \end{align*}

 \item % diff stirling
  Put $y=\sqrt{2\pi}x^{x-1/2}e^{-x}$, so 
  \[ \log(y) = \log(\sqrt{2\pi}) + (x-1/2)\log(x) - x, \]
  so
  \begin{align*}
   \frac{y'}{y} &= \log(y)' \\
    &= 0 + 1.\log(x) + (x-1/2).\log'(x) - 1 \\
    &= \log(x) + (x-1/2)/x - 1 = \log(x) + 1 - 1/(2x) - 1 \\
    &= \log(x) - 1/(2x).  
  \end{align*}

 \item % int subs 1
  Put $u=\log(x)$, so $du=x^{-1}dx$.  Then
  \begin{align*}
   \int\frac{(1+\log(x))^2}{x} \,dx &=
    \int (1+u)^2 du = (1+u)^3/3 \\
    &= (1+\log(x))^3/3.
  \end{align*}

 \item % int poly exp
  The general form is
  \[ \int (4x^2+2x+1)e^{2x}\,dx = (Ax^2+Bx+C)e^{2x} \]
  for some constants $A$, $B$ and $C$.  Differentiating, we find that 
  \begin{align*}
   (4x^2+2x+1)e^{2x} &= \frac{d}{dx}((Ax^2+Bx+C)e^{2x}) \\
     &= (2Ax+B)e^{2x} + (Ax^2+Bx+C).2e^{2x} \\
     &= (2Ax^2+(2A+2B)x+(B+2C))e^{2x},
  \end{align*}
  so $2A=4$ and $2A+2B=2$ and $B+2C=1$.  It follows that $A=2$ and
  $B=-1$ and $C=1$, so
  \[  \int (4x^2+2x+1)e^{2x}\,dx = (2x^2-x+1)e^{2x}. \]

 \item % linear 2
  The augmented matrix for this problem is
  \[ \left[\begin{array}{ccc|c}
      0&1&2 & 1 \\ -1&0&3 & 2 \\ -2&-3&0 & 1 
     \end{array}\right].
  \]
  This can be row-reduced as follows:
  {\tiny
   \[ \left[\begin{array}{ccc|c}
       0&1&2 & 1 \\ -1&0&3 & 2 \\ -2&-3&0 & 1 
      \end{array}\right] \xra{}
      \left[\begin{array}{ccc|c}
       1&0&-3 & -2 \\ 0&1&2 & 1 \\ -2&-3&0 & 1 
      \end{array}\right] \xra{}
    \] \[
      \left[\begin{array}{ccc|c} 
       1&0&-3 & -2 \\ 0&1&2 & 1 \\ 0&-3&-6 & -3 
      \end{array}\right] \xra{}
      \left[\begin{array}{ccc|c}
       1&0&-3 & -2 \\ 0&1&2 & 1 \\ 0&0&0 & 0 
      \end{array}\right] \xra{}
   \]}
  There is no pivot in the third column, so the variable $z$ is
  independent.  The final matrix corresponds to the equations
  $x=3z-2$ and $y=-2z+1$, which give the general solution.
%   In
%   particular, there are infinitely many different solutions, so the
%   determinant of the matrix of coefficients must be zero, so
%   \[ \det\left[\begin{array}{ccc}
%       0 & 1 & 2 \\ -1 & 0 & 3 \\ -2 & -3 & 0 
%      \end{array}\right] = 0.
%   \]

 \item % char sym
  {\small \begin{align*}
   \det\bbm t & a & b \\ a & t & c \\ b & c & t \ebm &= 
    t\det\bbm t&c \\ c&t \ebm - 
    a\det\bbm a&c \\ b&t \ebm +
    b\det\bbm a&t \\ b&c \ebm \\
   &= t(t^2-c^2) - a(at-bc) + b(ac-bt) \\ 
   &= t^3 - c^2 t - a^2t + abc + abc - b^2 t \\
   &= t^3 - (a^2+b^2+c^2) t + 2abc.
  \end{align*}}

\end{enumerate}

\renewcommand{\theenumi}{B\arabic{enumi}}

\begin{enumerate}

\item % range 1
  As $x$ increases from zero to infinity, $e^{-x}$ decreases from $1$
  to $0$ \mk, without ever reaching $0$ \mk.  This means that
  $\pi e^{-x}/2$ decreases from $\pi/2$ to $0$ (without ever reaching
  $0$) \mks{2}, and thus that $\sin(\pi e^{-x}/2)$ decreases from
  $\sin(\pi/2)=1$ to $0$ (without ever reaching $0$) \mk.  This means
  that the range of $g$ is $(0,1]$ \mks{2}.

 \item % int poly trig
  First note that $8x\sin(x)\cos(x)=4x\sin(2x)$, so 
  \begin{align*}
   \int 8x\sin(x)\cos(x)\,dx &= 
    \int 4x\sin(2x)\,dx \\
    &= -2x\cos(2x) + \int 2\cos(2x)\, dx \\
    &= -2x\cos(2x) + \sin(2x).
  \end{align*}

 \item % int subs 2
  Put $u=x^n$, so $du=nx^{n-1}\,dx$, so $dx=du/(nx^{n-1})$.  The
  integral becomes
  \begin{align*}
   \int \frac{dx}{x\sqrt{x^{-2n}-1}} &= 
   \int \frac{du}{nx^{n-1}.x\sqrt{x^{-2n}-1}} = 
   \frac{1}{n}\int \frac{du}{x^n\sqrt{x^{-2n}-1}} \\
   &= \frac{1}{n}\int\frac{du}{u\sqrt{u^{-2}-1}} = 
   \frac{1}{n}\int\frac{du}{\sqrt{1-u^2}} = \arcsin(u)/n \\
   &= \arcsin(x^n)/n.
  \end{align*}

 \item % pfrac 4
  The efficient method is as follows: we have
  $x^5-1=(x^4+x^3+x^2+x+1)(x-1)$, so  
  \[ \frac{x^5-1}{x^2(x-1)} = 
     \frac{x^4+x^3+x^2+x+1}{x^2} = 
     x^2 + x + 1 + x^{-1} + x^{-2}, 
  \]
  so 
  \[ \int \frac{x^5-1}{x^2(x-1)} =
      x^3/3 + x^2/2 + x + \log(x) - x^{-1} \mks{2}.
  \]

  If we follow the partial fraction method more mechanically, the
  solution is as follows:
  \[ \frac{x^5-1}{x^2(x-1)} = 
      Ax^2 + Bx + C + \frac{D}{x} + \frac{E}{x^2} + \frac{F}{x-1}
      \mks{2}
  \]
  for some constants $A,\dotsc,F$.  Multiplying by $x^2(x-1)$ gives
  \begin{align*}
   x^5 - 1 &= 
    (Ax^2 + Bx + C)x^2(x-1) + Dx(x-1) + E(x-1) + Fx^2 \\
    &= Ax^5 + Bx^4 + Cx^3 - Ax^4 - Bx^3 - Cx^2 + Dx^2 - Dx + Ex - E + Fx^2 \\
    &= Ax^5 + (B-A)x^4 + (C-B)x^3 + (D+F-C)x^2 + (E-D)x - E, \mks{2}
  \end{align*}
  so 
  \begin{align*}
   A &= 1 \\
   B-A &= 0 \\
   C-B &= 0 \\
   D+F-C &= 0 \\
   E-D &= 0 \\
   -E &= -1 \mk.
  \end{align*}
  The first three equations give $A=B=C=1$, the last two give $D=E=1$,
  and the remaining equation then gives $F=0$ \mks{2}.  The conclusion is that 
  \[ \frac{x^5-1}{x^2(x-1)} = 
     x^2 + x + 1 + x^{-1} + x^{-2},
  \]
  so
  \[ \int \frac{x^5-1}{x^2(x-1)} =
      x^3/3 + x^2/2 + x + \log(x) - x^{-1} \mks{2}.
  \]
  as before.

 \item % inverse matrix
  We write down the augmented matrix and row-reduce it as follows:
  {\tiny \[ 
   \left[ \begin{array}{cccc|cccc}
    0&1&1&1 & 1&0&0&0 \\
    0&0&0&1 & 0&1&0&0 \\
    0&0&1&1 & 0&0&1&0 \\
    1&1&1&1 & 0&0&0&1 
   \end{array} \right] \xra{1}
   \left[ \begin{array}{cccc|cccc}
    1&1&1&1 & 0&0&0&1 \\
    0&1&1&1 & 1&0&0&0 \\
    0&0&0&1 & 0&1&0&0 \\
    0&0&1&1 & 0&0&1&0
   \end{array} \right] \xra{2}
   \left[ \begin{array}{cccc|cccc}
    1&1&1&1 & 0&0&0&1 \\
    0&1&1&1 & 1&0&0&0 \\
    0&0&1&1 & 0&0&1&0 \\
    0&0&0&1 & 0&1&0&0
   \end{array} \right] \xra{3}
  \] \[
   \left[ \begin{array}{cccc|cccc}
    1&0&0&0 & -1& 0& 0& 1 \\
    0&1&1&1 &  1& 0& 0& 0 \\
    0&0&1&1 &  0& 0& 1& 0 \\
    0&0&0&1 &  0& 1& 0& 0
   \end{array} \right] \xra{4}
   \left[ \begin{array}{cccc|cccc}
    1&0&0&0 & -1& 0& 0& 1 \\
    0&1&0&0 &  1& 0&-1& 0 \\
    0&0&1&1 &  0& 0& 1& 0 \\
    0&0&0&1 &  0& 1& 0& 0
   \end{array} \right] \xra{5}
   \left[ \begin{array}{cccc|cccc}
    1&0&0&0 & -1& 0& 0& 1 \\
    0&1&0&0 &  1& 0&-1& 0 \\
    0&0&1&0 &  0&-1& 1& 0 \\
    0&0&0&1 &  0& 1& 0& 0
   \end{array} \right]
  \]}
  In step 1 we reorder the rows by bringing $R_4$ to the top, then in
  step 2 we exchange $R_3$ and $R_4$.  The next three steps are
  $R_1\mapsto R_1-R_2$, $R_2\mapsto R_2-R_3$ and
  $R_3\mapsto R_3-R_4$.  At this stage the left hand block is the
  identity, so the right hand block is the inverse of our original
  matrix, or in other words 
  {\tiny \[
   \bbm
    0 & 1 & 1 & 1 \\
    0 & 0 & 0 & 1 \\
    0 & 0 & 1 & 1 \\
    1 & 1 & 1 & 1 
   \ebm^{-1} = 
   \bbm
    -1& 0& 0& 1 \\
     1& 0&-1& 0 \\
     0&-1& 1& 0 \\
     0& 1& 0& 0
   \ebm.
  \]}
\end{enumerate} 



\closegraphsfile
\end{document}


 \item % expand
  \begin{align*}
   (aw+by)(cx+dz) &=  acwx + adwz + bcxy + bdyz \\
  -(cw+dy)(ax+bz) &= -acwx - bcwz - adxy - bdyz \\ 
  -(aw+cx)(by+dz) &= -abwy - adwz - bcxy - cdxz \\
   (ay+cz)(bw+dx) &=  abwy + adxy + bcwz + cdxz.
  \end{align*}
  Adding all these together gives zero.

  
 \item % diff misc
  \begin{align*}
   \frac{d}{dx}\left(\frac{x}{\sqrt{1+x^2}}\right) 
    &= \frac{1.(1+x^2)^{1/2}-x.\frac{1}{2}(1+x^2)^{-1/2}.2x}
            {1+x^2} \\
    &= \frac{(1+x^2)^{1/2}-x^2(1+x^2)^{-1/2}}{1+x^2}, \\
  \intertext{so}
   (1+x^2)^{1/2}\frac{d}{dx}\left(\frac{x}{\sqrt{1+x^2}}\right) 
    &= \frac{(1+x^2)-x^2}{1+x^2} = \frac{1}{1+x^2} = (1+x^2)^{-1} \\
  \intertext{so}
   \frac{d}{dx}\left(\frac{x}{\sqrt{1+x^2}}\right) &= 
    (1+x^2)^{-3/2}.
  \end{align*}
  In other words, we have $c=-3/2$.


 \item % pfrac 2
  Efficient method:
  \[ \frac{x+1}{(x-1)^4} = 
     \frac{(x-1) + 2}{(x-1)^4} = 
     \frac{1}{(x-1)^3} + \frac{2}{(x-1)^4}.
  \]

  Mechanical method:
  \[ \frac{x+1}{(x-1)^4} = 
     \frac{A}{x-1} + \frac{B}{(x-1)^2} +
     \frac{C}{(x-1)^3} + \frac{D}{(x-1)^4}.
  \]
  Multiplying by $(x-1)^4$ gives
  \begin{align*}
   x+1 &= A(x-1)^3 + B(x-1)^2 + C(x-1) + D \\
       &= A(x^3-3x^2+3x-1) + B(x^2-2x+1) + C(x-1) + D \\
       &= Ax^3 + (B-3A)x^2 + (C-2B+3A)x + (D-C+B-A).
  \end{align*}
  By comparing coefficients, we deduce that
  \begin{align*}
   A &= 0 \\
   B-3A &= 0 \\
   C-2B+3A &= 1 \\
   D-C+B-A &= 1.
  \end{align*}
  These equations are easily solved to give $A=B=0$ and $C=1$ and
  $D=2$.  We conclude that
  \[ f(x) = \frac{1}{(x-1)^3} + \frac{2}{(x-1)^4} \]
  as before.
  
  Either way, we have
  \[ \int f(x)\,dx = \frac{-1/2}{(x-1)^2} + \frac{-2/3}{(x-1)^3}.
  \] 

\item % range 2
  The set $A$ is the union of the intervals $(-1,0)$ and $(0,1)$.  As
  $x$ runs from $-1$ to $0$, the function $1/x$ decreases from $-1$ to
  $\infty$.  As $x$ runs from $0$ to $1$, the function $1/x$ decreases
  from $\infty$ to $0$.  As $0$, $1$ and $-1$ are not included in $A$,
  we see that $\infty$, $1$ and $-1$ are not in the range.  It follows
  that the range is $(-\infty,-1)\cup(1,\infty)$.

 \item % inverse 2
  We have $f(x)^2+1=\sinh(x)^2+1=\cosh(x)^2$, and $\cosh(x)$ is always
  positive, so $\sqrt{1+f(x)^2}=\cosh(x)$.  This implies that 
  \[ f(x)+\sqrt{1+f(x)^2} = 
     \sinh(x) + \cosh(x) = 
     \frac{e^x+e^{-x}}{2} + \frac{e^x-e^{-x}}{2} = e^x,
  \]
  so
  \[ g(f(x)) = \log(f(x)+\sqrt{1+f(x)^2}) = \log(e^x) = x.  \]
  

 \item % int exp trig 1
  First note that $\sin(x)^2=(1-\cos(2x))/2$, so 
  \[ e^{-x}\sin(x)^2 = \half e^{-x} - \half e^{-x}\cos(2x). \]
  We know that 
  \[ \int e^{-x}\cos(2x)\,dx= e^{-x}(a\cos(2x)+b\sin(2x)) \]
  for some $a$ and $b$.  Differentiating gives
  \begin{align*}
   e^{-x}\cos(2x) &= -e^{-x}(a\cos(2x) + b\sin(2x)) + 
                     e^{-x}(-2a\sin(2x)+2b\cos(2x)) \\
    &= e^{-x}((2b-a)\cos(2x) - (b+2a)\sin(2x)),
  \end{align*}
  so $2b-a=1$ and $b+2a=0$, giving $a=-1/5$ and $b=2/5$.  Thus
  \begin{align*}
   \int e^{-x}\sin(x)^2\, dx &= 
    \half \int e^{-x}\,dx - \half \int e^{-x}\cos(2x)\,dx \\
    &= -\half e^{-x} - \half e^{-x}(-\cos(2x)/5+2\sin(2x)/5) \\
    &= \tfrac{1}{10} e^{-x} (2\sin(2x) - \cos(2x) - 5).
  \end{align*}
