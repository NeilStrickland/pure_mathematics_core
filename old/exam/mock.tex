\documentclass[12pt,twoside]{shefexam}
%\usepackage[metapost,mplabels,truebbox]{mfpic}

\newcommand{\half}      {{\textstyle\frac{1}{2}}}
\newcommand{\st}        {\;|\;}
\newcommand{\xra}       {\rightarrow}

\renewcommand{\:}{\colon}

\def\myrubric{Attempt \textbf{all} the questions. The allocation of 
marks is shown in brackets; Section A is worth 60 marks in total, and
Section B is worth 30 marks.}

\begin{document}

%\Pressmark{PMA101}
\Department{PURE MATHEMATICS}
\Title{Pure Mathematics Core --- Mock exam}
\Semester{\autumn}
\Duration{$2$ hours}
\Rubric{\myrubric}

\paper{PMA101}
\stepcounter{secnum}
\gdef\theenumi{\textbf{\sectionstyle{secnum}\arabic{enumi}}}%
\gdef\thequestion{\sectionstyle{secnum}\arabic{enumi}}%

 \question % pfrac 1
  Convert the function
  $\displaystyle f(x) = \frac{4}{(x+1)(x^2-1)}$
  to partial fraction form, and thus find $\int f(x)\,dx$.
  \marks{6}

 \question % inverse 3
  Let $f\:(-1,1)\xra{}(-1,1)$ be given by $f(x)=(5x+4)/(4x+5)$.  Find
  a formula for $f^{-1}(x)$.  (You need not check anything about the
  domain or range of $f$ or $f^{-1}$.)
  \marks{4}

 \question % composite 1
  Let $a$ and $b$ be constants, and define $f,g\:\R\xra{}\R$ by
  $f(x)=a+x$ and $g(x)=b-x$.  Find $f^{-1}(x)$, $g^{-1}(x)$ and
  $(g\circ f\circ g)(x)$.
  \marks{6}

 \question % log 2
  Simplify the expression 
  $\displaystyle{\log\left(\frac{e^a e^b}{(e^c)^d}\right)}$.
  \marks{2}

 \question % arctan
  Sketch the graphs of the functions $\tan(\theta)$ and $\sin(\theta)$.
  Find an angle $\theta$ such that $\tan(\theta)=1$ and $\sin(\theta)<0$.
  \marks{4}

 \question % identity 2 
  Show that
  $\displaystyle 8\cosh(x)^4 - 8 \cosh(x)^2 + 1 = \cosh(4x)$.
  \marks{5}

 \question % diff rat 1
  Let $a$, $b$ and $n$ be constants.  Find $f'(x)$, where 
  $f(x)=\displaystyle{\left(\frac{x-a}{x-b}\right)^n}$.
  \marks{3}

 \question % diff chain 
  Find
  $\displaystyle\frac{d}{dx}\cos\left(\left(\frac{x+1}{2}\right)^2\right)$.
  \marks{2}

 \question % diff log
  Find $\displaystyle\frac{d}{dx}\log(\cos(x))$.
  \marks{2}

 \question % diff rat 3
  Let $a$, $b$, $c$ and $d$ be constants.  Find 
  $\displaystyle{\frac{d}{dx}\left(\frac{ax+bx^{-1}}{cx+dx^{-1}}\right)}$.
  \marks{4}

 \question % diff stirling
  If $y=\sqrt{2\pi}x^{x-1/2}e^{-x}$, show that $y'/y=\log(x)-1/(2x)$.
  \marks{4}

 \question % int subs 1
  By putting $u=\log(x)$, find
  $\int\frac{(1+\log(x))^2}{x}\,dx$.
  \marks{4}

 \question % int poly exp
  Find $\int (4x^2+2x+1)e^{2x}\,dx$.
  \marks{4}

 \question % linear 2
  Find the general solution of the following system of equations:
  \begin{eqnarray*}
   y+2z &=& 1 \\ -x+3z &=& 2 \\ -2x-3y &=& 1
  \end{eqnarray*} \vspace{-2ex}
%   What can you deduce about the determinant of the matrix
%   \[ \left[\begin{array}{ccc}
%       0 & 1 & 2 \\ -1 & 0 & 3 \\ -2 & -3 & 0 
%      \end{array}\right] ?
%   \]
  \marks{6}

 \question % char sym
  Find the determinant of the following matrix, simplifying 
  your answer as much as possible.
  \[ \left[\begin{array}{ccc}
      t & a & b \\ a & t & c \\ b & c & t
     \end{array}\right]
  \]
  \marks{4}

\stepcounter{secnum}
\setcounter{enumi}{0}
\gdef\theenumi{\textbf{\sectionstyle{secnum}\arabic{enumi}}}%
\gdef\thequestion{\sectionstyle{secnum}\arabic{enumi}}%

 \question % range 1
  Let $g\:[0,\infty)\xra{}\R$ be given by
  $g(x)=\sin(\pi e^{-x}/2)$.  Find the range of $g$.
  \marks{4}

 \question % int poly trig
  Find $\int 8x\sin(x)\cos(x)\,dx$
  \marks{7}

 \question % int subs 2
  By substituting $u=x^n$, find 
  $\displaystyle\int\frac{dx}{x\sqrt{x^{-2n}-1}}$.
  \marks{7}

 \question % pfrac 4
  Find $\displaystyle{\int\frac{x^5-1}{x^2(x-1)}\,dx}$.
  \marks{6}

 \question % inverse matrix
  Find the inverse of the following matrix:
  {\small \[ \left[ \begin{array}{cccc}
    0 & 1 & 1 & 1 \\
    0 & 0 & 0 & 1 \\
    0 & 0 & 1 & 1 \\
    1 & 1 & 1 & 1 
   \end{array}\right]
  \]}
  \marks{6}

\endpaper

\end{document}


 \question % expand
  Simplify the expression
  \begin{eqnarray*}
   (aw+by)(cx+dz) &-& (cw+dy)(ax+bz) - \\
   (aw+cx)(by+dz) &+& (ay+cz)(bw+dx).
  \end{eqnarray*}
  \marks{3}



 \question % int exp trig 1
  Find $\int e^{-x}\sin(x)^2\,dx$
  \marks{9}

 \question % diff misc
  Put $f(x)=x/\sqrt{1+x^2}$.  Simplify $\sqrt{1+x^2}f'(x)$, and hence
  find a constant $c$ such that $f'(x)=(1+x^2)^c$.
  \marks{5}


 \question % pfrac 2
  Convert the function 
  \[ f(x) = \frac{x+1}{(x-1)^4} \]
  to partial fraction form, and thus find $\int f(x)\,dx$.
  \marks{6}


 \question % range 2
  Put $A=\{x\in\R\st x\neq 0 \text{ and } -1<x<1\}$.  Define
  $h\:A\xra{}\R$ by $h(x)=1/x$.  What is the range of $h$?
  \marks{6}

 \question % range 3
  Find the range of the function $f\:[0,1]\xra{}\R$ given by
  $f(x)=x-x^2$. 
  \marks{4}



 \question % inverse 2
  Define $f,g\:\R\xra{}\R$ by $f(x)=\sinh(x)$ and
  $g(x)=\log(x+\sqrt{x^2+1})$.  Check that $g(f(x))=x$.
  \marks{5}
